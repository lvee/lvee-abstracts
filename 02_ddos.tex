\documentclass[10pt, a5paper]{article}
\usepackage{ucs}
\usepackage[utf8]{inputenc}
\usepackage[T2A]{fontenc}
\usepackage[english, russian]{babel}
\usepackage{hyperref}
\usepackage{geometry}
\frenchspacing

\begin{document}

\title{Организация программной защиты от DDoS-атак }

\author{Олег Бойцев\\
\small Минск, \texttt{[Mega-Admin.com][, infosecurity@ya.ru}
}
\maketitle

\begin{abstract}
DDoS Mitigation Software Solutions. The report outlines experience in deploying software protection against DDoS attacks using OS Linux. The following  issues of organization of DDoS attacks are reviewed: the psychology of the attackers, the tools and techniques they use. In the report of the conference participants will receive an answer to the question: Is it possible to defeat DDoS without hardware tools and if so, how?
\end{abstract}

\section*{Введение}
Тема DDoS/antiDDoS бесспорно является интересной и актуальной. 

Для многих компаний предоставление сервисов и услуг  online является чуть ли не единственной формой ведения бизнеса (\url{www.mastercard.com}, \url{www.facebook.com}, \url{www.habrahabr.ru}) в связи с чем, доступность сервиса онлайн можно считать критически важным условием успешного развития компании. Простой онлайн сервиса, даже на не продолжительное время, для компании может оказаться слишком дорогим <<удовольствием>>\ldots

\section*{Экономика DDoS-атак}
Как и в реальной экономике, расчет стоимости ддос атаки на сайт определяется прежде всего соотношением спрос/предложение. Цена также определяется и <<весом>> сайта --- чем крупнее клиент, тем больше обычно просят за него.
Для примера: на момент написания тезисов доклада средняя стоимость ддос атаки на среднестатистический сайт составляла \$50. Приблизительно такова же и стоимость защиты в сутки. Стоимость покупки ботнета составляет \$1500--\$2000.

\section*{Какие DDoS-атаки бывают, какие из них используют чаще всего}
В зависимости от целей  чаще всего используют следующие типы атак:  
\begin{itemize}
\item HTTP flood,
\item TCP flood,
\item UDP/ICMP flood.
\end{itemize}

\section*{Инструменты для проведения DDoS-атак}
За последние годы инструменты проведения DDoS атак значительно видоизменились. То, что раньше было доступно узкому кругу избранным, в настоящее время поставлено на поток. Общей  тенденцией развития инструментов проведения ддос атак можно считать: использование децентрализованной структуры управления ботнетом, упрощение user-end интерфейсов управления, шифрование трафика боты >< командный центр.

\section*{Модель многоуровневой защиты от DDoS-атак}
\verb!Firewall > Linux kernel > scripts > nginx > apache!
\begin{itemize}
\item Firewall: ограничиваем количество соединений в единицу времени, режем ботов через  string module
\item Linux kernel: уменьшаем таймауты на обработку соединений, повышаем лимит общего количества возможных соединений, предел используемой памяти.
\item Scripts: режем сетевые аномалии
\item Nginx: тюнингуем, прикручиваем  limit\_conn
\item Apache: тюнингуем, прикручиваем mod-evasive
\end{itemize}

\section*{Что можно победить}
Все что не толще ширины канала сервера.

\bf{В случае если толщина DDoS-атаки больше ширина канала --- железно фильтруем трафик на входе в ДЦ.}
\end{document}




