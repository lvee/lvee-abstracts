\documentclass[10pt, a5paper]{article}
\usepackage{pdfpages}
\usepackage{parallel}
\usepackage[T2A]{fontenc}
\usepackage{ucs}
\usepackage[utf8x]{inputenc}
\usepackage[polish,english,russian]{babel}
\usepackage{hyperref}
\usepackage{rotating}
\usepackage[inner=2cm,top=1.8cm,outer=2cm,bottom=2.3cm,nohead]{geometry}
\usepackage{listings}
\usepackage{graphicx}
\usepackage{wrapfig}
\usepackage{longtable}
\usepackage{indentfirst}
\usepackage{array}
\newcolumntype{P}[1]{>{\raggedright\arraybackslash}p{#1}}
\frenchspacing
\usepackage{fixltx2e} %text sub- and superscripts
\usepackage{icomma} % коскі ў матэматычным рэжыме
\PreloadUnicodePage{4}

\newcommand{\longpage}{\enlargethispage{\baselineskip}}
\newcommand{\shortpage}{\enlargethispage{-\baselineskip}}

\def\switchlang#1{\expandafter\csname switchlang#1\endcsname}
\def\switchlangbe{
\let\saverefname=\refname%
\def\refname{Літаратура}%
\def\figurename{Іл.}%
}
\def\switchlangen{
\let\saverefname=\refname%
\def\refname{References}%
\def\figurename{Fig.}%
}
\def\switchlangru{
\let\saverefname=\refname%
\let\savefigurename=\figurename%
\def\refname{Литература}%
\def\figurename{Рис.}%
}

\hyphenation{admi-ni-stra-tive}
\hyphenation{ex-pe-ri-ence}
\hyphenation{fle-xi-bi-li-ty}
\hyphenation{Py-thon}
\hyphenation{ma-the-ma-ti-cal}
\hyphenation{re-ported}
\hyphenation{imp-le-menta-tions}
\hyphenation{pro-vides}
\hyphenation{en-gi-neering}
\hyphenation{com-pa-ti-bi-li-ty}
\hyphenation{im-pos-sible}
\hyphenation{desk-top}
\hyphenation{elec-tro-nic}
\hyphenation{com-pa-ny}
\hyphenation{de-ve-lop-ment}
\hyphenation{de-ve-loping}
\hyphenation{de-ve-lop}
\hyphenation{da-ta-ba-se}
\hyphenation{plat-forms}
\hyphenation{or-ga-ni-za-tion}
\hyphenation{pro-gramming}
\hyphenation{in-stru-ments}
\hyphenation{Li-nux}
\hyphenation{sour-ce}
\hyphenation{en-vi-ron-ment}
\hyphenation{Te-le-pathy}
\hyphenation{Li-nux-ov-ka}
\hyphenation{Open-BSD}
\hyphenation{Free-BSD}
\hyphenation{men-ti-on-ed}
\hyphenation{app-li-ca-tion}

\def\progref!#1!{\texttt{#1}}
\renewcommand{\arraystretch}{2} %Іначай формулы ў матрыцы зліпаюцца з лініямі
\usepackage{array}

\def\interview #1 (#2), #3, #4, #5\par{

\section[#1, #3, #4]{#1 -- #3, #4}
\def\qname{LVEE}
\def\aname{#1}
\def\q ##1\par{{\noindent \bf \qname: ##1 }\par}
\def\a{{\noindent \bf \aname: } \def\qname{L}\def\aname{#2}}
}

\def\interview* #1 (#2), #3, #4, #5\par{

\section*{#1\\{\small\rm #3, #4. #5}}

\def\qname{LVEE}
\def\aname{#1}
\def\q ##1\par{{\noindent \bf \qname: ##1 }\par}
\def\a{{\noindent \bf \aname: } \def\qname{L}\def\aname{#2}}
}


\begin{document}

\title{Организация программной защиты от DDoS-атак }

\author{Олег Бойцев\footnote{Минск, [Mega-Admin.com][, \url{infosecurity@ya.ru}}}
\maketitle

\begin{abstract}
DDoS Mitigation Software Solutions. The report outlines experience in deploying software protection against DDoS attacks using OS Linux. The following  issues of organization of DDoS attacks are reviewed: the psychology of the attackers, the tools and techniques they use. In the report of the conference participants will receive an answer to the question: Is it possible to defeat DDoS without hardware tools and if so, how?
\end{abstract}

\section*{Введение}
Тема DDoS/antiDDoS бесспорно является интересной и актуальной. 

Для многих компаний предоставление сервисов и услуг  online является чуть ли не единственной формой ведения бизнеса (\url{www.mastercard.com}, \url{www.facebook.com}, \url{www.habrahabr.ru}) в связи с чем, доступность сервиса онлайн можно считать критически важным условием успешного развития компании. Простой онлайн сервиса, даже на не продолжительное время, для компании может оказаться слишком дорогим <<удовольствием>>\ldots

\section*{Экономика DDoS-атак}
Как и в реальной экономике, расчет стоимости DDoS-атаки на сайт определяется прежде всего соотношением спрос/предложение. Цена также определяется и <<весом>> сайта "--- чем крупнее клиент, тем больше обычно просят за него.
Для примера: на момент написания тезисов доклада средняя стоимость ддос атаки на среднестатистический сайт составляла \$50. Приблизительно такова же и стоимость защиты в сутки. Стоимость покупки ботнета составляет \$1500--\$2000.

\section*{Какие DDoS-атаки бывают, какие из них используют чаще всего}
В зависимости от целей  чаще всего используют следующие типы атак:  
\begin{itemize}
\item HTTP flood,
\item TCP flood,
\item UDP/ICMP flood.
\end{itemize}

\section*{Инструменты для проведения DDoS-атак}
За последние годы инструменты проведения DDoS-атак значительно видоизменились. То, что раньше было доступно узкому кругу избранным, в настоящее время поставлено на поток. Общей  тенденцией развития инструментов проведения ддос атак можно считать: использование децентрализованной структуры управления \linebreak ботнетом, упрощение user-end интерфейсов управления, шифрование трафика боты "--- командный центр.

\section*{Модель многоуровневой защиты от DDoS-атак}
Firewall → Linux kernel → scripts → nginx → apache
\begin{itemize}
\item Firewall: ограничиваем количество соединений в единицу времени, режем ботов через  string module
\item Linux kernel: уменьшаем таймауты на обработку соединений, повышаем лимит общего количества возможных соединений, предел используемой памяти.
\item Scripts: режем сетевые аномалии
\item Nginx: тюнингуем, прикручиваем  limit\_conn
\item Apache: тюнингуем, прикручиваем mod-evasive
\end{itemize}

\section*{Что можно победить}
Все что не толще ширины канала сервера.

В случае если толщина DDoS-атаки больше ширина канала "--- железно фильтруем трафик на входе в ДЦ.
\end{document}




