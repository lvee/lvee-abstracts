\documentclass[10pt, a5paper]{article}
\usepackage{pdfpages}
\usepackage{parallel}
\usepackage[T2A]{fontenc}
\usepackage{ucs}
\usepackage[utf8x]{inputenc}
\usepackage[polish,english,russian]{babel}
\usepackage{hyperref}
\usepackage{rotating}
\usepackage[inner=2cm,top=1.8cm,outer=2cm,bottom=2.3cm,nohead]{geometry}
\usepackage{listings}
\usepackage{graphicx}
\usepackage{wrapfig}
\usepackage{longtable}
\usepackage{indentfirst}
\usepackage{array}
\newcolumntype{P}[1]{>{\raggedright\arraybackslash}p{#1}}
\frenchspacing
\usepackage{fixltx2e} %text sub- and superscripts
\usepackage{icomma} % коскі ў матэматычным рэжыме
\PreloadUnicodePage{4}

\newcommand{\longpage}{\enlargethispage{\baselineskip}}
\newcommand{\shortpage}{\enlargethispage{-\baselineskip}}

\def\switchlang#1{\expandafter\csname switchlang#1\endcsname}
\def\switchlangbe{
\let\saverefname=\refname%
\def\refname{Літаратура}%
\def\figurename{Іл.}%
}
\def\switchlangen{
\let\saverefname=\refname%
\def\refname{References}%
\def\figurename{Fig.}%
}
\def\switchlangru{
\let\saverefname=\refname%
\let\savefigurename=\figurename%
\def\refname{Литература}%
\def\figurename{Рис.}%
}

\hyphenation{admi-ni-stra-tive}
\hyphenation{ex-pe-ri-ence}
\hyphenation{fle-xi-bi-li-ty}
\hyphenation{Py-thon}
\hyphenation{ma-the-ma-ti-cal}
\hyphenation{re-ported}
\hyphenation{imp-le-menta-tions}
\hyphenation{pro-vides}
\hyphenation{en-gi-neering}
\hyphenation{com-pa-ti-bi-li-ty}
\hyphenation{im-pos-sible}
\hyphenation{desk-top}
\hyphenation{elec-tro-nic}
\hyphenation{com-pa-ny}
\hyphenation{de-ve-lop-ment}
\hyphenation{de-ve-loping}
\hyphenation{de-ve-lop}
\hyphenation{da-ta-ba-se}
\hyphenation{plat-forms}
\hyphenation{or-ga-ni-za-tion}
\hyphenation{pro-gramming}
\hyphenation{in-stru-ments}
\hyphenation{Li-nux}
\hyphenation{sour-ce}
\hyphenation{en-vi-ron-ment}
\hyphenation{Te-le-pathy}
\hyphenation{Li-nux-ov-ka}
\hyphenation{Open-BSD}
\hyphenation{Free-BSD}
\hyphenation{men-ti-on-ed}
\hyphenation{app-li-ca-tion}

\def\progref!#1!{\texttt{#1}}
\renewcommand{\arraystretch}{2} %Іначай формулы ў матрыцы зліпаюцца з лініямі
\usepackage{array}

\def\interview #1 (#2), #3, #4, #5\par{

\section[#1, #3, #4]{#1 -- #3, #4}
\def\qname{LVEE}
\def\aname{#1}
\def\q ##1\par{{\noindent \bf \qname: ##1 }\par}
\def\a{{\noindent \bf \aname: } \def\qname{L}\def\aname{#2}}
}

\def\interview* #1 (#2), #3, #4, #5\par{

\section*{#1\\{\small\rm #3, #4. #5}}

\def\qname{LVEE}
\def\aname{#1}
\def\q ##1\par{{\noindent \bf \qname: ##1 }\par}
\def\a{{\noindent \bf \aname: } \def\qname{L}\def\aname{#2}}
}


\begin{document}

\title{Способы повышения производительности высоконагруженных проектов на CMS Drupal}

\author{Виталий Иоскевич\footnote{Минск, Беларусь}}
\date{}
\maketitle
\renewcommand{\abstractname}{Abstract}
\begin{abstract}
Drupal is a popular CMS/CMF used to power the different types of web-projects worldwide. The desire of developers to make their system competitive in terms of functionality and flexibility sooner or later leads to increase of server load and reduced performance. Extensive functionality, flexibility in configu\-ration and extensibility through third"=party modules are undoubtedly a huge advantage, enabling to carry out projects of varying complexity and purpose, but the price often is a poor perfor\-mance of high loaded Drupal-powered sites.
In our review we are going to revise existing standard methods of Drupal perfor\-mance optimi\-za\-tion (both server- and client-side) and  demonstrate cus\-tom solutions that can be applied to some of the critical areas of Drupal"=powered high"=loaded website. In our real"=life example we are going to show how to build Google Map page, capable to show thousands of markers based on Drupal content nodes.
\end{abstract}
  
Drupal (7-ая, текущая версия) является достаточно развитой системой управления контентом. Стремление разработчиков сделать свою систему универсальной и более функциональной, чем у конкурентов, рано или поздно приводит к тому, что возрастают нагрузки на сервер, снижается быстродействие. Обширный функционал, гибкость в конфигурировании и расширяемость посредством сторонних модулей несомненно являются огромным преимуществом, благодаря которому позволяет выполнять проекты различной сложности и назначения, но плата за него --- низкая производительность готового решения. 

Чем больше число использованных на сайте сторонних модулей (в т.~ч. собственной разработки), тем ниже производительность сайта при больших нагрузках (большом количестве посетителей), а именно при построении сайтов с предполагаемым большим числом  посетителей никак не обойтись стандартным функционалом CMS. Таким образом, данная CMS оказывается весьма требовательной к ресурсам сервера, и в большинстве случаев проблема  решается выбором специализированных дорогих хостингов --- выделенный сервер или несколько серверов. 

Но и при ограниченных возможностях сервера есть способы улучшения производительности сайта. Стандартной системы кэширования Drupal достаточно для нормальной работы среднего сайта при неинтенсивной посещаемости, однако для случая высокой нагрузки на сайт этой системы недостаточно.

\section*{Основные методы повышения производительности Drupal:}
\begin{itemize}
	\item включить кэш в модулях, позволяющих кэширование (таких как Views, Panels, Feeds, SWF Tools и т.п.);
	\item увеличить время жизни кэша для стандартной системы кэширования Drupal. При этом надо иметь ввиду, что посетители будут видеть обновления содержимого значительно позже;
	\item если нет большой необходимости в статистике, можно выключить стандартные модули statistics и database logging. Они используют дополнительные запросы к БД при загрузке страницы. Кроме того, можно найти альтернативу данным модулям;
	\item использовать модуль CSS Gzip. Уменьшает размер файлов и количество http-запросов;
	\item использовать модуль Global Redirect. Предотвращает кэширование дубликатов содержимого сайта по ссылкам псевдонимам (синонимам) в случае использования <<Clean URL>>;
	\item использовать модуль Javascript Aggregator. Настройки Drupal позволяют объединить js-файлы, однако этот модуль позволяет еще и уменьшить конечный размер передаваемого файла;
	\item заменить системный cron на более функциональный модуль Elysia Cron. В отличие от системного, Elysia Cron позволяет распределить задачи по времени одна за одной, а не все вместе, при этом снижая пиковую нагрузку на сервер;
	\item перенести вызов javascript вниз станицы;
	\item помещать сторонние библиотеки вне каталога модулей \linebreak /sites/all/modules "--- например, в /sites/all/libraries. В таком \linebreak случае Drupal не будет просматривать ненужные ему файлы;
	\item конвертировать таблицы MySQL в UTF8 InnoDB;
	\item создать индексы для медленных запросов Views. Желательно проанализировать такие запросы и увеличить производительность созданием индексов для полей;
	\item использовать ImageMagick вместо стандартной GD image \linebreak library, т.~к. он лучше использует ресурсы сервера и дает лучшее качество изображений;
	\item использовать модуль Content Delivery Network integration. Он позволяет распределить файлы на множество разделенных серверов. При этом снижается динамическая нагрузка на сервер, но требуется дополнительная плата за каждый сервер;
\end{itemize}

Наиболее продуктивный способ увеличения производительности "--- отдельная эффективная система кэширования.

{\bf Boost}. Одна из лучших систем кэширования для shared hosts. Cоздает копии html-станиц, которые генерирует Drupal, и хранит их в каталоге cache. Используя правила <<.htaccess>>, Boost проверяет, существует ли необходимый файл. Если да, то загружает статичный html, полностью избегая Drupal/PHP/MySQL. Если нет,  генерирует файл. Старые файлы удаляются по cron для того, чтобы выводимое содержимое сайта оставалось свежим (актуальным).

{\bf Memcache}. Серверная система кэширования, сохраняющая \linebreak страницы сайта в оперативной памяти сервера. Все что необходимо для эффективной работы в данном случае --- быстрое ОЗУ большого объема. Его можно успешно использовать как для кэширования анонимных пользователей, так и для зарегистрированных. Но возможен и вариант, когда Memcache кэширует зарегистрированных пользователей, а Boost используется для кэширования анонимов. К преимуществам варианта можно отнести возможность кэширования анонимов и зарегистрированных пользователей, а к недостаткам --- то, что нужен хотя бы VPS, c достаточной оперативной памятью и невозможность работы на shared hosting.

{\bf APC(Alternate PHP Cache)}. Система кэширования op-code для серверов Apache. Подход весьма эффективен при большом числе активных модулей на сайте. Суть системы заключается в кэшировании компилированного байт-кода PHP-скрипта для того, чтобы избежать расходов ресурсов сервера на разбор и компиляцию исходного кода при каждом запросе. Для дальнейшего повышения производительности кэшированный код хранится в общей памяти и непосредственно выполняется оттуда, тем самым сводя к минимуму количество медленных операций чтения файлов и копирования в памяти во время выполнения.
К преимуществам относится ускорение компиляции скриптов Drupal и уменьшение использования ресурсов сервера, к недостаткам --- необходимость произвести специальное конфигурирование сервера.

{\bf Varnish (reverse proxy http-accelerator)}. Данная система выступает на первом плане над сервером Apache, PHP и Drupal. \linebreak Varnish сохраняет кэшированное содержимое в оперативной памяти, и таким образом экономит ресурсы на загрузке Apache и Drupal. В результате этого Varnish предоставляет высокую производительность для кэшированных страниц для анонимных пользователей и является предпочтительным вариантом для сложных ресурсоемких сайтов, требующих значительной  масштабируемости. Для работы с Drupal 6 необходима была отдельная <<оптимизированная>> сборка дистрибутива под названием PressFlow и установка модуля интеграции Varnish HTTP Accelerator Integration. Для текущей версии Drupal достаточно установки модуля.

\section*{Пример проекта с высокой нагрузкой --- 350.org (Drupal 6)}

К характеристикам проекта можно отнести следующие: 
\begin{itemize}
	\item 150 000+  уникальных посетителей в день
	\item 6000+ ивентов, созданных пользователями
	\item ок. 500 авторизованных пользователей постоянно на сайте
	\item самая посещаемая страница: карта ивентов (/map)
\end{itemize}

Кэширование критично как для анонимов, так и для авторизованных пользователей. Используется Pressflow/Varnish и специальные меры для повышения производительности:
\longpage
\begin{itemize}
	\item Отказ от Views для генерации больших массивов данных
	\item XML"=файл на диск вместо DB"=кэша (плюсы/минусы)
	\item Кэширование на диск всех RSSЭ=фидов
	\item Ajax"=запросы информации для маркеров на карте.
\end{itemize}

\end{document}




