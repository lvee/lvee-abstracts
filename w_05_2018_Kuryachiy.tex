\documentclass[10pt, a5paper]{article}
\usepackage[T2A]{fontenc}
\usepackage{ucs}
\usepackage[utf8x]{inputenc}
\usepackage[polish,english,russian]{babel}
\usepackage{hyperref}
\usepackage[inner=2cm,top=1.8cm,outer=2cm,bottom=2.3cm,nohead]{geometry}
\usepackage{listings}
\usepackage{graphicx}
\usepackage{wrapfig}
\usepackage{longtable}
\usepackage{indentfirst}
\frenchspacing
\usepackage{fixltx2e} %text sub- and superscripts
\usepackage{icomma} % коскі ў матэматычным рэжыме
\PreloadUnicodePage{4}

\newcommand{\longpage}{\enlargethispage{\baselineskip}}
\newcommand{\shortpage}{\enlargethispage{-\baselineskip}}

\def\switchlang#1{\expandafter\csname switchlang#1\endcsname}
\def\switchlangbe{
\let\saverefname=\refname%
\def\refname{Літаратура}%
\def\figurename{Іл.}%
}
\def\switchlangen{
\let\saverefname=\refname%
\def\refname{References}%
\def\figurename{Fig.}%
}
\def\switchlangru{
\let\saverefname=\refname%
\let\savefigurename=\figurename%
\def\refname{Литература}%
\def\figurename{Рис.}%
}

\hyphenation{admi-ni-stra-tive}
\hyphenation{ex-pe-ri-ence}
\hyphenation{fle-xi-bi-li-ty}
\hyphenation{Py-thon}
\hyphenation{ma-the-ma-ti-cal}
\hyphenation{re-ported}
\hyphenation{imp-le-menta-tions}
\hyphenation{pro-vides}
\hyphenation{en-gi-neering}
\hyphenation{com-pa-ti-bi-li-ty}
\hyphenation{im-pos-sible}
\hyphenation{desk-top}
\hyphenation{elec-tro-nic}
\hyphenation{com-pa-ny}
\hyphenation{de-ve-lop-ment}
\hyphenation{de-ve-loping}
\hyphenation{de-ve-lop}
\hyphenation{da-ta-ba-se}
\hyphenation{plat-forms}
\hyphenation{or-ga-ni-za-tion}
\hyphenation{pro-gramming}
\hyphenation{in-stru-ments}
\hyphenation{Li-nux}
\hyphenation{en-vi-ron-ment}
\hyphenation{Te-le-pathy}
\hyphenation{Li-nux-ov-ka}

\def\progref!#1!{\texttt{#1}}
\renewcommand{\arraystretch}{2} %Іначай формулы ў матрыцы зліпаюцца з лініямі
\usepackage{array}

\def\interview #1 (#2), #3, #4, #5\par{

\section[#1, #3, #4]{#1, #5}
\def\qname{LVEE}
\def\aname{#1}
\def\q ##1\par{{\noindent \bf \qname: ##1 }\par}
\def\a{{\noindent \bf \aname: } \def\qname{L}\def\aname{#2}}
}

%\switchlang{ru}
\begin{document}
\title{Инструментальная поддержка преподавания дисциплины <<Архитектура ЭВМ и язык ассемблера>> на ВМК МГУ\footnote{\url{frbrgeorge@gmail.com}, \url{https://lvee.org/en/abstracts/267}}}
\author{Курячий Георгий Владимирович, <<Базальт СПО>>; \\ Рудаченко Михаил Евгеньевич, ГБС РАН}
\maketitle
\begin{abstract}
Moscow State University CMC department basic course <<Com\-puter Architecture and Assembly Language>> has two official implementations: one pretty outdated (16bit based) and one too complex (x86\underline{ }64 based). In pursuit of simplicity, actuality and practical support we have developed, approved and used a course based on MIPS32 architecture using MARS simulator as practice platform, along with semi-authomatic homework verification.\\ Although successful, this approach opened a list of technological challenges, like to implement some modern features in simulator or develop an illustrative tool showing that features at real hard\-ware.
\end{abstract}

\subsection*{О курсе <<Архитектура ЭВМ и язык ассемблера>>}

Это базовая дисциплина, которую читают на первом курсе во втором семестре на факультете ВМК МГУ. Плотность достаточная (4 часа теории + 4 часа практики в неделю), но объём сведений, которые хочется (следует!) туда втиснуть, год от года растёт.

\textbf{Цель}~--- сформировать понимание, что такое <<архитектуры \\ЭВМ>> и почему они такие.

\textbf{Задачи}~--- теория + фактология + практика:

\begin{itemize}
  \item Изучить основы логической организации ЭВМ; смоделировать и сравнить несколько различных архитектур
  \item Рассмотреть современное состояние ЭВМ и принципы, по которым развивается их архитектура, например:
\begin{itemize}
  \item конвенции и ABI;
  \item базис: система команд, виды адресации, регистры, подпрограммы, флаги и т. д.
  \item системные вызовы, прерывания, ловушки; внешние устройства, порты, MMIO, DMA и прочее; сопроцессоры;
  \item аппаратная оптимизация~--- кеш, конвейер, упреждающие вычисления и зависимости вычислительных потоков;
  \item аппаратная изоляция: режимы процессора, виртуальная память, виртуализация;
\end{itemize}


  \item Освоить язык ассемблера, в котором по возможности эти принципы применяются на практике
  \item Изучить некоторые приёмы низкоуровневого программирования и моделирования данных
\end{itemize}

К сожалению, два варианта этого базового курса, на наш взгляд, упираются в две крайности: один воспроизводит курс прошлого тысячелетия по 16-битной архитектуре (\textbf{MASM6} + \textbf{DOS}), другой основан на \textbf{nasm} и \textbf{x86\underline{ }64}. Сейчас готовится новый курс на базе \textbf{Masm32} и \textbf{Windows XP}, сочетающий, на наш взгляд, худшие качества первых двух вариантов~--- моральную и актуальную устарелость и усложнённость изучаемого материала.

\subsection*{Выбор архитектуры}

Мы провели много времени в спорах, что же взять за основу~--- идеальный учебный процессор в вакууме или настоящее живое железо, в котором пропустить всё лишнее. В итоге разработку идеального процессора мы оставили команде RiscV\footnote{\url{http://riscv.org}}, а сами пришли к такому компромиссу:

\begin{itemize}
  \item В качестве модельных машин использовать факультетские наработки\footnote{\url{http://al.cs.msu.su/files/ModComp.pdf}}, только дописать к ним эмулятор\footnote{\url{https://github.com/vslutov/modelmachine}}.
  \item В качестве базовой архитектуры выбрать MIPS32, как наиболее подходящую по соотношению <<современность>>/\linebreak<<адекватность системы команд>>. В виде бонуса мы получаем простой и понятный конвейер. Российские реалии: процессор <<Байкал Т>> имеет архитектуру MIPS32.
  \item В качестве среды программирования выбрать MIPS Assembler and Runtime Simulator\footnote{\url{http://courses.missouristate.edu/KenVollmar/mars/}} (MARS). Недостатки того, что это эмулятор, компенсируются наглядностью:

  \item Если хватит времени/интеллектуальных усилий, прокинуть мостик между языком ассемблера и Си в стиле <<Си~--- это такой суперудобный макроассемблер>>, перейдя либо на настоящее железо, либо на полный эмулятор Qemu\footnote{\url{http://qemu.org}}.
\end{itemize}

\subsection*{Инструменты}

Базовой LMS у нас на факультете является Moodle\footnote{\url{https://moodle.cs.msu.ru/course/view.php?id=42}}. В основном использовались три вида модулей~--- урок, задание, форум-семинар и контрольная. В контрольных мы старались давать параметрические задачи.

Для изучения модельных машин Владимир Лютов написал эмулятор\footnote{\url{https://github.com/vslutov/modelmachine}}, программы для которого задаются в шестнадцатеричных кодах. Эмулятор поддерживает пять архитектур из рассмотренных в методичке\footnote{\url{http://al.cs.msu.su/files/ModComp.pdf}}, отладчик, а также простой ассемблер для одной из них. Написан на Python.

Эмулятор MARS оказался неплох (правда, только в своей области):

\begin{itemize}
  \item Простая IDE
  \item Отладчик и дизассемблер, редактора памяти, регистров, сопроцессора
  \item поддержка текстового В/В и выполнения программы в режиме эмулятора без UI
  \item Поддержка нескольких простейших внешних устройства, в т. ч. работающих по прерываниям и растрового графического дисплея
  \item Наглядные модули, описывающие работу конвейера, кеша и предсказателя перехода
\end{itemize}

Поскольку практических заданий было много (в среднем, 3 к лекции), в какой-то момент в их проверке подключилась система проведения олимпиад Ejudge\footnote{\url{https://ejudge.ru/}}. И модельные машины, и MARS хорошо встраиваются в EJudge, но в силу простоты и краткости программ для первых, мы решили ограничиться автоматической проверкой только программ для Mars.

Пришлось разработать три варианта <<обвязки>> Д/З\footnote{\url{https://moodle.cs.msu.ru/mod/page/view.php?id=1877}}:

\begin{itemize}
  \item типа <<из памяти в память>>, когда входные и выходные данные лежат в заданных местах памяти, а необходимый для EJudge ввод-вывод делает программа-footer
  \item типа\,<<полная программа>>\,(после\,изучения\,темы\,<<ввод-вывод>>)
  \item типа <<подпрограмма>> (после изучения темы <<подпрограммы и конвенции по передаче параметров>>), в которой программа-footer, помимо формирования ввода и вывода, проверяла соблюдение конвенций по вызову
\end{itemize}

Задачи для контрольных каждый студент получал так: скачивал поргамму-гененратор (на Python), запускал её и получал текст условия и <<номер варианта>>. Контрольные также пришлось проверять вручную.

\subsection*{Выводы}

Модельные машины в том виде, в котором они описаны в методичке, слишком похожи друг на друга, и не похожи на MIPS. Возможно, стоит отказаться от конкретно этого варианта и разработать более общий фреймворк, позволяющий моделировать более широкий спектр архитектур, причём так, чтобы плавно переходить к системе команд MIPS32.

Следующие темы отсутствуют (ещё не запрограммированы или принципиально невозможны) в MARS:

\begin{itemize}
  \item Виртуальная память
  \item Многопроцессорность
  \item DMA и иные контроллеры
  \item Упреждающие вычисления, псевдоскалярность, векторность и микропрограммы
  \item Виртуализация
\end{itemize}

Все такие темы пришлось изучать <<всухую>>, есть подозрение, что контроль остаточных знаний по ним разочарует. Некоторые из этих тем можно проиллюстрировать, написав соответствующие модули для MARS (это несложно), а некоторые~--- нет.

Курс читался дважды. Оба раза в основное время к теме про Си вырулить не удалось за недостатком ресурсов (времени и плотности материала). Был подготовлен факультативный курс по Си на базе виртуальной машины с Linux и эмулятором Qemu-user-mipsel для изучения сгенерированного кода. К сожалению, формат курса (форум) оказался неудобным для изложения, потому что по факту получился учебник; требуются ресурсы для превращения одного во второе.

Вполне возможно, что стратегически правильное решение~--- отказаться от MIPS в пользу RiscV, как доступной, достаточно несложной, перспективной и ультрасовременной архитектуры. Ранее останавливало отсутствие наглядных инструментов, наподобие MARS, что там сейчас творится~--- надо исследовать.

\begin{thebibliography}{20}
\bibitem{Kuryachiy-1} Рудаченко М. Е. Свободный эмулятор MARS в курсе <<Архитектура ЭВМ и язык ассемблера>>. // Материалы XXII конференции <<Свободные программы в высшей школе>>.
\end{thebibliography}

\end{document}
