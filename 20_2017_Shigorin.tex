\documentclass[10pt, a5paper]{article}
\usepackage[T2A]{fontenc}
\usepackage{ucs}
\usepackage[utf8x]{inputenc}
\usepackage[polish,english,russian]{babel}
\usepackage{hyperref}
\usepackage[inner=2cm,top=1.8cm,outer=2cm,bottom=2.3cm,nohead]{geometry}
\usepackage{listings}
\usepackage{graphicx}
\usepackage{wrapfig}
\usepackage{longtable}
\usepackage{indentfirst}
\frenchspacing
\usepackage{fixltx2e} %text sub- and superscripts
\usepackage{icomma} % коскі ў матэматычным рэжыме
\PreloadUnicodePage{4}

\newcommand{\longpage}{\enlargethispage{\baselineskip}}
\newcommand{\shortpage}{\enlargethispage{-\baselineskip}}

\def\switchlang#1{\expandafter\csname switchlang#1\endcsname}
\def\switchlangbe{
\let\saverefname=\refname%
\def\refname{Літаратура}%
\def\figurename{Іл.}%
}
\def\switchlangen{
\let\saverefname=\refname%
\def\refname{References}%
\def\figurename{Fig.}%
}
\def\switchlangru{
\let\saverefname=\refname%
\let\savefigurename=\figurename%
\def\refname{Литература}%
\def\figurename{Рис.}%
}

\hyphenation{admi-ni-stra-tive}
\hyphenation{ex-pe-ri-ence}
\hyphenation{fle-xi-bi-li-ty}
\hyphenation{Py-thon}
\hyphenation{ma-the-ma-ti-cal}
\hyphenation{re-ported}
\hyphenation{imp-le-menta-tions}
\hyphenation{pro-vides}
\hyphenation{en-gi-neering}
\hyphenation{com-pa-ti-bi-li-ty}
\hyphenation{im-pos-sible}
\hyphenation{desk-top}
\hyphenation{elec-tro-nic}
\hyphenation{com-pa-ny}
\hyphenation{de-ve-lop-ment}
\hyphenation{de-ve-loping}
\hyphenation{de-ve-lop}
\hyphenation{da-ta-ba-se}
\hyphenation{plat-forms}
\hyphenation{or-ga-ni-za-tion}
\hyphenation{pro-gramming}
\hyphenation{in-stru-ments}
\hyphenation{Li-nux}
\hyphenation{en-vi-ron-ment}
\hyphenation{Te-le-pathy}
\hyphenation{Li-nux-ov-ka}

\def\progref!#1!{\texttt{#1}}
\renewcommand{\arraystretch}{2} %Іначай формулы ў матрыцы зліпаюцца з лініямі
\usepackage{array}

\def\interview #1 (#2), #3, #4, #5\par{

\section[#1, #3, #4]{#1, #5}
\def\qname{LVEE}
\def\aname{#1}
\def\q ##1\par{{\noindent \bf \qname: ##1 }\par}
\def\a{{\noindent \bf \aname: } \def\qname{L}\def\aname{#2}}
}

\switchlang{ru}
\begin{document}
\title{<<Эльбрус>> на альте}
\author{Михаил Шигорин, Москва, Russian Federation \footnote{\url{mike@altlinux.org}, \url {https://lvee.org/ru/abstracts/251}}}
\maketitle
\begin{abstract}
The report covers state of progress of porting ALT Linux to Elbrus machines, which is currently self-hosted and targeted at rebuilding the Sisiphus package base.
\end{abstract}
Как упоминалось в докладе прошлого года \cite{Shigorin1}, в <<Базальт СПО>> начали эксперименты по переносу своего дистрибутива на новую аппаратную платформу e2k в виде рабочей станции <<Эльбрус-401>>. Ранней весной 2017 года она уже была переведена на загрузку с альтовского корневого раздела, и в основном начальная <<раскрутка>> была завершена с формированием пакетного репозитория ёмкостью более 1200 пакетов с исходным кодом в основном из Sisyphus. На данный момент число пакетов уже превысило 1500.

Изначально работа велась в chroot под управлением штатной ОС Эльбрус (OSL). После пересборки репозитория в hasher \cite{Shigorin1} и запуска mkimage вместе с mkimage-profiles \cite{Shigorin2} получилось изготовить архив корневой файловой системы, каковой и был в итоге развёрнут на отдельном диске. Отдельное спасибо разработчикам удобной в этом плане базовой прошивки, умеющей загружать ядро и образ initrd с файловой системы Ext2, что позволяет отказаться от GRUB. 

По мере расширения репозитория менялись и проблемы, с которыми приходилось иметь дело --- от жёстко заданного пути самых первых шагов к минимальной возможности подумать о том, куда и как двигаться дальше, а затем --- с выходом на <<оперативный простор>> базовых сборочных зависимостей --- скорее <<какие подсистемы в каком порядке удобней брать в работу>> (Qt? Java?).

Судя по текущему положению дел, в итоге получается первая собранная на нынешнем <<Эльбрусе>> без применения кросс"=компиляции операционная система общего назначения.

На момент подачи тезисов (начало лета 2017 года) сборка переносится на четырёхпроцессорный <<Эльбрус 4.4>> с практически линейным ростом скорости сборки --- и эта машина, разумеется, тоже загружена под альтом c уже собранным нами ядром.

\begin{thebibliography}{99}
  \bibitem{Shigorin1} Шигорин М. Альт на <<Эльбрусе>>. Материалы LVEE Winter 2016. \url{http://lvee.org/ru/abstracts/180}
  \bibitem{Shigorin2} Шигорин М. Макраме из дистрибутивов: mkimage-profiles. Материалы LVEE 2012 \url{https://lvee.org/ru/reports/LVEE_2012_11}
\end{thebibliography}

\end{document}
