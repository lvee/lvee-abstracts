\documentclass[10pt, a5paper]{article}
\usepackage{pdfpages}
\usepackage{parallel}
\usepackage[T2A]{fontenc}
\usepackage{ucs}
\usepackage[utf8x]{inputenc}
\usepackage[polish,english,russian]{babel}
\usepackage{hyperref}
\usepackage{rotating}
\usepackage[inner=2cm,top=1.8cm,outer=2cm,bottom=2.3cm,nohead]{geometry}
\usepackage{listings}
\usepackage{graphicx}
\usepackage{wrapfig}
\usepackage{longtable}
\usepackage{indentfirst}
\usepackage{array}
\newcolumntype{P}[1]{>{\raggedright\arraybackslash}p{#1}}
\frenchspacing
\usepackage{fixltx2e} %text sub- and superscripts
\usepackage{icomma} % коскі ў матэматычным рэжыме
\PreloadUnicodePage{4}

\newcommand{\longpage}{\enlargethispage{\baselineskip}}
\newcommand{\shortpage}{\enlargethispage{-\baselineskip}}

\def\switchlang#1{\expandafter\csname switchlang#1\endcsname}
\def\switchlangbe{
\let\saverefname=\refname%
\def\refname{Літаратура}%
\def\figurename{Іл.}%
}
\def\switchlangen{
\let\saverefname=\refname%
\def\refname{References}%
\def\figurename{Fig.}%
}
\def\switchlangru{
\let\saverefname=\refname%
\let\savefigurename=\figurename%
\def\refname{Литература}%
\def\figurename{Рис.}%
}

\hyphenation{admi-ni-stra-tive}
\hyphenation{ex-pe-ri-ence}
\hyphenation{fle-xi-bi-li-ty}
\hyphenation{Py-thon}
\hyphenation{ma-the-ma-ti-cal}
\hyphenation{re-ported}
\hyphenation{imp-le-menta-tions}
\hyphenation{pro-vides}
\hyphenation{en-gi-neering}
\hyphenation{com-pa-ti-bi-li-ty}
\hyphenation{im-pos-sible}
\hyphenation{desk-top}
\hyphenation{elec-tro-nic}
\hyphenation{com-pa-ny}
\hyphenation{de-ve-lop-ment}
\hyphenation{de-ve-loping}
\hyphenation{de-ve-lop}
\hyphenation{da-ta-ba-se}
\hyphenation{plat-forms}
\hyphenation{or-ga-ni-za-tion}
\hyphenation{pro-gramming}
\hyphenation{in-stru-ments}
\hyphenation{Li-nux}
\hyphenation{sour-ce}
\hyphenation{en-vi-ron-ment}
\hyphenation{Te-le-pathy}
\hyphenation{Li-nux-ov-ka}
\hyphenation{Open-BSD}
\hyphenation{Free-BSD}
\hyphenation{men-ti-on-ed}
\hyphenation{app-li-ca-tion}

\def\progref!#1!{\texttt{#1}}
\renewcommand{\arraystretch}{2} %Іначай формулы ў матрыцы зліпаюцца з лініямі
\usepackage{array}

\def\interview #1 (#2), #3, #4, #5\par{

\section[#1, #3, #4]{#1 -- #3, #4}
\def\qname{LVEE}
\def\aname{#1}
\def\q ##1\par{{\noindent \bf \qname: ##1 }\par}
\def\a{{\noindent \bf \aname: } \def\qname{L}\def\aname{#2}}
}

\def\interview* #1 (#2), #3, #4, #5\par{

\section*{#1\\{\small\rm #3, #4. #5}}

\def\qname{LVEE}
\def\aname{#1}
\def\q ##1\par{{\noindent \bf \qname: ##1 }\par}
\def\a{{\noindent \bf \aname: } \def\qname{L}\def\aname{#2}}
}

\begin{document}
\title{О методах динамического встраивания в ядро операционной системы (на примере Linux)}
\author{Илья Матвейчиков, Москва, РФ\footnote{\url{matveychikov@gmail.com}, \url{http://lvee.org/en/abstracts/123}}}
\maketitle
\begin{abstract}
The article presents an overview of methods for dynamic integ\-ration into the Linux kernel to modify (add, change) its functio\-nality. Both traditional methods of integration based on changing kernel's code, such as patching, and methods based on using other capabilities are considered. Special attention is paid to bypassing integrity mechanisms while doing the interception. Data invalidation method is proposed.
\end{abstract}
\subsection*{Постановка задачи}

Под встраиванием в программную систему понимается процесс внедрения в неё дополнительных (сторонних) программных элементов, осуществляемый таким образом, чтобы с одной стороны сохранялось её функционирование, а с другой "--- расширялись или изменялись её функциональные возможности.

Говоря о встраивании, будем рассматривать следующую практическую задачу. Пусть есть некоторый целевой компонент программной системы, функционирующий в соответствии с заданным (базовым) алгоритмом. Необходимо осуществить модификацию работы данного компонента таким образом, чтобы иметь возможность вносить конкретные изменения в этот базовый алгоритм.

Таким образом, основной целью встраивания является получение возможности контроля, модификации и расширения функций компонентов программной системы, тогда как основной задачей встраивания является обеспечение внедрения в существующую программную систему при сохранении её работоспособности. Успешно выполненное встраивание характеризуется сохранением функционирования программной системы при наличии в её составе нового компонента.

\subsection*{Терминология}

При рассмотрении программной системы как совокупности взаимосвязанных программных модулей (компонент), образующих общую вычислительную систему, ее центральной частью становится ядро ОС. Оно выполняет функции посредника между приложениями и устройствами, осуществляющими обработку данных на аппаратном уровне. При этом, основной его задачей является эффективное управление ресурсами.

Рассматривая компоненты программных систем в качестве разного рода прикладных и системных программ, выполняющихся на соответствующем оборудовании, можно установить связь между встраиванием и получением возможности перехвата управления в ходе выполнения участков этих программ. При этом программный код, который обрабатывает подобные ситуации, называется кодом-перехватчиком или проще "--- хуком (англ., hooking). Сами же термины перехват (управления) и встраивание считаются схожими и, если это не оговаривается отдельно, используются для обозначения одного и того же. Однако следует иметь в виду существующее различие между ними: встраивание представляет собой процесс внедрения в общем смысле, тогда как перехват скорее указывает на конкретный методический приём.

Отличительной чертой методов динамического встраивания является отсутствие необходимости перезагрузки целевой системы \linebreak для того, чтобы ожидаемые изменения вступили в силу. Как правило, объектами перехвата являются функции "--- элементы кода ядра, реализующие тот или иной алгоритм. Реже встраивание происходит в такие системные механизмы как обработчики исключений и, в частности, диспетчер системных вызовов.

\subsection*{Специфика ядра Linux}

По своей архитектуре Linux представляет собой монолитное ядро с поддержкой возможности расширения функциональности за счёт модулей, по необходимости загружаемых в процессе работы. Учитывая данную особенность, для ядра Linux существует возможность разрабатывать расширения, которые, фактически являясь частью ядра, могут переопределять/дополнять различные его функции, т.е.  изменять порядок работы его подсистем.

Перехват функций ядра является базовым методом, позволяющим переопределять/дополнять различные его механизмы. Исходя из того, что ядро Linux почти полностью написано на языке C, за исключением небольших архитектурно-зависимых частей, можно утверждать, что для осуществления встраивания в большинство компонентов ядра достаточно иметь возможность перехвата соответствующих функций.

Обработка исключений лежит в основе функционирования множества системных механизмов. Вследствие этого перехват обработчиков исключений ядра Linux позволяет повысить степень контроля над системой, а перехват диспетчера системных вызовов даёт возможность осуществлять регуляцию запросов прикладного ПО к сервисам ядра Linux.

\subsection*{Техника патчинга}

В основе традиционных методов встраивания лежит патчинг (англ., patching) "--- техника внесения изменений в код или данные, позволяющая модифицировать поведение целевого алгоритма требуемым образом. Технически, результатом патчинга является изменение содержимого ячеек в оперативной памяти. Однако модификация кода, в отличие от модификации данных, имеет свои особенности, связанные прежде всего с фундаментальным отличием кода от данных.

Реализация перехвата с использованием техники патчинга требует квалификации, а также понимания принципов работы не только самого ядра, но и особенностей используемой аппаратной платформы. При модификации кода стоит особо обращать внимание на корректность встраивания в случае многопроцессорных систем, ведь в результате изменений не должна нарушаться когерентность. Кроме того следует учитывать необходимость обхода механизмов защиты кода ядра от модификации, а также особенности поиска и использования скрытых и не экспортируемых символов. Так или иначе, в большинстве случаев патчинг позволяет решить задачу встраивания.

\subsection*{Платформо-зависимые подходы}

Недостатком патчинга можно считать необходимое нарушение целостности компонентов целевой системы. Внесение изменений в код может быть легко обнаружено, что в некоторых контекстах является принципиальным ограничением. В этом случае следует использовать методы, лишённые такого рода ограничений. Как правило, часть таких методов использует аппарантые возможности платформы (например, аппаратные точки останова), что в принципе не может являться универсальным, учитывая хотя бы ограничения на число устанавливаемых перехватов. С другой стороны, всегда остаётся возможность использования разного рода виртуальных функций и прочих динамически заменяемых указателей, позволяющих переопределять в известных пределах поведение системы. Последнее, в частности, распространено для перехвата операций, осуществляемых в рамках виртуальной файловой системы (VFS), когда для операций с объектами используются таблицы виртуальных методов, замена которых может быть выполнена без модификации кода. Однако данный подход также имеет ряд ограничений, главное из которых заключается в том, что нет возможности контролировать то, контроль чего архитектурно не предусмотрен.

\subsection*{Метод инвалидации данных}

В ходе исследования вопросов осуществления встраивания без модификации кода была отмечена возможность управления обработкой исключений в ядре Linux. На базе этого был разработан метод встраивания, получивший название метода инвалидации данных, суть которого заключается в том, что модификации (инвалидации) подвергается внутренняя переменная, используемая в коде целевой подсистемы. Вследствие того, что значение этой переменной инвалидируется, создаются условия для возникновения исключения при доступе к ней частей алгоритма. Обработка таких ситуаций позволяет получить управление, необходимое для исправления ошибки, что в свою очередь используется для перехвата управления, а следовательно и встраивания.

Примером применения метода инвалидации данных является осуществление встраивания в ключевые механизмы ядра Linux, контролируемые с использованием фреймворка LSM (Linux Security Modules) для архитекруры x86\_64. При этом операция инвалидации данных заключается в изменении одного единственного бита в ключевой для LSM переменной "--- security\_ops.

\subsection*{Заключение}

Таким образом, для встраивания в ядро ОС существуют способы, использование которых применительно к конкретной задаче является более или менее целесообразным. Патчинг является базовым методом встраивания и может быть применим, если отсутствуют ограничения на сохранение целостности кода. В противном случае, в зависимости от ситуации, могут применяться специфичные для архитектуры решения (такие, как использование аппаратных точек останова), перегрузка виртуальных функций, а также метод инвалидации данных, который является в достаточной степени универсальным и может быть реализован для широкого круга систем.

\end{document}
