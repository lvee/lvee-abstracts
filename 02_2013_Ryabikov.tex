\documentclass[10pt, a5paper]{article}
\usepackage{pdfpages}
\usepackage{parallel}
\usepackage[T2A]{fontenc}
\usepackage{ucs}
\usepackage[utf8x]{inputenc}
\usepackage[polish,english,russian]{babel}
\usepackage{hyperref}
\usepackage{rotating}
\usepackage[inner=2cm,top=1.8cm,outer=2cm,bottom=2.3cm,nohead]{geometry}
\usepackage{listings}
\usepackage{graphicx}
\usepackage{wrapfig}
\usepackage{longtable}
\usepackage{indentfirst}
\usepackage{array}
\newcolumntype{P}[1]{>{\raggedright\arraybackslash}p{#1}}
\frenchspacing
\usepackage{fixltx2e} %text sub- and superscripts
\usepackage{icomma} % коскі ў матэматычным рэжыме
\PreloadUnicodePage{4}

\newcommand{\longpage}{\enlargethispage{\baselineskip}}
\newcommand{\shortpage}{\enlargethispage{-\baselineskip}}

\def\switchlang#1{\expandafter\csname switchlang#1\endcsname}
\def\switchlangbe{
\let\saverefname=\refname%
\def\refname{Літаратура}%
\def\figurename{Іл.}%
}
\def\switchlangen{
\let\saverefname=\refname%
\def\refname{References}%
\def\figurename{Fig.}%
}
\def\switchlangru{
\let\saverefname=\refname%
\let\savefigurename=\figurename%
\def\refname{Литература}%
\def\figurename{Рис.}%
}

\hyphenation{admi-ni-stra-tive}
\hyphenation{ex-pe-ri-ence}
\hyphenation{fle-xi-bi-li-ty}
\hyphenation{Py-thon}
\hyphenation{ma-the-ma-ti-cal}
\hyphenation{re-ported}
\hyphenation{imp-le-menta-tions}
\hyphenation{pro-vides}
\hyphenation{en-gi-neering}
\hyphenation{com-pa-ti-bi-li-ty}
\hyphenation{im-pos-sible}
\hyphenation{desk-top}
\hyphenation{elec-tro-nic}
\hyphenation{com-pa-ny}
\hyphenation{de-ve-lop-ment}
\hyphenation{de-ve-loping}
\hyphenation{de-ve-lop}
\hyphenation{da-ta-ba-se}
\hyphenation{plat-forms}
\hyphenation{or-ga-ni-za-tion}
\hyphenation{pro-gramming}
\hyphenation{in-stru-ments}
\hyphenation{Li-nux}
\hyphenation{sour-ce}
\hyphenation{en-vi-ron-ment}
\hyphenation{Te-le-pathy}
\hyphenation{Li-nux-ov-ka}
\hyphenation{Open-BSD}
\hyphenation{Free-BSD}
\hyphenation{men-ti-on-ed}
\hyphenation{app-li-ca-tion}

\def\progref!#1!{\texttt{#1}}
\renewcommand{\arraystretch}{2} %Іначай формулы ў матрыцы зліпаюцца з лініямі
\usepackage{array}

\def\interview #1 (#2), #3, #4, #5\par{

\section[#1, #3, #4]{#1 -- #3, #4}
\def\qname{LVEE}
\def\aname{#1}
\def\q ##1\par{{\noindent \bf \qname: ##1 }\par}
\def\a{{\noindent \bf \aname: } \def\qname{L}\def\aname{#2}}
}

\def\interview* #1 (#2), #3, #4, #5\par{

\section*{#1\\{\small\rm #3, #4. #5}}

\def\qname{LVEE}
\def\aname{#1}
\def\q ##1\par{{\noindent \bf \qname: ##1 }\par}
\def\a{{\noindent \bf \aname: } \def\qname{L}\def\aname{#2}}
}


\begin{document}

\title{<<Золотая пуля>>, или как приручить программного монстра}%\footnote{Текст данных и последующих тезисов, кроме специально оговоренных случаев, доступен под лицензией Creative Commons Attribution-ShareAlike 3.0}

\author{Александр Рябиков\footnote{Москва, Россия; \url{9109803341@mail.ru}}}
\maketitle

\begin{abstract}
Free and Open Source software development model is very advantageous to use when creating large software projects. This development model is suitable, including, for software products used in the business. But Free Software business applications require specific organizational forms of Free Software\linebreak Communities. For example, Free Software Community based on non"=profit organization has an opportunity to protect its\linebreak developments from unfair competition.
\end{abstract}

Фредерик Брукс сравнил большой программный проект с оборотнем, который по мере своего роста требует все больше и больше ресурсов на свое развитие, и со временем такой программный монстр легко может съесть своего автора. Для <<лечения>> подобных проектов"=монстров должно помочь мифическое средство, <<серебряная пуля>>, которая многократно увеличит производительность труда программиста и тем самым уменьшит требуемые ресурсы на разработку и поддержку большого программного проекта.

Практически все поиски «серебряной пули» были сосредоточены среди технических средств, облегчающих программирование. Но снизить стоимость программных продуктов можно и организационными методами "--- достаточно использовать свободную или открытую модель разработки. К сожалению, количество открытых и тем более свободных программных проектов для бизнеса катастрофически мало. Такая ситуация легко объяснима: или у компании нет ресурсов на разработку достойного продукта, или ресурсы есть, но тогда программный продукт делать свободным уже ненужно.

Но даже если несколько разработчиков сумеют договориться между собой о совместной разработке программного продукта для бизнеса, риск недобросовестной конкуренции очень велик. И неорганизованное сообщество, т.~е. сообщество без юридического оформления, всегда будет проигрывать коммерческим компаниям"=конкурентам. Организационные формы сообществ могут быть разными, но если программный продукт, разрабатываемый сообществом, предназначен для получения прибыли, то юридическая регистрация сообщества необходима.

Регистрация некоммерческой организации "--- долгое и затратное мероприятие. Дополнительную сложность при регистрации может создавать различные требования законодательства и опасения, что регистрация НКО была политически мотивированной. По этим причинам, большинство СПО сообществ либо никак не оформлены, либо действуют в рамках правил, разработанных коммерческой организацией "--- собственником программного продукта.

Выходом из подобной ситуации может стать создание независимых СПО сообществ в рамках одной некоммерческой организации. При этом государственная регистрация такого СПО сообщества не потребуется, и в тоже время оно станет частью юридического лица, т.~е. у него появляется возможность привлечения финансирования, судебной защиты и т.~д. Такой способ организации СПО сообществ особенно хорошо подходит для начального этапа развития проекта, а после того, как сообщество окрепнет, оно может принять решение о регистрации отдельного юридического лица.

В 2012 году зарегистрирован Фонд поддержки и развития делового свободного программного обеспечения <<Адемпиере>>, реализующий описанные выше принципы. Название Фонда взято по названию ERP"=системы «Адемпиере» (итал. <<выполнять, исполнять>>), которая дала начальный импульс к сотрудничеству компаний и отдельных разработчиков в данной области. В работе фонда участвуют представители России, Украины, Белоруссии, Узбекистана и Казахстана, т.~о. его деятельность можно считать международной.

Активную работу фонд начал вести в 2013 году, и сейчас в его рамках работает сообщество разработчиков системы управления предприятием Adempiere/iDempiere. В стадии организации находится Юридическая рабочая группа и Образовательный проект с использованием свободного программного обеспечения. Развитие этих проектов в рамках одной организации позволяет их участникам взаимовыгодно сотрудничать между собой.

Фонд является членом РАСПО, что позволяет быть услышанными на самом высоком уровне и быть в курсе последних тенденций развития законодательных инициатив. Из завершённых мероприятия участников фонда можно отметить организацию двух конференции разработчиков ADempiere ERP в Москве, участие в конференции ROSS 2013 и организацию посещения России Redhuan Daniel Oon (Рэд1) "--- лидера международного сообщества ADempiere.

\end{document}




