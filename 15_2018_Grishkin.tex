\documentclass[10pt, a5paper]{article}
\usepackage[T2A]{fontenc}
\usepackage{ucs}
\usepackage[utf8x]{inputenc}
\usepackage[polish,english,russian]{babel}
\usepackage{hyperref}
\usepackage[inner=2cm,top=1.8cm,outer=2cm,bottom=2.3cm,nohead]{geometry}
\usepackage{listings}
\usepackage{graphicx}
\usepackage{wrapfig}
\usepackage{longtable}
\usepackage{indentfirst}
\frenchspacing
\usepackage{fixltx2e} %text sub- and superscripts
\usepackage{icomma} % коскі ў матэматычным рэжыме
\PreloadUnicodePage{4}

\newcommand{\longpage}{\enlargethispage{\baselineskip}}
\newcommand{\shortpage}{\enlargethispage{-\baselineskip}}

\def\switchlang#1{\expandafter\csname switchlang#1\endcsname}
\def\switchlangbe{
\let\saverefname=\refname%
\def\refname{Літаратура}%
\def\figurename{Іл.}%
}
\def\switchlangen{
\let\saverefname=\refname%
\def\refname{References}%
\def\figurename{Fig.}%
}
\def\switchlangru{
\let\saverefname=\refname%
\let\savefigurename=\figurename%
\def\refname{Литература}%
\def\figurename{Рис.}%
}

\hyphenation{admi-ni-stra-tive}
\hyphenation{ex-pe-ri-ence}
\hyphenation{fle-xi-bi-li-ty}
\hyphenation{Py-thon}
\hyphenation{ma-the-ma-ti-cal}
\hyphenation{re-ported}
\hyphenation{imp-le-menta-tions}
\hyphenation{pro-vides}
\hyphenation{en-gi-neering}
\hyphenation{com-pa-ti-bi-li-ty}
\hyphenation{im-pos-sible}
\hyphenation{desk-top}
\hyphenation{elec-tro-nic}
\hyphenation{com-pa-ny}
\hyphenation{de-ve-lop-ment}
\hyphenation{de-ve-loping}
\hyphenation{de-ve-lop}
\hyphenation{da-ta-ba-se}
\hyphenation{plat-forms}
\hyphenation{or-ga-ni-za-tion}
\hyphenation{pro-gramming}
\hyphenation{in-stru-ments}
\hyphenation{Li-nux}
\hyphenation{en-vi-ron-ment}
\hyphenation{Te-le-pathy}
\hyphenation{Li-nux-ov-ka}

\def\progref!#1!{\texttt{#1}}
\renewcommand{\arraystretch}{2} %Іначай формулы ў матрыцы зліпаюцца з лініямі
\usepackage{array}

\def\interview #1 (#2), #3, #4, #5\par{

\section[#1, #3, #4]{#1, #5}
\def\qname{LVEE}
\def\aname{#1}
\def\q ##1\par{{\noindent \bf \qname: ##1 }\par}
\def\a{{\noindent \bf \aname: } \def\qname{L}\def\aname{#2}}
}

%\switchlang{ru}
\begin{document}
\title{Свободное программное обеспечение в образовательном учреждении Сергиево-Посадский филиал ВГИК им. С.А. Герасимова\footnote{\url{grishkin@mail.ru}, \url{https://lvee.org/en/abstracts/287}}}
\author{А.В. Гришкин, г. Сергиев Посад,  Россия, \\ А.А. Маркина, г. Брест, Республика Беларусь}
\maketitle
\begin{abstract}
The questions of the use of free software are considered both in the process of preparing the student and in the form of a means of supporting the educational process (educational environment) in the Sergiev Posad branch of the All-Russian State Institute of Cinematography. S.A. Gerasimov (VGIK). An overview is given of the free software used to organize a single information and educational environment for the institution to implement the requirements of the current legislation in the Russian Federation in the implementation of training programs for specialists in secondary vocational education and higher education.
\end{abstract}
В 2018 Сергиево-Посадскому филиалу Всероссийского\linebreak Государственного Института кинематографии им. С.А. Герасимова году исполнилось 70 лет. Учебное заведение много лет реализовывало программы среднего профессионального образования. В 2011 году, после присоединения к ВГИК, в филиале возникла необходимость реализации программ высшего образования.

Программное обеспечение, используемое в филиале, можно разделить на две категории: средства поддеркжи учебного процесса и изучаемые программные системы.

\subsection*{Средства поддержки учебного процесса}

Начиная с федеральных государственных образовательных стандартов (ФГОС) третьего поколения, при реализации программ высшего образования, обязательным является наличие в образовательной организации единой информационно-образовательной среды\linebreak (ЕИОС).

ФГОС выдвигает следующие цели использования ЕИОС при реализации программ высшего образования:

\begin{itemize}
  \item доступ к учебным планам, рабочим программам дисциплин (модулей), практик, к изданиям электронных библиотечных систем и электронным образовательным ресурсам, указанным в рабочих программах;
  \item фиксацию хода образовательного процесса, результатов промежуточной аттестации и результатов освоения основной образовательной программы;
  \item проведение всех видов занятий, процедур оценки результатов обучения, реализация которых предусмотрена с применением электронного обучения, дистанционных образовательных технологий;
  \item формирование электронного портфолио обучающегося, в том числе сохранение работ обучающегося, рецензий и оценок на эти работы со стороны любых участников образовательного процесса;
  \item взаимодействие между участниками образовательного процесса, в том числе синхронное и (или) асинхронное взаимодействия посредством сети <<Интернет>>.
\end{itemize}

Функционирование электронной информационно-\linebreak образовательной среды обеспечивается соответствующими средствами информационно-коммуникационных технологий и квалификацией работников, ее использующих и поддерживающих. Функционирование электронной информационно-образовательной среды\linebreak должно соответствовать законодательству Российской Федерации. Т.е. при организации ЕИОС необходимо обеспечить соблюдение Федеральный закон от 27 июля 2006 г. N 149-ФЗ <<Об информации, информационных технологиях и о защите информации>> и Федеральный закон от 27 июля 2006 г. N 152-ФЗ <<О персональных данных>> (Собрание законодательства Российской Федерации. (п.7.1.2. стандарта).

При обеспечении доступа к учебным планам, рабочим программам дисциплин (модулей), практик, к изданиям электронных библиотечных систем и электронным образовательным ресурсам, указанным в рабочих программах следует учесть, что аналогичные требования по доступу к перечисленным документам также установлены Постановлением Правительства РФ от 10.07.2013 N 582 (ред. от 07.08.2017) "Об утверждении Правил размещения на официальном сайте образовательной организации в информационно-телекоммуникационной сети <<Интернет>> и обновления информации об образовательной организации". Т.о. для однократности размещения и минимизации ошибок при обновлении, частью ЕИОС можно считать официальный сайт учебного учреждения.

Организация сайта учебного учреждения регламентируется приказом Рособрнадзора от 29.05.2014 N 785 <<Об утверждении требований к структуре официального сайта образовательной организации в информационно-телекоммуникационной сети <<Интернет>> и формату представления на нем информации>> (Зарегистрировано в Минюсте России 04.08.2014 N 33423). Приказ устанавливает не только перечень, наименование, адрес и минимальное содержание разделов, но и требования к микроразметке, которая используется для выделения ключевых элементов содержания страниц и контролируется Рособрнадзором в автоматизированном режиме. Правила применения микроразметки установлены актуализированными методическими рекомендациями представления информации об образовательной организации высшего образования в открытых источниках с учетом соблюдения требований законодательства в сфере образования распространяемыми Рособрнадзором.

В Сергиево-Посадском филиале ВГИК официальный сайт построен на CMS Joomla! с доработкой отдельных модулей и реализацией автоматизированного добавления микроразметки.

Стоит отметить, что применение микроразметки позволило многим сайтам-агрегаторам в автоматическом режиме собирать информацию о учебных заведениях (например сайт ucheba.ru).

Для фиксации хода образовательного процесса, результатов промежуточной аттестации и результатов освоения основной образовательной программы, формировании электронного портфолио обучающегося, в том числе сохранение работ обучающегося, рецензий и оценок на эти работы со стороны любых участников образовательного процесса был предложено создать сайт на основе CMS Joomla! доступный исключительно внутри учебного заведения содержащий требуемую информацию. Дневным отделением для каждого обучающегося создается учетная запись с необходимыми правами доступа к материалам, по окончании обучения права доступа отбираются. В настоящее время ход образовательного процесса фиксируется скан-копиями ведомостей оценок и результатов промежуточной аттестации. В будущем планируется разработка системы хранящей структурированную информацию о успеваемости обучающихся (аналог <<электронных журналов>> для средней школы).

Остро стоит вопрос о виде и способах представления портфолио студентов с возможностью рецензий и оценок. В настоящее время результаты творческой работы студентов в филиала неструктурированно размещаются на внутреннем хранилище, при этом на специально выделенной страницы внутреннего портала размещается информация о выполненных работах. Размещение творческих работ на внешних ресурсах (Youtube, Rutube, фотохостинги) порождает проблемы соблюдения авторских и смежных прав как студентов так и владельцев материала используемого студентами.

Для проведения всех видов занятий, процедур оценки результатов обучения, реализация которых предусмотрена с применением электронного обучения, дистанционных образовательных технологий используется среда Moodle, в которой размещаются учебно-методические комплексы преподаваемых предметов (модулей). В перспективе планируется использование возможностей Moodle для проведения оценки усвоения материала предмета (модуля) как с помощью тестовых заданий, так и в результате проверки представленных студентами работ в системе.

Также на Moodle возлагается обеспечение взаимодействия между участниками образовательного процесса, но т.к. система расположена исключительно внутри филиала, доступ к ней, а следовательно и взаимодействие внутри нее, при помощи сети <<Интернет>> невозможны.

В помощь преподавателям и студентам составлен каталог библиотеки филиала в электронном виде. Система построена на базе CMS Wordpress с использованием доработанного плагина <<Book Review Library>>. Плагин регистрирует новый тип поста, определяет его таксономию и реквизиты. За отображение данных отвечает разработанная тема с разбиением объемного содержания на отдельные страницы. Сформированные описания имеющихся в библиотеке единиц полностью соответствуют стандартам оформления литературных источников для выполняемых преподавателями и студентами работ. В перспективе в электронный каталог будут включены разработанные преподавателями материалы, представленные в системе Moodle.

Формирование учебно-методического комплекса предмета (модуля) осуществляется при помощи самостоятельно разработанного на базе Node.js приложения, в качестве среды хранения в котором применяется документоориентированная база данных MongoDB.\linebreak При наполнении приложение имеет возможность получения информации из электронного каталога библиотеки филиала через\linebreak Wordpress API. Приложение позволяет сформировать рекомендуемый состав методических материалов для специальностей, курсов обучения, предметов (модулей), учитывая при этом год обучения. Основной сложностью при разработке приложения стало изменение подхода к хранению данных от реляционного к документному. Каждая запись о методическом материале хранит весь набор характеристик материала, т.о. отсутствует необходимость во время поиска формировать и использовать соединение таблиц и разные виды подзапросов.

Применение приложения позволяет агрегировать в одном месте как ссылки на описание литературы из каталога библиотеки филиала, так и ссылки на литературу в электронном виде, представленную в электронных библиотеках <<Лань>> и iBooks.ru, доступ к которым осуществляется из внутренней сети филиала.

Таким образом, можно подытожить, что применение свободного программного обеспечения в целях создания единой\linebreak информационно-образовательной среды позволяет полностью выполнить требования ФГОС, хотя и вызывает необходимость разработки дополнительного программного обеспечения. Также следует отметить, что применение ЕИОС возможно и при реализации программ среднего профессионального образования в филиале.

В перспективе в филиале ставятся задачи повышения интеграции составных частей ЕИОС между собой, более объемное применение структурированной информации, получаемой, в том числе и от внешних систем.

\subsection*{Программное обеспечение, применяемое в учебном процессе}

В настоящее время в филиале свободное программное обеспечение применяется и непосредственно в учебном процессе.

В лабораторном практикуме активно применяются:

\begin{itemize}
  \item при изучении технологий хранения данных СУБД MySQL, осуществляются первые попытки использования СУБД\linebreak MongoDB;
  \item при изучении сетевых технологий~--- сетевой анализатор\linebreak WireShark, реализации сервисов электронной почты, FTP,\linebreak DNS на базе FreeBSD или Fedora;
  \item при изучении технологий разработки программ и информационных систем~--- IDE NetBeans, Node.js, git, Notepad++, сервис draw.io.
  \item при изучении 3D моделирования, в том числе и в анимации~--- Blender3D.
\end{itemize}

\begin{thebibliography}{20}
\bibitem{Grishkin-1} Федеральный закон от 29 декабря 2012 года N 273-ФЗ <<Об образовании в Российской Федерации>> (Ст. 29).
\bibitem{Grishkin-2} Постановление Правительства Российской Федерации от 10.07.2013 N 582 <<Об утверждении Правил размещения на официальном сайте образовательной организации в информационно-телекоммуникационной сети <<Интернет>> и обновления информации об образовательной организации>>.
\bibitem{Grishkin-3} Приказ Рособрнадзора от 29.05.2014 N 785 <<Об утверждении требований к структуре официального сайта образовательной организации в информационно-телекоммуникационной сети <<Интернет>> и формату представления на нем информации>>.
\bibitem{Grishkin-4} Приказ Минобрнауки России от 16.11.2016 N 1427 <<Об утверждении федерального государственного образовательного стандарта высшего образования по специальности 55.05.01 Режиссура кино и телевидения (уровень специалитета)>> (Зарегистрировано в Минюсте России 09.12.2016 N 44634)
\end{thebibliography}

\end{document}
