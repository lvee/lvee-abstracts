\documentclass[10pt, a5paper]{article}
\usepackage{pdfpages}
\usepackage{parallel}
\usepackage[T2A]{fontenc}
\usepackage{ucs}
\usepackage[utf8x]{inputenc}
\usepackage[polish,english,russian]{babel}
\usepackage{hyperref}
\usepackage{rotating}
\usepackage[inner=2cm,top=1.8cm,outer=2cm,bottom=2.3cm,nohead]{geometry}
\usepackage{listings}
\usepackage{graphicx}
\usepackage{wrapfig}
\usepackage{longtable}
\usepackage{indentfirst}
\usepackage{array}
\newcolumntype{P}[1]{>{\raggedright\arraybackslash}p{#1}}
\frenchspacing
\usepackage{fixltx2e} %text sub- and superscripts
\usepackage{icomma} % коскі ў матэматычным рэжыме
\PreloadUnicodePage{4}

\newcommand{\longpage}{\enlargethispage{\baselineskip}}
\newcommand{\shortpage}{\enlargethispage{-\baselineskip}}

\def\switchlang#1{\expandafter\csname switchlang#1\endcsname}
\def\switchlangbe{
\let\saverefname=\refname%
\def\refname{Літаратура}%
\def\figurename{Іл.}%
}
\def\switchlangen{
\let\saverefname=\refname%
\def\refname{References}%
\def\figurename{Fig.}%
}
\def\switchlangru{
\let\saverefname=\refname%
\let\savefigurename=\figurename%
\def\refname{Литература}%
\def\figurename{Рис.}%
}

\hyphenation{admi-ni-stra-tive}
\hyphenation{ex-pe-ri-ence}
\hyphenation{fle-xi-bi-li-ty}
\hyphenation{Py-thon}
\hyphenation{ma-the-ma-ti-cal}
\hyphenation{re-ported}
\hyphenation{imp-le-menta-tions}
\hyphenation{pro-vides}
\hyphenation{en-gi-neering}
\hyphenation{com-pa-ti-bi-li-ty}
\hyphenation{im-pos-sible}
\hyphenation{desk-top}
\hyphenation{elec-tro-nic}
\hyphenation{com-pa-ny}
\hyphenation{de-ve-lop-ment}
\hyphenation{de-ve-loping}
\hyphenation{de-ve-lop}
\hyphenation{da-ta-ba-se}
\hyphenation{plat-forms}
\hyphenation{or-ga-ni-za-tion}
\hyphenation{pro-gramming}
\hyphenation{in-stru-ments}
\hyphenation{Li-nux}
\hyphenation{sour-ce}
\hyphenation{en-vi-ron-ment}
\hyphenation{Te-le-pathy}
\hyphenation{Li-nux-ov-ka}
\hyphenation{Open-BSD}
\hyphenation{Free-BSD}
\hyphenation{men-ti-on-ed}
\hyphenation{app-li-ca-tion}

\def\progref!#1!{\texttt{#1}}
\renewcommand{\arraystretch}{2} %Іначай формулы ў матрыцы зліпаюцца з лініямі
\usepackage{array}

\def\interview #1 (#2), #3, #4, #5\par{

\section[#1, #3, #4]{#1 -- #3, #4}
\def\qname{LVEE}
\def\aname{#1}
\def\q ##1\par{{\noindent \bf \qname: ##1 }\par}
\def\a{{\noindent \bf \aname: } \def\qname{L}\def\aname{#2}}
}

\def\interview* #1 (#2), #3, #4, #5\par{

\section*{#1\\{\small\rm #3, #4. #5}}

\def\qname{LVEE}
\def\aname{#1}
\def\q ##1\par{{\noindent \bf \qname: ##1 }\par}
\def\a{{\noindent \bf \aname: } \def\qname{L}\def\aname{#2}}
}

\begin{document}
\title{Luwrain — адаптированная ОС с речевым интерфейсом}
\author{Михаил Пожидаев \footnote{Томск, Россия, \url{msp@altlinux.ru}}}
\maketitle
\begin{abstract}
The report covers new design of operating system for blind and visual impaired persons offered by the Luwrain project. Suggested conception includes text-based user interface without GUI elements as well as a number of system services managed through D-Bus. New approach to user interface implies implementation with Java virtual machine. The project is based on conclusions of ALT Linux Homeros distribution respecting all positive parts of its experience.
\end{abstract}
Наиболее распространённый подход к использованию персонального компьютера людьми с проблемами зрения заключается в установке так называемых экранных чтецов, которые в виде речи описывают каждое действие пользователя и связанные с ним изменения на экране. Использование мыши обычно невозможно, и все манипуляции выполняются только при помощи клавиатуры. Графический пользовательский интерфейс (GUI) ориентирован главным образом на взаимодействие при помощи мыши, и в такой ситуации применение клавиатуры приводит к существенному увеличению времени работы. Обычно при работе дома или в тихом офисе дополнительные затраты времени несущественны, но в шумной и напряжённой обстановке крупного аэропорта или конференции навигация по элементам GUI с использованием клавиатуры становится утомительной и некомфортной.

Функции экранных чтецов позволяют организовать доступ к огромному числу приложений, но для людей с проблемами зрения круг необходимых задач намного уже, поскольку в него не входят решения, связанные с обработкой растровой и векторной графики, визуальным проектированием, видеомонтажом и пр. Закон Парето, согласно которому 80\% потребностей пользователей удовлетворяется 20\% функций приложений, приобретает более выраженный вид, что делает большое количество ПО на традиционных компьютерах незрячих пользователей невостребованным.

Проект Luwrain, проходящий в настоящий момент фазу подготовки первых демонстрационных прототипов, занимается разработкой новой среды для незрячих и слабовидящих пользователей, ориентированной на представление всех рабочих объектов только в текстовом виде. Главное требование, которому должен удовлетворять продукт, — предельно высокая скорость работы, а также хороший уровень комфорта и удобства пользователя даже в шумном и многолюдном помещении.

Среда функционирует внутри виртуальной машины Java с задействованием доступных библиотек для решения конкретных прикладных задач (JavaMail для чтения почты, Rome для чтения новостей, Apache POI для работы с форматами офисных документов и т. д.). Экран разделяется на фрагменты, расположенные в виде тайлов, которые условно называются ``области''. Каждая область позволяет выводить некоторую текстовую информацию, отображая её моноширинным шрифтом. Размер шрифта и его цвет можно менять в зависимости от уровня зрения слабовидящего человека. Пользователь имеет возможность свободной навигации по каждой области с получением соответствующих голосовых оповещений аналогично тому, как он перемещается внутри текстового файла. Содержание и поведение каждой области зависит от типа решаемой задачи. Например, если требуется составить электронное сообщение, то на первой строке под специальным заголовком указывается получатель, строкой ниже — тема сообщения и далее текст. Как показывает опыт, большинство необходимых задач может быть пересмотрено в описанном виде. В качестве экспериментальной платформы выступает дистрибутив ALT Linux Homeros, предлагающий похожую концепцию на базе текстового редактора GNU Emacs, но из-за ограничений среды неспособного стать продуктом для массового пользователя.

Виртуальная машина Java должна функционировать в GNU/Linux, которая обеспечивает все необходимые системные компоненты. Особенное внимание планируется уделить сервисам, доступ к которым осуществляется через системную шину D-Bus. Примером таких сервисов может служить Network Manager, обеспечивающий управление сетевыми интерфейсами, Udisks, обеспечивающий управление носителями информации и пр. Особенно перспективным выглядит задействование службы Systemd, которая находится в активной разработке. Все указанные сервисы, управляемые централизованно при помощи D-Bus, позволяют сделать среду целостной и интегрированной, предоставляя пользователю удобные инструменты взаимодействия со всеми компонентами. Программа установки может быть выполнена при помощи технологии клонирования LiveCD.

Некоторые сервисы предполагается реализовать отдельно вне виртуальной машины Java. К ним относятся речевой сервер и служба управления медиапроигрывателем. Речевой сервер получает команды на воспроизведение фрагментов текста и управляет вызовом речевых синтезаторов для  генерации звукового сигнала и посылки его в аудио-устройство в режиме реального времени. Назначение службы управления медиапроигрывателем очень близко к функциям привычного проигрывателя, но требуется добавление особого механизма закладок, нужных для чтения ``говорящих книг''. Помимо этого, существует специализированный формат книг для незрячих людей — Daisy, который также должен обслуживаться описанным сервисом.

Некоторые задачи не могут быть решены в рамках приведённой концепции. Это прежде всего справедливо для веб-браузера, природа которого не позволяет представлять содержимое страниц в текстовом виде, и для закрытых коммерческих приложений, таких как, например, Skype. Для их работы можно воспользоваться готовой службой AT-SPI, запуская приложения без какой-либо модификации, но внутри специального оконного менеджера, оснащённого речевыми оповещениями и управляемого при помощи комбинаций ``горячих'' клавиш.

\end{document}
