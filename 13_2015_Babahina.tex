\documentclass[10pt, a5paper]{article}
\usepackage{pdfpages}
\usepackage{parallel}
\usepackage[T2A]{fontenc}
\usepackage{ucs}
\usepackage[utf8x]{inputenc}
\usepackage[polish,english,russian]{babel}
\usepackage{hyperref}
\usepackage{rotating}
\usepackage[inner=2cm,top=1.8cm,outer=2cm,bottom=2.3cm,nohead]{geometry}
\usepackage{listings}
\usepackage{graphicx}
\usepackage{wrapfig}
\usepackage{longtable}
\usepackage{indentfirst}
\usepackage{array}
\newcolumntype{P}[1]{>{\raggedright\arraybackslash}p{#1}}
\frenchspacing
\usepackage{fixltx2e} %text sub- and superscripts
\usepackage{icomma} % коскі ў матэматычным рэжыме
\PreloadUnicodePage{4}

\newcommand{\longpage}{\enlargethispage{\baselineskip}}
\newcommand{\shortpage}{\enlargethispage{-\baselineskip}}

\def\switchlang#1{\expandafter\csname switchlang#1\endcsname}
\def\switchlangbe{
\let\saverefname=\refname%
\def\refname{Літаратура}%
\def\figurename{Іл.}%
}
\def\switchlangen{
\let\saverefname=\refname%
\def\refname{References}%
\def\figurename{Fig.}%
}
\def\switchlangru{
\let\saverefname=\refname%
\let\savefigurename=\figurename%
\def\refname{Литература}%
\def\figurename{Рис.}%
}

\hyphenation{admi-ni-stra-tive}
\hyphenation{ex-pe-ri-ence}
\hyphenation{fle-xi-bi-li-ty}
\hyphenation{Py-thon}
\hyphenation{ma-the-ma-ti-cal}
\hyphenation{re-ported}
\hyphenation{imp-le-menta-tions}
\hyphenation{pro-vides}
\hyphenation{en-gi-neering}
\hyphenation{com-pa-ti-bi-li-ty}
\hyphenation{im-pos-sible}
\hyphenation{desk-top}
\hyphenation{elec-tro-nic}
\hyphenation{com-pa-ny}
\hyphenation{de-ve-lop-ment}
\hyphenation{de-ve-loping}
\hyphenation{de-ve-lop}
\hyphenation{da-ta-ba-se}
\hyphenation{plat-forms}
\hyphenation{or-ga-ni-za-tion}
\hyphenation{pro-gramming}
\hyphenation{in-stru-ments}
\hyphenation{Li-nux}
\hyphenation{sour-ce}
\hyphenation{en-vi-ron-ment}
\hyphenation{Te-le-pathy}
\hyphenation{Li-nux-ov-ka}
\hyphenation{Open-BSD}
\hyphenation{Free-BSD}
\hyphenation{men-ti-on-ed}
\hyphenation{app-li-ca-tion}

\def\progref!#1!{\texttt{#1}}
\renewcommand{\arraystretch}{2} %Іначай формулы ў матрыцы зліпаюцца з лініямі
\usepackage{array}

\def\interview #1 (#2), #3, #4, #5\par{

\section[#1, #3, #4]{#1 -- #3, #4}
\def\qname{LVEE}
\def\aname{#1}
\def\q ##1\par{{\noindent \bf \qname: ##1 }\par}
\def\a{{\noindent \bf \aname: } \def\qname{L}\def\aname{#2}}
}

\def\interview* #1 (#2), #3, #4, #5\par{

\section*{#1\\{\small\rm #3, #4. #5}}

\def\qname{LVEE}
\def\aname{#1}
\def\q ##1\par{{\noindent \bf \qname: ##1 }\par}
\def\a{{\noindent \bf \aname: } \def\qname{L}\def\aname{#2}}
}

\begin{document}
\title{Использование Blender при создании анимационного проекта}
\author{Виктория Бабахина, Рязань, РФ\footnote{\url{vitorry@gmail.com}, \url{http://lvee.org/ru/abstracts/158}}}
\maketitle
\begin{abstract}
The article gives an overview of the free software usage in film"=making industry. It covers all the basic steps of creating an animated film with specifics related to Blender.
\end{abstract}

При работе над анимационным проектом перед художником"=аниматором часто возникает ряд весьма нетривиальных задач. Если речь идет не о крупной студии, то аниматор часто становится и художником по фонам, и моделлером, и компоузером, и даже специалистом по видео"=монтажу. В таких условиях очень важно иметь под рукой удобное программное обеспечение, позволяющее выполнять как можно большее количество задач из разных областей создания готового мультипликата, обладающее при этом достаточной интерактивностью, удобством в обращении и, что весьма важно, адекватной ценой. Здесь мы рассмотрим, как с этой ролью справляется пакет Blender.

\section*{Задачи}

Вот лишь небольшой перечень задач, с которыми может столкнуться в работе художник"=аниматор: создание фонов, моделирование и анимация трехмерных объектов, анимация двухмерных объектов и кривых, создание эффектов,  работа с motion"=графикой, 
композитинг.

\section*{Выбор рендера}

Перед началом работы часто встает вопрос о том, каким рендером пользоваться. Blender предоставляет для работы следующие:

\begin{description}
  \item[Internal.] Рендер, более старый, чем Cycles, не фотореалистичный. Internal "--- это рендер с допущениями. К его несомненным плюсам можно отнести то, что он позволяет рендерить дым, волосы, ночные пейзажи и т. д., при рендере на CPU он значительно быстрее, отлично справляется со стилизованной картинкой, motion"=графикой, стилизацией под 2D и со всем, где не нужен реалистичный рендер. Из минусов "--- больше работы с настройками материалов, рендер с допущениями, и в нем нет глобального освещения и цветных рефлексов.
  \item[Cycles.] Плюсы: фотореалистичный рендер, быстрая и более удобная настройка материалов, просчет на GPU, что более удобно для графики, рендер, удобный для работы с интерьерами. Минусы "--- много проблем с шумом при рендере, трудности с рендером частиц и OSL.
  \item[Freestyle render.] Предназначен для стилизации под 2D"=изображение. 
Freestyle генерирует двумерные линии из набора объектов, линии могут быть стилизованы различными способами (разные цвета и толщины или добавление случайной неровности) для создания художественного (рисунка от руки) или технического (чертёжного) стилей. 
Freestyle может использоваться как дополнение к другим рендерам.
Freestyle для Blender имеет два дополнительных режима для стилизации линий: параметрический редактор и режим Python"=скриптов.
\end{description}

\section*{3D "--- моделинг, риггинг, текстуры, анимация}

Основная функция пакета Blender "--- это работа с 3D"=графикой. При работе над мультипликационным проектом, работа с трехмерной графикой бывает нужна даже в тех случаях, когда речь идет исключительно о 2D"=анимации. К примеру, при создании фонов для упрощения и ускорения рабочего процесса бывает удобно сделать трехмерную модель сцены и рендерить нужные ракурсы при необходимости и в нужном количестве. Для сглаживания разницы между трехмерным фоном и нарисованными плоскими персонажами удобно использовать Freestyle"=рендер, о котором шла речь ранее.

Часто более трудоемким является создание трехмерной модели персонажа. 
Обычные этапы создания трехмерного персонажа выглядят так: 
Подготовительный этап "--- сбор референсов, работа над образом персонажа в эскизах, создание листов персонажа.

\begin{itemize}
  \item Моделирование и текстуры "--- непосредственная работа над моделью.
  \item Риг и лицевой риг "--- оснащение персонажа скелетом и мимикой.
  \item Липсинк "--- создание речевой анимации и фонем.
\end{itemize}

В блендере есть возможность автоматизировать процесс, соответствующий последнему пункту. Для этого необходимо провести подготовительный этап "--- создать шейпы движения губ, соответствующие определенным буквам, и грамотно их переименовать. Затем требуется помощь стороннего приложения для кодирования звука "--- такой например, как Papagayo. В стороннее приложение загружается звуковой файл с текстом, который считывается и кодируется в формат, пригодный для чтения Blender. Результат загружается в Blender, с помощью специального плагина привязывается к модели и автоматически создает ключевые кадры с анимацией.

\section*{2D"=анимация и motion"=графика}

Помимо работы над 3D"=анимацией, Blender располагает обширным инструментарием для работы с 2D"=графикой, который постепенно расширяется и развивается "--- порой в весьма неожиданных направлениях.

Ранее уже было сказано о рендере Freestyle, позволяющем создать имитацию рисованной анимации.  Помимо стилизации во \linebreak Freestyle, в Blender возможна работа над настоящими 2D"=марионетками. С помощью плагина <<Import image as plane>> в блендер загружается персонаж "--- это можно сделать либо по частям, либо и одним изображением, чтобы выполнить его нарезку уже в Blender. Затем марионетка оснащается скелетом и анимируется.

Еще один прием, который способен сильно выручить при работе не только с двумерной анимацией, но и с motion"=графикой "--- это анимация процедурных текстур. С ее помощью сложно получить реалистичные эффекты, какие получились бы при работе с симуляторами огня, дыма или воды; однако когда речь идет о более стилизованной графике,  процедурные текстуры становятся незаменимы. К несомненным плюсам работы с ними относится высокая скорость просчета, отсутствие рывка движения при окончании цикла анимации, возможность сделать анимацию любого размера без потерь качества.

В последнее время появился еще один, весьма неожиданный способ работы с 2D"=анимацией в Blender "--- grease pencil. Изначально это была довольно удобная, но далекая от анимации функция рисования в окне программы. С ее помощью можно было выделить для себя важные моменты при моделировании, наметить траекторию будущей анимации и другие вещи, полезные в работе. И так было до выхода мультипликационного фильма <<For You>>, созданного в технике покадровой анимации исключительно с помощью grease pencil.

\section*{Композитинг}

Изображение, получаемое непосредственно после рендера "--- далеко не финальный результат. Огромное количество работы над изображением ведется на этапе постобработки "--- композитинга.

Финальная картинка при рендере "--- это финальная работа сочетания огромного количества слоев. То есть, даже если не ведется специальная работа над композитингом, он все равно ведется глубоко на уровне кода. В Blender эти слои называются \emph{пассы}.  Пассы позволяют влиять на свет, тени, переотражения, блики, цвета, материалы, маски, глубину и многое другое.

Работа с пассами "--- это не все, что можно осуществить при композитинге. С помощью различных входных нодов возможно, к примеру, заменить фон на изображение, сократив таким образом время рендеринга в разы. Также возможно разделить объекты рендера из одного изображения в разные слои и рендерить по отдельности все или несколько объектов сцены.

Помимо прочего, при композитинге есть возможность работы с масками (когда необходимо убрать, вставить или выделить какую"=то определенную часть изображения), кеингом (замена однотонного яркого фона на что"=то иное), трекингом (определение  местоположения объектов с помощью камеры и последующая работа с полученными точками).

Таким образом, от раннего этапа создания фильма до постобработки Blender будет полезен художнику"=аниматору и, обладая огромным количеством полезных функций, поможет справиться в том числе с довольно сложными и нетривиальными задачами.

\end{document}
