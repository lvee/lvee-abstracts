\documentclass[10pt, a5paper]{article}
\usepackage[T2A]{fontenc}
\usepackage{ucs}
\usepackage[utf8x]{inputenc}
\usepackage[polish,english,russian]{babel}
\usepackage{hyperref}
\usepackage[inner=2cm,top=1.8cm,outer=2cm,bottom=2.3cm,nohead]{geometry}
\usepackage{listings}
\usepackage{graphicx}
\usepackage{wrapfig}
\usepackage{longtable}
\usepackage{indentfirst}
\frenchspacing
\usepackage{fixltx2e} %text sub- and superscripts
\usepackage{icomma} % коскі ў матэматычным рэжыме
\PreloadUnicodePage{4}

\newcommand{\longpage}{\enlargethispage{\baselineskip}}
\newcommand{\shortpage}{\enlargethispage{-\baselineskip}}

\def\switchlang#1{\expandafter\csname switchlang#1\endcsname}
\def\switchlangbe{
\let\saverefname=\refname%
\def\refname{Літаратура}%
\def\figurename{Іл.}%
}
\def\switchlangen{
\let\saverefname=\refname%
\def\refname{References}%
\def\figurename{Fig.}%
}
\def\switchlangru{
\let\saverefname=\refname%
\let\savefigurename=\figurename%
\def\refname{Литература}%
\def\figurename{Рис.}%
}

\hyphenation{admi-ni-stra-tive}
\hyphenation{ex-pe-ri-ence}
\hyphenation{fle-xi-bi-li-ty}
\hyphenation{Py-thon}
\hyphenation{ma-the-ma-ti-cal}
\hyphenation{re-ported}
\hyphenation{imp-le-menta-tions}
\hyphenation{pro-vides}
\hyphenation{en-gi-neering}
\hyphenation{com-pa-ti-bi-li-ty}
\hyphenation{im-pos-sible}
\hyphenation{desk-top}
\hyphenation{elec-tro-nic}
\hyphenation{com-pa-ny}
\hyphenation{de-ve-lop-ment}
\hyphenation{de-ve-loping}
\hyphenation{de-ve-lop}
\hyphenation{da-ta-ba-se}
\hyphenation{plat-forms}
\hyphenation{or-ga-ni-za-tion}
\hyphenation{pro-gramming}
\hyphenation{in-stru-ments}
\hyphenation{Li-nux}
\hyphenation{en-vi-ron-ment}
\hyphenation{Te-le-pathy}
\hyphenation{Li-nux-ov-ka}

\def\progref!#1!{\texttt{#1}}
\renewcommand{\arraystretch}{2} %Іначай формулы ў матрыцы зліпаюцца з лініямі
\usepackage{array}

\def\interview #1 (#2), #3, #4, #5\par{

\section[#1, #3, #4]{#1, #5}
\def\qname{LVEE}
\def\aname{#1}
\def\q ##1\par{{\noindent \bf \qname: ##1 }\par}
\def\a{{\noindent \bf \aname: } \def\qname{L}\def\aname{#2}}
}

\begin{document}
\title{Особенности коррекции оптических искажений в цифровой фотографии}
\author{Алексей Бабахин, Рязань, РФ\footnote{\url{tamerlan311@mail.ru}, \url{http://lvee.org/ru/abstracts/136}}}
\maketitle
\begin{abstract}
The article talks about the distortion correction in digital images, as well as cases where correction is used to solve applied problems.
\end{abstract}
Любой объектив вносит искажения в формируемое им изображение. Чем дороже оптика, тем меньше искажений. Но полностью избавиться от них невозможно.

\begin{itemize}
  \item Виньетирование "--- затемнение изображения по краям кадра.
  \item Хроматические аберрации "--- «расслоение» изображения по цветовым каналам из"=за различных углов преломления у света с разной длиной волны. Проявляется в виде цветного ореола вокруг контрастных мест.
  \item Дисторсия "--- искривление изображения, вызванное неравномерным линейным увеличением при отклонении от оптической оси. Из"=за дисторсии прямые линии на кадре становятся изогнутыми.
\end{itemize}

С развитием цифровой техники появилась возможность строить математические модели оптических искажений и исправлять их. Помимо общего повышения качества фотографий, расчет и устранение аберраций критически необходимы для решения множества практических задач, таких как компьютерное зрение (CV "--- Compu\-ter Vision), фотограмметрия, объединение нескольких фотографий и создание панорам. Точные расчеты на основании фотографий без коррекции искажений невозможны.

С точки зрения различных расчетов наиболее важным является исправление дисторсии. виньетированием и хроматическими аберрации зачастую можно либо пренебречь, либо они исправляются некоторыми камерами прямо в процессе съемки, если камера знает калибровки для текущего объектива.

Существует несколько математических моделей, описывающих дисторсию. Пожалуй, самой популярной моделью является PTLens, изначально разработанная доктором Хельмутом Дерша (Helmut \linebreak Dersch) в Panorama Tools. На данный момент эта модель является основной в библиотеке LensFun. В свою очередь эту библиотеку используют множество популярных открытых фоторедакторов "--- UFRaw, Darktable, Rawstudio, Digikam/Kipi, GimpLensfun, Photivo и оболочка для создания панорам Hugin. Библиотека имеет постоянно пополняемую базу объективов, которая облегчает исправление искажений. Однако есть возможность и самостоятельно создать профиль для своего объектива.

Немного особняком стоит библиотека компьютерного зрения \linebreak OpenCV, которая использует свою собственную математическую модель, описывающую дисторсию. Ту же самую модель использует и Blender для реконструкции и привязки живого видео к 3D"=сцене.

Проблема заключается в том, что LensFun не поддерживает модель дисторсии, используемую в OpenCV. Да и при наличии такой поддержки, прямой конвертации одной модели в другую добиться невозможно. Поэтому на данный момент нет возможности использовать обширную базу объективов и инструменты для профилирования при работе с видео в Blender и в других разработках, использующих библиотеку OpenCV. Перспективной выглядит идея подбора коэффициентов одной модели на основании коэффициентов другой модели методом наименьших квадратов "--- например, при помощи библиотеки ceres"=solver.

Редакторы фотографий, как правило, могут использовать только уже готовые данные калибровки объективов. Сшиватель панорам Hugin или Blender могут подбирать приблизительные коэффициенты для коррекции дисторсии в процессе своей работы. Такое поведение обусловлено работой сразу с несколькими фотографиями (или видео), которые позволяют сопоставлять между собой разные ракурсы. Тем не менее, предварительная аккуратная калибровка объектива специальными мишенями позволяет повысить точность расчетов, качество результата и снизить суммарные трудозатраты.

\end{document}
