\documentclass[10pt, a5paper]{article}
\usepackage{pdfpages}
\usepackage{parallel}
\usepackage[T2A]{fontenc}
\usepackage{ucs}
\usepackage[utf8x]{inputenc}
\usepackage[polish,english,russian]{babel}
\usepackage{hyperref}
\usepackage{rotating}
\usepackage[inner=2cm,top=1.8cm,outer=2cm,bottom=2.3cm,nohead]{geometry}
\usepackage{listings}
\usepackage{graphicx}
\usepackage{wrapfig}
\usepackage{longtable}
\usepackage{indentfirst}
\usepackage{array}
\newcolumntype{P}[1]{>{\raggedright\arraybackslash}p{#1}}
\frenchspacing
\usepackage{fixltx2e} %text sub- and superscripts
\usepackage{icomma} % коскі ў матэматычным рэжыме
\PreloadUnicodePage{4}

\newcommand{\longpage}{\enlargethispage{\baselineskip}}
\newcommand{\shortpage}{\enlargethispage{-\baselineskip}}

\def\switchlang#1{\expandafter\csname switchlang#1\endcsname}
\def\switchlangbe{
\let\saverefname=\refname%
\def\refname{Літаратура}%
\def\figurename{Іл.}%
}
\def\switchlangen{
\let\saverefname=\refname%
\def\refname{References}%
\def\figurename{Fig.}%
}
\def\switchlangru{
\let\saverefname=\refname%
\let\savefigurename=\figurename%
\def\refname{Литература}%
\def\figurename{Рис.}%
}

\hyphenation{admi-ni-stra-tive}
\hyphenation{ex-pe-ri-ence}
\hyphenation{fle-xi-bi-li-ty}
\hyphenation{Py-thon}
\hyphenation{ma-the-ma-ti-cal}
\hyphenation{re-ported}
\hyphenation{imp-le-menta-tions}
\hyphenation{pro-vides}
\hyphenation{en-gi-neering}
\hyphenation{com-pa-ti-bi-li-ty}
\hyphenation{im-pos-sible}
\hyphenation{desk-top}
\hyphenation{elec-tro-nic}
\hyphenation{com-pa-ny}
\hyphenation{de-ve-lop-ment}
\hyphenation{de-ve-loping}
\hyphenation{de-ve-lop}
\hyphenation{da-ta-ba-se}
\hyphenation{plat-forms}
\hyphenation{or-ga-ni-za-tion}
\hyphenation{pro-gramming}
\hyphenation{in-stru-ments}
\hyphenation{Li-nux}
\hyphenation{sour-ce}
\hyphenation{en-vi-ron-ment}
\hyphenation{Te-le-pathy}
\hyphenation{Li-nux-ov-ka}
\hyphenation{Open-BSD}
\hyphenation{Free-BSD}
\hyphenation{men-ti-on-ed}
\hyphenation{app-li-ca-tion}

\def\progref!#1!{\texttt{#1}}
\renewcommand{\arraystretch}{2} %Іначай формулы ў матрыцы зліпаюцца з лініямі
\usepackage{array}

\def\interview #1 (#2), #3, #4, #5\par{

\section[#1, #3, #4]{#1 -- #3, #4}
\def\qname{LVEE}
\def\aname{#1}
\def\q ##1\par{{\noindent \bf \qname: ##1 }\par}
\def\a{{\noindent \bf \aname: } \def\qname{L}\def\aname{#2}}
}

\def\interview* #1 (#2), #3, #4, #5\par{

\section*{#1\\{\small\rm #3, #4. #5}}

\def\qname{LVEE}
\def\aname{#1}
\def\q ##1\par{{\noindent \bf \qname: ##1 }\par}
\def\a{{\noindent \bf \aname: } \def\qname{L}\def\aname{#2}}
}

\begin{document}
\title{Organic software. Как это работает}
\author{Константин Лепихов\footnote{Mozilla.Россия}}
\date{}
\maketitle

\begin{abstract}
The `organic software' term was declared by Mozilla for their software.
Its main idea is patent and blob-free software that respects the user. But it also
respects open standards, which create a level playing field for any
individual, company or orga\-ni\-za\-tion to create Web content for others. And it is a
manifestation of Mozilla's core belief in the importance of providing vehicles
for participation on the Internet (\url{http://www.mozilla.org/about/mozilla-manifesto.html}).
\end{abstract}

Концепция <<органического>> программного обеспечения была \linebreak рождена в недрах Mozilla как 
способ существовать на рынке закрытых программных решений и вести успешную
конкуренцию с ними. Главная идея подхода "--- программное обеспечение, свободное от
патентов и встроенных бинарных объектов, которое с уважением относится к пользователю.
Однако кроме пользователя, уважительное отношение задекларировано также к открытым стандартам,
дающим пространство для того, чтобы любой индивидуум, компания или организация могли создавать
для других веб-контент. Это проявление веры Mozilla в важность предоставления средств для
участия в развитии Интернет.
Ключевая идея изложена в \url{http://www.mozilla.org/about/mozilla-manifesto.html}.

Рассмотрим современный рынок браузеров и мобильных платформ. Инфраструктура
Mozilla является в полной мере и платформой для разработки, нет только вендора,
который сможет воплотить идеи в конечный продукт, будь то телефон или MID.

{\bf Microsoft} развивает платформу WM7 и браузер IE. Подход рассматривает 
технологии всего лишь повод увеличить сумму для заказчика. Потенциал платформы WM7 пока 
не очень неясен. Браузер является аутсайдером по реализации стандартов и чемпионом
по количеству ошибок проектирования и скрытых дефектов. 

{\bf Opera} стремится реализовать все первой. Реализация может быть
далека от идеала и в ущерб другим более полезным вещам, но должна появиться на месяц
раньше конкурентов (оставим за рамками и код браузера, поскольку нам
его никогда не покажут).

{\bf Google Android/Chrome} создается компанией, у которой есть платформа, 
копирующая все самое лучшее, что есть в iOS, WM или Symbian, но открытая для разработки и допускающая
возможность создания закрытых форков. Есть движок рендеринга, который также
открыт и допускает возможность создания закрытых форков. По отзывам, в создании этого 
движка участвовали люди из Apple и Symbian. У компании есть собственный браузер, который
очень быстр и очень хорош. Но, к сожалению, бранч с кодом браузера компании закрыт, 
а в ОС много блобов и DRM.

Mozilla Foundation преследует совсем другие цели: у сообщества не может быть
конкурентов, поскольку сообщество ничем не торгует. Но у него может быть PR,
чтобы быть услышанным на рынке коммерческих продуктов. И у сообщества есть
инфраструктура, чтобы разработка самого лучшего и удобного для пользователя
бразуера ничем не стесняла разработчика.

Основные компоненты инфраструктуры разработки:
\begin{itemize}
	\item Система отслеживания ошибок (Bugzilla)
	\item SDK и публичные репозитарии (Jetpack и Incubator)
	\item Документация на языках, доступных разработчикам любого уровня (MDN и mozhacks)
	\item Каналы поддержки (irc/web/social media)
	\item Система QA сборок и автоматического тестирования
	\item Правовая поддержка и защита интеллектуальных прав.
\end{itemize}


\end{document}


