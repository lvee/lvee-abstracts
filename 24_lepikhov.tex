\documentclass[10pt, a5paper]{article}
\usepackage{ucs}
\usepackage[utf8]{inputenc}
\usepackage[T2A]{fontenc}
\usepackage[english, russian]{babel}
\usepackage{hyperref}
\usepackage{geometry}
\frenchspacing
\begin{document}
\title{Organic software. Как это работает}
\author{Константин Лепихов\footnote{Mozilla.Россия}}
\date{}
\maketitle

\begin{abstract}
The "organic software" term was declared by Mozilla for their software.
Its main idea is patent and blob-free software that respects the user. But it also
respects open standards, which create a level playing field for any
individual, company or organization to create Web content for others. And it is a
manifestation of Mozilla's core belief in the importance of providing vehicles
for participation on the Internet (\url{http://www.mozilla.org/about/mozilla-manifesto.html}).
\end{abstract}

Концепция <<органического>> программного обеспечения была рождена в недрах Mozilla как 
способ существовать на рынке закрытых программных решений и вести успешную
конкуренцию с ними. Главная идея подхода "--- программное обеспечение, свободное от
патентов и встроенных бинарных объектов, которое с уважением относится к пользователю.
Однако кроме пользователя, уважительное отношение задекларировано также к открытым стандартам,
дающим пространство для того, чтобы любой индивидуум, компания или организация могли создавать
для других веб-контент. Это проявление веры Mozilla в важность предоставления средств для
участия в развитии Интернет.
Ключевая идея изложена в \url{http://www.mozilla.org/about/mozilla-manifesto.html}.

Рассмотрим современный рынок браузеров и мобильных платформ. Инфраструктура
Mozilla является в полной мере и платформой для разработки, нет только вендора,
который сможет воплотить идеи в конечный продукт, будь то телефон или MID.

{\bf Microsoft} развивает платформу WM7 и браузер IE. Подход рассматривает 
технологии всего лишь повод увеличить сумму для заказчика. Потенциал платформы WM7 пока 
не очень неясен. Браузер является аутсайдером по реализации стандартов и чемпионом
по количеству ошибок проектирования и скрытых дефектов. 

{\bf Opera} стремится реализовать все первой. Реализация может быть
далека от идеала и в ущерб другим более полезным вещам, но должна появиться на месяц
раньше конкурентов (оставим за рамками и код браузера, поскольку нам
его никогда не покажут).

{\bf Google Android/Chrome} создается компанией, у которой есть платформа, 
копирующая все самое лучшее, что есть в iOS, WM или Symbian, но открытая для разработки и допускающая
возможность создания закрытых форков. Есть движок рендеринга, который также
открыт и допускает возможность создания закрытых форков. По отзывам, в создании этого 
движка участвовали люди из Apple и Symbian. У компании есть собственный браузер, который
очень быстр и очень хорош. Но, к сожалению, бранч с кодом браузера компании закрыт, 
а в ОС много блобов и DRM.

Mozilla Foundation преследует совсем другие цели: у сообщества не может быть
конкурентов, поскольку сообщество ничем не торгует. Но у него может быть PR,
чтобы быть услышанным на рынке коммерческих продуктов. И у сообщества есть
инфраструктура, чтобы разработка самого лучшего и удобного для пользователя
бразуера ничем не стесняла разработчика.

Основные компоненты инфраструктуры разработки:
\begin{itemize}
	\item Система отслеживания ошибок (Bugzilla)
	\item SDK и публичные репозитарии (Jetpack и Incubator)
	\item Документация на языках, доступных разработчикам любого уровня (MDN и mozhacks)
	\item Каналы поддержки (irc/web/social media)
	\item Система QA сборок и автоматического тестирования
	\item Правовая поддержка и защита интеллектуальных прав.
\end{itemize}


\end{document}


