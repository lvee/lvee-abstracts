\documentclass[10pt, a5paper]{article}
\usepackage{pdfpages}
\usepackage{parallel}
\usepackage[T2A]{fontenc}
\usepackage{ucs}
\usepackage[utf8x]{inputenc}
\usepackage[polish,english,russian]{babel}
\usepackage{hyperref}
\usepackage{rotating}
\usepackage[inner=2cm,top=1.8cm,outer=2cm,bottom=2.3cm,nohead]{geometry}
\usepackage{listings}
\usepackage{graphicx}
\usepackage{wrapfig}
\usepackage{longtable}
\usepackage{indentfirst}
\usepackage{array}
\newcolumntype{P}[1]{>{\raggedright\arraybackslash}p{#1}}
\frenchspacing
\usepackage{fixltx2e} %text sub- and superscripts
\usepackage{icomma} % коскі ў матэматычным рэжыме
\PreloadUnicodePage{4}

\newcommand{\longpage}{\enlargethispage{\baselineskip}}
\newcommand{\shortpage}{\enlargethispage{-\baselineskip}}

\def\switchlang#1{\expandafter\csname switchlang#1\endcsname}
\def\switchlangbe{
\let\saverefname=\refname%
\def\refname{Літаратура}%
\def\figurename{Іл.}%
}
\def\switchlangen{
\let\saverefname=\refname%
\def\refname{References}%
\def\figurename{Fig.}%
}
\def\switchlangru{
\let\saverefname=\refname%
\let\savefigurename=\figurename%
\def\refname{Литература}%
\def\figurename{Рис.}%
}

\hyphenation{admi-ni-stra-tive}
\hyphenation{ex-pe-ri-ence}
\hyphenation{fle-xi-bi-li-ty}
\hyphenation{Py-thon}
\hyphenation{ma-the-ma-ti-cal}
\hyphenation{re-ported}
\hyphenation{imp-le-menta-tions}
\hyphenation{pro-vides}
\hyphenation{en-gi-neering}
\hyphenation{com-pa-ti-bi-li-ty}
\hyphenation{im-pos-sible}
\hyphenation{desk-top}
\hyphenation{elec-tro-nic}
\hyphenation{com-pa-ny}
\hyphenation{de-ve-lop-ment}
\hyphenation{de-ve-loping}
\hyphenation{de-ve-lop}
\hyphenation{da-ta-ba-se}
\hyphenation{plat-forms}
\hyphenation{or-ga-ni-za-tion}
\hyphenation{pro-gramming}
\hyphenation{in-stru-ments}
\hyphenation{Li-nux}
\hyphenation{sour-ce}
\hyphenation{en-vi-ron-ment}
\hyphenation{Te-le-pathy}
\hyphenation{Li-nux-ov-ka}
\hyphenation{Open-BSD}
\hyphenation{Free-BSD}
\hyphenation{men-ti-on-ed}
\hyphenation{app-li-ca-tion}

\def\progref!#1!{\texttt{#1}}
\renewcommand{\arraystretch}{2} %Іначай формулы ў матрыцы зліпаюцца з лініямі
\usepackage{array}

\def\interview #1 (#2), #3, #4, #5\par{

\section[#1, #3, #4]{#1 -- #3, #4}
\def\qname{LVEE}
\def\aname{#1}
\def\q ##1\par{{\noindent \bf \qname: ##1 }\par}
\def\a{{\noindent \bf \aname: } \def\qname{L}\def\aname{#2}}
}

\def\interview* #1 (#2), #3, #4, #5\par{

\section*{#1\\{\small\rm #3, #4. #5}}

\def\qname{LVEE}
\def\aname{#1}
\def\q ##1\par{{\noindent \bf \qname: ##1 }\par}
\def\a{{\noindent \bf \aname: } \def\qname{L}\def\aname{#2}}
}

\begin{document}
\title{Как пропатчить KDE4 под OpenBSD}
\author{Вадим Жуков, Москва, РФ\footnote{\url{zhuk@openbsd.org}, \url{http://lvee.org/en/abstracts/111}}}
\maketitle
\begin{abstract}
For a long time OpenBSD did not ship modern KDE versions. But during two last years situation changed. The talk is about what happened behind the scenes to make the working KDE4 platform usable by OpenBSD users. There will be covered such things as tweaking CMake modules, authentication support and so on. Also, there was a success in making KDE 3 and 4 co-exist that involved solving assorted technical problems. The experience gained during process could be useful for porting other software on OpenBSD as well as for porting in general.
\end{abstract}
KDE "--- крупный открытый проект, включающий в себя множество различных компонентов. Определённое множество этих компонентов составляет KDE Software Compilation (KDE SC):
\begin{itemize}
 \item основные библиотеки, например: kdelibs, kdepimlibs;
 \item служебные приложения, например: lnusertemp, kded, kread\-con\-fig;
 \item основные пользовательские приложения, например: Kon\-quer\-or, Dolphin, KMail;
 \item компоненты среды KDE, например: kde-workspace (включает, среди прочего, KDM и KWin), kde-artwork;
 \item дополнительные компоненты: игры, обучающие приложения, всевозможные полезные утилиты.
\end{itemize}

За пределами KDE SC находится ряд связанных проектов, которые можно условно разделить на две группы:
\begin{itemize}
 \item KDE Support: полусамостоятельные проекты, необходимые\linebreak для сборки и работы KDE, например: сервер Akonadi, lib\-at\-ti\-ca, набор модулей для CMake.
 \item базирующиеся на KDE проекты, как-то: Calligra Suite, Digikam SC, KMyMoney, Amarok и т.д.
\end{itemize}

Портирование KDE состоит из следующих этапов:

\begin{enumerate}
 \item Определение и портирование недостающих зависимостей. В случае OpenBSD это были весь KDE Support (включая сервер Akonadi), Virtuoso и ряд небольших библиотек. Моменты, которые хочется отметить:
\begin{itemize}
 \item Akonadi: ни с одним бэкендом до сих пор не наблюдается 100\% стабильной работы. В лучшем случае очередь запросов к серверу забивается из-за плохого планирования. Патчи, внесённые разработчиками Akonadi в SQLite"=бэкенд, выглядят не совсем корректными, поэтому в OpenBSD добавлена возможность использовать штатный, из состава Qt4.
 \item Virtuoso: ряд встроенных в дистрибутив Virtuoso тестов на регрессии до сих пор не проходит, по различным причинам. Однако, во-первых, имеющейся функциональности вполне хватает для нужд Ne\-po\-muk "--- единственного пользователя Virtuoso; а во-вторых, в KDE в данный момент ведётся работа над следующим поколением Nepomuk под кодовым названием Baloo, в которой значительно переделана и упрощена внутренняя архитектура данной подсистемы, в результате чего хранилище RDF-данных вроде Virtuoso становится не нужным.
 \item Модули CMake: часть этих модулей дублировала имеющиеся в поставке CMake (в портах OpenBSD силами dcoppa@ всегда есть актуальная версия), некоторые потребовали полного переписывания. А в случае с Find\-Get\-text.cma\-ke пришлось писать обвязку для KDE4-портов, которая исправляет на лету файлы CMake\-Lists.txt, в которых производится компиляция файлов локализация.
\end{itemize}

 \item Портирование kdelibs и других базовых компонентов. Этот процесс в OpenBSD, на самом деле, продолжается до сих пор. Все известные критичные проблемы устранены, для ряда подсистем обновлены или написаны «с нуля» специфичные для ОС реализации. На данный момент под OpenBSD в KDE недоступны следующие возможности:
\begin{itemize}
 \item Solid: отсутствует поддержка OpenBSD, поэтому KDE не умеет получать данные об устройствах в системе.
 \item KDM: отсутствует поддержка multi-seat X (т.н. «быстрое переключение пользователей») из-за ограничений базовой ОС.
 \item Незначительно ограничены возможности управления звуковой подсистемой из-за привязки некоторых компонентов Phonon к ALSA. Учитывая сравнительно малую проблемность звуковой подсистемы в OpenBSD (в большинстве случаев не требуется делать абсолютно ничего), данный пункт упомянут скорее для полноты картины.
\end{itemize}
 \item Портирование остальных частей KDE SC. Выполнено практически полностью, со следующими оговорками:
\begin{itemize}
 \item Отсутствует поддержка Web-камер и Jingle/GoogleTalk в Kopete.
 \item Порт Kalzium отмечен как BROKEN из-за проблем с зависимостями. Проблема обходится при использовании официального репозитория WIP-портов (https://github.com/\linebreak{}jas\-per\-la/open\-bsd-wip/), в котором имеются обновлённые, но не до конца проработанные порты.
 \item Не собираются биндинги к libattica. Проблема скорее в самом фреймворке Smoke, но, поскольку эти биндинги не требуются ни для одного портированного приложения, изучение проблемы отложено до лучших времён.
\end{itemize}
 \item Обеспечение совместной установки приложений KDE~3 и 4. В случае с OpenBSD для этого была переименована часть каталогов, в которых хранятся данные приложений KDE~3. Благодаря централизованности управления списками каталогов с данными в KDE, данное решение удалось разработать и внедрить в течение всего двух месяцев, включая отладку и тестирование. Единственный побочный эффект "--- некоторое разрастание пользовательского профиля KDE, из-за того, что приложение может считать данные из одного локального конфигурационного файла, а записать данные уже по другому пути, внутри того же профиля "--- это связано с тем, что иерархия «системных» и «пользовательских» каталогов в KDE едина.

Следует отметить, что, в отличие от других немногочисленных ОС, предоставлявших возможность параллельной установки KDE~3 и 4, в OpenBSD это сделано без ущерба для целостности системной структуры каталогов: никаких /opt или /usr/local/kde3.
 \item Обеспечение совместной работы приложений KDE~3 и 4. Это потребовало внесения изменений в пакеты kdelibs, kde-runtime и kde-workspaces, чтобы приложения KDE~4 использовали альтернативные пути к различным хранилищам временных файлов: KDE создаёт каталоги вида «/var/tmp/kde\-cache-user\-na\-me», а для доступа к ним использует символические ссылки в каталоге профиля пользователя. Данное изменение, в отличие от предыдущего, было внесено в KDE 4 с целью избежать ненужных проблем у пользователей KDE~3.
 \item Обеспечение совместной сборки KDE~3 и 4. Пользователям, не собирающим пакеты самостоятельно, эта проблема совершенно не очевидна и не видна. Однако для менйтейнеров она составляла заметную головную боль: невозможность собрать KDE~3 при установленном KDE~4 и наоборот ставила крест на используемой в OpenBSD технике непрерывной сборки пакетов. Для решения этой проблемы пришлось убедить в её серьёзности Марка Эспи, главного разработчика инфраструктуры портов и, в прошлом, мейнтейнера портов KDE. Это было сделано на Euro\-BSD\-Con 2013, а уже через несколько недель появился патч для dpb(1), позволявший разграничивать сборку портов по так называемым тегам. Фактически, это самый крупный «костыль», который был добавлен в OpenBSD в связи с портированием KDE.

В последние пару месяцев KDE~4 были подготовлены и частично внесены в дерево портов патчи, позволяющие KDE 3 собираться в присутствии KDE 4 и наоборот. Однако уже подготовленное полное решение в OpenBSD 5.5 не попадёт из-за близости момента заморозки дерева, которая должна произойти буквально в дни проведения нынешней конференции.
 \item Портирование приложений за пределами KDE SC. На текущий момент подгтовлены в WIP-репозитории и по большей части отлажены порты для Calligra Suite, Digikam SC, K3b, Kdenlive, Kile, KMyMoney, KTorrent, Tellico и Yakuake. Их импортирование так же ожидается на следующей итерации цикла разработки OpenBSD. Желающие же могут уже сейчас подключить WIP-репозитории и собрать интересующие их порты самостоятельно.
\end{enumerate}

\end{document}
