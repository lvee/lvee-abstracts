\documentclass[10pt, a5paper]{article}
\usepackage[T2A]{fontenc}
\usepackage{ucs}
\usepackage[utf8x]{inputenc}
\usepackage[polish,english,russian]{babel}
\usepackage{hyperref}
\usepackage[inner=2cm,top=1.8cm,outer=2cm,bottom=2.3cm,nohead]{geometry}
\usepackage{listings}
\usepackage{graphicx}
\usepackage{wrapfig}
\usepackage{longtable}
\usepackage{indentfirst}
\frenchspacing
\usepackage{fixltx2e} %text sub- and superscripts
\usepackage{icomma} % коскі ў матэматычным рэжыме
\PreloadUnicodePage{4}

\newcommand{\longpage}{\enlargethispage{\baselineskip}}
\newcommand{\shortpage}{\enlargethispage{-\baselineskip}}

\def\switchlang#1{\expandafter\csname switchlang#1\endcsname}
\def\switchlangbe{
\let\saverefname=\refname%
\def\refname{Літаратура}%
\def\figurename{Іл.}%
}
\def\switchlangen{
\let\saverefname=\refname%
\def\refname{References}%
\def\figurename{Fig.}%
}
\def\switchlangru{
\let\saverefname=\refname%
\let\savefigurename=\figurename%
\def\refname{Литература}%
\def\figurename{Рис.}%
}

\hyphenation{admi-ni-stra-tive}
\hyphenation{ex-pe-ri-ence}
\hyphenation{fle-xi-bi-li-ty}
\hyphenation{Py-thon}
\hyphenation{ma-the-ma-ti-cal}
\hyphenation{re-ported}
\hyphenation{imp-le-menta-tions}
\hyphenation{pro-vides}
\hyphenation{en-gi-neering}
\hyphenation{com-pa-ti-bi-li-ty}
\hyphenation{im-pos-sible}
\hyphenation{desk-top}
\hyphenation{elec-tro-nic}
\hyphenation{com-pa-ny}
\hyphenation{de-ve-lop-ment}
\hyphenation{de-ve-loping}
\hyphenation{de-ve-lop}
\hyphenation{da-ta-ba-se}
\hyphenation{plat-forms}
\hyphenation{or-ga-ni-za-tion}
\hyphenation{pro-gramming}
\hyphenation{in-stru-ments}
\hyphenation{Li-nux}
\hyphenation{en-vi-ron-ment}
\hyphenation{Te-le-pathy}
\hyphenation{Li-nux-ov-ka}

\def\progref!#1!{\texttt{#1}}
\renewcommand{\arraystretch}{2} %Іначай формулы ў матрыцы зліпаюцца з лініямі
\usepackage{array}

\def\interview #1 (#2), #3, #4, #5\par{

\section[#1, #3, #4]{#1, #5}
\def\qname{LVEE}
\def\aname{#1}
\def\q ##1\par{{\noindent \bf \qname: ##1 }\par}
\def\a{{\noindent \bf \aname: } \def\qname{L}\def\aname{#2}}
}

\begin{document}
\title{Консольно-ориентированные сервисы: wttr.in, cheat.sh, rate.sx~--- Идея, использвание, создание\footnote{\url{igor@chub.in}, \url{https://lvee.org/en/abstracts/278}}}
\author{Игорь Чубин (Igor Chubin)}
\maketitle
\begin{abstract}
Console orineted services: wttr.in, cheat.sh, rate.sx: idea, usage, and creation. 
The presentation is devoted to console oriented services, such as: wttr.in, cheat.sh, rate.sx. Which popular console oriented services exist currently and how can they be used in everyday life; what advantages and disadvantages do they have; how services like that could be created.
\end{abstract}


\subsubsection*{Консольно-ориентированные сервисы}

На рубеже 2015 и 2016 годов, в дополнение к традиционно существующим типам
программ и служб, доступным пользователям UNIX/Linux, а именно локально инсталлируеммым программам, доступным для использования в консоли или в графической оболочке, и, программам, работающим на внешних серверах, и доступным через web-интерфейс, появился третий, новый, тип программ, совмещающих в себе свойства первых двух типов: так называемые консольно-ориентированные сервисы, которые не требуют инсталляции и доступны к использованию как из терминала, так и из браузера.

Благодаря своей простоте использования, полному отсутствию необходимости
инсталляции и конфигурирования и ряду других преимуществ, они начали быстро завоёвывать популярность среди пользователей
консоли UNIX/Linux систем.

Характерной особенностью таких сервисов является то, что их использование, с точки зрения пользователя, напоминает использование обыкновенного простого веб-сайта, но только в отличие от веб-сайта, для них наличие браузера
не обязательно, вместо него можно использовать простые HTTP-клиенты,
такие как curl, httpie или wget. Путешествие по гиперссылкам при этом
заменяется манипуляциями с URL, который в случае консольных сервисов,
как правило, чрезвычайно прост и интуитивно понятен.

Важным аспектом сервисов является и то, что они построены таким образом,
что отображаются в консоли UNIX/Linux системы и в браузере одинаково.
Это достигается при помощи анализа заголовка User-Agent запроса,
в зависимости от которого ответ генерируется в форме HTML, пригодной для браузера, или в ANSI, пригодной для терминала. Такой подход существенно облегчает новым пользователям порог вхождения и начало использования консольных сервисов.

Следующие команды, выполненные в терминале, позволют получить первое впечатление о том, что такое консольные сервисы, и как они выглядят с точки зрения пользователя:

\ldots

    curl be.wttr.in/Minsk

    curl rate.sx/btc

    curl cheat.sh/lua/copy+file

\ldots

Если использовать эти же строки запроса в браузере, то можно увидеть ответы
аналогичные тем, что получены при запросе из терминала.

Как уже было сказано, сервисы созданные в соответствии с этим таким подходом
обладают множеством преимуществ как в сравнении с сервисами, созданными для использования из Web-браузера, так и в сравнении с традиционными консольными приложениями:

\begin{itemize}
  \item скорость;
  \item совместимость;
  \item очень низкие требования к клиенту;
  \item прекрасная возможность интеграции;
  \item простота и краткость;
  \item анонимность использования;
  \item и так далее.
\end{itemize}

В рамках популяризации идеи создания консольных сервисов, было создано
несколько типичных консольных сервисов, некоторые из которых получили большую
известность, и сейчас уже знакомы большому числу активных пользователей консоли UNIX/Linux во всём мире. Некоторые из них описаны ниже.

Кроме того, был создан специальный фреймворк, curlator, который существенно
упрощает создание консольных сервисов, и делает задачу создания консольных
сервисов доступным любому пользователю UNIX/Linux-систем, не требуя от него
никаких специальных знаний. Создание сервиса при этом по сложности соизмеримо с инсталляцией и начальным конфигурированием обычной UNIX/Linux-программы.

\subsubsection*{Примеры популярных консольно-ориентированных сервисов}

\paragraph{wttr.in}

wttr.in~--- сервис прогноза погоды, позволяет получить информацию о погоде в любой точке земного шара на одном из более 50 мировых языков; как и любой консольный сервис не требует никакой инсталляции и конфигурирования.

Примеры использования:

\ldots

    curl wttr.in

    curl ru.wttr.in

    curl be.wttr.in/Minsk

    curl uk.wttr.in/Москва

\ldots

\paragraph{cheat.sh}

cheat.sh~--- сервис подсказок по UNIX/Linux-командам и языкам программирования.

С его помощью можно получить подсказку с наиболее популярными примерами 
использования основных программ UNIX/Linux (сейчас сервис покрывает более 1000 команд), а так же получить ответы на практически любой вопрос по практически любому языку программирования.

Примеры использования:

\ldots

    curl cheat.sh

    curl cheat.sh/btrfs

    curl cheat.sh/az\~{}snapshot

    curl cheat.sh/lua/copy+file

    curl cheat.sh/ruby/скопировать+файл

    curl cheat.sh/python/створити+дерево+каталогів

\ldots

\paragraph{rate.sx}

rate.sx~--- сервис отслеживания обменных курсов валюты и криптовалюты.

С его помощью можно получить информацию о текущей и исторической стоимости
любой (из 500 наиболее популярных) криптовалюты на рынке, а так же её рыночную капитализацию, объём торгов и множество других характеристик.

\ldots

    curl rate.sx

    curl rate.sx/btc

    curl rate.sx/btc@1w

    curl rate.sx/btc/eth@1w

    curl eur.rate.sx/btc

\ldots

\paragraph{Другие сервисы}

Существует ряд других, менее популярных консольных сервисов, популярность которых, однако, растёт. Актуальный список сервисов доступен по адресу:  \url{https://github.com/chubin/awesome-console-services}

\subsubsection*{Об авторе}

Игорь Чубин~--- разработчик программного обеспечения, активный убеждённый пользователь и энтузиаст программного обеспечения. Основная его работа на протяжении последних десяти лет, это разработка высокопроизводительной распределённой реляционной базы данных Exasol.

В свободное от работы время он занимается разработкой и продвижением консольно-ориентированных сервисов.

Github: \url{https://github.com/chubin}

Twitter: \url{https://twitter.com/igor\_chubin}

StackOverflow: \url{https://stackoverflow.com/users/1458569/}

\end{document}
