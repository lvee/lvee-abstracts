\documentclass[10pt, a5paper]{article}
\usepackage{pdfpages}
\usepackage{parallel}
\usepackage[T2A]{fontenc}
\usepackage{ucs}
\usepackage[utf8x]{inputenc}
\usepackage[polish,english,russian]{babel}
\usepackage{hyperref}
\usepackage{rotating}
\usepackage[inner=2cm,top=1.8cm,outer=2cm,bottom=2.3cm,nohead]{geometry}
\usepackage{listings}
\usepackage{graphicx}
\usepackage{wrapfig}
\usepackage{longtable}
\usepackage{indentfirst}
\usepackage{array}
\newcolumntype{P}[1]{>{\raggedright\arraybackslash}p{#1}}
\frenchspacing
\usepackage{fixltx2e} %text sub- and superscripts
\usepackage{icomma} % коскі ў матэматычным рэжыме
\PreloadUnicodePage{4}

\newcommand{\longpage}{\enlargethispage{\baselineskip}}
\newcommand{\shortpage}{\enlargethispage{-\baselineskip}}

\def\switchlang#1{\expandafter\csname switchlang#1\endcsname}
\def\switchlangbe{
\let\saverefname=\refname%
\def\refname{Літаратура}%
\def\figurename{Іл.}%
}
\def\switchlangen{
\let\saverefname=\refname%
\def\refname{References}%
\def\figurename{Fig.}%
}
\def\switchlangru{
\let\saverefname=\refname%
\let\savefigurename=\figurename%
\def\refname{Литература}%
\def\figurename{Рис.}%
}

\hyphenation{admi-ni-stra-tive}
\hyphenation{ex-pe-ri-ence}
\hyphenation{fle-xi-bi-li-ty}
\hyphenation{Py-thon}
\hyphenation{ma-the-ma-ti-cal}
\hyphenation{re-ported}
\hyphenation{imp-le-menta-tions}
\hyphenation{pro-vides}
\hyphenation{en-gi-neering}
\hyphenation{com-pa-ti-bi-li-ty}
\hyphenation{im-pos-sible}
\hyphenation{desk-top}
\hyphenation{elec-tro-nic}
\hyphenation{com-pa-ny}
\hyphenation{de-ve-lop-ment}
\hyphenation{de-ve-loping}
\hyphenation{de-ve-lop}
\hyphenation{da-ta-ba-se}
\hyphenation{plat-forms}
\hyphenation{or-ga-ni-za-tion}
\hyphenation{pro-gramming}
\hyphenation{in-stru-ments}
\hyphenation{Li-nux}
\hyphenation{sour-ce}
\hyphenation{en-vi-ron-ment}
\hyphenation{Te-le-pathy}
\hyphenation{Li-nux-ov-ka}
\hyphenation{Open-BSD}
\hyphenation{Free-BSD}
\hyphenation{men-ti-on-ed}
\hyphenation{app-li-ca-tion}

\def\progref!#1!{\texttt{#1}}
\renewcommand{\arraystretch}{2} %Іначай формулы ў матрыцы зліпаюцца з лініямі
\usepackage{array}

\def\interview #1 (#2), #3, #4, #5\par{

\section[#1, #3, #4]{#1 -- #3, #4}
\def\qname{LVEE}
\def\aname{#1}
\def\q ##1\par{{\noindent \bf \qname: ##1 }\par}
\def\a{{\noindent \bf \aname: } \def\qname{L}\def\aname{#2}}
}

\def\interview* #1 (#2), #3, #4, #5\par{

\section*{#1\\{\small\rm #3, #4. #5}}

\def\qname{LVEE}
\def\aname{#1}
\def\q ##1\par{{\noindent \bf \qname: ##1 }\par}
\def\a{{\noindent \bf \aname: } \def\qname{L}\def\aname{#2}}
}


\begin{document}

\switchlang{be}
\title{Алгарытмы паляпшэння выяваў у ВПЗ: падвышэнне рэзкасці}%\footnote{Текст данных и последующих тезисов, кроме специально оговоренных случаев, доступен под лицензией Creative Commons Attribution-ShareAlike 3.0}

\author{Антон Літвіненка\footnote{Кіеў, Украіна; \url{tenebrosus.scriptor@gmail.com}. Пашыраная версія артыкула з прыкладамі апрацоўкі: \url{http://lvee.org/be/abstracts/76}}}
\maketitle

\begin{abstract}
A review of image sharpening enhancement algorithms devilered by FLOSS projects is provided together with discussion of their end-user characteristics and some theoretical aspects.
\end{abstract}

Крыніцы нярэзкасці ў выяве:
\begin{itemize}
  \item Памылка факусіроўкі;
  \item Нізкая якасць/«мяккасць» аб’ектыва;
  \item Характарыстыкі планшэтных сканэраў;
  \item Дрыжанне рук (у цемры) ці іншыя прычыны руху здымача;
\end{itemize}

Далей будзе разгледжаны шэраг алгарытмаў для выдалення шумоў з выявы і іх рэалізацыю ў вольных праграмных прадуктах: \href{http://www.gimp.org/}{GIMP} (GPL3+), \href{http://www.imagemagick.org/}{ImageMagick} (Apache 2.0), \href{http://gmic.sourceforge.net/}{G’MIC} (CeCILL license), \href{http://krita.org/}{Krita} (GPL2).

\paragraph*{USM.} Unsharp mask, альбо маска нярэзкасці. Алгарытм: змешванне выявы з яе гаўсавым размыццём. Наяўная амаль ва ўсіх рэдактарах, якія заяўляюць магчымасць апрацоўкі выяваў (нават пры аптычным друку здымкаў з плёнак) — {GIMP}, {ImageMagick}, {Krita}, {G'MIC}.

Перавагі:

\begin{itemize}
  \item Універсальнасць і распаўсюджанасць;
  \item Магчымасць застасавання парогу (рэзкасць павялічваецца толькі для тых фрагментаў выявы, якія адрозніваюцца ад навакольных на пэўную парогавую велічыню (застасоўваецца селектыўнае гаўсава размыццё);
\end{itemize}

Недахопы:

\begin{itemize}
  \item Нізкая сэлектыўнасць, не ўлічваецца марфалогія выявы;
  \item Метад накіраваны не на кампенсацыю эфектаў, якія прывялі да нярэзкасці, а на візуальнае успрыняцце здымку як больш рэзкага;
  \item Пры моцным павялічэнні рэзкасці каля краёў выявы з’яўляюцца артэфакты.
\end{itemize}

\paragraph*{Алгарытмы, заснаваныя на аналізе марфалогіі.} Шэраг метадаў, якія шукаюць контуры на выяве і імкнуцца падвысіць рэзкасць контуру, не кранаючы абласцей павольнага пераходу значэнняў пікселаў.

Рэалізацыі:
\begin{itemize}
  \item ImageMagick (складанне выявы з вынікам працы аналізатара марфалогіі LoG (лапласіян гаўсіяна)\cite{litv1})
\texttt{convert 1.png -define convolve:scale='100,100\%' -morphology Convolve 'Log:0x2' 1\_sharpen.png}
\end{itemize}

\begin{itemize}
  \item GIMP-плагін «erosion sharpening» (змешванне выявы з вынікамі застасавання да яе аперацый «dilate» ды «erode»).
\end{itemize}

Перавагі:
\begin{itemize}
  \item Больш селектыўныя ў параўнанні з USM.
\end{itemize}

Недахопы:
\begin{itemize}
  \item Складаней застасаваць парогавае значэнне, таму разам з рэзкасцю павялічваюць шум (патрабуе папярэдняга падаўлення шуму).
\end{itemize}

\paragraph*{Wavelet.} Алгарытм, у нечым падобны на папярэдні, дзе для выдзялення краёў ужываюцца вэйвлеты. Выконваецца вэйвлетны расклад выявы, і ўзмацняюцца некаторыя ягоныя складнікі.

Рэалізацыя: плагін для GIMP.

Перавагі:

\begin{itemize}
  \item Адзін з найбольш эфектыўных метадаў падвышэння рэзкасці;
  \item Можа ўжывацца разам з вэйвлетным падаўленнем шумоў (камбінаванне двух застасаванняў вэйвлетнага раскладу дае неблагія вынікі);
\end{itemize}

Недахопы:

\begin{itemize}
  \item Немагчыма застасаванне парогу.
\end{itemize}

\paragraph*{Алгарытмы, заснаваныя на пошуку ці ўгадванні функцыі распаўсюджання кропкі.}

Прычыны, якія выклікаюць размыццё выявы, маюць як стахастычную частку (якая вызначаецца выпадковымі дэфектамі, разбалансіроўкай частак аптычнае схемы, ці проста ўкладам шумоў) і функцыянальна-заканамерную (напрыклад, у выпадку размыцця рухам ці памылкі факусіроўкі значэнні пікселаў выніковай выявы знаходзяцца ў функцыянальнай залежнасці ад значэнняў пікселаў арыгінальнай). У выпадку, калі функцыянальна-заканамернае размыццё дамінуе (альбо калі такую функцыянальную залежнасць можна знайсці), веданне дакладнай функцыі размеркавання кропкі (PSF, point spread function) тэарэтычна дазваляе цалкам аднавіць арыгінальную выяву (з агаворкамі наконт краёў).

На практыцы ідэальнаму аднаўленню замінае брак ведаў дакладнай PSF і шум. Тым не менш, існуе шэраг рэалізацый, якія тым ці іншым спосабам імкнуцца ацаніць PSF (кіруючыся пэўнымі меркаваннямі наконт прычыны размыцця ці з агульных меркаванняў), і з меншым ці большым поспехам аднавіць выяву.

\begin{itemize}
  \item Deconvolution sharpening у {G'MIC} (алгарытм Рычардсана-Люсі \cite{litv2});
  \item Плагін Refocus у {GIMP} (алгарытм Вінера \cite{litv3}) — расфакусіроўка etc.;
  \item Refocus-it у {GIMP} (нэйронная сетка Хопфілда \cite{litv4}) — расфакусіроўка, размыццё рухам, гаўсава размыццё;
\end{itemize}

Перавагі:

\begin{itemize}
  \item Скіраваныя на тое, каб прыбраць прычыны ўзнікнення нярэзкасці;
  \item Могуць дазволіць выцягнуць інфармацыю з выявы там, дзе іншыя метады няздольныя нават тэарэтычна;
\end{itemize}

Недахопы:

\begin{itemize}
  \item Патрабуюць шмат вылічальных рэсурсаў;
  \item Часта атрымліваюцца выявы са скажэннямі.
\end{itemize}

Паводле асабістага досведу аўтар можа рэкамендаваць выкарыстанне вэйвлетнага алгарытму ў абсалютнай большасці выпадкаў.

Такім чынам, вольнае праграмнае забеспячэнне рэалізуе шырокі спектр магутных тэарэтычна абгрунтаваных алгарытмаў падвышэння рэзкасці выяваў, якія грунтуюцца як на візуальным падвышэнні рэзкасці, так і на супрацьдзеянні фізічным прычынам, якія вядуць да яе зніжэння. У адрозненне ад алгарытмаў падаўлення шумоў, алгарытмы падвышэння рэзкасці заўважна адрозніваюцца ад праграмы да праграмы і ў асноўным рэалізаваныя як плагіны альбо скрыпты. Дакладным лідэрам у колькасці рэалізаваных алгарытмаў з’яўляецца {GIMP} (асабліва калі браць да ўвагі {G'MIC} for {GIMP}).

\begin{thebibliography}{9}
\bibitem{litv1} \url{http://www.imagemagick.org/Usage/convolve/#sharpen}
\bibitem{litv2} JOSA, 62, 1, pp. 55-59 (1972); Astronomical Journal, 79, p. 745 (1974).
\bibitem{litv3} N. Wiener. Extrapolation, Interpolation, and Smoothing of Stationary Time Series. New York: Wiley, 1949.
\bibitem{litv4} PNAS, 79, 8, pp. 2554—2558 (1982).
\end{thebibliography}

\end{document}
