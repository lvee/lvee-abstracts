\documentclass[10pt, a5paper]{article}
\usepackage[T2A]{fontenc}
\usepackage{ucs}
\usepackage[utf8x]{inputenc}
\usepackage[polish,english,russian]{babel}
\usepackage{hyperref}
\usepackage[inner=2cm,top=1.8cm,outer=2cm,bottom=2.3cm,nohead]{geometry}
\usepackage{listings}
\usepackage{graphicx}
\usepackage{wrapfig}
\usepackage{longtable}
\usepackage{indentfirst}
\frenchspacing
\usepackage{fixltx2e} %text sub- and superscripts
\usepackage{icomma} % коскі ў матэматычным рэжыме
\PreloadUnicodePage{4}

\newcommand{\longpage}{\enlargethispage{\baselineskip}}
\newcommand{\shortpage}{\enlargethispage{-\baselineskip}}

\def\switchlang#1{\expandafter\csname switchlang#1\endcsname}
\def\switchlangbe{
\let\saverefname=\refname%
\def\refname{Літаратура}%
\def\figurename{Іл.}%
}
\def\switchlangen{
\let\saverefname=\refname%
\def\refname{References}%
\def\figurename{Fig.}%
}
\def\switchlangru{
\let\saverefname=\refname%
\let\savefigurename=\figurename%
\def\refname{Литература}%
\def\figurename{Рис.}%
}

\hyphenation{admi-ni-stra-tive}
\hyphenation{ex-pe-ri-ence}
\hyphenation{fle-xi-bi-li-ty}
\hyphenation{Py-thon}
\hyphenation{ma-the-ma-ti-cal}
\hyphenation{re-ported}
\hyphenation{imp-le-menta-tions}
\hyphenation{pro-vides}
\hyphenation{en-gi-neering}
\hyphenation{com-pa-ti-bi-li-ty}
\hyphenation{im-pos-sible}
\hyphenation{desk-top}
\hyphenation{elec-tro-nic}
\hyphenation{com-pa-ny}
\hyphenation{de-ve-lop-ment}
\hyphenation{de-ve-loping}
\hyphenation{de-ve-lop}
\hyphenation{da-ta-ba-se}
\hyphenation{plat-forms}
\hyphenation{or-ga-ni-za-tion}
\hyphenation{pro-gramming}
\hyphenation{in-stru-ments}
\hyphenation{Li-nux}
\hyphenation{en-vi-ron-ment}
\hyphenation{Te-le-pathy}
\hyphenation{Li-nux-ov-ka}

\def\progref!#1!{\texttt{#1}}
\renewcommand{\arraystretch}{2} %Іначай формулы ў матрыцы зліпаюцца з лініямі
\usepackage{array}

\def\interview #1 (#2), #3, #4, #5\par{

\section[#1, #3, #4]{#1, #5}
\def\qname{LVEE}
\def\aname{#1}
\def\q ##1\par{{\noindent \bf \qname: ##1 }\par}
\def\a{{\noindent \bf \aname: } \def\qname{L}\def\aname{#2}}
}

%\switchlang{en}
\begin{document}
\title{Платформа .NET Core 2.0 в Linux: миграция, разработка, графический интерфейс\footnote{\url{lav@etersoft.ru}, \url{https://lvee.org/en/abstracts/265}}}
\author{Виталий Липатов, St. Petersburg, Russian Federation}
\maketitle
\begin{abstract}
.NET Core platform is a cross-platform analogue of .NET\linebreak Framework on a base class library (BCL) level. It implements the .NET Standard 2.0 specification. This report is a review of development tools, development environment options and methods of creating graphical interfaces that are aimed at creating cross-platform user applications.
\end{abstract}
\section*{Платформа .NET Core}

По сути .NET Core является повторной разработкой .NET\linebreak Framework, но имеющей открытую лицензию (MIT) и работающей на всех платформах (Windows, Linux, MacOS). Обеспечивается совместимость c .NET Framework на уровне базовых классов  (BCL), согласно стандарту .NET Standard 2.0, к которому, по замыслу архитекторов Microsoft, должна стремиться любая существующая реализация .NET. На сегодняшний день .NET Core претендует на то, чтобы быть полноценной платформой промышленного уровня.

В составе SDK поставляется компилятор C\# с открытым исходным кодом из состава .NET Compiler Platform — Roslyn. Как и полагается современной платформе, имеется менеджер пакетов (Nuget), и репозиторий пакетов. Платформа очень быстро развивается (сотни коммитов различных разработчиков каждый день), появляются и исчезают проблемы, меняется схема сборки, учитываются нюансы различных дистрибутивов Linux.

К сожалению, во многом платформа выглядит чуждой на Linux-системах. Главные моменты: проект обязательно должен быть собран под .NET Core (в отличие от Mono, совместимого с .NET Framework и запускающего его программы без пересборки), а также отсутствует встроенная поддержка графического интерфейса.

Отдельно графический интерфейс может быть реализован следующими способами:

\begin{itemize}
  \item Создание приложения в модели ASP.NET Core и отображения клиенту интерфейса через произвольный браузер;
  \item Использование Electron для отображения интерфейса. Это придаст приложению свойства десктопности (отдельность в панели задач, нахождение в трее, взаимодействие с операционной системой);
  \item Использование Avalonia (свободный межплатформенный GUI для .NET);
  \item Использование экзотичного проекта electron-edge, выполняющего .NET и Node.js в одном процессе в Electron. На данный момент проект поддерживает только старую версию .NET Core 1.0 и не рассматривается.
\end{itemize}

\section*{Среды разработки и средства разработки GUI}

\subsection*{Electron}

Electron – это фреймворк для создания нативных приложений с использованием веб-технологий. Electron позволяет использовать браузерный движок в качестве основы для построения нативных приложений с GUI, добавляя к нему набор своих API, недоступных в обычном браузере.

Electron построен на основе Node.js (который в свою очередь использует v8) и Chromium. Приложение на Electron имеет основной процесс (Main), который является точкой входа в приложение, и по меньшей мере один процесс браузера (Renderer), в котором производится отрисовка GUI.

Main – это Node.js-приложение, которое управляет окнами\linebreak (Renderer), а также имеет доступ к API, связанным с системным GUI: строка меню, контекстное меню, работа с треем, диалоги, модальные окна и т. д. В Main также доступны встроенные модули Node.js (fs, http, net\ldots{}) и сторонние, устанавливаемые через пакетный менеджер npm.

В Renderer загружается HTML-страница, таблицы стилей и JS-скрипты. От обычного браузера отличается наличием доступа к модулям из Node.js и возможность обмена сообщениями с Main при помощи встроенного в Electron механизма IPC.

Примером реального приложения, написанного на Electron, является редактор Visual Studio Code.

\subsection*{Visual Studio Code}

VS Code – это кроссплатформенный редактор для кода с открытым исходным кодом от компании Microsoft, не имеющий ничего общего с небезызвестной IDE Visual Studio.

VS Code позиционируется как редактор, но по сути это нечто среднее между редактором и IDE. Основные возможности:

\begin{itemize}
  \item подсветка синтаксиса;
  \item автодополнение с учётом языковых конструкций;
  \item поддержка отладки;
  \item интеграция с системами контроля версий – поддержка Git встроена.
VS Code не завязан на какую-либо конкретную платформу или язык, а благодаря механизму плагинов можно добавлять поддержку различных языков и дополнительных инструментов. Поддержка JavaScript и TypeScript есть «из коробки». Редактор хорошо интегрируется со всеми популярными языками: C/C++, Java, Python, C\#, Ruby, PHP, Go.
\end{itemize}

\subsection*{Avalonia}

Avalonia – свободный кроссплатформенный GUI-фреймворк для .NET Framework / .NET Core. Avalonia во многом похож на WPF, который реализован для Windows, использует XAML и подразумевает использование MVVM-архитектуры приложений, но при этом не привязан к Windows-платформе и Microsoft.

Avalonia работает не только на десктопе (Linux, Windows,\linebreak MacOS), но и на мобильных устройствах (Android, iOS). Фреймворк имеет свой набор GUI-элементов, которые одинаково выглядят на всех системах.
Фреймворк пока что находится в alpha-версии, но он активно развивается и выглядит довольно перспективным решением для создания кроссплатформенных графических приложений. Его важным преимуществом является сходство с WPF, вплоть до использования тех же XAML-файл с описанием интерфейса, что упрощает портирование существующих WPF-приложений.

\end{document}
