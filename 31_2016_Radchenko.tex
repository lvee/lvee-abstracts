\documentclass[10pt, a5paper]{article}
\usepackage{pdfpages}
\usepackage{parallel}
\usepackage[T2A]{fontenc}
\usepackage{ucs}
\usepackage[utf8x]{inputenc}
\usepackage[polish,english,russian]{babel}
\usepackage{hyperref}
\usepackage{rotating}
\usepackage[inner=2cm,top=1.8cm,outer=2cm,bottom=2.3cm,nohead]{geometry}
\usepackage{listings}
\usepackage{graphicx}
\usepackage{wrapfig}
\usepackage{longtable}
\usepackage{indentfirst}
\usepackage{array}
\newcolumntype{P}[1]{>{\raggedright\arraybackslash}p{#1}}
\frenchspacing
\usepackage{fixltx2e} %text sub- and superscripts
\usepackage{icomma} % коскі ў матэматычным рэжыме
\PreloadUnicodePage{4}

\newcommand{\longpage}{\enlargethispage{\baselineskip}}
\newcommand{\shortpage}{\enlargethispage{-\baselineskip}}

\def\switchlang#1{\expandafter\csname switchlang#1\endcsname}
\def\switchlangbe{
\let\saverefname=\refname%
\def\refname{Літаратура}%
\def\figurename{Іл.}%
}
\def\switchlangen{
\let\saverefname=\refname%
\def\refname{References}%
\def\figurename{Fig.}%
}
\def\switchlangru{
\let\saverefname=\refname%
\let\savefigurename=\figurename%
\def\refname{Литература}%
\def\figurename{Рис.}%
}

\hyphenation{admi-ni-stra-tive}
\hyphenation{ex-pe-ri-ence}
\hyphenation{fle-xi-bi-li-ty}
\hyphenation{Py-thon}
\hyphenation{ma-the-ma-ti-cal}
\hyphenation{re-ported}
\hyphenation{imp-le-menta-tions}
\hyphenation{pro-vides}
\hyphenation{en-gi-neering}
\hyphenation{com-pa-ti-bi-li-ty}
\hyphenation{im-pos-sible}
\hyphenation{desk-top}
\hyphenation{elec-tro-nic}
\hyphenation{com-pa-ny}
\hyphenation{de-ve-lop-ment}
\hyphenation{de-ve-loping}
\hyphenation{de-ve-lop}
\hyphenation{da-ta-ba-se}
\hyphenation{plat-forms}
\hyphenation{or-ga-ni-za-tion}
\hyphenation{pro-gramming}
\hyphenation{in-stru-ments}
\hyphenation{Li-nux}
\hyphenation{sour-ce}
\hyphenation{en-vi-ron-ment}
\hyphenation{Te-le-pathy}
\hyphenation{Li-nux-ov-ka}
\hyphenation{Open-BSD}
\hyphenation{Free-BSD}
\hyphenation{men-ti-on-ed}
\hyphenation{app-li-ca-tion}

\def\progref!#1!{\texttt{#1}}
\renewcommand{\arraystretch}{2} %Іначай формулы ў матрыцы зліпаюцца з лініямі
\usepackage{array}

\def\interview #1 (#2), #3, #4, #5\par{

\section[#1, #3, #4]{#1 -- #3, #4}
\def\qname{LVEE}
\def\aname{#1}
\def\q ##1\par{{\noindent \bf \qname: ##1 }\par}
\def\a{{\noindent \bf \aname: } \def\qname{L}\def\aname{#2}}
}

\def\interview* #1 (#2), #3, #4, #5\par{

\section*{#1\\{\small\rm #3, #4. #5}}

\def\qname{LVEE}
\def\aname{#1}
\def\q ##1\par{{\noindent \bf \qname: ##1 }\par}
\def\a{{\noindent \bf \aname: } \def\qname{L}\def\aname{#2}}
}

\begin{document}
\title{Связанные открытые данные в университете\footnote{\url{iradche@gmail.com}, \url{http://lvee.org/ru/abstracts/230}}}
\author{Ирина Радченко, Михаил Навроцкий \\Санкт-Петербург, Russian Federation}
\maketitle
\begin{abstract}
LinkedUniversities.org is one of important projects in the field of Linked Open Data in European universities.
ITMO \linebreak University joined linked open data in universities initiative in 2014. ISST Labs is an organization responsible for developing LOD-IFMO project. I am going to give a talk on project progress and outcomes.
\end{abstract}
Уже несколько лет существует европейский проект связанных университетов LinkedUniversities.org, в рамках которого несколько европейских университетов выкладывают наборы связанных открытых данных.

В 2014 году в лаборатории ISST университета ИТМО~\cite{Radchenko1} был запущен проект LOD-IFMO~\cite{Radchenko2} "--- проект связанных открытых данных в российском университете. Этот проект был инициирован совместно с коллегами из исследовательской группы Agile Knowledge Engineering and Semantic Web (AKSW) Лейпцигского университета~\cite{Radchenko3}.

Одной из важных целей этого проекта было присоединение к европейской инициативе университетских связанных открытых данных. Для этого университетские открытые данные (данные по учебным курсам, публикациям, исследовательским проектам, \linebreak профессорско-преподавательскому составу и зданиям университета) необходимо сделать связанными открытыми данными, а точнее "--- перевести в форматы представления данных по модели RDF (Resource Description Framework) при помощи таких онтологий, как FOAF, BIBO, VIVO, AIISO и др. и связать их со связанными открытыми данными европейских университетов. Для перевода данных из университетских баз данных в RDF-форматы представления данных в проекте LOD-IFMO используется kOre framework (kOre: Lin*k*ed Data for *O*penScience Info*r*mation Int*e*gration).

Cерверная часть системы поддерживается при помощи платформы управления данными Virtuoso Universal Server~\cite{Radchenko4}. Это платформа с гибридной серверной архитектурой, позволяющая управлять как реляционными базами данных, так и системами связанных открытых данных, осуществлять управление контентом и веб-приложениями.

Фронт-энд для развертывания интерфейса доступа SPARQL-endpoints разрабатывался при помощи Pubby~\cite{Radchenko5}, который позволяет организовать интерфейс связанных данных на основе SPARQL-endpoints. Исходный код проекта выложен на GitHub~\cite{Radchenko6}.

Помимо технологической сложности развертывания подобных проектов (проектов с использованием связанных открытых данных) существует сложность в организации работы пользователей с данными, представленными по модели RDF. Для решения этой задачи необходимо разработать учебный курс по работе со связанными открытыми данными, а также систему понятной документации.

К сожалению, на настоящий момент времени русскоязычная документация на подобные системы почти не представлена. В силу этой причины участниками проекта было принято решение о создании раздела сайта университета ИТМО, посвященного особенностям использования связанных открытых данных в университете. В этом разделе планируется также выкладывать инструкции пользователя, словари данных, описание интерфейса доступа SPARQL-endpoints, документация на систему и тд. Предварительные макеты проекта расположены по следующему адресу: lu-itmo.herokuapp.com \url{https://lu-itmo.herokuapp.com/}.

\begin{thebibliography}{99}
\bibitem{Radchenko1} Лаборатория ISST университета ИТМО \url{http://semantics.ifmo.ru/}{semantics.ifmo.ru}.
\bibitem{Radchenko2} Проект LOD-IFMO \url{http://lod.ifmo.ru}.
\bibitem{Radchenko3} Группа Agile Knowledge Engineering and Semantic Web Лейпцигского университета \url{http://aksw.org/About.html}.
\bibitem{Radchenko4} Virtuoso Universal Server \url{http://virtuoso.openlinksw.com/}.
\bibitem{Radchenko5} Pubby \url{http://wifo5-03.informatik.uni-mannheim.de/pubby}.
\bibitem{Radchenko6} Исходный код проекта на GitHub \url{https://github.com/AKSW/itmolod}.
\end{thebibliography}
\end{document}
