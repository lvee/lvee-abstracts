\documentclass[10pt, a5paper]{article}
\usepackage{pdfpages}
\usepackage{parallel}
\usepackage[T2A]{fontenc}
\usepackage{ucs}
\usepackage[utf8x]{inputenc}
\usepackage[polish,english,russian]{babel}
\usepackage{hyperref}
\usepackage{rotating}
\usepackage[inner=2cm,top=1.8cm,outer=2cm,bottom=2.3cm,nohead]{geometry}
\usepackage{listings}
\usepackage{graphicx}
\usepackage{wrapfig}
\usepackage{longtable}
\usepackage{indentfirst}
\usepackage{array}
\newcolumntype{P}[1]{>{\raggedright\arraybackslash}p{#1}}
\frenchspacing
\usepackage{fixltx2e} %text sub- and superscripts
\usepackage{icomma} % коскі ў матэматычным рэжыме
\PreloadUnicodePage{4}

\newcommand{\longpage}{\enlargethispage{\baselineskip}}
\newcommand{\shortpage}{\enlargethispage{-\baselineskip}}

\def\switchlang#1{\expandafter\csname switchlang#1\endcsname}
\def\switchlangbe{
\let\saverefname=\refname%
\def\refname{Літаратура}%
\def\figurename{Іл.}%
}
\def\switchlangen{
\let\saverefname=\refname%
\def\refname{References}%
\def\figurename{Fig.}%
}
\def\switchlangru{
\let\saverefname=\refname%
\let\savefigurename=\figurename%
\def\refname{Литература}%
\def\figurename{Рис.}%
}

\hyphenation{admi-ni-stra-tive}
\hyphenation{ex-pe-ri-ence}
\hyphenation{fle-xi-bi-li-ty}
\hyphenation{Py-thon}
\hyphenation{ma-the-ma-ti-cal}
\hyphenation{re-ported}
\hyphenation{imp-le-menta-tions}
\hyphenation{pro-vides}
\hyphenation{en-gi-neering}
\hyphenation{com-pa-ti-bi-li-ty}
\hyphenation{im-pos-sible}
\hyphenation{desk-top}
\hyphenation{elec-tro-nic}
\hyphenation{com-pa-ny}
\hyphenation{de-ve-lop-ment}
\hyphenation{de-ve-loping}
\hyphenation{de-ve-lop}
\hyphenation{da-ta-ba-se}
\hyphenation{plat-forms}
\hyphenation{or-ga-ni-za-tion}
\hyphenation{pro-gramming}
\hyphenation{in-stru-ments}
\hyphenation{Li-nux}
\hyphenation{sour-ce}
\hyphenation{en-vi-ron-ment}
\hyphenation{Te-le-pathy}
\hyphenation{Li-nux-ov-ka}
\hyphenation{Open-BSD}
\hyphenation{Free-BSD}
\hyphenation{men-ti-on-ed}
\hyphenation{app-li-ca-tion}

\def\progref!#1!{\texttt{#1}}
\renewcommand{\arraystretch}{2} %Іначай формулы ў матрыцы зліпаюцца з лініямі
\usepackage{array}

\def\interview #1 (#2), #3, #4, #5\par{

\section[#1, #3, #4]{#1 -- #3, #4}
\def\qname{LVEE}
\def\aname{#1}
\def\q ##1\par{{\noindent \bf \qname: ##1 }\par}
\def\a{{\noindent \bf \aname: } \def\qname{L}\def\aname{#2}}
}

\def\interview* #1 (#2), #3, #4, #5\par{

\section*{#1\\{\small\rm #3, #4. #5}}

\def\qname{LVEE}
\def\aname{#1}
\def\q ##1\par{{\noindent \bf \qname: ##1 }\par}
\def\a{{\noindent \bf \aname: } \def\qname{L}\def\aname{#2}}
}

\begin{document}
\title{Параўнальны аналіз сродкаў кросплатформеннага праграмавання}
\author{Г. Злобін, А. Чмихало, Львів, Украіна\footnote{\url{zlobingg@gmail.com}, \url{http://lvee.org/en/abstracts/148}}}
\maketitle
\begin{abstract}
The article provides a comparative analysis of crossplatform programming tools. Crossplatform programming tools are divided into three groups: crossplatform compiled languages (Table 1), crossplatform programming languages at the execution level \linebreak (Table 2) and crossplatform interpreters (Table 3). The informa\-tion in Tables 1--3 is ordered by number of supported operating systems (descending). Table 4 provides information about \linebreak standardized libraries and frameworks used in crossplatform programming. To substantiate the analysis results, the index of popularity of programming languages ​​TIOBE is used.
\end{abstract}
Кросплатформеннасць "--- гэта здольнасць праграмнага забеспячэння працаваць больш чым на адной платформе або аперацыйнай сістэме. Кросплатформннасць праграмнага забеспячэння набыла асаблівае значэнне пасля завяршэння эры практычна непадзельнага панавання платформы Wintel (x86 + Microsoft Windows). Як вынікае з [1],   колькасць працоўных месцаў з не"=Wintel платформай у 2012 перавысіла 65\% і працягвае павялічвацца. Гэта зрабіла эканамічна прывабным кросплатформеннае праграмаванне ў галіне распрацоўкі прыкладнога праграмнага забеспячэння.

Мовы праграмавання, якія можна выкарыстоўваць для крос\-платформеннай распрацоўкі праграм, дзеляцца на тры групы:

\begin{itemize}
  \item кросплатформавыя мовы праграмавання на ўзроўні кампіляцыі "--- для гэтых моў існуюць кампілятары для розных платформаў (C, C++, Pascal, Fortran, Ада і г.д.);
  \item кросплатформавыя мовы на ўзроўні выканання (Java і C\#) "--- вынікам працы кампілятара ў гэтых мовах ёсць байт"=код, які можна запускаць на розных платформах з дапамогай віртуальных машын (Java VM для Java і CLR для C\#);
  \item кросплатформавыя інтэрпрэтатары "--- для гэтых моў \linebreak з'яўляюцца інтэрпрэтатары (PHP, Perl, Python, Tcl, Ruby і т.д.) для розных платформаў.
\end{itemize}

\begin{table}
\label{z_table1}
\caption{~}
	\centering
  \begin{tabular}{ p{1.6cm} p{4.6cm} p {3.6cm} }
\hline
    Інструмен\-тальная абалонка & Падтрымліваюцца кампілятары/колькасць моў праграмавання & Падтрымліваюцца АС / іх колькасць \\ \hline
    Qt Creator  & GCC, Clang, MinGW, MSVC, Linux ICC, GCCE, RVCT, WINSCW / 8  & Linux, OS X, Windows, Unix,  iOS, Android, Blackberry 10, WinRT, Embed. Linux, QNX / 10  \\
    Eclipse  & C/С++, Fortran/3  & AIX, FreeBSD, HP-UX, Linux, Mac OS X, OpenSolaris, Solaris, QNX, Windows, Android (AR) / 10 \\
    Free Pascal  & Free Pascal Compiler, Object Pascal, част. GNU Pascal, ISO Extended Pascal / 4  & DOS, FreeBSD, Linux, Mac OS X, Windows, Sun Solaris, Haiku / 7  \\
    Lazarus & Free Pascal Compiler / 1 & Linux, FreeBSD, Mac OS X, Windows, Android / 5  \\
    Code:: Blocks  & MinGW,~GCC~C/C++,~GNU GCC~(PowerPC,~ARM,~AVR), SDCC, Digital Mars (C/C++, D), Visual C++, Borland C++, Watcom, Intel C++, GNU Fortran, GNU ARM, GNU GDC / 15  & Windows, Linux, Mac OS X, Unix / 4 \\
    NetBeans IDE & C, C++, Ада / 3  & Windows, Linux, ~~ FreeBSD,  Solaris / 4  \\
    Embarcade\-ro~RAD~Stu\-dio XE7 & Delphi, С, C++ / 3 & Windows, Mac OS X, iOS, Android / 4  \\ \hline
  \end{tabular}
\end{table}

Разгледзім кароткія характарыстыкі кросплатформавых моў \linebreak праграмавання на ўзроўні кампіляцыі ў таблiцы 1.

\begin{table}
\label{z_table2}
\caption{~}
	\centering
  \begin{tabular}{ p{2cm} p {3.5cm} p{4cm} }
\hline
    Інстру\-ментальная абалонка & Падтрымліваюцца кампілятары/колькасць моў праграмавання & Падтрымліваюцца АС / іх колькасць \\ \hline
    Eclipse\footnotemark[1] & Java & AIX, FreeBSD, HP-UX, Linux, Mac OS X, OpenSolaris, Solaris, QNX, Microsoft Windows, Android (ARM) / 10  \\
    NetBeans IDE\footnotemark[1] & Java & Windows, Linux, FreeBSD,  Solaris / 4  \\
    IntelliJ IDEA\footnotemark[1] & Java & Linux, Mac OS X,  Windows / 3  \\
    AIDE\footnotemark[1] & Java &  Android \\
    Microsoft Visual Studio Code & C\#, Java/2 & Linux, Mac OS X / 2 \\
    .NET Core 5 & C\# &  Linux, Mac OS X / 2 \\
    Mono & C\# & Linux, MacOS / 2  \\
\hline
  \end{tabular}
\end{table}


У табліцы 2 прадстаўлены кароткія характарыстыкі  кросплатформавых моў  на ўзроўні  выканання\footnote{Праз вялікую колькасць інструментальных сродкаў для Java іх пералік няпоўны}.


У табліцы 3 прадстаўлены кароткія характарыстыкі  кросплатформавых інтэрпрэтатараў.

\begin{table}
\label{z_table3}  
\caption{~}
	\centering
  \begin{tabular}{ p{2cm} p {3.5cm} p{4cm} }
    \hline
    Інстру\-ментальная абаолонка & Падтрымліваюцца кампілятары~/ колькасць моў праграмавання & Падтрымліваюцца АС / іх колькасць \\ \hline
    Eclipse & Perl, PHP, JavaScript, Python, Ruby/5 & AIX, FreeBSD, HP-UX, Linux, Mac OS X, OpenSolaris, Solaris, QNX, Microsoft Windows, Android (ARM) / 10  \\
    NetBeans IDE & Java, JavaFX, PHP, JavaScript, HTML5, Python, Groovy / 7 & Windows, Linux, FreeBSD,  Solaris / 4  \\
    Embarcadero RAD Studio XE7 & HTML5 & Windows, Mac OS X, iOS, Android / 4  \\
    Xojo IDE Real  & Basic & OS X, Windows, Linux, iOS / 4  \\
    Komodo IDE/Komodo Edit & Perl, PHP, Python, Ruby, Tcl. JavaScript, CSS3, HTML5, XML,  XSLT / 10 & Linux, Mac OS X, Windows / 3  \\
    PyCharm & Python, JavaScript, HTML / 3 &  Windows, Linux, Mac OS X / 3 \\
    Aptana Studio 3 & JavaScript, PHP, Ruby, Python / 4 & Windows, Linux, Mac OS X / 3  \\
    Microsoft Visual Studio Code & Python, JavaScript, PHP, JSON, XML, HTML, CSS / 7 & Mac OS X, Linux / 2  \\ \hline
  \end{tabular}
\end{table}
На жаль, аўтарам невядомыя даследаванні папулярнасці сродкаў кросплатформеннага праграмавання, таму скарыстаемся індэксам TIOBE [2], па якім ацэньваюць папулярнасць моў праграмавання. Як вынікае з [2], найбольшай папулярнасцю больш за 10 гадоў карыстаюцца мовы праграмавання C (на 1.2015 г. 16,703\%) і Java (15,528\%), якія шырока выкарыстоўваюць для кросплатформеннага праграмавання. Менш папулярныя C++ (6,705\%), C\# (5,045\%), PHP (3,784\%), Python (2,613\%), Perl (2,256\%), Delphi / Object Pascal (0,837\%). Яшчэ адным фактарам адбору можа быць колькасць аперацыйных сістэм і колькасць апаратных платформаў, для якіх можна выкарыстоўваць сродкі кросплатформеннага распрацоўкі. Не менш важнымі для кросплатформеннага праграмавання з'яўляецца стандартызаваныя бібліятэкі часу выканання. У прыватнасці, стандартам стала бібліятэка мовы Сі. Cвои стандартныя бібліятэкі маюць C++, Java, Python, Ruby, якія падаюцца разам са сродкамі распрацоўкі і даступныя на падтрымоўваных платформах. Варта адзначыць таксама некаторыя вялікія кросплатформавыя бібліятэкі "--- такія як Qt, GTK+, FLTK, STL, Boost, OpenGL, SDL, OpenAL, OpenCL. У табліцы 4 стандартныя бібліятэкі часу выканання падзеленыя на бібліятэкі з адкрытым кодам і бібліятэкі з зачыненым кодам.

\begin{table}
\caption{Стандартныя бібліятэкі і праграмныя каркасы з адкрытым і зачыненым кодам}
\label{z_table4}
  \centering
  \begin{tabular}{p{6cm} p{3.5cm}}
     \hline
    Бібліятэкі і праграмныя каркасы з адкрытым кодам / адкрытыя стандарты & Бібліятэкі і праграмныя каркасы з зачыненым кодам \\ \hline
    Boost, GIMP ToolKit, GTK+, FLTK, Kivy, OpenCV, OpenCL, OpenGL, SDL, Apache Cordova,  Tk &  \\
    OpenAL (раннія версіі) & OpenAL (пазнейшыя версіі) \\
    Qt & Qt \\
    Simple DirectMedia Layer & Unity3D \\ \hline
  \end{tabular}
\end{table}
\subsection*{Высновы}

\begin{enumerate}
  \item Па папулярнасці моў праграмавання сярод кампілятараў першае месца займае мова праграмавання C, сярод інтэрпрэтатараў "--- мова Java.
  \item Па індэксе Tiobe ў студзені 2015 мова праграмавання Delpi / Object Pascal займае 20 месца.
  \item ППа колькасці аперацыйных сістэм, у якіх можна скарыстацца згаданымі ў аглядзе сродкамі распрацоўкі, яны размешчаны ў табліцах 1--3. Апошняя радок у табліцы 1 (кросплатформеннага мовы праграмавання на ўзроўні кампіляцыі) з двума падтрымоўванымі АС займае Microsoft Visual Studio Code, ў табліцы 2 (кросплатформенныя мовы праграмавання на ўзроўні выканання) тры апошнія радкі з двума падтрыманымі АС займаюць Microsoft Visual Studio Code, .NET Core 5, Mono, у табліцы 3 (кросплатформавыя інтэрпрэтатары) апошнi радок з двума падтрыманымі АС займае Microsoft Visual Studio Code.
  \item Як і варта было чакаць колькасць стандартных бібліятэк з адкрытым кодам амаль у чатыры разы перавышае колькасць стандартных бібліятэк з зачыненым кодам.
\end{enumerate}

\section*{Iнфармацыйныя сродкi}

\begin{enumerate}
\item D. Thompson. The 11 Most Fascinating Charts From Mary Meeker's Epic Slideshow of Internet Trends. \url{http://tinyurl.com/ozwck2k}
\item TIOBE Index for May 2015. \url{http://www.tiobe.com/index.php/content/paperinfo/tpci/index.html} \end{enumerate}
\end{document}
