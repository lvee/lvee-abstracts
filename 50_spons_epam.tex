\documentclass[10pt, a5paper]{article}
\usepackage{pdfpages}
\usepackage{parallel}
\usepackage[T2A]{fontenc}
\usepackage{ucs}
\usepackage[utf8x]{inputenc}
\usepackage[polish,english,russian]{babel}
\usepackage{hyperref}
\usepackage{rotating}
\usepackage[inner=2cm,top=1.8cm,outer=2cm,bottom=2.3cm,nohead]{geometry}
\usepackage{listings}
\usepackage{graphicx}
\usepackage{wrapfig}
\usepackage{longtable}
\usepackage{indentfirst}
\usepackage{array}
\newcolumntype{P}[1]{>{\raggedright\arraybackslash}p{#1}}
\frenchspacing
\usepackage{fixltx2e} %text sub- and superscripts
\usepackage{icomma} % коскі ў матэматычным рэжыме
\PreloadUnicodePage{4}

\newcommand{\longpage}{\enlargethispage{\baselineskip}}
\newcommand{\shortpage}{\enlargethispage{-\baselineskip}}

\def\switchlang#1{\expandafter\csname switchlang#1\endcsname}
\def\switchlangbe{
\let\saverefname=\refname%
\def\refname{Літаратура}%
\def\figurename{Іл.}%
}
\def\switchlangen{
\let\saverefname=\refname%
\def\refname{References}%
\def\figurename{Fig.}%
}
\def\switchlangru{
\let\saverefname=\refname%
\let\savefigurename=\figurename%
\def\refname{Литература}%
\def\figurename{Рис.}%
}

\hyphenation{admi-ni-stra-tive}
\hyphenation{ex-pe-ri-ence}
\hyphenation{fle-xi-bi-li-ty}
\hyphenation{Py-thon}
\hyphenation{ma-the-ma-ti-cal}
\hyphenation{re-ported}
\hyphenation{imp-le-menta-tions}
\hyphenation{pro-vides}
\hyphenation{en-gi-neering}
\hyphenation{com-pa-ti-bi-li-ty}
\hyphenation{im-pos-sible}
\hyphenation{desk-top}
\hyphenation{elec-tro-nic}
\hyphenation{com-pa-ny}
\hyphenation{de-ve-lop-ment}
\hyphenation{de-ve-loping}
\hyphenation{de-ve-lop}
\hyphenation{da-ta-ba-se}
\hyphenation{plat-forms}
\hyphenation{or-ga-ni-za-tion}
\hyphenation{pro-gramming}
\hyphenation{in-stru-ments}
\hyphenation{Li-nux}
\hyphenation{sour-ce}
\hyphenation{en-vi-ron-ment}
\hyphenation{Te-le-pathy}
\hyphenation{Li-nux-ov-ka}
\hyphenation{Open-BSD}
\hyphenation{Free-BSD}
\hyphenation{men-ti-on-ed}
\hyphenation{app-li-ca-tion}

\def\progref!#1!{\texttt{#1}}
\renewcommand{\arraystretch}{2} %Іначай формулы ў матрыцы зліпаюцца з лініямі
\usepackage{array}

\def\interview #1 (#2), #3, #4, #5\par{

\section[#1, #3, #4]{#1 -- #3, #4}
\def\qname{LVEE}
\def\aname{#1}
\def\q ##1\par{{\noindent \bf \qname: ##1 }\par}
\def\a{{\noindent \bf \aname: } \def\qname{L}\def\aname{#2}}
}

\def\interview* #1 (#2), #3, #4, #5\par{

\section*{#1\\{\small\rm #3, #4. #5}}

\def\qname{LVEE}
\def\aname{#1}
\def\q ##1\par{{\noindent \bf \qname: ##1 }\par}
\def\a{{\noindent \bf \aname: } \def\qname{L}\def\aname{#2}}
}

\begin{document}
\title{Голос спонсора: EPAM Systems}
%\author{}
\date{}
\maketitle

Компания EPAM Systems не первый год является спонсором международной конференции разработчиков и пользователей свободного программного обеспечения LVEE (Linux Vacation / Eastern Europe). Этот год также не стал исключением. Пожалуй, LVEE является самым значимым событием для русскоязычных разработчиков и тестировщиков Open Source. Каждое лето здесь встречаются начинающие специалисты и «ветераны»"=разработчики из десятка стран для обмена опытом и общения на профессиональные темы. Наши специалисты также активно участвуют в данной конференции: в качестве докладчиков и организаторов/волонтёров. Это уникальная в своём роде конференция, и именно поэтому EPAM Systems очередной раз принимает участие в LVEE в качестве спонсора.


EPAM Systems "--- одна из крупнейших компаний"=поставщиков\linebreak услуг в области разработки программного обеспечения и решений на территории СНГ и Центральной и Восточной Европы. Созданная в 1993 году, сегодня она имеет представительства в 12 странах мира, в штате работают более 9 тыс. сотрудников, из которых более 3 тыс. "--- в Беларуси. Рост компании обеспечивается за счет собственных обучающих программ и передаче опыта от больших специалистов до начинающих разработчиков. Компания EPAM Systems выполняет проекты более чем в 30 странах мира. Основные направления деятельности: разработка, тестирование, сопровождение и поддержка заказного программного обеспечения и бизнес"=приложений, а также ИТ"=консалтинг с учетом отраслевой специфики бизнеса.

Наша компания участвует в проектах с такими крупными, хорошо известными заказчиками как Google, Novell, Infoblox, Parallels, 10Gen и др., так и с небольшими, в том числе и с начинающими свой путь в софтверном бизнесе.


К примеру, для Infoblox была реализована связка между WebUI с BIND и DHCP. Для этого был разработан комплекс решений под управлением Shell и Python скриптов, а также механизм позволяющий вносить правки в BIND и DHCP на языке C. Также была разработана развернутая функциональность, автоматизирующая инсталляцию новых устройств и их эксплуатацию, что позволяет значительно упростить управление данными. Встроенный Web"=интерфейс позволяет разворачивать, управлять сервисами DNS, DNSSEC, DHCP, IPAM, устанавливать новые версии ПО, архивировать и восстанавливать из архивов необходимые данные, восстанавливать их после аварии, проводить мониторинг сети и создавать отчеты без необходимости обращения к командной строке.


Еще одним решением, реализованным для компании Infoblox, являлся программный продукт, позволяющий контролировать сетевые изменения, таким образом, облегчая идентификацию трудноуловимых проблем конфигурации и соответствие требованиям. Вместо того чтобы просто регистрировать изменения, система использует внесенную информацию для проверки, анализа и автоматической обработки сетевых изменений. Благодаря инновационной, квалифицированной, глубокой технике логического анализа, программа изолирует проблемы исправности и конфигурации до того, как они могут вызвать более серьезные сбои.


Разработанная для анализа сложных сетей система изучает сеть, собирает ключевую информацию, применяет встроенную технику логического анализа и создает оценку исправности сети и список проблем, требующих принятие мер для улучшения качества работы сети.


Правильное использование свободного ПО в разработках сокращает и расходы на покупку лицензионных программ, и трудозатраты при создании коммерческого ПО. Немалую роль для достижения превосходного результата играет привлечение к разработке опытных специалистов. LVEE способствует появлению таких специалистов, развитию их навыков и расширению кругозора. Хотелось бы пожелать участникам конференции интересных проектов и максимум пользы от участия в LVEE.


\end{document}


