\documentclass[10pt, a5paper]{article}
\usepackage{ucs}
\usepackage[utf8]{inputenc}
\usepackage[T2A]{fontenc}
\usepackage[english, russian]{babel}
\usepackage{hyperref}
\usepackage{geometry}
\frenchspacing
\begin{document}
\title{Голос спонсора: EPAM Systems}
%\author{}
\date{}
\maketitle

Компания EPAM Systems не первый год является спонсорами научно-практической конференции разработчиков и тестировщиков свободного программного обеспечения LVEE (Linux Vacation / Eastern Europe). Этот год также не стал исключением. LVEE является самым значимым событием для разработчиков и пользователей Open Source, по скольку именно там, в лесной глуши, встречаются для обмена опытом с отечественными и зарубежными коллегами как начинающие специалисты, так и “ветераны”-разработчики.  Наши специалисты также активно участвуют в данной конференции: в качестве докладчиков и организаторов/волонтёров. Это уникальная в своем роде конференция, и именно по этому EPAM Systems очередной раз принимает участие в LVEE в качестве спонсора. 

EPAM Systems "--- одна из крупнейших компаний-поставщиков услуг в области разработки программного обеспечения и решений на территории СНГ и  Центральной и Восточной Европе. Созданная в 1993 году, сегодня она имеет 17 представительств в 8 странах мира, в штате работают более 6 тыс. сотрудников, из которых 2,5 тыс. "--- в Беларуси. Рост компании обеспечивается засчет собственных обучающих программ и передаче опыта от больших специалистов до начинающих разработчиков. Компания EPAM Systems выполняет проекты более чем в 30 странах мира. Основные направления деятельности: разработка, тестирование, сопровождение и поддержка заказного программного обеспечения и бизнес-приложений, а также ИТ-консалтинг с учетом отраслевой специфики бизнеса.

Наша компания участвует в проектах с такими крупными, хорошо известными заказчиками как Google, Novell, Infoblox, Parallels и др. так и с небольшими Neterion, Viaclix, WinBox, в том числе и начинающими свой путь в софтверном бизнесе.

К примеру, для Infoblox была реализована связка между WebUI с BIND и DHCP. Для этого был разработан комплекс решений под управлением Shell и Python скриптов, а также механизм позволяющий вносить правки в BIND и DHCP на языке C. Также был разработан развернутый функционал, автоматизирующий инсталляцию новых устройств и их эксплуатацию, что позволяет значительно упростить управление данными.  Встроенный WEB интерфейс позволяет разворачивать, управлять сервисами DNS, DNSSEC, DHCP, IPAM, устанавливать новые версии ПО, архивировать и восстанавливать из архивов необходимые данные, восстанавливать их после аварии, проводить мониторинг сети и создавать отчеты без необходимости обращения к командной строке. 

Еще одним решением реализованным для компании Infoblox являлся программный продукт позволяющий контролировать сетевые изменения, таким образом облегчая идентификацию трудноуловимых проблем конфигурации и соответствие требованиям. Вместо того, чтобы просто регистрировать изменения, система использует внесенную информацию для проверки, анализа и автоматической обработке сетевых изменений. Благодаря инновационной квалифицированной глубокой технике логического анализа,  программа изолирует проблемы исправности и конфигурации до того, как они могут вызвать более серьезные проблемы. 

Разработанная для анализа сложных сетей, система изучает сеть, собирает ключевую информацию, применяет встроенную технику логического анализа и создает оценку исправности сети и список проблем, требующих принятие мер для улучшения качества работы сети.  

Правильное использование свободного ПО в разработках сокращает и расходы на покупку лицензионных программ и трудозатраты при создании коммерческого ПО. Не малую роль для достижения превосходного результата играет привлечение к разработке опытных специалистов. LVEE способствует появлению таких специалистов, развитию их навыков и расширению кругозора. Хотелось бы пожелать участникам конференции интересных проектов и получить максимум пользы от участия в LVEE.
\end{document}


