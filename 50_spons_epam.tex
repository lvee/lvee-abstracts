\documentclass[10pt, a5paper]{article}
\usepackage{pdfpages}
\usepackage{parallel}
\usepackage[T2A]{fontenc}
\usepackage{ucs}
\usepackage[utf8x]{inputenc}
\usepackage[polish,english,russian]{babel}
\usepackage{hyperref}
\usepackage{rotating}
\usepackage[inner=2cm,top=1.8cm,outer=2cm,bottom=2.3cm,nohead]{geometry}
\usepackage{listings}
\usepackage{graphicx}
\usepackage{wrapfig}
\usepackage{longtable}
\usepackage{indentfirst}
\usepackage{array}
\newcolumntype{P}[1]{>{\raggedright\arraybackslash}p{#1}}
\frenchspacing
\usepackage{fixltx2e} %text sub- and superscripts
\usepackage{icomma} % коскі ў матэматычным рэжыме
\PreloadUnicodePage{4}

\newcommand{\longpage}{\enlargethispage{\baselineskip}}
\newcommand{\shortpage}{\enlargethispage{-\baselineskip}}

\def\switchlang#1{\expandafter\csname switchlang#1\endcsname}
\def\switchlangbe{
\let\saverefname=\refname%
\def\refname{Літаратура}%
\def\figurename{Іл.}%
}
\def\switchlangen{
\let\saverefname=\refname%
\def\refname{References}%
\def\figurename{Fig.}%
}
\def\switchlangru{
\let\saverefname=\refname%
\let\savefigurename=\figurename%
\def\refname{Литература}%
\def\figurename{Рис.}%
}

\hyphenation{admi-ni-stra-tive}
\hyphenation{ex-pe-ri-ence}
\hyphenation{fle-xi-bi-li-ty}
\hyphenation{Py-thon}
\hyphenation{ma-the-ma-ti-cal}
\hyphenation{re-ported}
\hyphenation{imp-le-menta-tions}
\hyphenation{pro-vides}
\hyphenation{en-gi-neering}
\hyphenation{com-pa-ti-bi-li-ty}
\hyphenation{im-pos-sible}
\hyphenation{desk-top}
\hyphenation{elec-tro-nic}
\hyphenation{com-pa-ny}
\hyphenation{de-ve-lop-ment}
\hyphenation{de-ve-loping}
\hyphenation{de-ve-lop}
\hyphenation{da-ta-ba-se}
\hyphenation{plat-forms}
\hyphenation{or-ga-ni-za-tion}
\hyphenation{pro-gramming}
\hyphenation{in-stru-ments}
\hyphenation{Li-nux}
\hyphenation{sour-ce}
\hyphenation{en-vi-ron-ment}
\hyphenation{Te-le-pathy}
\hyphenation{Li-nux-ov-ka}
\hyphenation{Open-BSD}
\hyphenation{Free-BSD}
\hyphenation{men-ti-on-ed}
\hyphenation{app-li-ca-tion}

\def\progref!#1!{\texttt{#1}}
\renewcommand{\arraystretch}{2} %Іначай формулы ў матрыцы зліпаюцца з лініямі
\usepackage{array}

\def\interview #1 (#2), #3, #4, #5\par{

\section[#1, #3, #4]{#1 -- #3, #4}
\def\qname{LVEE}
\def\aname{#1}
\def\q ##1\par{{\noindent \bf \qname: ##1 }\par}
\def\a{{\noindent \bf \aname: } \def\qname{L}\def\aname{#2}}
}

\def\interview* #1 (#2), #3, #4, #5\par{

\section*{#1\\{\small\rm #3, #4. #5}}

\def\qname{LVEE}
\def\aname{#1}
\def\q ##1\par{{\noindent \bf \qname: ##1 }\par}
\def\a{{\noindent \bf \aname: } \def\qname{L}\def\aname{#2}}
}

\begin{document}
\title{Голос спонсора: EPAM Systems}
%\author{}
\date{}
\maketitle

\subsection*{Infoblox в вопросах и ответах, или Шанс вступить на путь хакера и сделать при этом карьеру}

\textbf{Infoblox:  Кто это? Что это? Где это?}

Infoblox "--- довольно известная в кремниевой долине компания, производящая стоечные 
сервера, призванные помочь в управлении крупными сетями со сложной инфраструктурой. 
Основной программный продукт компании, NIOS, является довольно сложной, с технической 
точки зрения, системой состоящей из множества взаимодействующих компонент, что 
автоматически делает его интересным для серьезных разработчиков.

\textbf{А причем тут хакеры?}

Фундаментом всей системы служит разрабатываемый собственными силами дистрибутив 
Linux, основанный на Fedora 12, и имеющий кодовое название Granville.  Infoblox делает 
Granville 64-битным, чтобы следовать в ногу со временем. В EPAM Systems есть специальная 
команда, которая занимается разработкой и поддержкой Granville, ведь NIOS имеет в среднем 
около 4 релизов в год. Работа с Granville "--- это шанс вступить на путь настоящего хакера.

\textbf{И это все, что делают инженеры? Не маловато будет?}

Infoblox вносит свой вклад в развитие ПО с открытым кодом: все модификации open-source 
кода публикуются под лицензией GPL. Кроме того высокий профессиональный уровень 
инженеров EPAM Systems, работающих с Infoblox, позволяет им в свободное время принимать 
участие в разработке других проектов с открытым кодом, среди которых есть даже ядро 
Linux.

\textbf{Сказали «А»--говорите и «Б»!}

И если Granville является лишь основой, то ядро системы это, во-первых, распределенное 
(сервера имеют возможность организации в сетку с центральным управлением) серверное 
приложение для обработки запросов управления сетью и сетевыми службами, построенное 
по трехуровневой архитектуре, и во-вторых ряд служб-демонов (например DNS или DHCP 
службы), обрабатывающих специфические запросы возникающие в инфраструктуре сети.

\textbf{Огласите весь список.. инструментария!}

Инженеры EPAM Systems постоянно используют несколько языков программирования и множество 
различных инструментов для решения каждодневных задач. Кроме богатого инструментария, 
доступного разработчикам на Unix подобных системах, достойны упоминания такие 
классические вещи, как C, Perl и Bash, без которых эффективная разработка под Unix системы 
просто немыслима, современные, перспективные и быстроразвивающиеся Ruby и Python, 
которые выбрали для себя Google, NASA, CERN и Motorola "--- и это список лишь наиболее 
часто используемых инструментов, с которыми приходится иметь дело.

Ключевым компонентом NIOS можно назвать движок управления базами данных oneDB, 
реализованный поверх Berkeley DB. В Infoblox добавили возможность работать с 
подмножеством SQL, а также подобие слоя объектно-реляционного отображения, при это не 
потеряв гибкости и легковесности BDB.

\textbf{А кому все это нужно? Кто это покупает?}

Развитие новых технологий является важной составляющей деятельности Infoblox.  Так, 
совместно с Boeing, IBM, HP, Intel, \linebreak Microsoft, US National Security Agency и другими заинтересованными сторонами компания участвует в разработке нового протокола под 
названием IF-MAP, к которому уже сегодня проявляют большой интерес многие компании 
как изнутри, так и извне IT индустрии. Infoblox, к тому же, поставляет реализацию IF-MAP сервера, как один из своих продуктов. Разработчики EPAM Systems имеют тесное отношение к успешному продвижению этого продукта.

\textbf{Так а чем вы лучше других?}

Больным местом многих отечественных компаний разработчиков ПО является полное, либо 
частичное, отсутствие четкого процесса изготовления продукта. Это часто приводит к 
конфликтам, снижению производительности, потере квалифицированного персонала. 
Работающие с Infoblox инженеры EPAM Systems полностью интегрированы в процессы Infoblox. 
Здесь практикуют модульное тестирование и рецензирование кода, есть четкое расписание, 
которое не изменяется от одного только желания менеджмента добавить новую возможность 
в следующий релиз. Проектная документация исправно ведется и находится в свободном 
доступе. Специалисты рабочей группы имеют возможность руководить проектами Infoblox, разрабатывая проектную документацию и дизайн самостоятельно, тем самым поднимая сотрудничество EPAM Systems и Infoblox на более высокий уровень.


\textbf{«Выдавали тайны» Infoblox:}

\noindent Михаил Бойко, Senior Software Engineering Manager в EPAM Systems;
Олег Орёл,  Lead Software Engineer в EPAM Systems;
Даниил Прищепа, Software Engineering Manager в EPAM Systems.



\textbf{Готова к вашим вопросам служба рекрутинга EPAM Systems}

Проекту срочно нужны:
\begin{itemize}
\item Perl разработчики.
\item С/С++ разработчики.
\item Linux Kernel разработчики.
\end{itemize}


\textbf{Команда ждет своих новых героев!}


\end{document}


