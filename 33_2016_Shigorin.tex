\documentclass[10pt, a5paper]{article}
\usepackage{pdfpages}
\usepackage{parallel}
\usepackage[T2A]{fontenc}
\usepackage{ucs}
\usepackage[utf8x]{inputenc}
\usepackage[polish,english,russian]{babel}
\usepackage{hyperref}
\usepackage{rotating}
\usepackage[inner=2cm,top=1.8cm,outer=2cm,bottom=2.3cm,nohead]{geometry}
\usepackage{listings}
\usepackage{graphicx}
\usepackage{wrapfig}
\usepackage{longtable}
\usepackage{indentfirst}
\usepackage{array}
\newcolumntype{P}[1]{>{\raggedright\arraybackslash}p{#1}}
\frenchspacing
\usepackage{fixltx2e} %text sub- and superscripts
\usepackage{icomma} % коскі ў матэматычным рэжыме
\PreloadUnicodePage{4}

\newcommand{\longpage}{\enlargethispage{\baselineskip}}
\newcommand{\shortpage}{\enlargethispage{-\baselineskip}}

\def\switchlang#1{\expandafter\csname switchlang#1\endcsname}
\def\switchlangbe{
\let\saverefname=\refname%
\def\refname{Літаратура}%
\def\figurename{Іл.}%
}
\def\switchlangen{
\let\saverefname=\refname%
\def\refname{References}%
\def\figurename{Fig.}%
}
\def\switchlangru{
\let\saverefname=\refname%
\let\savefigurename=\figurename%
\def\refname{Литература}%
\def\figurename{Рис.}%
}

\hyphenation{admi-ni-stra-tive}
\hyphenation{ex-pe-ri-ence}
\hyphenation{fle-xi-bi-li-ty}
\hyphenation{Py-thon}
\hyphenation{ma-the-ma-ti-cal}
\hyphenation{re-ported}
\hyphenation{imp-le-menta-tions}
\hyphenation{pro-vides}
\hyphenation{en-gi-neering}
\hyphenation{com-pa-ti-bi-li-ty}
\hyphenation{im-pos-sible}
\hyphenation{desk-top}
\hyphenation{elec-tro-nic}
\hyphenation{com-pa-ny}
\hyphenation{de-ve-lop-ment}
\hyphenation{de-ve-loping}
\hyphenation{de-ve-lop}
\hyphenation{da-ta-ba-se}
\hyphenation{plat-forms}
\hyphenation{or-ga-ni-za-tion}
\hyphenation{pro-gramming}
\hyphenation{in-stru-ments}
\hyphenation{Li-nux}
\hyphenation{sour-ce}
\hyphenation{en-vi-ron-ment}
\hyphenation{Te-le-pathy}
\hyphenation{Li-nux-ov-ka}
\hyphenation{Open-BSD}
\hyphenation{Free-BSD}
\hyphenation{men-ti-on-ed}
\hyphenation{app-li-ca-tion}

\def\progref!#1!{\texttt{#1}}
\renewcommand{\arraystretch}{2} %Іначай формулы ў матрыцы зліпаюцца з лініямі
\usepackage{array}

\def\interview #1 (#2), #3, #4, #5\par{

\section[#1, #3, #4]{#1 -- #3, #4}
\def\qname{LVEE}
\def\aname{#1}
\def\q ##1\par{{\noindent \bf \qname: ##1 }\par}
\def\a{{\noindent \bf \aname: } \def\qname{L}\def\aname{#2}}
}

\def\interview* #1 (#2), #3, #4, #5\par{

\section*{#1\\{\small\rm #3, #4. #5}}

\def\qname{LVEE}
\def\aname{#1}
\def\q ##1\par{{\noindent \bf \qname: ##1 }\par}
\def\a{{\noindent \bf \aname: } \def\qname{L}\def\aname{#2}}
}

\begin{document}
\title{Восьмая платформа BaseALT\footnote{\url{mike@altlinux.org}, \url{http://lvee.org/ru/abstracts/225}}}
\author{Михаил Шигорин, Москва, Россия}
\maketitle
\begin{abstract}
BaseALT p8 Platform has been released Summer 2016 following the three-years-old ALT Linux p7 Platform; let's look into the news of the release and the corresponding distributions with spins including starterkits, ALT Workstation, and Education.
\end{abstract}
Летом 2016 года выпущена восьмая платформа BaseALT (p8) как следующая за седьмой платформой ALT Linux (p7), опубликованной три года тому; доклад посвящён новостям в подходе к выпуску и публикации дистрибутивов и иных сборок, включая стартовые наборы, Альт Рабочая станция, комплект для образования.

\subsection*{Репозиторий}

BaseALT p8 "--- новая стабильная инфраструктура репозиториев пакетов СПО, созданная и развиваемая в рамках проекта Sisyphus командой ALT Linux Team (инструментарий сам является свободным ПО).
Стандартными архитектурами пакетов являются i586 и x86\_64; кроме того, добавлен репозиторий для архитектуры \linebreak AArch64 (64-битный ARM), а к осени планируется выпуск репозитория для аппаратной платформ «Эльбрус».

\subsection*{Стартовые наборы}

Мы предлагаем пользователям, предпочитающим самостоятельно определять состав и оформление системы, загрузочные образы стартеркитов для архитектур i586 и x86\_64. Поддерживается 13 вариантов окружений рабочего стола (Cinnamon, Enlightenment, GNOME, GNUstep, IceWM, KDE4, KDE5, LXDE, LXQt, MATE, TDE, WindowMaker, Xfce), а также 9 специализированных вариантов дистрибутива (rescue для восстановления систем, основной server, jeos с минималистичным инсталятором, builder для воспроизводимой сборки пакетов, vm-net для виртуальных машин, cloud с cloud-init для облачных виртуальных машин, ovz-generic для \linebreak OpenVZ, server-openstack с OpenStack и server-pve с Proxmox). Среди пакетов, ориентированных на корпоративное применение, также сервер контроллера домена Active Directory (Samba-DC) и сервер групповой работы с функциональностью Microsoft Exchange (SOGo).

При этом стартеркиты являются стартовыми наборами, а не  дистрибутивами в прямом смысле этого слова, т.е. не представляют из себя законченное решение.

\subsection*{Дистрибутивы}

Дистрибутивы, существующие в рамках Восьмой платформы, частично основаны на представленных в стартовых наборах наработках и на сегодняшний день их насчитывается три: Альт Сервер, Базальт Рабочая станция и Альт Образование.

\textbf{Альт Сервер 8.0} "--- многофункциональный дистрибутив для серверов с  удобным пользовательским интерфейсом для настройки и с возможностью использования в качестве рабочей станции разработчика. Дистрибутив построен на основе графической среды MATE со стандартным набором пользвоательских приложений. Основу его функциональности составляет комплект «BaseALT-домен»: взаимосвязанные серверы LDAP, Kerberos, DNS, Samba, DHCP, \linebreak Postfix, Dovecot, сервер сетевой загрузки, сервер обновлений. Предусмотрена возможность развернуть только определённые службы или использовать их отдельно, без Alterator.

\textbf{Базальт Рабочая станция 8.0} включает ПО для типовых пользовательских задач: электронная почта, работа с документами и презентациями, прослушивание аудиофайлов и просмотр видео, работа в сети Интернет. На этапе установки пользователю предоставляется выбор разворачиваемых решений по области применения (например, виртуализация, мультимедиа и т.д.). Основу пользовательского окружения составляют рабочая среда  MATE, офисный пакет LibreOffice версии 5.1 и браузер Firefox.

\textbf{Альт Образование 8.0} "--- это простая в установке система, ориентированная на учреждения общего и среднего образования. В состав входят средства управление классом, возможность stateless-использования (с восстановлением состояния рабочего места после завершения сеанса), средства централизованной аутентификации по сети (через Active Directory и LDAP/Kerberos) с графическими средствами настройки, а также стандартные средства синхронизации времени, управления пользователями, группами и т.д. В качестве рабочей среды предоставляется XFCE 4.12 (опционально KDE 5.7.1), также предусмотрен стандартный набор пакетов работы в сети Интернет, с документами, графикой, анимацией, для обработки звука и видео, разработки ПО. Предусмотрено ПО для образования (Moodle, РУЖЕЛЬ, Mediawiki и др.), WINE и средства антивирусной защиты.

\end{document}
