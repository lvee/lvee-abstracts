\documentclass[10pt, a5paper]{article}
\usepackage{pdfpages}
\usepackage{parallel}
\usepackage[T2A]{fontenc}
\usepackage{ucs}
\usepackage[utf8x]{inputenc}
\usepackage[polish,english,russian]{babel}
\usepackage{hyperref}
\usepackage{rotating}
\usepackage[inner=2cm,top=1.8cm,outer=2cm,bottom=2.3cm,nohead]{geometry}
\usepackage{listings}
\usepackage{graphicx}
\usepackage{wrapfig}
\usepackage{longtable}
\usepackage{indentfirst}
\usepackage{array}
\newcolumntype{P}[1]{>{\raggedright\arraybackslash}p{#1}}
\frenchspacing
\usepackage{fixltx2e} %text sub- and superscripts
\usepackage{icomma} % коскі ў матэматычным рэжыме
\PreloadUnicodePage{4}

\newcommand{\longpage}{\enlargethispage{\baselineskip}}
\newcommand{\shortpage}{\enlargethispage{-\baselineskip}}

\def\switchlang#1{\expandafter\csname switchlang#1\endcsname}
\def\switchlangbe{
\let\saverefname=\refname%
\def\refname{Літаратура}%
\def\figurename{Іл.}%
}
\def\switchlangen{
\let\saverefname=\refname%
\def\refname{References}%
\def\figurename{Fig.}%
}
\def\switchlangru{
\let\saverefname=\refname%
\let\savefigurename=\figurename%
\def\refname{Литература}%
\def\figurename{Рис.}%
}

\hyphenation{admi-ni-stra-tive}
\hyphenation{ex-pe-ri-ence}
\hyphenation{fle-xi-bi-li-ty}
\hyphenation{Py-thon}
\hyphenation{ma-the-ma-ti-cal}
\hyphenation{re-ported}
\hyphenation{imp-le-menta-tions}
\hyphenation{pro-vides}
\hyphenation{en-gi-neering}
\hyphenation{com-pa-ti-bi-li-ty}
\hyphenation{im-pos-sible}
\hyphenation{desk-top}
\hyphenation{elec-tro-nic}
\hyphenation{com-pa-ny}
\hyphenation{de-ve-lop-ment}
\hyphenation{de-ve-loping}
\hyphenation{de-ve-lop}
\hyphenation{da-ta-ba-se}
\hyphenation{plat-forms}
\hyphenation{or-ga-ni-za-tion}
\hyphenation{pro-gramming}
\hyphenation{in-stru-ments}
\hyphenation{Li-nux}
\hyphenation{sour-ce}
\hyphenation{en-vi-ron-ment}
\hyphenation{Te-le-pathy}
\hyphenation{Li-nux-ov-ka}
\hyphenation{Open-BSD}
\hyphenation{Free-BSD}
\hyphenation{men-ti-on-ed}
\hyphenation{app-li-ca-tion}

\def\progref!#1!{\texttt{#1}}
\renewcommand{\arraystretch}{2} %Іначай формулы ў матрыцы зліпаюцца з лініямі
\usepackage{array}

\def\interview #1 (#2), #3, #4, #5\par{

\section[#1, #3, #4]{#1 -- #3, #4}
\def\qname{LVEE}
\def\aname{#1}
\def\q ##1\par{{\noindent \bf \qname: ##1 }\par}
\def\a{{\noindent \bf \aname: } \def\qname{L}\def\aname{#2}}
}

\def\interview* #1 (#2), #3, #4, #5\par{

\section*{#1\\{\small\rm #3, #4. #5}}

\def\qname{LVEE}
\def\aname{#1}
\def\q ##1\par{{\noindent \bf \qname: ##1 }\par}
\def\a{{\noindent \bf \aname: } \def\qname{L}\def\aname{#2}}
}

\switchlang{be}
\begin{document}
\title{Выкарыстанне вольнага праграмнага забеспячэння ў ЛНУ імя Івана Франка і ЛНМУ імя Данііла Галіцкага \footnote{\url{zlobingg@gmail.com} \url{http://lvee.org/ru/abstracts/248}}}
\author{С. Апуневич, Г. Злобін, Р. Рикалюк, П. Риковський, \\ Р. Шувар , Ľviv, Ukraine}
\maketitle
\begin{abstract}
The report analyzes 20 years of experience in the implementation of free software at Lviv Ivan I. Franko University and Lviv D. Galitsky University
\end{abstract}

На працягу многіх гадоў львоўская група карыстальнікаў Linux праводзіла агітацыйную працу па ўкараненні вольнага праграмнага забеспячэння ў навучальны працэс навучальных устаноў г. Львова. З 1998 г. пачалося ўкараненне АС Linux у навучальны працэс ЛНМУ імя Данііла Галіцкага. У 2002-2003 гг. было прынята рашэнне дэпутацкай камісіі па пытаннях адукацыі Львоўскага гарадскога савета аб закупцы класаў вучэбнай камп'ютарнай тэхнікі з прадусталяванай АС Linux. Для настаўнікаў школ, якія атрымлівалі камп'ютарную тэхніку з АС Linux, былі праведзены навучальныя курсы па асновах працы ў АС Linux ~\cite{Zlobin-1}. У 2004 г. пачаўся праект па аптымізацыі дыстрыбутыва Debian для карыстання камп'ютарнай тэхнікі з Еўропы, якая падавалася экалагічным арганізацыям Украіны на бязвыплатнай аснове ~\cite{Zlobin-2}. У той жа час пачалося карыстанне АС Linux у навучальным працэсe факультэта электронікі ЛНУ імя Івана Франка. У 2013 г. праз цяжкую сітуацыю з фінансаваннем факультэта з аднаго боку і ў сувязі з пачаткам ціску прадстаўніцтва Microsoft ва Украіне па выкарыстанні ліцэнзійнага ПЗ фірмы Microsoft у навучальных установах Украіны кіраўніцтва факультэта электронікі прыняло рашэнне аб пераводзе навучальных лабараторый на АС Linux.

На жаль, з-за разнамаснай камп'ютарнай тэхнікі ў школах (пачынаючы з ПЭВМ з працэсарам Intel 80486) і супраціў настаўнікаў інфарматыкі адбыўся пераход школ да неліцэнзійнай АС Microsoft Windows і праграмнага забеспячэння для яе. Затое ў ЛНМУ імя Данііла Галіцкага і ЛНУ імя Івана Франка выкарыстоўваецца па сённяшні дзень. Разгледзім сітуацыю па выкарыстанні ВПЗ ў гэтых навучальных установах.

\textbf{І.} Варта падкрэсліць, што пераход навучальнага працэсу ў ЛНМУ імя Данііла Галіцкага на ВПЗ адбыўся пад пагрозай праверкі ліцэнзій ўсталяванага ПЗ -- <<добразычліўцы>> папярэдзілі рэктара ЛНМУ за паўгода да запланаванай праверкі. Сёння ва універсітэце налічваецца больш за 800 камп'ютарных працоўных месцаў,  да некаторых з іх падлучаныя прылады друку, сканеры і шматфункцыянальныя прылады, ёсць 20 камп'ютарных класаў, 10 лекцыйных аўдыторый, якія выкарыстоўваюцца для правядзення ліцэнзійных экзаменаў КРОК і відэаканферэнцый, імітацыйны клас, у якім знаходзяцца 4 праграмуемых манекенаў.

Напрамкі выкарыстання СПО у ЛНМУ:

\begin{itemize}
  \item \textbf{серверныя прымянення} ~--- Linux  (Debian, Ubuntu), Unix (FreeBSD);
  \item \textbf{навучальны працэс} ~--- аперацыйная сістэма \linebreak(Debian,  Ubuntu), офісны пакет (LibreOffice), сродкі праграмавання \linebreak (Bluefish), тэхналогіі тэлемедыцыны, відэаканферэнцсувя-
зі, кантроль за напісаннем ліцэнзійных экзаменаў КРОК, імітацыйны медыцынскі навучальны клас, элементы сістэмы дыстанцыйнага навучання;
  \item \textbf{студэнцкая навуковая праца} ~--- аперацыйная сістэма \linebreak  (Debian, Ubuntu), офісны пакет (LibreOffice), сродкі праграмавання \linebreak (Bluefish), сістэмы кіравання базамі дадзеных (MySQL, Base).
\end{itemize}

Вынікам сказанага з'яўляецца той факт, што:

\begin{enumerate}
  \item У цяперашні час універсітэт цалкам функцыянуе на ліцэнзійным ПЗ, праведзена адпаведная дакументацыйная і арганізацыйная праца.
  \item Шырокі спектр даступнага вольнага праграмнага забеспячэння паказаў магчымасць поўнага забеспячэння працоўнага, навучальнага і навуковага працэсаў у універсітэце.
  \item Пераход на свабоднае праграмнае забеспячэнне дазволіў не толькі пазбегнуць выкарыстання пірацкага праграмнага забеспячэння ў навучальным працэсе, але і паменшыць рызыку нападаў на працоўныя станцыі з боку зламыснікаў і аператараў станцый, паменшылася пагроза віруснай паразы і неабходнасць перманентнай пераўсталёўкі АС.
  \item Працягваецца праца па выкарыстанні элементаў сістэмы дыстанцыйнага навучання.
\end{enumerate}

\textbf{ІІ.} У адрозненне ад ЛНМУ ў ЛНУ імя Івана Франка выкарыстанне ВПЗ не мае ўсёабдымнага характару, а адбываецца ў асобных падраздзяленнях універсітэта. Сёння ў ЛНУ імя Івана Франка налічваецца больш за 2000 камп'ютарных працоўных месцаў, некаторыя з падлучанымі прыладамі друку, сканерамi і шматфункцыянальнымi прыладамi, ёсць 37 камп'ютарных класаў, 20 лекцыйных аўдыторый, якія выкарыстоўваюцца для чытання лекцый з відэапраектарам і правядзення відэаканфэрэнцый. З мэтай забеспячэння ліцэнзійнай АС Microsoft Windows у 2013 была праведзена аплата двухгадовай падпіскі Dream Spark для дзесяці натуральных факультэтаў, аднак два факультэты (фізічны і электронікі і камп'ютарных тэхналогій) гэтай падпіскі не атрымалі з-за грубых памылак у аплікацыйных формах. На гуманітарных факультэтах забяспечанасць ліцэнзійным ПЗ з'яўляецца частковай. На факультэце электронікі і камп'ютарных тэхналогій ВПЗ прысутнiчае ў пяці навучальных лабараторыях, у трох навучальных лабараторыях працуюць дзве аперацыйныя сістэмы: Linux і Microsoft Windows 7 або 10 дзякуючы бясплатнай падпісцы Dream Spark, якую ўдалося аформіць на кафедру радыёфізікі і камп'ютарных тэхналогій пасля правалу платнай падпісцы Dream Spark.

Напрамкі выкарыстання ВПЗ у ЛНУ імя Івана Франка:

\begin{itemize}
  \item \textbf{серверы} ~--- Linux (Debian, Open SuSE,  CentOS), Unix FreeBSD;
  \item \textbf{навучанне} ~--- аперацыйная сістэма (Debian, Open SuSE,\linebreak CentOS, SlackWare), офісныя пакеты (OpenOffice, LibreOffice, Planner), сістэма дыстанцыйнага навучання Moodle, сродкі праграмавання (gcc, IDE Kuzya, IDE CodeBlocks, Qt Creator), матэматычныя пакеты (Octave, SciLab, SMathStudio, Maxima, Labplot), пакеты мадэлявання (Packettracer, Altera Quartus, KiCAD, \linebreak Proteus), кліент-серверныя тэхналогіі;
  \item \textbf{студэнцкая навуковая праца} ~--- аперацыйная сістэма \linebreak (Debian, Open SuSE), офісны пакет (OpenOffice), сродкі праграмавання (gcc, Kuzya IDE, Qt Creator), матэматычныя пакеты (Octave, SciLab, Maxima, Labplot), сістэмы кіравання базамі дадзеных (MySQL), эмулятары апаратных сродкаў і аперацыйных сістэм;
  \item \textbf{навуковыя даследаванні} ~--- аперацыйная сістэма (Debian, Open SuSE), офісны пакет (OpenOffice), сродкі праграмавання (gcc, IDE Kuzya, Qt Creator), матэматычныя пакеты (Octave, SciLab, Maxima, Labplot), арганізацыя вылічальных кластараў і кластараў высокай даступнасці (Scientific Linux, CentOS PaceMaker, Kickstarter, Webmin, SGE, Ganglia, OpenMPI, \linebreak MPICH2, BLAS, FFTW, NorduGrid ARC, Condor, CUDA 5.0 production release).
\end{itemize}

Высновы:

\begin{enumerate}
  \item Нягледзячы на працяглы і няпросты працэс пераходу на ВПЗ у ЛНМУ імя Данііла Галіцкага апынуўся ўдалым ~\cite{Zlobin-3}. Варта адзначыць актыўную падтрымку гэтага працэсу з боку кіраўніцтва ЛНМУ, якая, на думку аўтараў дакладу, мела асабліва матэрыяльны характар. Дзякуючы пераходу на ВПЗ, ЛНМУ імя Данііла Галіцкага не толькі зэканоміў значныя грашовыя сродкі на набыцці ліцэнзій на ПЗ, але і пазбег \linebreak штрафных санкцый за выкарыстанне неліцэнзійнага ПЗ.
  \item Нягледзячы на тое, што ў ЛНУ імя Івана Франка ёсць неабходны кадравы патэнцыял, свабоднае праграмнае забеспячэнне выкарыстоўваецца не ва ўсіх падраздзяленнях універсітэта. Абумоўлена гэта тым, што кіраўніцтва універсітэта не надае належнай увагі ліцэнзійнай чысціні  праграм.
  \item Прыходзіцца канстатаваць, што выкарыстанне прапрыетарнага (і нават неліцэнзійнага) праграмнага забеспячэння прыводзіць да разрыву паміж навучальным працэсам у ЛНУ імя Івана Франка і вытворчым працэсам у ІТ-фірмах Украіны, што дае падставы некаторым аўтарам сцвярджаць пра састарэласць навучальнага працэсу ў навучальных установах Украіны і у прыватнасці ў ЛНУ імя Івана Франко ~\cite{Zlobin-4}.
\end{enumerate}

\begin{thebibliography}{20}

\bibitem{Zlobin-1}Матеріали науково-методичного семінару <<Використання ОС Linux у навчальному процесі середніх і вищих навчальних закладів>>. Львів-2003, 60 с.
\bibitem{Zlobin-2}С. Апуневич, В. Бойко, Г. Злобін, В. Семенюк, С. Кудрик <<Linux — це просто як Borsch>> Львів-2006, 116 с.
\bibitem{Zlobin-3}Курьянович О.В. Використання вільного програмного забезпечення в ЛНМУ імені Данила Галицького. Збірник наукових праць  сьомої науково-практичної конференції FOSS Lviv 2017. -- Львів, Львівський національний університет імені Івана Франка, с. 57
\bibitem{Zlobin-4} \url{http://igate.com.ua/news/13725-top-5-mifov-kotorye-} \url{meshayut-zhit-ukrainskim-programmistam}
\end{thebibliography}
\end{document}
