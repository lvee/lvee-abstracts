\documentclass[10pt, a5paper]{article}
\usepackage{pdfpages}
\usepackage{parallel}
\usepackage[T2A]{fontenc}
\usepackage{ucs}
\usepackage[utf8x]{inputenc}
\usepackage[polish,english,russian]{babel}
\usepackage{hyperref}
\usepackage{rotating}
\usepackage[inner=2cm,top=1.8cm,outer=2cm,bottom=2.3cm,nohead]{geometry}
\usepackage{listings}
\usepackage{graphicx}
\usepackage{wrapfig}
\usepackage{longtable}
\usepackage{indentfirst}
\usepackage{array}
\newcolumntype{P}[1]{>{\raggedright\arraybackslash}p{#1}}
\frenchspacing
\usepackage{fixltx2e} %text sub- and superscripts
\usepackage{icomma} % коскі ў матэматычным рэжыме
\PreloadUnicodePage{4}

\newcommand{\longpage}{\enlargethispage{\baselineskip}}
\newcommand{\shortpage}{\enlargethispage{-\baselineskip}}

\def\switchlang#1{\expandafter\csname switchlang#1\endcsname}
\def\switchlangbe{
\let\saverefname=\refname%
\def\refname{Літаратура}%
\def\figurename{Іл.}%
}
\def\switchlangen{
\let\saverefname=\refname%
\def\refname{References}%
\def\figurename{Fig.}%
}
\def\switchlangru{
\let\saverefname=\refname%
\let\savefigurename=\figurename%
\def\refname{Литература}%
\def\figurename{Рис.}%
}

\hyphenation{admi-ni-stra-tive}
\hyphenation{ex-pe-ri-ence}
\hyphenation{fle-xi-bi-li-ty}
\hyphenation{Py-thon}
\hyphenation{ma-the-ma-ti-cal}
\hyphenation{re-ported}
\hyphenation{imp-le-menta-tions}
\hyphenation{pro-vides}
\hyphenation{en-gi-neering}
\hyphenation{com-pa-ti-bi-li-ty}
\hyphenation{im-pos-sible}
\hyphenation{desk-top}
\hyphenation{elec-tro-nic}
\hyphenation{com-pa-ny}
\hyphenation{de-ve-lop-ment}
\hyphenation{de-ve-loping}
\hyphenation{de-ve-lop}
\hyphenation{da-ta-ba-se}
\hyphenation{plat-forms}
\hyphenation{or-ga-ni-za-tion}
\hyphenation{pro-gramming}
\hyphenation{in-stru-ments}
\hyphenation{Li-nux}
\hyphenation{sour-ce}
\hyphenation{en-vi-ron-ment}
\hyphenation{Te-le-pathy}
\hyphenation{Li-nux-ov-ka}
\hyphenation{Open-BSD}
\hyphenation{Free-BSD}
\hyphenation{men-ti-on-ed}
\hyphenation{app-li-ca-tion}

\def\progref!#1!{\texttt{#1}}
\renewcommand{\arraystretch}{2} %Іначай формулы ў матрыцы зліпаюцца з лініямі
\usepackage{array}

\def\interview #1 (#2), #3, #4, #5\par{

\section[#1, #3, #4]{#1 -- #3, #4}
\def\qname{LVEE}
\def\aname{#1}
\def\q ##1\par{{\noindent \bf \qname: ##1 }\par}
\def\a{{\noindent \bf \aname: } \def\qname{L}\def\aname{#2}}
}

\def\interview* #1 (#2), #3, #4, #5\par{

\section*{#1\\{\small\rm #3, #4. #5}}

\def\qname{LVEE}
\def\aname{#1}
\def\q ##1\par{{\noindent \bf \qname: ##1 }\par}
\def\a{{\noindent \bf \aname: } \def\qname{L}\def\aname{#2}}
}

\begin{document}
\title{Программирование на C для PostgreSQL}
\author{Александр Коротков, Москва, РФ\footnote{\url{aekorotkov@gmail.com}, \url{http://lvee.org/en/abstracts/153}}}
\maketitle
\begin{abstract}
PostgreSQL offers great extendability. Users can add literally everything on their own: data types, functions, operators, index types, procedural languages and so on. But in order to use the full power of these features one should write C code for \linebreak PostgreSQL. Traditionally it's assumed that barrier to entry of C programming for PostgreSQL is very high. That's why extendability of PostgreSQL is no as demanded as it could be. The goal of present talk is to overcome this circumstances.
\end{abstract}
PostgreSQL обладает отличной расширяемостью, пользователи могут добавлять сами буквально всё: типы данных, функции, операторы, типы индексов, языки хранимых процедур и т.д. Но для того, чтобы использовать многие из этих возможностей, нужно уметь программировать под PostgreSQL на C.

Традиционно считается, что это сложно и порог вхождения \linebreak очень велик. Из"=за этого вся мощь расширяемости PostgreSQL оказывается не так востребована, как могла бы быть. Мне хотелось бы это обстоятельство постепенно преодолевать. Действительно, как и в любом большом проекте, написанном на языке C, программирование под PostgreSQL имеет свои особенности.

В PostgreSQL используется свой calling convention, благодаря которому функции, написанные как на C, так и на процедурных языках, могут быть видны из SQL. Есть универсальный тип данных "--- Datum, к которому может быть приведено значение любого типа. Параметры и результат передаются как Datum. Ниже приведён пример функции, использующей PostgreSQL calling convention.

\begin{verbatim}
PG_FUNCTION_INFO_V1(increment);

Datum
increment(PG_FUNCTION_ARGS)
{
    int32   arg = PG_GETARG_INT32(0);

    PG_RETURN_INT32(arg + 1);
}\end{verbatim}
В PostgreSQL есть свой менеджер памяти: любое выделение памяти осуществляется в рамках некоторого контекста памяти. Контексты памяти, в свою очередь, образуют иерархическую структуру. Ниже приведён шаблон функции, которая возвращает набор строк (set-returning function). Такая функция вызывается для каждой отдельной строки. Память, которую эта функция выделяет, будут освобождена перед следующим вызовом, но также доступен контекст памяти, которых сохраняется между вызовами.

\begin{verbatim}
Datum
my_set_returning_function(PG_FUNCTION_ARGS)
{
    FuncCallContext  *funcctx;
    Datum             result;
    
    if (SRF_IS_FIRSTCALL())
    {
        MemoryContext oldcontext;

        funcctx = SRF_FIRSTCALL_INIT();
        oldcontext = MemoryContextSwitchTo(funcctx->
multi_call_memory_ctx);
        /* Инициализация структур памяти, которая 
выполняется только один раз */
        пользовательский код
        MemoryContextSwitchTo(oldcontext);
    }

    /* Инициализация структур памяти, которая выполняется 
каждый раз */
    пользовательский код
    funcctx = SRF_PERCALL_SETUP();
    пользовательский код

   /* Нужно ли вернуть ещё одну строку или все строки уже 
возвращены? */
    if (funcctx->call_cntr < funcctx->max_calls)
    {
        /* Возврат следующей строки результата (result) */
        пользовательский код
        SRF_RETURN_NEXT(funcctx, result);
    }
    else
    {
        /* Все строки уже были возвращены, выполняется 
освобождение использованных ресурсов, если нужно */
        пользовательский код
        SRF_RETURN_DONE(funcctx);
    }
}\end{verbatim}
Благодаря тому, что любой контекст памяти можно очистить в любой момент времени, во многих случаях можно не освобождать отдельно каждый участок памяти. При этом накладные расходы несравнимо малы по сравнению с применением сборщика мусора.

Посылать к БД SQL"=запросы можно напрямую в тот же backend, из которого вызвана функция. Для этого есть специальный интерфейс "--- SPI. Ниже приведён пример функции, который выполняет запрос и выводит его результаты в виде INFO сообщений. Её аргументами являются текст запроса в виде text, и максимальное число выводимых строк ответа.

\begin{verbatim}
int
execq(text *sql, int cnt)
{
    char *command;
    int ret;
    int proc;

    /* Преобразуем text в C-строку */
    command = text_to_cstring(sql);

    SPI_connect();

    ret = SPI_exec(command, cnt);

    proc = SPI_processed;
    /*
     * Если удалось получить строки, 
     * то выводим их через elog(INFO).
     */
    if (ret > 0 && SPI_tuptable != NULL)
    {
        TupleDesc tupdesc = SPI_tuptable->tupdesc;
        SPITupleTable *tuptable = SPI_tuptable;
        char buf[8192];
        int i, j;

        /* Цикл по строкам ответа */
        for (j = 0; j < proc; j++)
        {
            HeapTuple tuple = tuptable->vals[j];

            /* Цикл по полям строки ответа */
            for (i=1, buf[0] = 0; i <= tupdesc->natts; i++)
                snprintf(buf + strlen (buf), sizeof(buf) - 
strlen(buf), " %s%s",
                        SPI_getvalue(tuple, tupdesc, i),
                        (i == tupdesc->natts) ? " " : " |");
            elog(INFO, "EXECQ: %s", buf);
        }
    }

    SPI_finish();
    pfree(command);

    return (proc);
}\end{verbatim}
Но если немного привыкнуть, то для того, кто уже имеет опыт на C, программировать под PostgreSQL не так уж и сложно. А благодаря тому, что уже есть готовые удобные макросы, функции, структуры данных, это может быть даже проще, чем на чистом C.

\end{document}
