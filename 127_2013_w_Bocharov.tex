\documentclass[10pt, a5paper]{article}
\usepackage{pdfpages}
\usepackage{parallel}
\usepackage[T2A]{fontenc}
\usepackage{ucs}
\usepackage[utf8x]{inputenc}
\usepackage[polish,english,russian]{babel}
\usepackage{hyperref}
\usepackage{rotating}
\usepackage[inner=2cm,top=1.8cm,outer=2cm,bottom=2.3cm,nohead]{geometry}
\usepackage{listings}
\usepackage{graphicx}
\usepackage{wrapfig}
\usepackage{longtable}
\usepackage{indentfirst}
\usepackage{array}
\newcolumntype{P}[1]{>{\raggedright\arraybackslash}p{#1}}
\frenchspacing
\usepackage{fixltx2e} %text sub- and superscripts
\usepackage{icomma} % коскі ў матэматычным рэжыме
\PreloadUnicodePage{4}

\newcommand{\longpage}{\enlargethispage{\baselineskip}}
\newcommand{\shortpage}{\enlargethispage{-\baselineskip}}

\def\switchlang#1{\expandafter\csname switchlang#1\endcsname}
\def\switchlangbe{
\let\saverefname=\refname%
\def\refname{Літаратура}%
\def\figurename{Іл.}%
}
\def\switchlangen{
\let\saverefname=\refname%
\def\refname{References}%
\def\figurename{Fig.}%
}
\def\switchlangru{
\let\saverefname=\refname%
\let\savefigurename=\figurename%
\def\refname{Литература}%
\def\figurename{Рис.}%
}

\hyphenation{admi-ni-stra-tive}
\hyphenation{ex-pe-ri-ence}
\hyphenation{fle-xi-bi-li-ty}
\hyphenation{Py-thon}
\hyphenation{ma-the-ma-ti-cal}
\hyphenation{re-ported}
\hyphenation{imp-le-menta-tions}
\hyphenation{pro-vides}
\hyphenation{en-gi-neering}
\hyphenation{com-pa-ti-bi-li-ty}
\hyphenation{im-pos-sible}
\hyphenation{desk-top}
\hyphenation{elec-tro-nic}
\hyphenation{com-pa-ny}
\hyphenation{de-ve-lop-ment}
\hyphenation{de-ve-loping}
\hyphenation{de-ve-lop}
\hyphenation{da-ta-ba-se}
\hyphenation{plat-forms}
\hyphenation{or-ga-ni-za-tion}
\hyphenation{pro-gramming}
\hyphenation{in-stru-ments}
\hyphenation{Li-nux}
\hyphenation{sour-ce}
\hyphenation{en-vi-ron-ment}
\hyphenation{Te-le-pathy}
\hyphenation{Li-nux-ov-ka}
\hyphenation{Open-BSD}
\hyphenation{Free-BSD}
\hyphenation{men-ti-on-ed}
\hyphenation{app-li-ca-tion}

\def\progref!#1!{\texttt{#1}}
\renewcommand{\arraystretch}{2} %Іначай формулы ў матрыцы зліпаюцца з лініямі
\usepackage{array}

\def\interview #1 (#2), #3, #4, #5\par{

\section[#1, #3, #4]{#1 -- #3, #4}
\def\qname{LVEE}
\def\aname{#1}
\def\q ##1\par{{\noindent \bf \qname: ##1 }\par}
\def\a{{\noindent \bf \aname: } \def\qname{L}\def\aname{#2}}
}

\def\interview* #1 (#2), #3, #4, #5\par{

\section*{#1\\{\small\rm #3, #4. #5}}

\def\qname{LVEE}
\def\aname{#1}
\def\q ##1\par{{\noindent \bf \qname: ##1 }\par}
\def\a{{\noindent \bf \aname: } \def\qname{L}\def\aname{#2}}
}

\begin{document}
\title{Выбор информационных систем. Продвижение открытые системы в Enterprise}
\author{Вячеслав Бочаров \footnote{Минск, Беларусь}}
\maketitle
\begin{abstract}
The report adresses some aspects of chosing information systems. Steps are proposed to examine, which should provide help to such communities as the LVEE and BSOFT to increase the market share of open source software in the enterprise segment.
\end{abstract}
\section*{Время альтернатив}

Время, когда какая-либо информационная система была востребована только в силу своей исключительности, уже прошло. 
Всегда имеется несколько альтернативных решений.

Любая компания, которая приняла решение о внедрении той или иной информационной системы, стоит перед выбором, обусловленным многими факторами. На него влияют компании-поставщики, интеграторы, собственные технические специалисты и консультанты.
Открытые информационные системы -- также участники данного рынка. Рассмотрим, какие преимущества они имеют, и какие шаги позволят продвигать open-source  к корпоративном сегменте.

\section*{Выбор информационной системы. Как это происходит}

Почему в современной инфраструктуре так мала доля открытого ПО в бизнес-критичных приложениях, например СУБД, системах Unified Communication, почтовых системах, средствах виртуализации, и в тоже время уже пропорциональна в инфраструктурной системе?

Ответ прост: при выборе инфраструктуры, зарытой от бизнес-пользователей, окончательное решение принимает технический специалист. Он руководствуется опытом таких же технических специалистов и во многом собственными предпочтениями.

При выборе бизнес-критичных систем решение принимает владелец бизнес-процесса, директор или сам владелец бизнеса, и опираться он будет на многочисленные факторы, среди которых предпочтение технических специалистов будет иметь, как показывает опыт, не решающее значение.

Какие же факторы учитываются в первую очередь?

\begin{enumerate}
  \item А -- экономический эффект от внедрения системы,
  \item В – риски, связанные с внедрением.
\end{enumerate}

При $A \textgreater{} B$  принимается решение о внедрении. При выборе из нескольких систем соответственно $[A1-B1]\textgreater{}[A2-B2]$ , чем больше тем больше эффективность и меньше риски, тем лучше.

Также не стоит забывать затраты на функционирование системы, так называемое TCO (Total cost of ownership) или стоимость владения информационной системой. В нее входят зарплаты, стоимость серверного и клиентского аппаратного обеспечения, стоимость внедрения и затраты на доработку, затраты на техническую поддержку.
Соответвенно, формула выбора выглядит как $[A1 -- (TCO1 + B1)] \textgreater{} [A2-(TCO2 + B2)]$.

В эффективность входит прямая прибыль от внедрения, возможно, повышение выручки, снижение затрат, сокращение времени.
Перечисленное описывается термином ROI (Return on Investment) или возврат от инвестиций.

Еще один показатель, влияющий на выбор информационной системы, это срок возврата от инвестиций (Pay-Back Period). Например, рассматриваемая информационная система окупается и начинает приносить прибыль в неприемлемо большие сроки, т.е. в ней заморожены инвестиции и активы.

В набор рисков входят:

\begin{enumerate}
  \item Риск того, что система не будет соответствовать полностью или частично заявленным требованиям.
  \item Риск того, что расчетные показатели TCO были недооценены, и затраты превышают ожидания.
  \item Риски того, что система не может модернизироваться, либо ее модернизация стоит неоправданно дорого.
  \item Законодательные риски, т.е. соответствие всех элементов системы законодательству.
\end{enumerate}

Как ни странно, не существует эффективной методики расчета рисков как таковой. При выборе информационные системы приходится оценивать по методу аналогии. Где, у кого такая информационная система внедрена? Каковы позиции на рынке у производителя информационной системы? Какой у интегратора есть опыт внедрения этой информационной системы, и кто может дать оценку успешности внедрения? Как планируется техническая поддержка информационной системы и ее развитие?
Немаловажно и мнение системного интегратора, имеющего в своем портфеле решений систему с открытым кодом.

Все сказанное дает представление о том, какие особенности приходится учитывать при выборе, помимо качества кода и полноты соответствия ТЗ.

\section*{Сравнение преимуществ и недостатков решений open source на рынке РБ}

Представим, что на основании сказанного нам, как бизнес-пользователю,  предстоит выбрать систему.

\subsubsection*{Основные конкурентные преимущества Open-source:}

\begin{enumerate}
  \item Отсутствует плата за приобретение и использование.
  \item Открытый код может быть модернизирован и приспособлен к нуждам предприятия нашим собственным ИТ-подразделением либо привлеченными компаниями.
  \item Над совершенствованием кода работает сообщество независимых высокопрофессиональных специалистов, что позволяет своевременно выявлять и устранять ошибки и недостатки.
\end{enumerate}

\subsubsection*{Риски:}

\begin{enumerate}
  \item Отсутствие информации об успешных внедрениях информационной системы, либо ее недоступность для лиц, принимающих решение. Непрозрачность всех этапов внедрения.
  \item Отсутствие централизованной технической поддержки. Невозможность получить гарантированную договором поддержку информационной системы.
  \item Незаинтересованность системных интеграторов в предложениях открытых информационных систем, в виду отсутствия прибыли при их продаже.
  \item Устоявшиеся мифы о высоких затратах на оплату специалистов по поддержке Linux-систем, а также о небольшом количестве специалистов на рынке труда. Невозможность адекватно оценить уровень знаний специалистов.
\end{enumerate}

Несмотря на первоначальную привлекательность решения, риски также оказываются велики. Это и затраты на доработку / техническое обслуживание, и отсутствие интеграторов, и негарантированная техническая поддержка, и невозможность оценить опыт других внедрений.

\section*{Возможные действия}

В ситуации, когда сообщество пользователей открытых систем разрознено и не объединено, практически невозможно преодолеть опасения бизнес-пользователей по использованию открытых систем.
Но в связи с развитием сообществ и общественных организаций, таких как LVEE и создаваемое Белорусское общество открытых технологий, появляется шанс переломить ситуацию.

По мнению автора, для этого необходимо выполнять следующие действия:

\begin{enumerate}
  \item Аккумулировать знания об успешных проектах по внедрению открытых систем, описывать и производить рассылку участникам сообщества и заинтересовавшимся бизнес-пользователям, публиковать на тематических сайтах отчеты об успешных внедрениях, отзывы о достигнутых эффектах от внедрения. Это позволит оценить уровень сложности, затраты по внедрению.
  \item Более тесно взаимодействовать с системными интеграторами, предлагая новую схему получения прибыли на уровне сопровождения систем. Добиваться включения в их портфель решений на базе открытых систем. На данный момент системными интеграторами, например, такими как ``Белсофт'', высказана предварительная заинтересованность в открытых системах. Это поможет получить потенциальному пользователю уверенность в технической поддержке и сопровождении информационной системы.
  \item Совместно с государственными и коммерческими учебными центрами разработать программу оценки специалистов, проводить и публиковать анализ рынка труда по данным специальностям.
  \item При необходимости сообщества должны оказывать информационное сопровождение, услуги по консультированию проекта по внедрению.
\end{enumerate}

\end{document}
