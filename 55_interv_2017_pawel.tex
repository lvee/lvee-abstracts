\documentclass[10pt, a5paper]{article}
\usepackage{pdfpages}
\usepackage{parallel}
\usepackage[T2A]{fontenc}
\usepackage{ucs}
\usepackage[utf8x]{inputenc}
\usepackage[polish,english,russian]{babel}
\usepackage{hyperref}
\usepackage{rotating}
\usepackage[inner=2cm,top=1.8cm,outer=2cm,bottom=2.3cm,nohead]{geometry}
\usepackage{listings}
\usepackage{graphicx}
\usepackage{wrapfig}
\usepackage{longtable}
\usepackage{indentfirst}
\usepackage{array}
\newcolumntype{P}[1]{>{\raggedright\arraybackslash}p{#1}}
\frenchspacing
\usepackage{fixltx2e} %text sub- and superscripts
\usepackage{icomma} % коскі ў матэматычным рэжыме
\PreloadUnicodePage{4}

\newcommand{\longpage}{\enlargethispage{\baselineskip}}
\newcommand{\shortpage}{\enlargethispage{-\baselineskip}}

\def\switchlang#1{\expandafter\csname switchlang#1\endcsname}
\def\switchlangbe{
\let\saverefname=\refname%
\def\refname{Літаратура}%
\def\figurename{Іл.}%
}
\def\switchlangen{
\let\saverefname=\refname%
\def\refname{References}%
\def\figurename{Fig.}%
}
\def\switchlangru{
\let\saverefname=\refname%
\let\savefigurename=\figurename%
\def\refname{Литература}%
\def\figurename{Рис.}%
}

\hyphenation{admi-ni-stra-tive}
\hyphenation{ex-pe-ri-ence}
\hyphenation{fle-xi-bi-li-ty}
\hyphenation{Py-thon}
\hyphenation{ma-the-ma-ti-cal}
\hyphenation{re-ported}
\hyphenation{imp-le-menta-tions}
\hyphenation{pro-vides}
\hyphenation{en-gi-neering}
\hyphenation{com-pa-ti-bi-li-ty}
\hyphenation{im-pos-sible}
\hyphenation{desk-top}
\hyphenation{elec-tro-nic}
\hyphenation{com-pa-ny}
\hyphenation{de-ve-lop-ment}
\hyphenation{de-ve-loping}
\hyphenation{de-ve-lop}
\hyphenation{da-ta-ba-se}
\hyphenation{plat-forms}
\hyphenation{or-ga-ni-za-tion}
\hyphenation{pro-gramming}
\hyphenation{in-stru-ments}
\hyphenation{Li-nux}
\hyphenation{sour-ce}
\hyphenation{en-vi-ron-ment}
\hyphenation{Te-le-pathy}
\hyphenation{Li-nux-ov-ka}
\hyphenation{Open-BSD}
\hyphenation{Free-BSD}
\hyphenation{men-ti-on-ed}
\hyphenation{app-li-ca-tion}

\def\progref!#1!{\texttt{#1}}
\renewcommand{\arraystretch}{2} %Іначай формулы ў матрыцы зліпаюцца з лініямі
\usepackage{array}

\def\interview #1 (#2), #3, #4, #5\par{

\section[#1, #3, #4]{#1 -- #3, #4}
\def\qname{LVEE}
\def\aname{#1}
\def\q ##1\par{{\noindent \bf \qname: ##1 }\par}
\def\a{{\noindent \bf \aname: } \def\qname{L}\def\aname{#2}}
}

\def\interview* #1 (#2), #3, #4, #5\par{

\section*{#1\\{\small\rm #3, #4. #5}}

\def\qname{LVEE}
\def\aname{#1}
\def\q ##1\par{{\noindent \bf \qname: ##1 }\par}
\def\a{{\noindent \bf \aname: } \def\qname{L}\def\aname{#2}}
}

%\switchlang{be}
%\usepackage{color}
\begin{document}
\title{Интервью с участниками}
%\author{}
\date{}
\maketitle

По традиции в сборник материалов входят интервью, в которых активные участники
сообщества open source делятся своим мнением о свободном ПО, открытых
технологиях, роли и месте свободных лицензий, рассказывают, как видят проблематику
свободных проектов. В этот раз, из-за англоязычности всех интервьюируемых, интервью приводятся на двух языках - английском и русском.

\section{Pawel Chojnacki "---  Варшава, Польша}
%\begin{figure}[ht]
%\centering{\includegraphics[width=4cm]{49_spons_altoros.jpg}}
%\end{figure}
\begin{Parallel}[p]{}{}
     \ParallelLText{%
      \selectlanguage{english}
{\noindent \bf LVEE: Can you briefly introduce yourself?}

{\noindent \bf Pawel Chojnacki:} My name is Pawel, I live in Warsaw. I used to be a part of Warsaw hackerspace; right now I'm not engaged in any organizational activity, may be beside the Global Innovation Gathering (\url{http://www.globalinnovationgathering.com}), a community which also involves hackering. I’m a web developer, a freelancer, a frontend professional doing some Open Access Science in the meantime. I deal a lot with open source in my professional activity. And in addition to that I’m giving cybersecurity lectures, an introduction to cybersecurity: what to store, why should you use Firefox and Chrome but not Internet Explorer and Safari.  And I work with open source every day.

{\noindent \bf L: How did you get acquainted with open source software? Do you remember?} 

{\noindent \bf P:}  I’ve started using Firefox very early in my life. It was the first year of Firefox existence, I don’t remember exactly which year it was. But I didn't consider it open source, I knew nothing about it. So my first conscious meeting with open source was related to GIMP graphic editor. I just wanted some program which had more options than Paint, and I didn’t want payed Photoshop version. That’s why I learned GIMP. I looked through one tutorial, and the second, and started learning to work with GIMP. Than I met some people speaking Polish who were interested in GIMP and helped them to found Polish Gimp Users' Forum (messaging board for Polish-speaking GIMP users). And from that I started reading and got very interested in Linux and the open source philosophy, and that was, I think, in mid school and high school. 

{\noindent \bf L: By the way, did you have some previous experience with Photoshop, before starting GIMP, or not?}

{\noindent \bf P:}  I didn't have any previous experience. Nothing with Photoshop, or Corel, or anything else.

{\noindent \bf L: It's an interesting question, because we know a lot of examples, when different previous experience had complicated people using GIMP more or less. And you have also mentioned Hackerspace. It’s a hackerspace in Warsaw?}

{\noindent \bf P:} Yeah. Warsaw hackerspace. 

{\noindent \bf L: The topic of hackerspace itself is rather interesting, because we have finally officially registered Hackerspace in Minsk some time ago. How long does it work in Warsaw?}

{\noindent \bf P:} Four or five years. I was in the board of hackerspace, and was basically driving it full time as a community manager, or whatever you want to call it, for several months. It was founded basically by some enthusiasts but got wheel speed when Warsaw University of Technology decided to shut down their experimental technology division and all people who wanted to experiment went to the hackerspace. And right now Warsaw hackerspace is a club. It’s typical members are people who have quite similar background from the Internet, who chat there… Even with a similar sense of humor. Those who like specific areas of IT, the security, administration, and who want to contribute to some projects or just to use technology for fun.

{\noindent \bf L: Usually hackerspace communities have strong connections with some open hardware projects: they are either developing their own or using some devices previously developed under such open hardware licenses. Perhaps there is the same situation in Warsaw, isn’t it?}

{\noindent \bf P:} We don’t have any projects that the hackerspace is creating as a hackerspace, but several members have created several hacks, for example way to install OpenWrt on specific devices which where unknown to be able of doing this.

Also there is some tool for gathering money from the members, just to send e-mails with reminders, and show current status. So hackerspace has some infrastructure. It is open but not necessary all of it is open source. 
 
A lot of things are based on Arduino, but I wouldn’t call it open hardware with us, as not all this things are documented and shared. But at least few things are.

{\noindent \bf L: At the very beginning open hardware was exotic, it was even supposed that open source licenses are good for software but not for hardware and other areas, like art. But well, later several really popular projects have appeared published under such licenses. How do you think, what's the reason of their success?}

{\noindent \bf P:} I think it is because they bring right now technology to the people who have open hardware which is also standartized.

Because right now you can take Arduino, take Raspberry Pi, and you can be sure that the most of software written for the one piece will work on any similar kind of chip, anywhere in the world. 

For example we have OpenBCI brain amplifier created by a small startup in US, which is becoming a new Arduino for brain amplifying, for just brain signal analysis which is also a very important for the community. There are a lot of other projects...

So they are bringing the costs down, as you don't have any kind of overprotection from company that puts control and keeps the prices  so that they are profitable. 

This hardware projects are just tools for sharing knowledge and for sharing technical experience using them as building blocks in practice.

{\noindent \bf L: Thank you. 
I think that open source is the most powerful in software areas nowadays, but with open hardware and open media like Wikipedia for example, or creative Commons materials online – perhaps together they are three most widely used open directions.}

{\noindent \bf P:} I think you forgot about one thing. You’ve mentioned the media as Creative Commons and Wikipedia, but there's also open access and open notebook at science. 

Right now I’m working on the scientific project of my own, creating it with the totally open notebook access, so that I’m commenting every step of my experiment, with use of open source tools. This approach allows people to replicate the experiment in same environment. If I actually had the equipment which would be open hardware, it could be much easier to actually replicate the data gathering.

{\noindent \bf L: It’s really interesting, as it’s not very easy to reproduce results of other researchers sometimes.
And open source software has some powerful positions in scientific research because it allows to see sources and control how where all the calculations really done. }

{\noindent \bf P:} Yes especially, but there also may be problems. I don't know if you have heard, recently we have lost about 40,000 articles regarding neurology and fMRI, because someone put the wrong code into scientific tools used to monitor them. 

The package itself was open source, but nobody was actively testing it, nobody was actively looking. So it's not only about saying “Hey this is open”, as universities are saying - “This piece of software is open, but please don't look at it because only we understand how it works”. 

Yes, you have to document it. You have to put it to open environment and encourage people to test it, to modify, to understand how it works.

If you are the the owner and the creator, and also the only person who understands how it works, it's not really the open source. Somebody has to look at it, somebody has to analyze it. If that would happened to the FMRI tools that where used around the world for the last 15 years, we could have all the knowledge gathered with this tools intact. But now we have to scrape basically the 15 years of neurology, because the software was faulty. It contains different statistical methods than it was supposed to...

{\noindent \bf L: And what is the role of community in supporting such projects? For example there is the known problem of open source projects with only one person who really knows their internals. If such person goes away for some reasons and abandons the development process, the project is in real trouble. So what about the problem of the community involvement in the development?}

{\noindent \bf P:}I believe that in case of some projects community has to be more involved, test something, develop harder. Even if sometimes there is no need in development… For example we have that issue with the set of statistical tests that nobody else is going to implement because they are working, they are fast, they are okay. But we still need someone who actually got stuck in the code. Those who got check it from time to time and notice if anything does go wrong. 

Basically university workers can be maintainers in some cases. I think that if there's nobody from the community to volunteer, nobody able to make some package of software – then we should start looking at the university's officials, asking universities to maintain that. Actually start paying the university researchers to maintain some software and respond to all the people who requires something or would like to ask some questions. 

I am speaking about employees, who would do it instead of working on some of their own projects. Because I know how this is in computational neuroscience in Warsaw, and in Poland. More than 60\% of people aren’t contributing to science. They are mostly just read that old papers and do not creating anything but useless pieces of software that only they can use, but no one else. It is waste of money that are going to the universities. If they are to be useful, they should become maintainers of the scientific packages, they should start analyzing them, they should start being responsible for how they work.

I believe that would be one of the best uses of university money. Universities have actually started networking, because with several neurological projects around Europe you are just creating the same feature over and over again in several different universities: the French are doing the same as Germans, the same as Italians, the same as Polish and the same as Russians.
 
What's the point of replicating involved in it - they fit the same problems and they hope for each other to solve that. 

     }
     \ParallelRText{%
       \selectlanguage{russian}
{\noindent \bf LVEE: Для начала, несколько слов о себе.}

{\noindent \bf Pawel Chojnacki:} Меня зовут Павел, я живу в Варшаве. Раньше активно участвовал в Варшавском Хакерспейсе; в настоящий момент организационной активностью не занимаюсь, ну может быть кроме Global Innovation Gathering (\url{http://www.globalinnovationgathering.com}), сообщество, связанное в том числе с хакерством. Я веб-разработчик, фрилансер, специалист по разработке фронтэнда, а кроме того занимаюсь наукой и открытым доступом (Open Access). Мне много приходится работать со свободным ПО в профессиональной деятельности. А еще я читаю лекции по компьютерной безопасности, такое введение в информационную безопасность: что хранить, почему использовать Firefox и Chrome, а не Internet Explorer или Safari. Ну и каждый день имею дело со свободным ПО.

{\noindent \bf L: Не вспомнишь, как ты вообще познакомился со свободным ПО?} 

{\noindent \bf P:}  Я начал пользоваться Firefox в очень раннем возрасте. Это был первый год существования Firefox как такового, сейчас не вспомню, в каком именно году. Но я не воспринимал его как свободное ПО, просто ничего об этом не знал. Так что первая сознательная встреча со свободным ПО была связана с графическим редактором GIMP. Мне была нужна какая-нибудь программа с б\'{о}льшими чем у Paint возможностями, и мне не хотелось иметь дело с платной версией Photoshop. Поэтому я изучил GIMP. Заглянул в один туториал, потом в другой, и начал учиться с ним работать. Затем я познакомился с польскоговорящими людьми, которые интересовались GIMP, и помог им создать Polish Gimp Users' Forum (площадку для польскоговорящих пользователей GIMP). И начиная с этого момента я стал читать, заинтересовался Linux и философией свободного ПО, думаю, это было в средних или старших классах. 

{\noindent \bf L: А кстати, у тебя уже был какой-то опыт работы в Photoshop до GIMP или нет?}

{\noindent \bf P:}  Никакого прежнего опыта. Ни с Photoshop, ни с Corel, ни с чем-либо ещё.

{\noindent \bf L: Это интересный вопрос, потому что есть достаточно примеров, когда предыдущий опыт с чем-то другим в той или иной степени осложнял людям изучение GIMP. Итак, ты упоминал хакерспейс. Это хакерспейс в Варшаве?}

{\noindent \bf P:} Да. Варшавский хакерспейс. 

{\noindent \bf L: Тема хакерспейса сама по себе интересна, потому что не так давно мы наконец официально зарегистрировали хакерспейс в Минске. Как долго он существует в Варшаве?}

{\noindent \bf P:} Четыре или пять лет. Я был в совете хакерспейса, несколько  месяцев фактически тянул его на себе на штатной основе в качестве комьюнити-менеджера, или как ещё можно назвать эту должность. Он был сначала основан несколькими энтузиастами, а раскрутился, когда в Варшавском технологическом университете закрылось экспериментальное технологическое отделение, и тога все, кто хотел заниматься экспериментами, отправились в хакерспейс. А сейчас Варшавский хакреспейс --- это клуб. Его типичные члены --- это люди со схожим опытом в Интернет, которые в нём общаются… Вплоть до схожего чувства юмора. Те, кому нравятся определённые области IT, безопасность, администрирование, кто хотел бы вносить вклад в некоторые проекты или просто возиться с технологиями ради собственного удовольствия.

{\noindent \bf L: Обычно комьюнити хакерспейсов имеют сильные связи с проектами свободного аппаратного обеспечния: разрабатывают что-то своё, или используют устройства, созданные ранее под свободными лицензиями. В Варшаве наверное такая же ситуация?}

{\noindent \bf P:} У нас нет проектов, создаваемых в хакерспейсе именно от имени хакерспейса, но некоторые члены были инициаторами некоторых хаков --- например, способ установить прошивку OpenWrt на некоторые устройства, для которых прежде ничего не было известно о такой возможности.

Ещё есть инструмент для сбора денег с участников: напоминания по e-mails и т.~д., отображение текущего статуса. Понятно, у хакерспейса есть определенная инфраструктура. Она открыта, но не обязательно является свободной вся целиком. 
 
Многие вещи построены на Arduino, но в нашем случае я не назвал бы это свободным аппаратным обеспечением, потому что не все проекты задокументированы и выложены в общий доступ. Но несколько --- да.

{\noindent \bf L: В самом начале свободное аппаратное обеспечение считалось экзотикой, считалось даже, что свободные лицензии хороши для программ, но не для аппаратуры и некоторых других областей, как например искусство. Но однако, позже появилось несколько действительно популярных проектов именно под такими лицензиями. Как ты думаешь, в чём причина их успеха?}

{\noindent \bf P:} Думаю, в том, что они прямо сейчас несут технологию людям, в распоряжении которых есть аппаратура, свободная и при этом стандартизованная.

Прямо сейчас можно взять Arduino или Raspberry Pi и быть уверенным, что б\'{о}льшая часть ПО, написанного для одного устройства, будет работать на таком же чипе в любой точке планеты. 

Например, существует усилитель электрической активности мозга OpenBCI, созданный небольшим стартапом в США, который становится новым Arduino эля энцефалографии, для анализа сигналов мозга, и это тоже очень важно для комьюнити. Есть много других проектов...

Они снижают затраты, потому что нет никакой избыточной защиты от компаний которые всё контролируют и должны поддерживать цены на таком уровне, чтобы сохранить прибыльность. 

Эти аппаратные проекты --- просто инструменты, которые позволяют делиться знаниями и техническим опытом, используя их на практике в качестве готовых блоков.

{\noindent \bf L: Спасибо. 
Думаю, на текущий момент свободные технологии наиболее сильны в области ПО, аппаратного обеспечения и свободных медиа, таких как Википедия, например, или Creative Commons --- пожалуй, вместе это три наиболее сильные направления.}

{\noindent \bf P:} Я думаю, ты забыл одну важную вещь. Ты упомянул свободные медиа, такие как Creative Commons и Википедия, Но есть ведь еще в сфере науки и материалы с открытым доступом, и открытые лабораторные журналы (open notebook). 

Прямо сейчас я работаю над собственным научным проектом, создаю его в режиме открытого журнала, комментирую с помощью свободных инструментов каждый шаг эксперимента. Такой подход позволяет людям воспроизвести эксперимент в аналогичных условиях. И будь у меня оборудование, распространяемое на условиях свободных лицензий, было бы намного легче воспроизвести сбор данных.

{\noindent \bf L: Это очень интересная тема, результаты других исследователей вообще бывает не так уж легко воспроизвести.
And open source software has some powerful positions in scientific research because it allows to see sources and control how where all the calculations really done. }

{\noindent \bf P:} Yes especially, but there also may be problems. I don't know if you have heard, recently we have lost about 40,000 articles regarding neurology and fMRI, because someone put the wrong code into scientific tools used to monitor them. 

The package itself was open source, but nobody was actively testing it, nobody was actively looking. So it's not only about saying “Hey this is open”, as universities are saying - “This piece of software is open, but please don't look at it because only we understand how it works”. 

Yes, you have to document it. You have to put it to open environment and encourage people to test it, to modify, to understand how it works.

If you are the the owner and the creator, and also the only person who understands how it works, it's not really the open source. Somebody has to look at it, somebody has to analyze it. If that would happened to the FMRI tools that where used around the world for the last 15 years, we could have all the knowledge gathered with this tools intact. But now we have to scrape basically the 15 years of neurology, because the software was faulty. It contains different statistical methods than it was supposed to...

{\noindent \bf L: And what is the role of community in supporting such projects? For example there is the known problem of open source projects with only one person who really knows their internals. If such person goes away for some reasons and abandons the development process, the project is in real trouble. So what about the problem of the community involvement in the development?}

{\noindent \bf P:}I believe that in case of some projects community has to be more involved, test something, develop harder. Even if sometimes there is no need in development… For example we have that issue with the set of statistical tests that nobody else is going to implement because they are working, they are fast, they are okay. But we still need someone who actually got stuck in the code. Those who got check it from time to time and notice if anything does go wrong. 

Basically university workers can be maintainers in some cases. I think that if there's nobody from the community to volunteer, nobody able to make some package of software – then we should start looking at the university's officials, asking universities to maintain that. Actually start paying the university researchers to maintain some software and respond to all the people who requires something or would like to ask some questions. 

I am speaking about employees, who would do it instead of working on some of their own projects. Because I know how this is in computational neuroscience in Warsaw, and in Poland. More than 60\% of people aren’t contributing to science. They are mostly just read that old papers and do not creating anything but useless pieces of software that only they can use, but no one else. It is waste of money that are going to the universities. If they are to be useful, they should become maintainers of the scientific packages, they should start analyzing them, they should start being responsible for how they work.

I believe that would be one of the best uses of university money. Universities have actually started networking, because with several neurological projects around Europe you are just creating the same feature over and over again in several different universities: the French are doing the same as Germans, the same as Italians, the same as Polish and the same as Russians.
 
What's the point of replicating involved in it - they fit the same problems and they hope for each other to solve that. 


     }
   \end{Parallel}







\section{Мiхаiл Волчак "---  Мiнск, Беларусь}
%\begin{figure}[ht]
%\centering{\includegraphics[width=4cm]{49_spons_altoros.jpg}}
%\end{figure}

{\noindent \bf LVEE: Ты прымаў удзел у некалькіх папярэдніх LVEE ў досыць розных амплуа: ад прэзентацыі беларускай партыі піратаў да воркшопа па меш-сетках. Як бы ты ідэнтыфікаваў сябе ў першую чаргу?
}

{\noindent \bf Мiхаiл Волчак:} Перш за ўсё для мне цікавыя такія праекты, у якіх удзельнічае супольнасць, таму што, супольнасць дае разнастайнасць, а я лічу што разнастайнасць "--- гэта такая базавая ўмова для развіцця. Гэта датычыцца розных праектаў, у тым ліку і развіцця размеркаваных меш-сетак. Але вядома я "--- пірат.

{\noindent \bf L: Які менавіта тэрмін ты ўкладаеш у слова «пірат»?}

{\noindent \bf М:} Піраты для мяне "--- гэта пэўная ідэалогія, калі Інтэрнэт на сваю карысць выкарыстоўваюць людзі, а не «крывавы карпарэйшн», напрыклад\ldots Гэта пірацкая палітычная ідэалогія, якая зараз узнікае, яе тэзіс, калі лаканічна, дэмакратыя з выкарыстаннем лічбавых сродкаў, праграмных і апаратных.




{\noindent \bf L: І гэтая твая цікавасць неяк перасякаецца з вольнымі ліцэнзіямі і кантэнтам, да якіх ты маеш непасрэднае дачыненне?} 

{\noindent \bf М:}  Свабодны кантэнт у нашу эпоху ведаў, пост-інфармацыйную эпоху\ldots

{\noindent \bf L: А пост-інфармацыйная, гэта?..}

{\noindent \bf М:}  Усе кажуць што мы ў інфармацыйнай, а насамрэч зараз важна пост-інфармацыя, калі мы не толькі ўспрымаем дадзеныя, мы іх ўмеем пераўтвараць у веды. Для спажыўцоў "--- інфармацыйная. Але iнфармацыя гэта не мэта, а сродак або інструмент.

{\noindent \bf L: І як гэта і пірацтва звязана са свабодным кантэнтам?}

{\noindent \bf М:} А як пірацтва звязана с кантэнтам? Кантэнт можна не толькі спажываць, а яго можна яшчэ змяняць, i гэта асноўны падыход, які мне блізкі. Яго можа зменяць кожны, калі мы заходзім у сеціва і маем там вікі ў сваіх свабодных праектах, і такім часам мы не проста спажыўцы, а сустваральнікі. Сустварэнне "--- гэта такая важная штука для любога апенсорса і ўвогуле любой супольнасці, якая хоча развівацца. Ўключаць новае ў свой дыскурс.

Першы кампанент "--- гэта сумесны ўдзел і магчымасць змяняць кантэнт, і другі кампанент "--- абараніць гэты кантэнт у эпоху, калі яго могуць прыватызаваць. Першапачатковую маёмасць супольнасці часам могуць прыватызаваць праз такія інструменты, як інтэлектуальная ўласнасць, гэта цэлы набор законаў, які дазваляе эканамічным суб'ектам, фірмам там, вялікім, малым "--- прысвойваць. Другі бок "--- гэта юрыдычнае спараджэнне свабоднага кантэнта, так званыя свабодныя ліцэнзіі, якія мы маем. GNU, Creative Commons (Творчыя Суполкі), яшчэ шэраг. Пад свабодай кантэнту мы маем тое ж самае, што з сафтом, тыя знакамітыя рычардавскія свабоды: свабода вывучаць, змяняць і распаўсюджваць. Калі ў кантэнту адна свабода знікае "--- знікае і цэласнае разуменне гэтай свабоды. Палітычныя піраты на сёння гэта разумеюць. Напрыклад, з забаронай распаўсюджвання знікае магчымасць арганічнага развіцця супольнасці. Карпаратыўнае развіццё можа мець месца, але тады арганічнасць знікае\ldots


{\noindent \bf L: Аднак, твая цікавасць да свабодных ліцэнзіях з'явілася яшчэ раней? Як гэта было?}

{\noindent \bf М:} Упершыню я пачуў пра гэта дзесьці ў годзе 2009-2010. У той перыяд, калі я пачаў больш глыбока разумець вікі-тэнхалогію і пачаў працаваць. Гэта прыйшло ад сафта. Я доўга мучыўся ад дыхатаміі Wіndows-Lіnux, першы мой Lіnux быў у 1999 годзе, я яго роўна на паўгадзіны устанавіў\ldots

{\noindent \bf L: Які дыстрыбутыў?}

{\noindent \bf М:} Asp Lіnux, там было некалькі дыскаў, якія я засоўваў і высоўваў "--- проста набыў дыскі і пачаў з імі іграцца\ldots Але потым была Kubuntu дзесь у 2010 годзе, у мяне стабільна сталі дзве сістэмы. Другі крок гэта былi web-applіcatіon, Drupal. Я актыўна удзельнічаў у Drupal-супольнасці, актыўна вывучаў Drupal і пасля пачаў чытаць лiцэнзiйныя дамовы GNU GPL, разоў пяць пачынаў чытаць і не заканчваў, але мне спадабаўся канцэпт. Канцэпт не з узору легальнасцi насамрэч "--- тое, што ён выяўляўся не для вырашэння задач камерцыі, а для вырашэння задач супольнасці. Я ў Drupal-супольнасць пагрузіўся не як спажывец праграмы, ці там кода, а як удзельнік, ну і тады я пачаў вывучаць пакеты, з якімі яны працуюць, і тады пакацілася. 

Ліцэнзійныя прынцыпы рэфлексуюць прынцыпы супольнасці, і калі ты адымаеш ад супольнасці штосьці з гэтых прынцыпаў, атрымліваецца нейкая карпаратыўная культура, а калі ты працуеш з трыма свабодамі, атрымліваецца супольнасць. А потым з Creatіve Commons платнічок пайшоў ў 2012 годзе. Я чытаў шмат кніг пра капірайт, і так арганічна гэта пайшло туды\ldots
 

{\noindent \bf L: Лоўрэнс Лессіг распіярыў?}

{\noindent \bf М:} Ведаеш, магчыма ад яго пайшлі некаторыя словы. Яго «Свабодная культура» дала мне нейкі стартавы слоўнікавы запас, на які можна было абапірацца далей.

{\noindent \bf L: А раскажы трохі аб Вікіпедыі. Пра сваё знаёмства з самой тэхналогіяй, падыходам, калі кам'юніці рэгулюе артыкулы.}

{\noindent \bf М:} Знаёмства адбылося практычна. Напісаў артыкул, яго праз не\-калькі хвілін выдалілі, а я на яго патраціў тры дні. Тады я зрабіў паўзу, таму што не разумеў, як працуе гэта супольнасць\ldots Вядома, я выкарыстоўваў той жа падыход, які выкарыстоўваў у Drupal-супольнасці, але яно не спрацавала. І хутчэй па культурна-палі\-тычным плане\ldots

{\noindent \bf L: Як гэта?}

{\noindent \bf М:} Я напісаў артыкул, які не адпавядаў крытэрыям значнасці для рускай вікіпедыі\ldots Потым я зноў пачаў пісаць, напісаў артыкул у беларускай вікіпедыі, яго заапрувілі і жыццё пайшло\ldots І вось у гэты момант, калі заапрувілі (быў статус «на вычытку» і раптам змяніўся) "--- мне стала цікава, як адбываецца самарэгуляцыя, як кантэнт застаецца рэлевантным.

{\noindent \bf L: У нейкім сэнсе самарэгуляцыю ты адчуў на сабе, пры першай спробе\ldots}

{\noindent \bf М:} Не зусім так. Спачатку мне здавалася, што гэта самарэгуляцыя, але потым прыйшло разуменне. Моўныя раздзелы, якія маюць 100-200 тысяч артыкулаў, выкарыстоўваюць стратэгію «набраць колькасць», калі прымаюцца любыя артыкулы, якія маюць 2-3 крыніцы. Калі раздзел перасякае рысу 500-600 тысяч, яны трохі змяняюць стратэгію,але вось гэтая свабода дадавання колькасці існуе. А калі яны перасякаюць мільён, або 800-900 тысяч, яны кардынальна мяняюць стратэгію і павялічваюць крытэрый значнасці. Такую стратэгію спачатку выкарыстоўвала і руская вікіпедыя, але пасля гэтага парога яны пачалі касіць усё, што нязначна для Расіі і выкарыстоўваюць тую ж стратэгію для беларускіх артыкулаў. Іншымі словамі, для “свадобных” ведаў у рувікі выкарыстоўваюцца геаграфічныя і культурна-палітычныя межы.


{\noindent \bf L: Патлумач гэты момант.}

{\noindent \bf М:} Ну, мы ж з большасці рускамоўныя, таму і пішам там.

У вікі-супольнасці ёсць 5 прынцыпаў асноўных, адзін з іх "--- крытэрый значнасці, notable sources, і ў многіх вялікіх раздзелах яна дэфармавалася ў крытэрый мега-якасці, які стварае такі парог уваходу, што ў невялікіх супольнасцяў не знойдзецца столькі спецыялістаў, якія будуць пісаць на адмысловую тэму. Атрымліваецца што невялікая супольнасць, напрыклад беларуская, якая прымае такія ж стандарты, якія існуюць у польскай або ў рускай wіkі (у польскай мякчэй) "--- узнікае такі парог уваходу, што пачаткоўца, які захоча нешта напісаць, выкідваюць з удзелу.


{\noindent \bf L: А чаму такі падыход не назіраецца ў ангельскай?}

{\noindent \bf М:} Да, з ангельскай вікі такія праблемы не ўзнікалі, я публікаваў нават такі эксперыментальны матэрыял, які быў бы з рускай выключаны з тымі крытэрыямі. Кожная супольнасць, нават і анлайнавая, рэфлексуе, як бы ты не хацеў называць яе анлайнавай, пэўную мадэль лакацыі\ldots Ангельская "--- выключэнне, таму што гэта вельмі глабальная супольнасць, яна больш інтэрнацыянальная, там і Еўропа, і Амерыка, і Азія. Ўсюды англійскае камьюніці, і таму там культурнага рэлятывізму або нейкай палітычнай скіраванасці не існуе.
 
{\noindent \bf L: І напрыканцы, як ты ўяўляеш сабе перспектывы адкрытых ліцэнзій ў вобласці, не звязанай з софтам: у open media, open hardware?}

{\noindent \bf М:} Асноўная праблема кантэнтных і сафтовых ліцэнзій "--- тое, што яны не арыгінальныя для трох чвэрцей гэтай планеты. У іх вельмі класная задума, як хакнуць залішнія абмежаванні, напрыклад, капірайта, але яны вельмі арганічна глядзяцца ў краіне, дзе капірайт найбольш развіты і стаў часткай штодзеннага жыцця для многіх. Для Беларусі гэта нехарактэрна, Беларусь ніколі не была краінай, дзе капірайт вельмі прывіваўся\ldots Ад Францыска Скарыны, які быў піратам і правозіў кантрабанднае абсталяванне праз польскую мяжу для распаўсюду свабодных ведаў у Вялікім Княстве Літоўскім.


{\noindent \bf L: ?!}

{\noindent \bf М:} Так, яго двойчы затрымлівалі, і толькі на трэці раз яму ўдалося правезці гэта hardware, і толькі пасля гэтага ён змог заняцца тут кнігадрукаваннем\ldots Умоўна кажучы, гэтая ліцэнзійная фішка павінна перасэнсавацца нашым грамадствам, і ўсімі не Western Europe і амерыканскімі супольнасцямі. Заходняя Еўропа "--- гэта індывідуалістычны падыход да ўсяго, а open source "--- гэта communіty ownershіp, tradіtіonal knowledge, традыцыйныя веды, калі усё “племя” нешта ведае, і ніхто не робіць гэтыя веды прапрыятарнымі, каб нажыцца. Сеткавыя супольнасці зараз працуюць па гэтым жа прынцыпе як плямёны там\ldots з Лацінскай Амерыкі ці Аўстраліі, і таму тут пытанне з гэтымі ліцэнзіямі, якія індывідуальна накіраваныя.

{\noindent \bf L: Але яны досыць паспяхова прымяняюцца?}

{\noindent \bf М:} Не зусім так. Спачатку мне здавалася, што гэта самарэгуляцыя, але потым прыйшло разуменне. Моўныя раздзелы, якія маюць 100-200 тысяч артыкулаў, выкарыстоўваюць стратэгію «набраць колькасць», калі прымаюцца любыя артыкулы, якія маюць 2-3 крыніцы. Калі раздзел перасякае рысу 500-600 тысяч, яны трохі змяняюць стратэгію,але вось гэтая свабода дадавання колькасці існуе. А калі яны перасякаюць мільён, або 800-900 тысяч, яны кардынальна мяняюць стратэгію і павялічваюць крытэрый значнасці. Такую стратэгію спачатку выкарыстоўвала і руская вікіпедыя, але пасля гэтага парога яны пачалі касіць усё, што нязначна для Расіі і выкарыстоўваюць тую ж стратэгію для беларускіх артыкулаў. Іншымі словамі, для “свадобных” ведаў у рувікі выкарыстоўваюцца геаграфічныя і культурна-палітычныя межы.


{\noindent \bf L: Тут выразна чутны голас пірацкай партыі :-)}

{\noindent \bf М:} Я пакуль спрабую перасэнсаваць гэтае пытанне\ldots Я за тое, каб гэтыя ліцэнзіі прысутнічалі, я сам прасоўваю ліцензіі Creative Com\-mons ў Беларусі. Але калі займаешся іх прамоўшнам\ldots Так, пад вікіпедыяй ёсць Creatіve Commons, але 90\% вікіпедыстаў разумеюць гэтую ліцэнзію хутчэй праз паводзіны ў супольнасці, а не ў юрыдычным плане. Паовдзіны супольнасці фарміруюць правілы "--- змест гэтых ліцэнзій, а не наадварот. Гэта для мяне важный момант\ldots

А ліцэнзіі на жалеза "--- гэта больш накшталт патэнтаў. У патэнтаў ёсць адзін плюс\ldots У іх ёсць шмат мінусаў, але адзін плюс, якому пакуль не прыдумана альтэрнатыва. Яны ўсё ўключаюць у адну базу дадзеных, якую чалавек можа праглядаць. Вось калі open source community, якое працуе з hardware, выпрацуе нейкі падобны механізм з класіфікатарам, катэгарызацыяй "--- тады гэта можа быць годнай альтэрнатывай, але пакуль такога механізму няма. Гэта выклік "--- ці могуць актывісты open hardware такое зрабіць, але такое трэба. Першая ўмова росту папулярнасці вольных ліцэнзій для hardware "--- гэта стварыць сістэму, якая будзе больш эфектыўная, больш адкрытая, чым патэнтнае бюро, патэнтныя офісы па розных краінах.

\switchlang{ru}  

\section{Даниэль Надь "---  Будапешт, Венгрия}
%\begin{figure}[ht]
%\centering{\includegraphics[width=4cm]{49_spons_altoros.jpg}}
%\end{figure}

{\noindent \bf LVEE: Поскольку нынешние интервью связаны с комьюни\-ти-ориентированными и свободными технологиями за пределами непосредственно свободного программного обеспечения "--- поговорим о криптовалютах. Как возник твой интерес к этой сфере?}

{\noindent \bf Даниэль Надь:} Это уходит корнями в глубокое детство. В 1984 году была выпущена такая компьютерная игра, Elite...\ldots 

{\noindent \bf L: Ух ты! Всё началось с неё?}

{\noindent \bf Д:} Вот, ты её тоже помнишь. Я в неё вложил "--- т. е. закопал "--- многие тысячи часов, наверное. И когда уже появился Интернет, у меня была такая мечта "--- сделать что-то похожее, но для большого количества игроков. Сейчас уже такие игры существуют, а когда я занимался этим  своим юношеским проектом "--- тогда возникали две проблемы. Во-первых, достаточно быстро стало ясно, что на централизованном сервере хранить все это очень затратно. Хотелось использовать вычислительные мощности клиентов, сделать какую-то распределенную систему. А во-вторых, я заметил, что в Elite наступал такой момент, когда у игрока становилось очень много денег, и игра из-за этого делалась несколько скучной. Я задумался, почему так не случается в жизни, начал читать про денежную систему "--- как это работает, что такое деньги вообще, почему, например, в игре никогда не случается так, что у меня есть какой-то товар, а у покупателей на космической станции нет денег\ldots

{\noindent \bf L: Действительно :)} 

{\noindent \bf Д:} ...а в реальном мире это достаточно частая ситуация. В общем, я начал вчитываться в суть денег, потом наступил крах доткомов 2001 года, когда у многих появился интерес к этой теме, и появилось с кем об этом поговорить. Сначала я начал вчитываться в то, как можно вести распределенный бухгалтерский учет для большой онлайн-игры, а через некоторое время задумался: зачем это делать в игрушечном варианте, когда чего-то подобного миру не хватает по-настоящему. 

{\noindent \bf L: Как обстояли на тот момент дела с электронными платёжными системами?}

{\noindent \bf Д:}  В 1998 году в русскоязычном пространстве появился Webmoney, а за несколько лет до этого был DigiCash, так что ростки криптовалют "--- они появлялись в конце 90-х, а теоретические основы (академические статьи на эту тему) начали появляться ещё в 80-х.

{\noindent \bf L: А в распределенном, децентрализованном виде?}

{\noindent \bf Д:} Всем хотелось децентрализации. Это был такой Святой Грааль, который всем хотелось сделать. И, естественно, была проблема византийского консенсуса. Если описать простыми словами "--- это задача, как обеспечить, чтобы состояние бухгалтерского учета было для всех одинаково, с какой точки ни смотри на систему, т. е. чтобы всегда сохранялась целостность. Это была достаточно сложная проблема, и хорошего практического решения она не имела, пока вдруг из ниоткуда не появился Bitcoin.

{\noindent \bf L:А блокчейн?}

{\noindent \bf Д:} Он не существовал до Bitcoin. 

{\noindent \bf L: Насколько понимаю, распределенная децентрализованная система по идеологии и принципам достаточно близка открытому ПО. И что, Bitcoin оказался первой реально успешной системой электронных денег, по своим принципам родственной свободному ПО?}

{\noindent \bf Д:} Не сказал бы, что первой. Были свободные проекты до Bitcoin. И ePoint ведь тоже был основан на открытом ПО, и еще было много разных проектов. Это в начале двухтысячных была достаточно горячая тема, многие ею занимались, были разные эксперименты, но они были не очень успешны именно из-за той нерешенной проблемы, о которой мы только что говорили.

{\noindent \bf L: Кто первый сумел ее решить, тот завоевал мир?}

{\noindent \bf Д:} Да. Хотя я этот параметр и не очень высоко ценю, но тем не менее рыночная капитализация Bitcoin в разы превосходит рыночную капитализацию остальных криптовалют вместе взятых.

{\noindent \bf L: С технической точки зрения, со времён появления Bitcoin криптовалюты как-то развивались?}

{\noindent \bf Д:} Конечно. Есть разные направления развития. Кстати не все они положительные. Да и направление развития Bitcoin не совсем соответствует тем ожиданиям и предсказаниям, которые ему сопутствовали в 2009 году.  На сегодняшний день децентрализация Bitcoin во многом условна. На самом деле обработка транзакций происходит на достаточно маленьком количестве узлов, которое к тому же еще и сокращается.

{\noindent \bf L: И территориально, насколько я помню, есть проблема, что почти все они расположены там, где быстро делают микросхемы, и\ldots}

{\noindent \bf Д:} ...и где легко воровать электричество, называя вещи своими именами.

Когда эти проблемы Bitcoin стали более-менее очевидными, многие взялись их решать. Первая более-менее успешная альтернатива "--- Litecoin, базируется на коде Bitcoin, и тут как раз видно преимущество открытого ПО, потому что можно было не реализовывать всё с нуля, а просто взять Bitcoin и поправить там некоторые вещи, чтобы достичь тех целей, которые разработчики себе поставили. В случае Litecoin основной задачей была децентрализация майнинга. В отличие от ожиданий Сатоши Накамото, оказалось что идеальный майнер в Bitcoin "--- это не универсальный компьютер, а некая микросхема, созданная именно под эту задачу. И создатели Litecoin задались целью создать такой алгоритм майнинга, который лучше всего выполняется на компьютере общего назначения. Это им в какой-то степени удалось. 

А еще в тот момент, когда запустили Litecoin, люди уже были знакомы с Bitcoin, поэтому у этой системы не было ещё одной проблемы. Около трети общего количества биткоинов находится в руках максимум 18 юридических или физических лиц. Эта очень большая централизация денежной массы случилась из-за того, что в самом начале мало кто знал о Bitcoin, и ранние майнеры очень много себе намайнили. С Litecoin такой проблемы не было, т. к. когда его объявили, очень многие бросились заниматься майнингом. И сам майнинг там децентрализован благодаря тому, что очень долго не было (а по-моему, и до сих пор нет) специализированных микросхем. 

Это было первое развитие темы Bitcoin "--- достаточно успешный проект, у которого тоже существенная рыночная капитализация, и кроме того есть ещё некоторые преимущества перед Bitcoin. Например, существенно быстрее время блоков. 


{\noindent \bf L: Думаю, это нелишним будет пояснить.}

{\noindent \bf Д:} Для тех кто не знаком с основными принципами блокчейна, можно сказать, что Bitcoin "--- это такая большая публичная бухгалтерия, куда записываются все транзакции, и страница в этой бухгалтерской книге "--- это блок.

{\noindent \bf L: Копии всех страниц должны быть на каждом узле, и благодаря системе указателей с цифровыми подписями ни у кого нет возможности незаметно модифицировать часть блокчейна.}

{\noindent \bf Д:} Да. И пока транзакция не попала в эту общую книгу, она теоретически может быть ещё изменена или удалена. Транзакция становится точной и необратимой, когда попадает в этот децентрализованный распределенный блокчейн. В случае Bitcoin это занимает в среднем около 10 минут, а в случае Litecoin "--- 2.5 минуты. 

{\noindent \bf L: После этого были ещё похожие системы?}

{\noindent \bf Д:} Очень много. Многие пытались что-то улучшить. С технической точки зрения им это может быть и удалось, но в случае с криптовалютами очень силён эффект сети. Даже если какая-то криптовалюта по параметрам  лучше Bitcoin, для того, чтобы у нее появилась существенная база пользователей, она должна найти для себя нишу, потому что сам тот факт, что огромное количество людей и бизнесов пользуется Bitcoin, делает остальные альтернативы непривлекательными.
 
{\noindent \bf L: Кстати, по поводу блокчейна, который присутствует на всех узлах. Вроде бы криптовалюты считаются такими анонимными, но при этом все твои транзакции сохраняются навсегда в публичном доступе. Они вроде бы пока не привязаны к твоей персоне, но никогда нет гарантии что\ldots}

{\noindent \bf Д:} ...что кто-нибудь потом не привяжет.


{\noindent \bf L: Да. Так насколько корректно считать это способом анонимнизации платежей?}

{\noindent \bf Д:} По-моему, совершенно некорректно. Bitcoin я бы анонимным не назвал. Есть отдельные анонимизаторы, такие сервисы, принимающие грязные биткоины, а возвращающие отмытые. Потом есть, скажем так, теоретически разработанные технологии, при помощи которых можно создать действительно анонимную криптовалюту. Этим сейчас тоже занимаются люди, и думаю, это тоже одно из направлений, куда в будущем будут развиваться технологии.

{\noindent \bf L: По поводу анонимности. Получается, если какие-то правительства косо смотрят на Bitcoin по поводу возможности сокрытия доходов "--- то это скорее ошибочная точка зрения?}

{\noindent \bf Д:} Нет, с этим уже не соглашусь. Несмотря на то, что биткоины не очень сложно привязать к какому-то конкретному имени, их очень сложно отнять и очень сложно воспрепятствовать их движению. Само по себе использование Bitcoin еще не спасает в случае чего, но если у человека значительная часть сбережений в биткоинах "--- он, грубо говоря, бежит значительно быстрее в случае каких-то потрясений. Когда случаются какие-то финансовые катаклизмы, массовые неплатежи или финансовый кризис в банковской системе, тогда бывает паника, люди пытаются перемещать свои капиталы, но многим это не удаётся или удаётся только частично. И вот в этом случае Bitcoin очень сильно помогает "--- это было продемонстрировано в случае кипрского кризиса, потом греческого кризиса, последнее время в Восточной Европе тоже, в Украине, и в Беларуси тоже, по-моему, те кто сделал ставку на Bitcoin, не ошиблись. 

Но есть ещё и такое преимущество Bitcoin и вообще систем распределённого бухгалтерского учета: если надо, они достаточно хорошо сочетаются с системой налогообложения, контроля. Например, я сейчас получаю значительную часть моей прибыли в криптовалютах, получаю от фонда, фонд должен держать свою бухгалтерию открытой, тот факт что это получаю я, а не кто-то другой "--- это не секрет, я с этих денег конечно же плачу налоги. В налоговой было достаточно просто показать, что вот этот счет, вот транзакция "---  в общем, из-за этого не возникает подозрений, что где-то тут мошенничество или что я что-то скрываю. Т. е. если я хочу играть с открытыми картами "--- я имею такую возможность.

Подводя итог,  такой подход в значительной степени отдаёт контроль над средствами индивиду или организации, даёт возможность решать, какую часть своих операций держать в какой юрисдикции. Это ставит юрисдикции в положение конкуренции друг с другом.


{\noindent \bf L: Еще вопрос про применение блокчейна за пределами платёжных систем. Насколько понимаю, с тех пор появились и такие проекты?}

{\noindent \bf Д:} Да. Я работаю как раз над таким проектом. Сейчас наиболее известен и успешен среди таких проектов Etherium "--- проект, который действительно превратил распределенный бухгалтерский учет Bitcoin в распределенную базу данных общего назначения. Там можно хранить в блокчейне любые данные и записывать условия изменения этих данных на тьюринг-полном языке.
 То есть на этих данных можно проводить практически любые вычисления так, что все могут убедиться: действительно результатом этих вычислений стало то-то и то-то. Здесь открываются такие возможности, о которых даже трудно говорить. Это как пытаться в начале 80-х годов рассказать, что такое персональный компьютер. Подавляющее большинство будущих применений еще не видны из нашей перспективы, потому что их еще не изобрели.

{\noindent \bf L: Можешь привести хотя бы парочку примеров, случайно выбранных?}

{\noindent \bf Д:} Попытаюсь. Например, можно сделать страховку без страховой компании. Если какие-то люди хотят распределить между собой какие-то риски, они могут записать условие в умный контракт: тот, кто регулярно делал какие-то взносы и с ним что-то случилось, получает определённые деньги. Это записывается в контракт, и таким образом страхование становится распределенным и децентрализованным: нет страховой компании, есть только люди, которые между собой поделили риски. 

Или, что меня больше всего интересует "--- это общее пользование капиталом. Самое простое, то над чем я сейчас конкретно работаю "--- чтобы любой человек мог сдавать в аренду свои вычислительные ресурсы. В данном случае это хранение и передача данных,  такое распределенное хранилище, к которому любой человек может подключить свои накопители и зарабатывать на том, что его накопителями пользуются другие люди.

{\noindent \bf L: То есть мы говорим не строго о вычислительной мощности процессора, мы понимаем более широко.}

{\noindent \bf Д:} Да, более широко, но и процессор тоже. В будущем, я думаю, можно будет сдавать в аренду и процессор. Есть какая-то программа, которая записывается в блокчейн, и потом данные можно рассылать разным участникам, они эти данные будут обрабатывать. Конечно, должна быть некоторая избыточность. Если те же  самые данные два узла обработали иначе, автоматически без участия человека можно выяснить, кто из них обработал данные неправильно, и исключить этот узел. Можно, например, распределенно рендерить фильмы. Но это, мне кажется, только первые ласточки, это достаточно просто. А вот, например, есть такая проблема, что огромное количество капитала человечества существует в виде припаркованных автомобилей, которые ничего не делают, а просто загораживают дорогу.

{\noindent \bf L: Неожиданно.}

{\noindent \bf Д:} Из-за того, что мы друг другу не доверяем, я не могу подойти к какому-то припаркованному автомобилю, быстро доказать владельцу, что я надёжный и хороший человек и у меня есть достаточно сбережений, чтобы компенсировать ему ущерб, если я его машину сломаю. Несмотря на то что это так, я не могу легко и просто это доказать. Но при помощи блокчейн-технологий это все становится доказуемо: как репутация, так и финансовое положение. Можно сделать такую автоматическую систему, когда к любому автомобилю можно подойти, быстро доказать, что мне его можно дать в аренду, заплатить, поехать куда-то и там его оставить для следующего желающего. Таким образом, количество машин, которое должно существовать, оказывается значительно меньшим. Многим не нужно будет иметь собственный автомобиль. И мне кажется что в итоге от этого человечество станет значительно богачP:  90\% автомобилей не нужно будет производить, и эти мощности можно будет использовать для чего-то другого.

{\noindent \bf L: По существу, блокчейн "--- такой достаточно мощный механизм, который «перетащит» комьюнити-технолгоии в \linebreak осязаемую сферу?}

{\noindent \bf Д:} Да. Научная фантастика на эту тему существует, мы ее читали, и действительно сказку сейчас делаем былью, в том смысле, что, как мы  считаем, он перевернет мир так же, как в свое время это сделал доступный компьютер. Это очень амбициозный проект, и над ним очень приятно работать :)

 
\end{document}


