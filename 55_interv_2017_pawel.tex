\documentclass[10pt, a5paper]{article}
\usepackage{pdfpages}
\usepackage{parallel}
\usepackage[T2A]{fontenc}
\usepackage{ucs}
\usepackage[utf8x]{inputenc}
\usepackage[polish,english,russian]{babel}
\usepackage{hyperref}
\usepackage{rotating}
\usepackage[inner=2cm,top=1.8cm,outer=2cm,bottom=2.3cm,nohead]{geometry}
\usepackage{listings}
\usepackage{graphicx}
\usepackage{wrapfig}
\usepackage{longtable}
\usepackage{indentfirst}
\usepackage{array}
\newcolumntype{P}[1]{>{\raggedright\arraybackslash}p{#1}}
\frenchspacing
\usepackage{fixltx2e} %text sub- and superscripts
\usepackage{icomma} % коскі ў матэматычным рэжыме
\PreloadUnicodePage{4}

\newcommand{\longpage}{\enlargethispage{\baselineskip}}
\newcommand{\shortpage}{\enlargethispage{-\baselineskip}}

\def\switchlang#1{\expandafter\csname switchlang#1\endcsname}
\def\switchlangbe{
\let\saverefname=\refname%
\def\refname{Літаратура}%
\def\figurename{Іл.}%
}
\def\switchlangen{
\let\saverefname=\refname%
\def\refname{References}%
\def\figurename{Fig.}%
}
\def\switchlangru{
\let\saverefname=\refname%
\let\savefigurename=\figurename%
\def\refname{Литература}%
\def\figurename{Рис.}%
}

\hyphenation{admi-ni-stra-tive}
\hyphenation{ex-pe-ri-ence}
\hyphenation{fle-xi-bi-li-ty}
\hyphenation{Py-thon}
\hyphenation{ma-the-ma-ti-cal}
\hyphenation{re-ported}
\hyphenation{imp-le-menta-tions}
\hyphenation{pro-vides}
\hyphenation{en-gi-neering}
\hyphenation{com-pa-ti-bi-li-ty}
\hyphenation{im-pos-sible}
\hyphenation{desk-top}
\hyphenation{elec-tro-nic}
\hyphenation{com-pa-ny}
\hyphenation{de-ve-lop-ment}
\hyphenation{de-ve-loping}
\hyphenation{de-ve-lop}
\hyphenation{da-ta-ba-se}
\hyphenation{plat-forms}
\hyphenation{or-ga-ni-za-tion}
\hyphenation{pro-gramming}
\hyphenation{in-stru-ments}
\hyphenation{Li-nux}
\hyphenation{sour-ce}
\hyphenation{en-vi-ron-ment}
\hyphenation{Te-le-pathy}
\hyphenation{Li-nux-ov-ka}
\hyphenation{Open-BSD}
\hyphenation{Free-BSD}
\hyphenation{men-ti-on-ed}
\hyphenation{app-li-ca-tion}

\def\progref!#1!{\texttt{#1}}
\renewcommand{\arraystretch}{2} %Іначай формулы ў матрыцы зліпаюцца з лініямі
\usepackage{array}

\def\interview #1 (#2), #3, #4, #5\par{

\section[#1, #3, #4]{#1 -- #3, #4}
\def\qname{LVEE}
\def\aname{#1}
\def\q ##1\par{{\noindent \bf \qname: ##1 }\par}
\def\a{{\noindent \bf \aname: } \def\qname{L}\def\aname{#2}}
}

\def\interview* #1 (#2), #3, #4, #5\par{

\section*{#1\\{\small\rm #3, #4. #5}}

\def\qname{LVEE}
\def\aname{#1}
\def\q ##1\par{{\noindent \bf \qname: ##1 }\par}
\def\a{{\noindent \bf \aname: } \def\qname{L}\def\aname{#2}}
}

%\switchlang{be}
%\usepackage{color}
\begin{document}
\title{Интервью с участниками}
%\author{}
\date{}
\maketitle

По традиции в сборник материалов входят интервью, в которых активные участники
сообщества open source делятся своим мнением о свободном ПО, открытых
технологиях, роли и месте свободных лицензий, рассказывают, как видят проблематику
свободных проектов. В этот раз, из-за англоязычности всех интервьюируемых, интервью приводятся на двух языках - английском и русском.


%\begin{figure}[ht]
%\centering{\includegraphics[width=4cm]{49_spons_altoros.jpg}}
%\end{figure}
\begin{Parallel}[p]{}{}

     \ParallelLText{%
      \selectlanguage{english}
\interview* Paweł Chojnacki (P.), Warsaw, Poland, 

{\noindent \bf LVEE: Can you briefly introduce yourself?}

{\noindent \bf Paweł Chojnacki:} My name is Paweł, I live in Warsaw. I used to be a part of Warsaw hackerspace; right now I'm not engaged in any organizational activity, may be beside the Global Innovation Gathering (\url{globalinnovationgathering.com}), a community which also involves hackering. I’m a web developer, a freelancer, a frontend professional doing some Open Access Science in the meantime. I deal a lot with open source in my professional activity. And in addition to that I’m giving cybersecurity lectures, an introduction to cybersecurity: what to store, why should you use Firefox and Chrome but not Internet Explorer and Safari.  And I work with open source every day.

{\noindent \bf L: How did you get acquainted with open source software? Do you remember?} 

{\noindent \bf P:}  I’ve started using Firefox very early in my life. It was the first year of Firefox existence, I don’t remember exactly which year it was. But I didn't consider it open source, I knew nothing about it. So my first conscious meeting with open source was related to GIMP graphic editor. I just wanted some program which had more options than Paint, and I didn’t want payed Photoshop version. That’s why I learned GIMP. I looked through one tutorial, and the second, and started learning to work with GIMP. Than I met some people speaking Polish who were interested in GIMP and helped them to found Polish Gimp Users' Forum (messaging board for Polish-speaking GIMP users). And from that I started reading and got very interested in Linux and the open source philosophy, and that was, I think, in mid school and high school. 

{\noindent \bf L: By the way, did you have some previous experience with Photoshop, before starting GIMP, or not?}

{\noindent \bf P:}  I didn't have any previous experience. Nothing with Photoshop, or Corel, or anything else.

{\noindent \bf L: It's an interesting question, because we know a lot of examples, when different previous experience had complicated people using GIMP more or less. And you have also mentioned Hackerspace. It’s a hackerspace in Warsaw?}

{\noindent \bf P:} Yeah. Warsaw hackerspace. 

{\noindent \bf L: The topic of hackerspace itself is rather interesting, because we have finally officially registered Hackerspace in Minsk some time ago. How long does it work in Warsaw?}

{\noindent \bf P:} Four or five years. I was in the board of hackerspace, and was basically driving it full time as a community manager, or whatever you want to call it, for several months. It was founded basically by some enthusiasts but got wheel speed when Warsaw University of Technology decided to shut down their experimental technology division and all people who wanted to experiment went to the hackerspace. And right now Warsaw hackerspace is a club. It’s typical members are people who have quite similar background from the Internet, who chat there… Even with a similar sense of humor. Those who like specific areas of IT, the security, administration, and who want to contribute to some projects or just to use technology for fun.

{\noindent \bf L: Usually hackerspace communities have strong connections with some open hardware projects: they are either developing their own or using some devices previously developed under such open hardware licenses. Perhaps there is the same situation in Warsaw, isn’t it?}

{\noindent \bf P:} We don’t have any projects that the hackerspace is creating as a hackerspace, but several members have created several hacks, for example way to install OpenWrt on specific devices which where unknown to be able of doing this.

Also there is some tool for gathering money from the members, just to send e-mails with reminders, and show current status. So hackerspace has some infrastructure. It is open but not necessary all of it is open source. 
 
A lot of things are based on Arduino, but I wouldn’t call it open hardware with us, as not all this things are documented and shared. But at least few things are.

{\noindent \bf L: At the very beginning open hardware was exotic, it was even supposed that open source licenses are good for software but not for hardware and other areas, like art. But well, later several really popular projects have appeared published under such licenses. How do you think, what's the reason of their success?}

{\noindent \bf P:} I think it is because they bring right now technology to the people who have open hardware which is also standartized.

Because right now you can take Arduino, take Raspberry Pi, and you can be sure that the most of software written for the one piece will work on any similar kind of chip, anywhere in the world. 

For example we have OpenBCI brain amplifier created by a small startup in US, which is becoming a new Arduino for brain amplifying, for just brain signal analysis which is also a very important for the community. There are a lot of other projects...

So they are bringing the costs down, as you don't have any kind of overprotection from company that puts control and keeps the prices  so that they are profitable. 

This hardware projects are just tools for sharing knowledge and for sharing technical experience using them as building blocks in practice.

{\noindent \bf L: Thank you. 
I think that open source is the most powerful in software areas nowadays, but with open hardware and open media like Wikipedia for example, or creative Commons materials online – perhaps together they are three most widely used open directions.}

{\noindent \bf P:} I think you forgot about one thing. You’ve mentioned the media as Creative Commons and Wikipedia, but there's also open access and open notebook at science. 

Right now I’m working on the scientific project of my own, creating it with the totally open notebook access, so that I’m commenting every step of my experiment, with use of open source tools. This approach allows people to replicate the experiment in same environment. If I actually had the equipment which would be open hardware, it could be much easier to actually replicate the data gathering.

{\noindent \bf L: It’s really interesting, as it’s not very easy to reproduce results of other researchers sometimes.
And open source software has some powerful positions in scientific research because it allows to see sources and control how where all the calculations really done. }

{\noindent \bf P:} Yes especially, but there also may be problems. I don't know if you have heard, recently we have lost about 40,000 articles regarding neurology and fMRI, because someone put the wrong code into scientific tools used to monitor them. 

The package itself was open source, but nobody was actively testing it, nobody was actively looking. So it's not only about saying “Hey this is open”, as universities are saying -- “This piece of software is open, but please don't look at it because only we understand how it works”. 

Yes, you have to document it. You have to put it to open environment and encourage people to test it, to modify, to understand how it works.

If you are the the owner and the creator, and also the only person who understands how it works, it's not really the open source. Somebody has to look at it, somebody has to analyze it. If that would happened to the FMRI tools that where used around the world for the last 15 years, we could have all the knowledge gathered with this tools intact. But now we have to scrape basically the 15 years of neurology, because the software was faulty. It contains different statistical methods than it was supposed to\ldots

{\noindent \bf L: And what is the role of community in supporting such projects? For example there is the known problem of open source projects with only one person who really knows their internals. If such person goes away for some reasons and abandons the development process, the project is in real trouble. So what about the problem of the community involvement in the development?}

{\noindent \bf P:}I believe that in case of some projects community has to be more involved, test something, develop harder, even if sometimes there is no need in development. For example we have that issue with the set of statistical tests that nobody else is going to implement because they are working, they are fast, they are okay. But we still need someone who actually got stuck in the code. Those who got check it from time to time and notice if anything does go wrong. 

Basically university workers can be maintainers in some cases. I think that if there's nobody from the community to volunteer, nobody able to make some package of software -- then we should start looking at the university's officials, asking universities to maintain that. Actually start paying the university researchers to maintain some software and respond to all the people who requires something or would like to ask some questions. 

I am speaking about employees, who would do it instead of working on some of their own projects. Because I know how this is in computational neuroscience in Warsaw, and in Poland. More than 60\% of people aren’t contributing to science. They are mostly just read that old papers and do not creating anything but useless pieces of software that only they can use, but no one else. It is waste of money that are going to the universities. If they are to be useful, they should become maintainers of the scientific packages, they should start analyzing them, they should start being responsible for how they work.

I believe that would be one of the best uses of university money. 


\interviewfooter{Questions and Russian translation by Dmitriy Kostiuk.}
\vfill
     }
     \ParallelRText{%
       \selectlanguage{russian}
\interview* Paweł Chojnacki (P.), Варшава, Польша, 
       
{\noindent \bf LVEE: Для начала, несколько слов о себе.}

{\noindent \bf Paweł Chojnacki:} Меня зовут Павел, я живу в Варшаве. Раньше активно участвовал в Варшавском Хакерспейсе; в настоящий момент организационной активностью не занимаюсь, ну может быть кроме Global Innovation Gathering (\url{globalinnovationgathering.com}), сообщество, связанное в том числе с хакерством. Я веб"=разработчик, фрилансер, специалист по разработке фронтэнда, а кроме того занимаюсь наукой с открытым доступом (Open Access). Мне много приходится работать со свободным ПО в профессиональной деятельности. А еще я читаю лекции по компьютерной безопасности, такое введение в информационную безопасность: что хранить, почему использовать Firefox и Chrome, а не Internet Explorer или Safari. Ну и каждый день имею дело со свободным ПО.

{\noindent \bf L: Не вспомнишь, как ты вообще познакомился со свободным ПО?} 

{\noindent \bf P:}  Я начал пользоваться Firefox в очень раннем возрасте. Это был первый год существования Firefox как такового, сейчас не вспомню, в каком именно году. Но я не воспринимал его как свободное ПО, просто ничего об этом не знал. Так что первая сознательная встреча со свободным ПО была связана с графическим редактором GIMP. Мне была нужна какая-нибудь программа с б\'{о}льшими чем у Paint возможностями, и мне не хотелось иметь дело с платной версией Photoshop. Поэтому я изучил GIMP. Заглянул в один туториал, потом в другой, и начал учиться с ним работать. Затем я познакомился с польскоговорящими людьми, которые интересовались GIMP, и помог им создать Polish Gimp Users' Forum (площадку для польскоговорящих пользователей GIMP). И начиная с этого момента я стал читать, заинтересовался Linux и философией свободного ПО, думаю, это было в средних или старших классах. 

{\noindent \bf L: А кстати, у тебя уже был какой-то опыт работы в \linebreak Photoshop до GIMP или нет?}

{\noindent \bf P:}  Никакого прежнего опыта. Ни с Photoshop, ни с Corel, ни с чем-либо ещё.

{\noindent \bf L: Это интересный вопрос, потому что есть достаточно примеров, когда предыдущий опыт с чем-то другим в той или иной степени осложнял людям изучение GIMP. Итак, ты упоминал хакерспейс. Это хакерспейс в Варшаве?}

{\noindent \bf P:} Да. Варшавский хакерспейс. 

{\noindent \bf L: Тема хакерспейса сама по себе интересна, потому что не так давно мы наконец официально зарегистрировали хакерспейс в Минске. Как долго он существует в Варшаве?}

{\noindent \bf P:} Четыре или пять лет. Я был в совете хакерспейса, несколько  месяцев фактически тянул его на себе на штатной основе в качестве комьюнити-менеджера, или как ещё можно назвать эту должность. Он был сначала основан несколькими энтузиастами, а раскрутился, когда в Варшавском технологическом университете закрылось экспериментальное технологическое отделение, и тога все, кто хотел заниматься экспериментами, отправились в хакерспейс. А сейчас Варшавский хакреспейс --- это клуб. Его типичные члены --- это люди со схожим опытом в Интернет, которые в нём общаются\ldots Вплоть до схожего чувства юмора. Те, кому нравятся определённые области IT, безопасность, администрирование, кто хотел бы вносить вклад в некоторые проекты или просто возиться с технологиями ради собственного удовольствия.

{\noindent \bf L: Обычно комьюнити хакерспейсов имеют сильные связи с проектами свободного аппаратного обеспечения: разрабатывают что-то своё, или используют устройства, созданные ранее под свободными лицензиями. В Варшаве наверное такая же ситуация?}

{\noindent \bf P:} У нас нет проектов, создаваемых в хакерспейсе именно от имени хакерспейса, но некоторые члены были инициаторами некоторых хаков --- например, способ установить прошивку OpenWrt на некоторые устройства, для которых прежде ничего не было известно о такой возможности.

Ещё есть инструмент для сбора денег с участников: напоминания по e-mails и т.~д., отображение текущего статуса. Понятно, у хакерспейса есть определенная инфраструктура. Она открыта, но не обязательно является свободной вся целиком. 
 
Многие вещи построены на Arduino, но в нашем случае я не назвал бы это свободным аппаратным обеспечением, потому что не все проекты задокументированы и выложены в общий доступ. Но несколько --- да.

{\noindent \bf L: В самом начале свободное аппаратное обеспечение считалось экзотикой, считалось даже, что свободные лицензии хороши для программ, но не для аппаратуры и некоторых других областей, как например искусство. Но однако, позже появилось несколько действительно популярных проектов именно под такими лицензиями. Как ты думаешь, в чём причина их успеха?}

{\noindent \bf P:} Думаю, в том, что они прямо сейчас несут технологию людям, в распоряжении которых есть аппаратура, свободная и при этом стандартизованная.

Прямо сейчас можно взять Arduino или Raspberry Pi и быть уверенным, что б\'{о}льшая часть ПО, написанного для одного устройства, будет работать на таком же чипе в любой точке планеты. 

Например, существует усилитель электрической активности мозга OpenBCI, созданный небольшим стартапом в США, который становится новым Arduino эля энцефалографии, для анализа сигналов мозга, и это тоже очень важно для комьюнити. Есть много других проектов...

Они снижают затраты, потому что нет никакой избыточной защиты от компаний которые всё контролируют и должны поддерживать цены на таком уровне, чтобы сохранить прибыльность. 

Эти аппаратные проекты --- просто инструменты, которые позволяют делиться знаниями и техническим опытом, используя их на практике в качестве готовых блоков.

{\noindent \bf L: Спасибо. 
Думаю, на текущий момент свободные технологии наиболее сильны в области ПО, аппаратного обеспечения и свободных медиа, таких как Википедия, например, или Creative Commons --- пожалуй, вместе это три наиболее сильные направления.}

{\noindent \bf P:} Я думаю, ты забыл одну важную вещь. Ты упомянул свободные медиа, такие как Creative Commons и Википедия, Но есть ведь еще в сфере науки и материалы с открытым доступом, и открытые лабораторные журналы (open notebook). 

Прямо сейчас я работаю над собственным научным проектом, создаю его в режиме открытого журнала, комментирую с помощью свободных инструментов каждый шаг эксперимента. Такой подход позволяет людям воспроизвести эксперимент в аналогичных условиях. И будь у меня оборудование, распространяемое на условиях свободных лицензий, было бы намного легче воспроизвести сбор данных.

{\noindent \bf L: Это очень интересная тема, результаты других исследователей вообще бывает не так уж легко воспроизвести.
А свободное ПО в научных исследованиях имеет серьёзные преимущества, потому что позволяет посмотреть исходный код и проконтролировать, как на самом деле выполнялись все вычисления.}

{\noindent \bf P:} Да, именно, но здесь тоже могут быть проблемы. Не знаю, слышал ли ты, недавно мы потеряли около 40~000 статей по нейрологии и функциональной МРТ, потому что кто-то поместил неверный код в программные инструменты, использовавшиеся для мониторинга. 

Сам по себе пакет был свободным ПО, но никто его особенно не тестировал, никто не уделял ему особого внимания. Так что дело не только в том, чтобы сказать <<Привет, это открыто>>, как любят делать в университетах: <<Это ПО открытое, но даже не пытайтесь заглянуть в его код, всё равно только мы понимаем, как оно работает>>. 

Да, вам нужно заниматься документированием. Нужно разместить его в открытой среде и вовлекать людей в его тестирование, изменение, чтобы они понимали, как там всё работает.

Если вы владелец и создатель проекта, и при этом единственный, кто понимает, как он работает "--- это не настоящее свободное ПО. Кто-то должен за ним присматривать, анализировать его. Будь так с теми инструментами по функциональной МРТ, которые использовали во всем мире в течение 15 лет, знания, полученные с помощью этих инструментов, не пострадали бы. А сейчас нам надо вычистить примерно 15 лет неврологии из-за дефектного ПО. Статистические методы там оказались совсем не те, что предполагалось\ldots

{\noindent \bf L: А какую роль вообще играет комьюнити в поддержке таких проектов? Например, есть известная проблема свободных проектов, внутреннее устройство которых знает один единственный человек. Если такой разработчик по какой-то причине отходит от дел и прекращает работать над проектом, то проект оказывается в большой беде. Так как быть с проблемой вовлечения сообщества в разработку?}

{\noindent \bf P:}Я верю, что в некоторых проектах сообщество должно больше участвовать, что-то тестировать и больше вкладываться в разработку, даже если иногда в разработке нет особой нужды. Например, этот наш случай с набором статистических тестов, которые никто не собирается реализовывать заново, потому что они работают, они быстрые, и с ними всё в порядке. И всё же нужен кто-то, кто реально возится с кодом. Кто-то, кто проверяет его время от времени и заметит, если что-то пойдёт не так. 

Вообще-то в некоторых случаях сотрудники университетов могли бы быть мэйнтейнерами. Думаю, если не находится волонтёров от комьюнити, и никто не может подготовить некий программный пакет "--- тогда есть смысл посмотреть в сторону университетской администрации, обратиться к университетам с просьбой о поддержке таких проектов. Фактически, платить исследователям в университетах за поддержку какого-то ПО, и за общение с людьми, интересующимися каким-то функционалом или имеющими какие-то вопросы. 

Я говорю о сотрудниках, которые могли бы заниматься этим вместо работы над некоторыми своими проектами. Потому что я знаю как обстоят дела с вычислительной нейробиологией в Варшаве, и вообще в Польше. Более 60\% людей не вносят вклад в науку. Они в основном перечитывают старые статьи и не создают ничего кроме бесполезного ПО, которым не может пользоваться никто кроме них. Это пустая трата поступающих в университеты денег. Чтобы приносить пользу, они могли бы стать мэйнтейнерами каких-то научных пакетов, могли бы начать их анализировать, взять на себя ответственность за то, как эти пакеты работают.

Я верю, что это было бы одним из лучших способов тратить университетские бюджеты. 

\interviewfooter{Вопросы и русский перевод Дмитрия Костюка.}
\vfill

     }
   \end{Parallel}









 
\end{document}


