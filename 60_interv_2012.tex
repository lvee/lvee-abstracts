\documentclass[10pt, a5paper]{article}
\usepackage{pdfpages}
\usepackage{parallel}
\usepackage[T2A]{fontenc}
\usepackage{ucs}
\usepackage[utf8x]{inputenc}
\usepackage[polish,english,russian]{babel}
\usepackage{hyperref}
\usepackage{rotating}
\usepackage[inner=2cm,top=1.8cm,outer=2cm,bottom=2.3cm,nohead]{geometry}
\usepackage{listings}
\usepackage{graphicx}
\usepackage{wrapfig}
\usepackage{longtable}
\usepackage{indentfirst}
\usepackage{array}
\newcolumntype{P}[1]{>{\raggedright\arraybackslash}p{#1}}
\frenchspacing
\usepackage{fixltx2e} %text sub- and superscripts
\usepackage{icomma} % коскі ў матэматычным рэжыме
\PreloadUnicodePage{4}

\newcommand{\longpage}{\enlargethispage{\baselineskip}}
\newcommand{\shortpage}{\enlargethispage{-\baselineskip}}

\def\switchlang#1{\expandafter\csname switchlang#1\endcsname}
\def\switchlangbe{
\let\saverefname=\refname%
\def\refname{Літаратура}%
\def\figurename{Іл.}%
}
\def\switchlangen{
\let\saverefname=\refname%
\def\refname{References}%
\def\figurename{Fig.}%
}
\def\switchlangru{
\let\saverefname=\refname%
\let\savefigurename=\figurename%
\def\refname{Литература}%
\def\figurename{Рис.}%
}

\hyphenation{admi-ni-stra-tive}
\hyphenation{ex-pe-ri-ence}
\hyphenation{fle-xi-bi-li-ty}
\hyphenation{Py-thon}
\hyphenation{ma-the-ma-ti-cal}
\hyphenation{re-ported}
\hyphenation{imp-le-menta-tions}
\hyphenation{pro-vides}
\hyphenation{en-gi-neering}
\hyphenation{com-pa-ti-bi-li-ty}
\hyphenation{im-pos-sible}
\hyphenation{desk-top}
\hyphenation{elec-tro-nic}
\hyphenation{com-pa-ny}
\hyphenation{de-ve-lop-ment}
\hyphenation{de-ve-loping}
\hyphenation{de-ve-lop}
\hyphenation{da-ta-ba-se}
\hyphenation{plat-forms}
\hyphenation{or-ga-ni-za-tion}
\hyphenation{pro-gramming}
\hyphenation{in-stru-ments}
\hyphenation{Li-nux}
\hyphenation{sour-ce}
\hyphenation{en-vi-ron-ment}
\hyphenation{Te-le-pathy}
\hyphenation{Li-nux-ov-ka}
\hyphenation{Open-BSD}
\hyphenation{Free-BSD}
\hyphenation{men-ti-on-ed}
\hyphenation{app-li-ca-tion}

\def\progref!#1!{\texttt{#1}}
\renewcommand{\arraystretch}{2} %Іначай формулы ў матрыцы зліпаюцца з лініямі
\usepackage{array}

\def\interview #1 (#2), #3, #4, #5\par{

\section[#1, #3, #4]{#1 -- #3, #4}
\def\qname{LVEE}
\def\aname{#1}
\def\q ##1\par{{\noindent \bf \qname: ##1 }\par}
\def\a{{\noindent \bf \aname: } \def\qname{L}\def\aname{#2}}
}

\def\interview* #1 (#2), #3, #4, #5\par{

\section*{#1\\{\small\rm #3, #4. #5}}

\def\qname{LVEE}
\def\aname{#1}
\def\q ##1\par{{\noindent \bf \qname: ##1 }\par}
\def\a{{\noindent \bf \aname: } \def\qname{L}\def\aname{#2}}
}

\begin{document}
\title{Интервью с участниками}
%\author{}
\date{}
\maketitle

По сложившейся традиции в сборник материалов включены интервью, взятые представителями оргкомитета у участников сразу после предыдущей летней конференции. Участники рассказывают о себе и своей работе, делятся планами, высказывают мнения по актуальным вопросам, волнующим сообщество. 

\interview Алексей Новодворский (А.~Н.), Москва, РФ, заместитель генерального директора компании Альт Линукс

\q Для начала традиционный вопрос всех интервью на Linux Vacation: как у вас возник интерес к свободному ПО?

\a  Была такая питерская фирма Урбансофт, основанная Джоном Росмэном. Они занимались популяризацией свободного софта, и я как-то купил их диск-сборку. Кое-что из свободных систем я и до этого видел, например FreeBSD, но вообще было очень интересно, я купил их диск и мы с сыном установили его дома каждый на свой компьютер. У меня был по"=мощнее, и я установил RedHat,  у него был менее мощный, и он установил Debian. С тех пор так и получилось, что  я развивал направление  RPM, а сын стал Debian"=девелопером, хотя следует отметить, что у нас никогда не было по этому поводу какого-то противостояния :)  Но тем не менее, кто первый что установил "--- тому то и понравилось. 

\q ALT Linux "--- самый известный русскоязычный дистрибутив пост"=советского пространства, поэтому вопрос, который в принципе нельзя не задать: а как вообще рождаются национальные дистрибутивы? По крайней мере, как это получилось у вас?

\a Начиналось все с такой группы IPLabs Linux Team, которая ставила своей целью популяризацию Linux-софта. Вначале это для нас далеко не было основным занятием, мы готовили к изданию различные дистрибутивы, писали статьи о них и о развитии свободного софта, я в частности раз в неделю писал какие-то обзоры. Со временем стали обращать внимание на интернационализацию "--- не локализацию, не переводы, потому что тогда были проблемы большие именно с  интернационализацией. А потом уже вокруг этой группы  IPLabs Linux Team\ldots К нам приходили люди, заинтересованные в разработке, и мы стали заниматься разработкой. Сначала мы много работали с другими командами "--- в частности, делали базовую интернационализацию для SuSE, ездили в Нюрнберг, подружились с разработчиками\ldots Мы пробовали ввозить на продажу их коробочные версии, но это оказалось очень сложно из"=за всяких таможенных хитростей. Познакомились с Дювалем [\emph{Гаэль Дюваль "--- создатель дистрибутива, известного сейчас как Mandriva}], и появился с его благословения Mandrake Russian Edition. А дальше "--- у команды, которая сформировалась вокруг этого проекта, были свои представления о том, как нужно жить, куда двигаться дальше, что разрабатывать. В частности, это касалось вопросов безопасности, потому что среди нас были очень серьезные специалисты в этой области, и в меньшей степени интернационализации (хотя там тоже возникали проблемы),  состава пакетов дистрибутива и так далее. Короче говоря, когда выяснилось, что несмотря на прекрасные отношения, мы не можем сколько-нибудь серьезно влиять на политику коллег, мы решили попробовать сами. Главное в этом деле "--- определение стратегии развития и наличие сильной команды. Команда у нас сформировалась очень сильная, и в 2001 году мы организовали фирму Альт Линукс, на основе  IPLabs Linux Team и команды LRN, был тогда такой проект. В рамках второго проекта возникли такие интересные общественные начинания, как LinuxFest. Фирма Альт Линукс от дистрибутивостроения довольно быстро перешла к решению более на наш взгляд интересных задач, которые были связаны с построением инфраструктуры, и постепенно мы перешли к проблемам построения платформы как продукта. 

\q Получается, сначала было принято решение сделать собственный дистрибутив? Не было такого, что в один прекрасный момент вы посмотрели на свои наработки и подумали, что это больше похоже на\ldots

\a Нет, Сначала был  Mandrake Russian Edition 6.0, потом 7.0, а потом, когда образовался Альт Линукс, был уже готов практически и вышел сразу же Mandrake RE Spring 2001, который на самом деле был вполне самостоятельным дистрибутивом, но мы договорились с Дювалем, что в последний раз выступаем под их именем, потому что там было очень много все"=таки от проекта Mandriva. 

И мне, и Алексею Смирнову [\emph{генеральный директор Альт Линукс}]  больше всего нравилась именно идея свободной разработки, доступной для всеобщего развития: то, что мы могли править код, и то, что мы выпускали код, который могли править другие. Мы тогда не особо задумывались, как вокруг этого организовать бизнес. Проект  IPLabs Linux Team не преследовал изначально какие"=то коммерческие цели. У нас была  совершенно другая работа, мы тогда занимались наукой, логикой, но нам это было очень близко по духу, и мы хотели рассказать об этом широкому кругу своих знакомых "--- и незнакомых. А дальше уже, я думаю, практически как со всеми, кто занимается свободным софтом\ldots Это стало отнимать все больше времени, потому что было интересно, и мы стали думать: а как на развитие всех этих совершенно замечательных идей, на написание новых продуктов и модификацию старых, как на это находить деньги. Бизнес изначально был ориентирован на то, как сделать, чтоб и нам было что поесть, и разработчику, которого мы могли бы пригласить. 

\q И переходя к современности: в условиях, когда информационное сообщество становится более однородным, а локализация  достаточно хороша в базовой версии — как вам видится в этой ситуации роль или будущее направление развития для национальных дистрибутивов как таковых?

\a Честно говоря, ALT Linux мы не позиционировали никогда как национальный дистрибутив\ldots

\q \ldots Но все окружающее его так позиционировали.

\a Ну, я думаю, что, во"=первых, не все\ldots А, во"=вторых, как я уже говорил, мы сделали интернационализацию приличную, но не это было нашим основным преимуществом. И пока в сообществе была атмосфера "--- не желание озолотиться, а желание построить такой бизнес, который позволил бы развивать свое, всеми любимое "--- было очень хорошо и интересно.

Вопрос пиара на международном рынке стоит, конечно, остро, это действительно вопрос сложный. Но есть еще один момент: несмотря на то, что у нас в списке рассылки разработчиков английский язык разрешен, все предпочитают говорить на русском. Периодически появляются разработчики не из стран, в которых знают русский язык, но тем не менее их не становится достаточно много, мы общаемся только по отдельным проектам. У нас с ними идет интенсивный диалог, но не в основных списках рассылки. 

Я думаю, сейчас никакого будущего у национальных дистрибутивов нет, но весь вопрос в том, что такое «национальный». Дистрибутив, который умышленно себя позиционирует для русских, украинцев, Южной Африки — кого угодно — это несерьезно. Другое дело — влияние таких вот продуктовых фирм, которые развивают свободное программное обеспечение, на собственно атмосферу IT"=индустрии внутри страны. И вот это "--- вопрос очень серьезный, он примыкает к вопросу продуктовых фирм. Потому что страны, которые представлены, в частности, на LVEE — они могут похвастаться тем, что существуют мощные фирмы, которые занимаются оффшорным программированием, и в которых сотрудники хорошо зарабатывают. Это очень хорошо и на здоровье, но при этом  продуктовые фирмы все — даже не в Европе в основном. И это связано с другим вопросом: а согласны ли мы на такое международное разделение труда, когда не мы определяем направление развития? Мы хоть и хорошо зарабатывающие — но исполнители, которых привлекают, когда что"=то нужно. Или все"=таки нам интересно (просто интересно, никаких геополитических намерений) делать здесь какой"=то может быть небольшой, но все"=таки свой центр разработки, который тоже как"=то влиял бы на то, что происходит. Этот же вопрос был перед нами, когда мы создавали Альт Линукс: а можем мы создать такой проект, который нам тоже будет интересен, и который мы сможем развивать? И вот эта проблема "--- она проблема очень серьезная. К сожалению, в этой проблеме сейчас есть изрядное количество политики, причем не столько внешней, сколько внутренней: много лоббистов крупных фирм, которые хотят устроить такое разделение труда. Это неплохо, может быть так и нужно\ldots Но мне кажется, что как и нам в 2001 году, так и многим из нас здесь было бы интереснее делать свои продукты, пытаться выйти с ними на международный рынок, и может быть очень небольшую пока свою роль играть и показывать свою точку зрения. Тем более, что она есть. 

\interview Сергей Васильев (С.~В.), Таллин, Эстония, SQA"=инженер в компании Symantec, музыкант и энтузиаст свободного ПО из Таллина

\q Как получилось, что ты заинтересовался open source?

\a Рассказ будет неполным без некоторой предыстории. Рассказать надо не почему я заинтересовался open source, а почему я ушел от Windows. Все началось с операционной системы OS/2 от IBM. В 99 году благодаря преподавателю по операционным системам мой мир полностью изменился. OS/2 продержалась недолго, потому что пришло сознание, что это тоже проприетарная система, но я понял, что есть что"=то другое, нежели Windows. Потом пришла FreeBSD. Пришла, когда я начал заниматься Ethernet"=провайдингом, в начале 2000"=х. Тогда домашние сети процветали и надо было на чем"=то делать роутеры, потому что железных еще не было. Несколько лет я усиленно изучал FreeBSD.

\q А как вообще переход с OS/2 на FreeBSD?

\a Тяжел. Тяжел. Я тогда был молодой, глупый, а отсутствие, простите, дисков C и D "--- это было суровое потрясение :). Шучу конечно, это довольно быстро прошло. Стереотипы действительно пришлось ломать, но переход состоялся. А с Linux вообще проблем не возникло.

\q Сколько дней заняла установка FreeBSD, самая первая?

\a Полчаса установка и где"=то три часа компиляция ядра нового, потому что не было файрвола по умолчанию. 

\q А говоришь, прошло тяжело.

\a Нет, тяжело не установка прошла, а потом – когда надо было с ней работать.

\q То есть она установила себя сама, а потом пришлось с этим работать, и тут"=то начались проблемы? :)

\a Были не то что проблемы, были потрясения устоев :). Совершенно другая концепция, отсутствие UI, понятие монтирования, девайсы "--- это файлы, ну и <<многопользовательскость>>, конечно же, права и все такое.  Потом Linux, потому что мне дали заказ написать систему биллинга для интернет"=кафе на основе LTSP (Linux Terminal Server Project). Я сначала написал биллинговую систему, потом человек, который все это поддерживал, ушел, и все осталось на мне. Мне пришлось разбираться в Linux, и все пошло"=поехало. Потом стали появляться другие проекты, потом я в итоге перешел в компанию, в которой сейчас работаю и занимаюсь тестингом наших проектов на Linux, HP"=UX, Solaris, AIX и всем остальном.

\q У вас достаточно универсальный Unix"=софт?

\a Он не только Unix, просто я работаю в Unix"=team, и мы плотно работаем также с Linux.

\q А интерес к музыке?

\a Интерес к музыке "--- он был всегда. Окончена «музыкалка», потом несколько забросил, потом лет в 20 стал играть в группах, потом подзабросил, потом снова стал играть в группах, сейчас снова подзабросил. В основном играю для себя и друзей "--- квартирники, джемы. А когда стал увлекаться Linux, в 2005--2006 году, стал пробовать смотреть, как вообще можно open source использовать для музыки. В 2005 это было тяжело, очень неудобно. Сейчас намного проще. 

\q За прошедшие 4--5 лет все сильно изменилось?

\a Все сильно изменилось и продолжает изменяться в лучшую сторону. Как я раньше упоминал в докладе [на LVEE 2011] "--- рассказывайте своим знакомым музыкантам, пусть знакомятся со свободным софтом, пишут баги, давят на разработчиков, и тогда софт будет еще лучше. Потому что сейчас некоторые вещи еще не очень.

\q А вообще — музыкальный софт под Linux пишут видимо в основном те, кто им пользуется?

\a Нет. Музыкальный софт для Linux пишут не профессионалы"=музыканты, и в этом самая главная проблема софта. Музыкальный софт для Linux пишут музыканты"=энтузиасты, т.~е. программисты, которые владеют еще какими"=то музыкальными инструментами.

\q Например, закончили когда"=то музыкальную школу и раз в пять лет участвуют в какой"=то музыкальной группе?

\a Например, но не обязательно. Суть в том, что поэтому софт "--- он немного неюзабельный. А юзабилити у него страдает, потому что над коммерческим софтом работает в том числе много консультантов из профессиональных музыкантов, а под Linux он делается энтузиастами. Человек делает, как он видит, и не консультируется, или у него нет возможности проконсультироваться. И это "--- основная проблема музыкального софта под Linux.

\q А профессионального софта для создания музыки под Linux нету?

\a Смотря что понимать под словами «профессиональный софт». Это софт, который используется профессионалами? А кто есть профессионалы? Те люди, которые зарабатывают этим на жизнь, делают на этом деньги. В этом смысле профессионального софта под Linux нет, потому что музыканту удобнее взять MacBook или еще что"=то, где есть все и работает из коробки. А если попытаться кому"=то объяснить, что вот, тебе надо еще включить realtime"=ядро, немножко подпилить Jack, потом еще что"=нибудь сделать "--- у музыканта начинает нервно дергаться глаз, он открывает свой MacBook и забывает про Linux. То есть пока "--- нет, профессионалы не будут переходить. 

\q А дистрибутивы, ориентированные на музыкантов "--- они тоже имеют примерно такой уровень требований?

\a Они проще. Но все равно там есть некие проблемы. Есть конечно софт, который может вполне использоваться профессионалами "--- например, Ardour. Это супер"=крутая многодорожечная студия, и многие ее используют. Еще точно знаю, что используют: я был на концерте одного сэмпл"=гитариста, он использует Super Looper (есть под Linux, лицензия GPL). И он сказал, что это лучший Looper, который он только  видел вообще, потому что проприетарный софт не такой удобный. Вот, кстати, пример успешного использования профессионалами.

\q Теперь понятно, что ты имеешь в виду, когда говоришь, что ситуация меняется.

\a Меняется. И может быть есть что"=то еще. Я ведь говорю только то, что видел лично, о своем опыте, своем круге общения.

\q А если обобщить: Open Source и Эстония? Какова ситуация?

\a Смотря в каких сферах. По ощущениям "--- очень плотно. За последние годы стало много людей, которые разбираются в Linux, я стал замечать много людей, у кого на компьютерах появляется та же самая Ubuntu. Потому что это удобно, и в отличие от Windows, там очень много плюсов, а минусов не так много. В последнее время Linux стал дружелюбнее. Если говорить об Эстонии "--- я уже упоминал, что обслуживал сеть Интернет"=кафе, и все они были построены на LTSP. Очень много сетей было построено на использовании старого железа, для него это вторая жизнь. 

\q То есть технология использования тонких клиентов была популярна?

\a Да, была. Она популярна и сейчас в определенных сферах. Еще open source естественно использовался на полную катушку во время бума интернет"=провайдинга и домашних сетей. А еще могу сказать, что в библиотеках везде он используется. В национальной библиотеке стоит что"=то Ubuntu"=based, в образовательных учреждениях\ldots Кстати, неплохой пример распространенности Open Source в стране "--- наличие документации и драйверов под Linux для ID"=readera на официальном государственном сайте. ID"=reader "--- считывающее устройство для ID"=карты, идентификационного документа, внутри страны полностью заменившего паспорт, с помощью которого можно делать разновсяческие операции в интернете "--- интернет-банки, электронная подпись и так далее.

\q Если коротко назвать несколько основных преимуществ, за которые Linux, все ту же Ubuntu, начинают использовать простые пользователи?

\a Попытаюсь сформулировать. Я немножко не совсем простой пользователь. Но по ощущениям, мне кажется, некоторые начинают именно по идеологическим мотивам использовать. У меня есть такие знакомые.

\q Не айтишники?

\a Не айтишники, да, которые используют по идеологическим мотивам. Еще есть люди, которые используют, потому что это не требует лицензии, т.~е. бесплатный софт. Есть люди, которых привлекает то, что не требуется антивирус. И есть люди, которым нравится, что не надо искать и качать никакой дополнительный софт. И еще ситуация с драйверами: народу очень нравится ситуация с драйверами, потому что на большом количестве железа все работает из коробки, не надо ничего искать, обновлять, система все делает сама. Но это только в последнее время, еще 5 лет назад все было не так радужно.

\q А в чем идеология? Если это не айтишники?

\a Есть люди, которые понимают, что такое open source\ldots а есть люди, которые не хотят быть как все. 

\q Спасибо. Будем надеяться, что скоро это уже не будет поводом выделиться :)

\interview Лукаш Сверчевский (Л.~С.), Ломжа, Польша, студент Колледжа компьютерных наук и делового администрирования, Ломжа, Польша

\q Расскажи немного о себе, пожалуйста.

\a Начинаем с простых вопросов, да? Я из города Ломжа, это в 72 км на запад от Белостока, студент последнего курса бакалавриата, изучаю системы программирования,  по специальности Computer Engineering, операционные системы. 

\q Т.~е. специализируешься в области системного программирования?

\a Ну, с программированием я столкнулся раньше. До университета я получил еще диплом техника по информационным технологиям. Там было немного программирования, но меня тогда оно мало привлекало. А уже на первом курсе вдруг оказалось, что я совсем неплох в этом. Увлекся математикой, алгоритмами\ldots

\q Что было твоим первым опытом с open source?

\a Первый случай, когда я всерьез имел дело со свободным ПО "--- это система BOINC (Berkeley Open Infrastructure for Network Computing). 

\q Пару слов об этой системе?

\a Это система для распределенных вычислений, разработанная в Университете Беркли, в Калифорнии. Так вышло, что когда я начал программировать, то почти сразу столкнулся с достаточно сложной задачей из области дискретной математики, теории чисел\ldots получение простых чисел, это связано с криптографией. Требовалось выполнить как можно больше вычислений за как можно меньшее время, и я заинтересовался платформой BOINC, потому что она позволяет распределить задания на большое число компьютеров, подключенных к Интернет. Два года назад на конференции во Львове у меня был доклад об этой платформе "--- и пожалуй это был для меня первый серьезный случай.

\q А до этого тебе доводилось использовать какой"=либо дистрибутив Linux или что"=нибудь еще, как простому пользователю? Или ты начал сразу со специального программного обеспечения для вычислений?

\a Признаться, в этом я достаточно старомодный :) Еще несколько лет назад пользовался Windows. Позже нашел в Интернет "--- так, из любопытства "--- что если зарегистрироваться на сайте, то тебе вышлют дистрибутивы Ubuntu. Подумал: <<Ладно, зарегистрируюсь. Если действительно пришлют "--- установлю ее>>. И правда, через три месяца пришла посылка, кажется из Голландии, так и установил свой первый Linux. С тех пор программирую только на Linux "--- окружение там гораздо лучше подходит. Компилятор, свободная интегрированная среда "--- я пользуюсь Code::Blocks, достаточно широко известная IDE, доступна и на Windows. Но часто ограничиваюсь каким"=нибудь простым редактором, типа gedit, и компиляцией в консоли. Я привязан к консоли, потому что когда пользуюсь удаленным входом через ssh "--- например, из университета "--- то у меня есть только консоль, и я делаю все через нее. Не пользуюсь NetBeans или какими"=то другими «навороченными» инструментами. Ищу более простых и быстрых решений.

\q Расскажи немного о своем вузе?

\noindent Это высшее учебное заведение в городе Ломжа, очень молодое, существует около восьми лет. 

\q Какое"=нибудь свободное ПО у вас используется?

\a Должен сказать, не слишком много. Только на компьютерах в библиотеки есть Linux. А в лабораториях и учебных классах "--- Windows. Студентов учат на старой среде программирования под С++, которая уже некоторое время не развивается. В других вузах Польши "--- например, в университете Марии Склодовской"=Кюри в Люблине "--- там можно найти на компьютерах установленные и Windows и Linux, и пользователь может выбрать, что загружать. У нас в Ломже, однако, до этого еще не дошло.

\q Может через какое"=то время доля свободного ПО на компьютерах увеличится. А какова причина использования Ubuntu или какого"=то другого дистрибутива Linux в библиотеке?

\a Хороший вопрос! По"=моему просто для снижения стоимости компьютеров в библиотечных залах, чтобы не покупать лицензии на Windows. Туда приходят студенты с первого курса "--- не только с информатики, есть ведь и другие направления "--- бизнес, уход, какие"=то основы медицины тоже изучаются "--- и у них бывают некоторые проблемы. Они пользовались только Windows 7, а тут совсем другое окружение, Linux. Удивляются, конечно "--- просто тому, что\ldots оказывается, есть что"=то другое. 

\q У них бывают серьезные сложности?

\a Не могу сказать, я вообще"=то не вращаюсь в этой среде. Но мне доводилось помогать с подготовкой дипломных работ студентам"=инженерам "--- не с информатики, а с других специальностей. У них проблемы с оформлением работы в Word, что уже говорить о \TeX. И с Linux у них тоже проблемы. Но это не из"=за того, что Linux неинтуитивен. Вообще, проводилось даже такое исследование: было взято какое"=то число детей, которых обучили Windows, и какое"=то, которых обучали Linux, а позже их поменяли местами. Те, что учились Linux, начали пользоваться Windows, а тех, что пользовались Windows, пересадили на Linux. Обе группы имели одинаковые проблемы. Так что проблемы наших студентов "--- просто это оттого, что дома у них Windows. 

\q И последнее: как в Польше обстоит дело с мероприятиями, посвященными свободному ПО?

\a Есть вообще"=то несколько конференций, которые можно считать связанными со свободным ПО, в частности с программированием под свободные платформы. Не знаю почему, но был дважды на Украине, потом здесь, а в Польше ни на одной свободной конференции не был. Собирался выбраться на студенческий фестиваль информатики "--- но не сложилось по срокам. А так, есть еще несколько отдельных мероприятий по Питону, PHP. То есть более узкой тематики.

\end{document}


