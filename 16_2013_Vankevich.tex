\documentclass[10pt, a5paper]{article}
\usepackage{pdfpages}
\usepackage{parallel}
\usepackage[T2A]{fontenc}
\usepackage{ucs}
\usepackage[utf8x]{inputenc}
\usepackage[polish,english,russian]{babel}
\usepackage{hyperref}
\usepackage{rotating}
\usepackage[inner=2cm,top=1.8cm,outer=2cm,bottom=2.3cm,nohead]{geometry}
\usepackage{listings}
\usepackage{graphicx}
\usepackage{wrapfig}
\usepackage{longtable}
\usepackage{indentfirst}
\usepackage{array}
\newcolumntype{P}[1]{>{\raggedright\arraybackslash}p{#1}}
\frenchspacing
\usepackage{fixltx2e} %text sub- and superscripts
\usepackage{icomma} % коскі ў матэматычным рэжыме
\PreloadUnicodePage{4}

\newcommand{\longpage}{\enlargethispage{\baselineskip}}
\newcommand{\shortpage}{\enlargethispage{-\baselineskip}}

\def\switchlang#1{\expandafter\csname switchlang#1\endcsname}
\def\switchlangbe{
\let\saverefname=\refname%
\def\refname{Літаратура}%
\def\figurename{Іл.}%
}
\def\switchlangen{
\let\saverefname=\refname%
\def\refname{References}%
\def\figurename{Fig.}%
}
\def\switchlangru{
\let\saverefname=\refname%
\let\savefigurename=\figurename%
\def\refname{Литература}%
\def\figurename{Рис.}%
}

\hyphenation{admi-ni-stra-tive}
\hyphenation{ex-pe-ri-ence}
\hyphenation{fle-xi-bi-li-ty}
\hyphenation{Py-thon}
\hyphenation{ma-the-ma-ti-cal}
\hyphenation{re-ported}
\hyphenation{imp-le-menta-tions}
\hyphenation{pro-vides}
\hyphenation{en-gi-neering}
\hyphenation{com-pa-ti-bi-li-ty}
\hyphenation{im-pos-sible}
\hyphenation{desk-top}
\hyphenation{elec-tro-nic}
\hyphenation{com-pa-ny}
\hyphenation{de-ve-lop-ment}
\hyphenation{de-ve-loping}
\hyphenation{de-ve-lop}
\hyphenation{da-ta-ba-se}
\hyphenation{plat-forms}
\hyphenation{or-ga-ni-za-tion}
\hyphenation{pro-gramming}
\hyphenation{in-stru-ments}
\hyphenation{Li-nux}
\hyphenation{sour-ce}
\hyphenation{en-vi-ron-ment}
\hyphenation{Te-le-pathy}
\hyphenation{Li-nux-ov-ka}
\hyphenation{Open-BSD}
\hyphenation{Free-BSD}
\hyphenation{men-ti-on-ed}
\hyphenation{app-li-ca-tion}

\def\progref!#1!{\texttt{#1}}
\renewcommand{\arraystretch}{2} %Іначай формулы ў матрыцы зліпаюцца з лініямі
\usepackage{array}

\def\interview #1 (#2), #3, #4, #5\par{

\section[#1, #3, #4]{#1 -- #3, #4}
\def\qname{LVEE}
\def\aname{#1}
\def\q ##1\par{{\noindent \bf \qname: ##1 }\par}
\def\a{{\noindent \bf \aname: } \def\qname{L}\def\aname{#2}}
}

\def\interview* #1 (#2), #3, #4, #5\par{

\section*{#1\\{\small\rm #3, #4. #5}}

\def\qname{LVEE}
\def\aname{#1}
\def\q ##1\par{{\noindent \bf \qname: ##1 }\par}
\def\a{{\noindent \bf \aname: } \def\qname{L}\def\aname{#2}}
}


\begin{document}

\title{Эра post-PC в ВУЗах}%\footnote{Текст данных и последующих тезисов, кроме специально оговоренных случаев, доступен под лицензией Creative Commons Attribution"=ShareAlike 3.0}

\author{Дмитро Ванькевич\footnote{Львов, Украина; \url{dvankevich@gmail.com}}}
\maketitle

\begin{abstract}
Prognoses of Post"=PC era coming are discussed and some real statistics, to determine features of future computer"=based \linebreak workflow in higher education institutions. The «bring"=your"=own"=device» concept is reviewed as a possible infrastructure \linebreak modernization approach.
\end{abstract}

Про наступление эры post"=PC говорят с конца 90"=х годов XX"=века. Ведущие идеологи ИТ того времени "--- Лу Герстнер, Ларри Эллисон, Стив Джобс "--- предрекали скорый закат традиционным ПК. Но уже в 2000"=х годах сторонники <<концепции PostPC>> не отрицали, что компьютеры в смысле вычислительных и управляющих устройств не исчезнут. Имелся в виду лишь конец эры персональных компьютеров, в их Wintel"=воплощении\cite{Vank1}, их доминирования как в смысле технологий, так и (в первую очередь!) в смысле рынка \cite{Vank2}.

Исследования Paul Marsden для конференции Digital Innovation Day 2013 подтверждает правоту предыдущего утверждения\cite{Vank3}. Если в 1998--2005 платформа Wintel доминировала (96\%), то на конец 2012 года её доля составила около 35\%. Будучи одним из ведущих производителей материнских плат, фирма Intel, заявила о прекращении выпуска этого вида продукции \cite{Vank4}. Но несмотря на снижение доли ПК среди всех вычислительных устройств, говорить про их уход с рынка ещё очень рано. Скорее всего произойдёт переопределение термина ПК. Так например на смену традиционным настольным ПК Intel предлагает <<Next Unit of Computing>> "--- ультракомпактный системный блок размером $12\times11\times4$ см на основе процессоров Core i3, i5 \cite{Vank5}.

Скорее можно утверждать, что эра post"=PC "--- это эра многообразия форм компьютерных систем, в которой произойдёт освобождение рабочего места от привязки к конкретному ПК, стоящему на столе. При помощи виртуализации десктопов возможно превратить рабочее место в виртуальное, доступное с любого клиентского устройства \cite{Vank6}.  Виртуализация рабочих мест позволит использовать концепцию BYOD (англ. «bring"=your"=own"=device» "--- «принеси собственное устройство»), которая получила широкое распространение в течение последних нескольких лет. Она подразуемевает использование студентами собственных клиентских устройств в учебных целях. К 2017 году, по прогнозу Gartner, каждый второй работник будет пользоваться собственными гаджетами в рабочих целях \cite{Vank7}. И уже сейчас большинство студентов ВУЗов имеют в своём распоряжении ноутбуки, которые по своим параметрам зачастую превосходят находящиеся в учебном классе ПК. Это стоит учитывать при построеннии ИТ инфраструктуры ВУЗа. В частности, учитывая характерный для вузов развитый компьютерный парк и достаточно мощное подразделение по его обслуживанию, имеет смысл вкладывать ресурсы не столько в обновление ПК, сколько в построение частного облака, предоставляющего студентам необходимые приложения в качестве сервисов.

При формировании комплекса сервисов для ВУЗовской сети и переносе локальных задач в частное облако просматриваются два подхода. Первый заключается в применении технологии web desktops "--- окружений рабочего стола, написанных на основе веб"=технологий (преимущественно, PHP), запускаемых на сервере и взаимодействующих с пользователем через веб"=браузер \cite{Vank8}. Преимуществом такого подхода является нетребовательность к ресурсам, а недостатком "--- ограниченные возможности существующих систем этого типа, в особенности распространяемых под свободными лицензиями. Т.о. подход может быть рекомендован к использованию только совместно с альтернативными решениями.

Альтернативой интранет"=технологиям является использование виртуализации. Помимо поднятия сервера виртуальных машин с нуля, для построения частного облака можно использовать одно из готовых решений. В частности, одно из таких решений "--- дистрибутив Proxmox Virtual Environment \cite{Vank9} \cite{Vank10} "--- успешно использовался для создания учебного полигона при проведении лабораторных работ в рамках курса <<Системное администрирование ОС Linux>> во Львовском национальном университете имени Ивана Франко\cite{Vank11}.



\begin{thebibliography}{9}
\bibitem{Vank1} Wintel "--- маркетинговый термин, сокращение, образованное слиянием слов Windows и Intel, которое обозначает персональный компьютер, использующий центральный процессор с x86"=совместимой микроархитектурой и операционную систему семейства Microsoft Windows. \url{http://ru.wikipedia.org/wiki/Wintel}
\bibitem{Vank2} <<Еще раз о PostPC>> журнал PC Magazine RE август 2000г. \url{http://www.pcmag.ru/issues/detail.php?ID=6682}
\bibitem{Vank3} <<Post"=PC Era by the Numbers: List of Top Post"=PC Stats>> \url{http://socialcommercetoday.com/post-pc-era-by-the-numbers-list-of-top-post-pc-stats/}
\bibitem{Vank4} <<Intel Will Exit Desktop Motherboard Market After Haswell is Released>> \url{http://www.legitreviews.com/news/14994/}
\bibitem{Vank5} <<Intel® NUC>> \url{http://www.intel.com/content/www/us/en/motherboards/desktop-motherboards/nuc.html}
\bibitem{Vank6} <<ПК умер "--- да здравствует ПК!>> «Открытые системы», № 09, 2011 \url{http://www.osp.ru/os/2011/09/13011556/}
\bibitem{Vank7} <<Gartner прогнозирует бум BYOD в 2016 году>> \url{http://www.therunet.com/news/905-gartner-prognoziruet-bum-byod-v-2016-godu}
\bibitem{Vank8} <<Web desktop>> \url{http://en.wikipedia.org/wiki/Web\_desktop}
\bibitem{Vank9} \url{http://pve.proxmox.com}
\bibitem{Vank10} Д.Є.Ванькевич, Г.Г.Злобін. Використання приватної хмари на базі дистрибутиву PROXMOXVE в навчальному процесі, Хмарні технології в освіті: матеріали Всеукраїнського науково"=методичного Інтернет"=семінару (Кривий Ріг "--- Київ "--- Черкаси "--- Харків, 21 грудня 2012р.). "--- Кривий Ріг: Видавничий відділ КМІ, 2012. "--- 173 с.
\bibitem{Vank11} Д.Є.Ванькевич. Навчальний полігон на базі дистрибутиву PROXMOXVE для проведення лабораторних робіт з курсу <<Системне адміністрування ОС LINUX>>, Теорія та методика електронного навчання: збірник наукових праць. Випуск IV. "--- Кривий Ріг: Видавничий відділ КМІ, 2013. "--- 311 с.
\end{thebibliography}
\end{document}




