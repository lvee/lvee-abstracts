\documentclass[10pt, a5paper]{article}
\usepackage{pdfpages}
\usepackage{parallel}
\usepackage[T2A]{fontenc}
\usepackage{ucs}
\usepackage[utf8x]{inputenc}
\usepackage[polish,english,russian]{babel}
\usepackage{hyperref}
\usepackage{rotating}
\usepackage[inner=2cm,top=1.8cm,outer=2cm,bottom=2.3cm,nohead]{geometry}
\usepackage{listings}
\usepackage{graphicx}
\usepackage{wrapfig}
\usepackage{longtable}
\usepackage{indentfirst}
\usepackage{array}
\newcolumntype{P}[1]{>{\raggedright\arraybackslash}p{#1}}
\frenchspacing
\usepackage{fixltx2e} %text sub- and superscripts
\usepackage{icomma} % коскі ў матэматычным рэжыме
\PreloadUnicodePage{4}

\newcommand{\longpage}{\enlargethispage{\baselineskip}}
\newcommand{\shortpage}{\enlargethispage{-\baselineskip}}

\def\switchlang#1{\expandafter\csname switchlang#1\endcsname}
\def\switchlangbe{
\let\saverefname=\refname%
\def\refname{Літаратура}%
\def\figurename{Іл.}%
}
\def\switchlangen{
\let\saverefname=\refname%
\def\refname{References}%
\def\figurename{Fig.}%
}
\def\switchlangru{
\let\saverefname=\refname%
\let\savefigurename=\figurename%
\def\refname{Литература}%
\def\figurename{Рис.}%
}

\hyphenation{admi-ni-stra-tive}
\hyphenation{ex-pe-ri-ence}
\hyphenation{fle-xi-bi-li-ty}
\hyphenation{Py-thon}
\hyphenation{ma-the-ma-ti-cal}
\hyphenation{re-ported}
\hyphenation{imp-le-menta-tions}
\hyphenation{pro-vides}
\hyphenation{en-gi-neering}
\hyphenation{com-pa-ti-bi-li-ty}
\hyphenation{im-pos-sible}
\hyphenation{desk-top}
\hyphenation{elec-tro-nic}
\hyphenation{com-pa-ny}
\hyphenation{de-ve-lop-ment}
\hyphenation{de-ve-loping}
\hyphenation{de-ve-lop}
\hyphenation{da-ta-ba-se}
\hyphenation{plat-forms}
\hyphenation{or-ga-ni-za-tion}
\hyphenation{pro-gramming}
\hyphenation{in-stru-ments}
\hyphenation{Li-nux}
\hyphenation{sour-ce}
\hyphenation{en-vi-ron-ment}
\hyphenation{Te-le-pathy}
\hyphenation{Li-nux-ov-ka}
\hyphenation{Open-BSD}
\hyphenation{Free-BSD}
\hyphenation{men-ti-on-ed}
\hyphenation{app-li-ca-tion}

\def\progref!#1!{\texttt{#1}}
\renewcommand{\arraystretch}{2} %Іначай формулы ў матрыцы зліпаюцца з лініямі
\usepackage{array}

\def\interview #1 (#2), #3, #4, #5\par{

\section[#1, #3, #4]{#1 -- #3, #4}
\def\qname{LVEE}
\def\aname{#1}
\def\q ##1\par{{\noindent \bf \qname: ##1 }\par}
\def\a{{\noindent \bf \aname: } \def\qname{L}\def\aname{#2}}
}

\def\interview* #1 (#2), #3, #4, #5\par{

\section*{#1\\{\small\rm #3, #4. #5}}

\def\qname{LVEE}
\def\aname{#1}
\def\q ##1\par{{\noindent \bf \qname: ##1 }\par}
\def\a{{\noindent \bf \aname: } \def\qname{L}\def\aname{#2}}
}

\begin{document}
\title{Постобработка фотографий в программе darktable\footnote{\url{vle@gmx.net}, \url{https://lvee.org/en/abstracts/292}}}
\author{Aleksey Cheusov, Minsk, Belarus}
\maketitle
\begin{abstract}
Darktable is an open source application devoted to processing of RAW
files. The program can manage collections of RAWs with rating, color
labels and custom tags. A rich set of built-in filters, used at
processing, store their settings in a form of a history stack, which
is saved alongside original RAW, providing original RAW
untouched.
\end{abstract}
Программа обработки графических изображений darktabkle уже поднималась
на LVEE в 2011-м году. Поэтому настоящий доклад не столько посвящен
описанию его возможностей, а модулей в системе несколько десятков,
сколько представляет собой набор примеров того, в каких ситуациях
какой модуль или какие модули можно использовать и каким
образом. Привести исчерпывающие примеры для всех модулей и ситуаций
вряд ли возможно, ввиду огромного их количества, но я попытаюсь кратко
описать наиболее типичные ситуации.

\subsection*{Маски}

Все программы для обработки фото имеют возможность наложить какой-либо
фильтр или эффект на часть фотографии. Например, обрабатываемая
область может быть нарисована вручную кистью или ограничена
многоугольником. Такая же возможность есть и в darktable. Но я хотел
бы особенно отметить наличие так называемых parametric masks. С их
помощью можно задать область фотографии, ограниченную данными в
каналах Lightness, a и b в цветовом пространстве Lab, а также Chroma и
hue в цветовом пространстве LCh. В совокупности с возможностью
выполнения логических операций над масками это дает очень мощные
возможности по точной <<прорисовке>> области применения того или иного
эффекта.

\subsection*{Коррекция экспозиции и контрастность изображения}

Иногда бывает так, что при сьемке неправильно выставлена
экспозиция. Таким образом снимок получается частично
переэкспонированным или недоэкспонированным. Это также происходит в
случаях, когда динамического диапазона камеры не хватает для того,
чтобы полностью передать все детали снимаемого объекта, например,
яркого солнечного неба и глубоких теней. Самым простым способом
исправить это является модуль exposition. Что касается контрастности,
то здесь можно использовать модули levels, tone curve, contrast
brightness saturation, zone systems и shadows and highlights. Лично я
практически никогда не использую модули contrast brightness saturation
и shadows and highlights для этих целей, поскольку модули levels, zone
systems и tone curve предоставляют гораздо больше возможностей для
этого. Недостаток контраста возникает почти всегда при сЪемке, например,
<<типичной белорусской зимой>>, когда картинка по большому счету
представляет собой <<серое на сером>>. В этом случае необходимо
выставить точки белого и черного в модуле levels. Честно говоря, этот
модуль я использую практически всегда. Именно с него начинается
обработка практически любого снимка. Настройка точки белого и черного
также помогает избавиться от лишних деталей в тенях и свете, и придает
фотографии большую драматичность или концентрирует внимание на
основном объекте съемки. Модули tone curve и zone systems очень похожи
по функциональности, но я предпочитаю zone systems для обработки
черно-белых фотографий.

\subsection*{Кадрирование, выравнивание горизонта и коррекция геометрии}

Практически все кадры после сьемки нуждаются в кадрировании. За это
отвечает модуль crop and rotate. Этот же модуль можно использоать для
исправления <<заваленного>> горизонта. При съемке высотных зданий,
особенно с близкого расстояния, возникают геометрические искажения,
что так же можно исправить этим модулем или модулем perspective
correction.

\subsection*{Цветокоррекция, насыщенность}

Для придания фотографии нужной атмосферы можно не только
корректировать освещенность (lightness в цветовой схеме Lab), но и
влиять на цвета. Это можно сделать с помощью модулей white balance,
velvia, vibrance, color zones, contrast brightness saturation а также
color contrast, color mapping, color reconstruction и colorize. Так же
как и в fotoshop (PS), darktable позволяет изменять цвета по всему
цветовому кругу, что дает возможность создать монохромную фотографию
или оставить только хорошо сочетающиеся друг с другом цвета.

\subsection*{Шумоподавление, ХА, увеличение резкости, микроконтраст и т.п.}

Любая камера, особенно при повышенных ISO, добавляет шумы в кадр. Их
количество можно уменьшить с использованием модулей raw denoise,
denoise (profiled), denoise (bilateral filter) и denoise (non-local
means).  Хроматические аберрации можно убрать с помощью модулей
chromatic aberrations и defringe.  Скорректировать дисторсию, вносимую
объективом, можно исправить модулем lens correction. В этом случае
используется постоянно пополняемая база данных
объективов. Виньетирование, вносимое объективами на открытых
диафрагмах, можно исправить с помощью модуля vignetting. С его же
помощью можно его добавить, например, для портретов.

\subsection*{Обработка лица и кожи}

Часто возникает необходимость ретушировать кожу человека на снимках и
убрать недостатки, видимые на фотографии, особенно на снимках с
лицевым портретом. Для этого можно воспользоваться модулями spot
removal и equalizer. Equalizer может так же использоваться для
увеличения микро- и макро-контраста на уровнях света и цвета. Для
улучшения резкости изображения можно поспользоваться модулем sharpen.

\subsection*{Черно-белая фотография}

Кажется, черно-белая фотография никогда не потеряет своей
актуальности. В darktable есть много способов реализовать это, но я
предпочитаю два модуля: monochrome и color zones. Последний дает
возможность преобразовать различные цвета в разные оттенки серого.

\subsection*{Другие возможности}

Есть ряд других модулей, с помощью которых можно добавить подпись к
фотографии (watermark), добавить зернистости по аналогии с пленочной
фотографией (grain), поместить фотографию в рамочку (framing) и многое
другое.

\subsection*{Выводы}

На мой взгляд, darktable представляет собой профессиональный
инструмент для постобработки фотографий, по совокупности факторов не
уступающий по свои возможностям, таким известным программам как
photoshop, lightroom и capture one. Лично мне не хватает разьве что
поканальных RGB кривых для получения <<чистого>> черного цвета и более
богатых средств ретуширования.

\end{document}
