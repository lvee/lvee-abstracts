\documentclass[10pt, a5paper]{article}
\usepackage{ucs}
\usepackage[utf8]{inputenc}
\usepackage[T2A]{fontenc}
\usepackage[english, russian]{babel}
\usepackage{hyperref}
\usepackage{geometry}
\frenchspacing
\begin{document}
\title{Голос спонсора: SaM Solutions}
%\author{}
\date{}
\maketitle

Компания SaM Solutions выступает в роли системо-образующего спонсора конференции Linux Vacation Eastern Europe, с момента зарождения LVEE в 2005 году и на протяжении всех лет её проведения. 

Сложившаяся корпоративная практика не случайна. Продукты и решения, задействующие Linux и другие Free/Open Source Software проекты, составляют заметную часть пакета разработок SaM Solutions. Кадровая политика компании направлена на поощрение профессионального развития своих сотрудников, организацию их эффективного отдыха и привлечение хорошо мотивированных кандидатов к работе на компанию. Формат конференции LVEE успешно позволяет решать все три задачи. 

Одним из направлений нашей деятельности является направление Linux/Unix разработок. Специалисты компании на протяжении десятилетий работают с СПО. Компанией реализован ряд проектов по адаптации ОС GNU/Linux для работы в различных устройствах, построенных на таких платформах как ARM, PowerPC, x86, MIPS. В последние годы - на ведущие позиции выходит разработка управляющего ПО для серверов Enterprise-класса, от низкоуровнего BMC Firmware на основе Linux до высокоуровневых систем контроля виртуализации и графических интерфейсов управления. Надёжность, качество и широкий функционал множества свободных проектов позволяет строить нам системы любого уровня и сложности, опираясь на высококачественные готовые компоненты.

Мы разрабатываем, модифицируем и адаптируем различное свободное програмное обеспечение для наших заказчиков, но не забываем о нём и для своих собственных нужд - наши сотрудники используют в своей работе существующие програмные продукты и вносят вклад в их развитие. Часть внутренней инфраструктуры, а именно интранет-сеть компании, тестовые стенды отдела контроля качества, рабочие места сотрудников профильных подразделений - также работает под управлением СПО (серверные и десктопные платформы GNU/Linux и FreeBSD).

По указанным причинам компания заинтересована в развитии движения open source, в продвижении свободных программных продуктов, и участие в конференции разработчиков и пользователей СПО.

В рамках Unix/Linux направления успешно выполнены проекты для таких знаковых заказчиков, как  Novell/SUSE, Fujitsu Technology Solutions  и осуществляется партнёрство с компаниями IBM и Oracle/Sun в области Open Source решений.

Линукс-отдел компании SaM Solutions (Dept6/DDC3) уже более десяти лет является настоящей неофициальной кузницей кадров для специалистов в области Free/Open Source Software (FOSS). Код наших сотрудников есть в таких известных свободных проектах как Linux Kernel, Debian GNU/Linux, Alt Linux, Samba, wxWidgets, языке Ruby и его библиотеках и многих других. И SaM Solutions продолжает двигаться вперёд, не останавливаясь на достигнутом.

\end{document}


