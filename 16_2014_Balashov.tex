\documentclass[10pt, a5paper]{article}
\usepackage{pdfpages}
\usepackage{parallel}
\usepackage[T2A]{fontenc}
\usepackage{ucs}
\usepackage[utf8x]{inputenc}
\usepackage[polish,english,russian]{babel}
\usepackage{hyperref}
\usepackage{rotating}
\usepackage[inner=2cm,top=1.8cm,outer=2cm,bottom=2.3cm,nohead]{geometry}
\usepackage{listings}
\usepackage{graphicx}
\usepackage{wrapfig}
\usepackage{longtable}
\usepackage{indentfirst}
\usepackage{array}
\newcolumntype{P}[1]{>{\raggedright\arraybackslash}p{#1}}
\frenchspacing
\usepackage{fixltx2e} %text sub- and superscripts
\usepackage{icomma} % коскі ў матэматычным рэжыме
\PreloadUnicodePage{4}

\newcommand{\longpage}{\enlargethispage{\baselineskip}}
\newcommand{\shortpage}{\enlargethispage{-\baselineskip}}

\def\switchlang#1{\expandafter\csname switchlang#1\endcsname}
\def\switchlangbe{
\let\saverefname=\refname%
\def\refname{Літаратура}%
\def\figurename{Іл.}%
}
\def\switchlangen{
\let\saverefname=\refname%
\def\refname{References}%
\def\figurename{Fig.}%
}
\def\switchlangru{
\let\saverefname=\refname%
\let\savefigurename=\figurename%
\def\refname{Литература}%
\def\figurename{Рис.}%
}

\hyphenation{admi-ni-stra-tive}
\hyphenation{ex-pe-ri-ence}
\hyphenation{fle-xi-bi-li-ty}
\hyphenation{Py-thon}
\hyphenation{ma-the-ma-ti-cal}
\hyphenation{re-ported}
\hyphenation{imp-le-menta-tions}
\hyphenation{pro-vides}
\hyphenation{en-gi-neering}
\hyphenation{com-pa-ti-bi-li-ty}
\hyphenation{im-pos-sible}
\hyphenation{desk-top}
\hyphenation{elec-tro-nic}
\hyphenation{com-pa-ny}
\hyphenation{de-ve-lop-ment}
\hyphenation{de-ve-loping}
\hyphenation{de-ve-lop}
\hyphenation{da-ta-ba-se}
\hyphenation{plat-forms}
\hyphenation{or-ga-ni-za-tion}
\hyphenation{pro-gramming}
\hyphenation{in-stru-ments}
\hyphenation{Li-nux}
\hyphenation{sour-ce}
\hyphenation{en-vi-ron-ment}
\hyphenation{Te-le-pathy}
\hyphenation{Li-nux-ov-ka}
\hyphenation{Open-BSD}
\hyphenation{Free-BSD}
\hyphenation{men-ti-on-ed}
\hyphenation{app-li-ca-tion}

\def\progref!#1!{\texttt{#1}}
\renewcommand{\arraystretch}{2} %Іначай формулы ў матрыцы зліпаюцца з лініямі
\usepackage{array}

\def\interview #1 (#2), #3, #4, #5\par{

\section[#1, #3, #4]{#1 -- #3, #4}
\def\qname{LVEE}
\def\aname{#1}
\def\q ##1\par{{\noindent \bf \qname: ##1 }\par}
\def\a{{\noindent \bf \aname: } \def\qname{L}\def\aname{#2}}
}

\def\interview* #1 (#2), #3, #4, #5\par{

\section*{#1\\{\small\rm #3, #4. #5}}

\def\qname{LVEE}
\def\aname{#1}
\def\q ##1\par{{\noindent \bf \qname: ##1 }\par}
\def\a{{\noindent \bf \aname: } \def\qname{L}\def\aname{#2}}
}

\begin{document}
\title{Гарантированное уничтожение информации}
\author{Виталий Балашов, Харьковский НИИ \\ судебных экспертиз, Харьков, Украина\footnote{\url{vitaly.balashov@gmail.com}, \url{http://lvee.org/ru/abstracts/141}}}
\maketitle
\begin{abstract}
Today HDDs help people to save huge volumes of information.  People often use already used HDDs to save money. If the infor\-mation stored in the HDD should not be accessible to a new HDD owner (or other persons), it should be utilized by special methods. Use of a special utilization method is important for information security, because a lot of filesystems actually don`t wipe information, but only remove (delete) it. Information that was not wiped, is always at risk of recovery.  
\end{abstract}
Человек издавна старался сохранять информацию. Использовал для этого все возможные способы и всё время их совершенствовал в соответствии со своими потребностями в коммуникации. Рисунок на скале в пещере не поддавался транспортировке и человек придумал использовать различные каменные дощечки, которые уже легко перенести и показать своему собрату. Но дощечки тяжелые и неудобные, поэтому человек решил использовать более легкие варианты "--- из дерева или аналогичного материала.  И весь этот долгий процесс эволюции носителей информации привел нас к широко распространенным на сегодняшний день способам хранения информации с помощью фиксации состояния магнитного поля  и дальнейшего его считывания.  К сожалению, в таких носителях человек уже не может самостоятельно считывать информацию, обязательным становится использование устройства, умеющего интерпретировать магнитное поле в информацию, воспринимаемую человеком. Но эта жертва простительна, учитывая современные требования к уровню коммуникации.

Мы научились хранить огромные объемы самых разнообразнейших типов информации десятками лет и считывать их практически в режиме реального времени, находясь при этом на другом континенте, а то и за пределами планеты.

Оперируя носителями информации, человек всегда хотел иметь возможность ее гарантированного уничтожения. И если с краской на скале, березовой берестой, бумагой и другими носителями, хранящими информацию в виде, доступном прямому восприятию человеком, все достаточно просто, то с магнитными и транзисторными накопителями возникает вопрос: а действительно ли информация полностью и безвозвратно уничтожена?

Практически все существующие файловые системы не уничтожают информацию фактически, когда ее уничтожают пользователи  штатными средствами. Это позволяет добиться более высоких показателей производительности носителя и продления срока его службы. Информация не уничтожена, а удалена от пользователя, ему более не доступна, результат достигнут. Когда, с технической точки зрения, разрушить эту информацию будет целесообразным, тогда она и будет уничтожена. Бросить что"=либо как есть всегда дешевле, чем проводить утилизацию (а если в ней нет критической необходимости, то и вообще не целесообразно).

Само собой, этот эффект имеет и обратную сторону.  Информация, которая не должна попасть после ее удаления пользователем в руки третьих лиц, остается доступной для восстановления и использования.

Среди решений данной проблемы различают два основных метода гарантированного уничтожения информации:
\begin{enumerate}
      \item Физическое уничтожение носителя;
      \item Гарантированное уничтожение с сохранением работоспособности носителя и возможностью дальнейшей его эксплуатации.
\end{enumerate}

Первый метод не нуждается в комментариях, носитель просто механически или химически разрушают (например, разбивают диски НЖМД молотком, используют мощный электромагнитный импульс и т. п.).

Рассмотрим подробнее метод №2. При больших количествах носителей, хранящих информацию, подлежащую уничтожению, их разрушение становится экономически накладным (пусть даже \linebreak оправданным). Стоимость НЖМД позволяет уничтожать их большими объемами разве что некоторым очень хорошо спонсируемым государственным ведомствам, например, военным или правоохранительным, но даже военные ведомства далеко не в каждой стране могут позволить себе подобное расточительство.

Гарантированно уничтожить информацию, не хуже чем при физическом уничтожении, позволяет ее перезапись. На перезаписи базируются все существующие способы уничтожения информации с сохранением работоспособности носителя.

Но даже перезапись иногда может не спасти от полного уничтожения данных. К примеру, если данные были записаны в секторах, которые в последствии были помечены как сбойные, не являясь таковыми по сути, то контроллер НЖМД не допустит его перезаписи, т.~к. уже считает сектор сбойным и не использует в работе.

Иным способом восстановления, лишь частично предотвращаемым с помощью перезаписи, является считывание данных между дорожками разметки НЖМД. Запись в этих пространствах появляется за счет рассеивания магнитного поля в зазоре между записывающей головкой и поверхностью диска. Когда ширина поля рассеивания становится больше ширины дорожки разметки, появляется теоретическая возможность считывания сигналов, сохранившихся на поверхности диска между дорожками.

Если же восстановление информации из перемещенных (bad) секторов может дать полезный эффект, в случае, когда этих секторов много и они подлежат считыванию, то восстановление из междорожечного пространства, учитывая плотность размещения дорожек в современных НЖМД, носит более теоретический характер.

Другой фактор, который является больше теорией "--- подход с использованием остаточной намагниченности магнитных доменов. Суть этого метода заключается в том, что уровень намагниченности не всегда соответствует уровню нуля или уровню единицы. Грубо говоря, если в магнитном домене содержался уровень 0 и был перезаписан уровнем единицы, то по факту уровень намагниченности будет равен, к примеру, 0,75. Или же наоборот, после содержания единицы, перезаписанной нулем, в домене уровень намагниченности будет ближе к 0,25. Контроллером НЖМД такие уровни будут округляться и декодироваться как 0 и 1, но если считать эти уровни в обход контроллера, например, с помощью магнитного силового микроскопа, то можно интерпретировать их по"=своему: 0,25 воспринимать как 1, а 0,75 как 0. Теоретически, в результате должна получиться информация, содержащаяся на данном участке накопителя до его перезаписи. Также, теоретически, есть вероятность восстановления сигнала вплоть до нескольких перезаписей, к примеру, если сигналы имели различные характеристики частоты магнитного поля.

На данный момент такой подход имеет достаточное количество как теоретических, так и практических проблем, и остается теорией, имеющей право на жизнь. Реально работающие устройства либо компании, реально гарантирующие восстановление данных после перезаписи, пока неизвестны. Сам же Питер Гутман, разработчик одного из наиболее эффективных методов гарантированного уничтожения данных с помощью перезаписи, утверждает, что у спецслужб такие устройства есть. Спецслужбы в свою очередь не дают подтверждения, но и, как правило, их процедуры безопасности данных считают перезаписанный диск ненадежным. К примеру, в США жесткие диски, на которых хранилась информация с грифом «СОВЕРШЕННО СЕКРЕТНО», подлежат только размагничиванию или физическому уничтожению (что по сути почти одно и тоже).

Питер Гутман предложил метод, охватывающий все возможные НЖМД, а также предусматривающий эффективное уничтожение данных с НЖМД, использующих MFM и RLL кодирование, для которых были предназначены 27 проходов. В силу тех факторов, что пользователь, как правило, не знает, какое магнитное кодирование используется в утилизируемом НЖМД, метод Гутмана предусматривает в сумме 35 проходов перезаписи. Каждый проход записывает различные шаблоны данных в каждый байт каждого сектора, 8 из которых используют в качестве шаблонов случайные последовательности. При точном знании метода магнитного кодирования НЖМД, количество проходов перезаписи можно сократить без потери эффективности уничтожения данных. Также следует учитывать, что метод был предложен в 1996 году, и некоторые методы кодирования уже являются устаревшими, что также позволяет сократить количество проходов без потери качества уничтожения данных.

Существует несколько свободных проектов, реализующих метод Гутмана. Наиболее популярна, по всей видимости, утилита  shred, входящая в состав GNU Core Utilities.

Для унификации и оптимизации методов уничтожения информации в различных странах были выпущены различные стандарты, которыми все на сегодняшний день и пользуются.

Министерством Обороны США был выпущен стандарт DoD \linebreak 5220.22"=M. Стандарт имеет различные модификации для различных областей применения. Количество проходов различными шаблонами данных, включая случайные шаблоны, колеблется от 2 до 7.

В Российской Федерации разработан стандарт ГОСТ P50739"=95. Данный государственный стандарт предоставляет свободу в выборе шаблона перезаписи и предусматривает, в зависимости от класса автоматизированной системы, от 1 до 2 циклов перезаписи.

В Германии разработан стандарт VSITR. Стандарт предусматривает 7 циклов перезаписи, но не предусматривает случайных последовательностей в шаблонах перезаписи и использует всего три заранее заданных шаблона.

К сожалению, белорусских нормативных актов, регулирующих процессы уничтожения цифровой информации, найти не удалось, а украинские не описывают методы, которые должны применяться, и лишь указывают на необходимость применения гарантированного уничтожения информации с цифровых носителей.

Как правило, на практике зачастую достаточно одного"=трех проходов для вполне успешного уничтожения данных и дальнейшего использования НЖМД без практической возможности восстановления информации, содержащейся на нем ранее.

\end{document}
