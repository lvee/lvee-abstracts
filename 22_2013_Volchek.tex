\documentclass[10pt, a5paper]{article}
\usepackage{pdfpages}
\usepackage{parallel}
\usepackage[T2A]{fontenc}
\usepackage{ucs}
\usepackage[utf8x]{inputenc}
\usepackage[polish,english,russian]{babel}
\usepackage{hyperref}
\usepackage{rotating}
\usepackage[inner=2cm,top=1.8cm,outer=2cm,bottom=2.3cm,nohead]{geometry}
\usepackage{listings}
\usepackage{graphicx}
\usepackage{wrapfig}
\usepackage{longtable}
\usepackage{indentfirst}
\usepackage{array}
\newcolumntype{P}[1]{>{\raggedright\arraybackslash}p{#1}}
\frenchspacing
\usepackage{fixltx2e} %text sub- and superscripts
\usepackage{icomma} % коскі ў матэматычным рэжыме
\PreloadUnicodePage{4}

\newcommand{\longpage}{\enlargethispage{\baselineskip}}
\newcommand{\shortpage}{\enlargethispage{-\baselineskip}}

\def\switchlang#1{\expandafter\csname switchlang#1\endcsname}
\def\switchlangbe{
\let\saverefname=\refname%
\def\refname{Літаратура}%
\def\figurename{Іл.}%
}
\def\switchlangen{
\let\saverefname=\refname%
\def\refname{References}%
\def\figurename{Fig.}%
}
\def\switchlangru{
\let\saverefname=\refname%
\let\savefigurename=\figurename%
\def\refname{Литература}%
\def\figurename{Рис.}%
}

\hyphenation{admi-ni-stra-tive}
\hyphenation{ex-pe-ri-ence}
\hyphenation{fle-xi-bi-li-ty}
\hyphenation{Py-thon}
\hyphenation{ma-the-ma-ti-cal}
\hyphenation{re-ported}
\hyphenation{imp-le-menta-tions}
\hyphenation{pro-vides}
\hyphenation{en-gi-neering}
\hyphenation{com-pa-ti-bi-li-ty}
\hyphenation{im-pos-sible}
\hyphenation{desk-top}
\hyphenation{elec-tro-nic}
\hyphenation{com-pa-ny}
\hyphenation{de-ve-lop-ment}
\hyphenation{de-ve-loping}
\hyphenation{de-ve-lop}
\hyphenation{da-ta-ba-se}
\hyphenation{plat-forms}
\hyphenation{or-ga-ni-za-tion}
\hyphenation{pro-gramming}
\hyphenation{in-stru-ments}
\hyphenation{Li-nux}
\hyphenation{sour-ce}
\hyphenation{en-vi-ron-ment}
\hyphenation{Te-le-pathy}
\hyphenation{Li-nux-ov-ka}
\hyphenation{Open-BSD}
\hyphenation{Free-BSD}
\hyphenation{men-ti-on-ed}
\hyphenation{app-li-ca-tion}

\def\progref!#1!{\texttt{#1}}
\renewcommand{\arraystretch}{2} %Іначай формулы ў матрыцы зліпаюцца з лініямі
\usepackage{array}

\def\interview #1 (#2), #3, #4, #5\par{

\section[#1, #3, #4]{#1 -- #3, #4}
\def\qname{LVEE}
\def\aname{#1}
\def\q ##1\par{{\noindent \bf \qname: ##1 }\par}
\def\a{{\noindent \bf \aname: } \def\qname{L}\def\aname{#2}}
}

\def\interview* #1 (#2), #3, #4, #5\par{

\section*{#1\\{\small\rm #3, #4. #5}}

\def\qname{LVEE}
\def\aname{#1}
\def\q ##1\par{{\noindent \bf \qname: ##1 }\par}
\def\a{{\noindent \bf \aname: } \def\qname{L}\def\aname{#2}}
}

\begin{document}
\title{Пиратское движение в Беларуси}
\author{Михаил Волчек \footnote{Минск, Беларусь; \url{fannrm@gmail.com}}}
\maketitle
\begin{abstract}
This presentation opens belarusian pirates' agenda, describes what they do and ups some actual issues for free software \linebreak community. Do we ready to comminucate and build bridges for overcoming common challenges?
\end{abstract}
Беларусь не является тихой гаванью и не может считаться полностью избавленной от проблем, которые имеют интернет"=пользователи по всему миру: несбалансированное копирайт"=законодательство, нелегальный сбор данных коммерческими и правительственными структурами и т.д.

В 2013 году в Беларуси появился Пиратский Центр, имеющий веб"=сайт по адресу pirates.by и ориентированный на то, чтобы активно заниматься проблемами беларусских интернет"=пользователей (в т.ч. через просветительскую деятельность).
Пиратский центр является полноправным членом Pirate Party International (PPI) -- международной организации, действующей с 2009 года и ставящей своей целью способствовать взаимодействию появляющихся пиратских партий, групп и движений по всему миру.

Пиратские партии концентрируются на цифровых правах, но не только. В этот список входят реформа копирайта (патентной системы), анонимность и приватность в сети, доступный интернет (как право), нейтралитет сети, вовлеченность людей в принятие решений, прозрачность управления в обществе, экономика, ориентированная на человека, а не рост цифр ВВП.

Как можно видеть по анонсам на pirates.by, центр на текущий момент занимается:

\begin{enumerate}
  \item Продвижением среди пользователей интернета криптографических
инструментов как средств для приватного и анонимного интернета. Форма популяризации -- криптовечеринки.
  \item Раскрытием темы свободных лицензий как одной из альтернатив копирайту. Форма популяризации --  открытые лекции с различными творческими сообществами и людьми, аудиоконференции, образовательный клуб, просмотры фильмов о сетевых сообществах.
  \item Изучением инструментов, реализованных на открытых платформах для
выработки и принятия решений в больших группах (например,
Liquidfeedback, adhocracy, wikiarguments).
\end{enumerate}

Перечисленный набор активностей направлен на популяризацию знаний о цифровых правах человека, расширение кругозора и повышение уровня сетевой и компьютерной грамотности пользователей.

Следует учитывать, что под цифровыми правами в данном случае понимаются права, описанные во Всеобщей Декларации Прав Человека, применённые в цифровых технологиях (осн. Сеть), а под копирайтом понимается набор законов и международных договоров, регулирующих культуру, медиа, науку и другие сферы общества, связанные с производством, потреблением и распределением продуктов индустрии контента (фильмы, музыка, программы, игры, книги и т.д).

\end{document}
