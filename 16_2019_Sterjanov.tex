\documentclass[10pt, a5paper]{article}
\usepackage[T2A]{fontenc}
\usepackage{ucs}
\usepackage[utf8x]{inputenc}
\usepackage[polish,english,russian]{babel}
\usepackage{hyperref}
\usepackage[inner=2cm,top=1.8cm,outer=2cm,bottom=2.3cm,nohead]{geometry}
\usepackage{listings}
\usepackage{graphicx}
\usepackage{wrapfig}
\usepackage{longtable}
\usepackage{indentfirst}
\frenchspacing
\usepackage{fixltx2e} %text sub- and superscripts
\usepackage{icomma} % коскі ў матэматычным рэжыме
\PreloadUnicodePage{4}

\newcommand{\longpage}{\enlargethispage{\baselineskip}}
\newcommand{\shortpage}{\enlargethispage{-\baselineskip}}

\def\switchlang#1{\expandafter\csname switchlang#1\endcsname}
\def\switchlangbe{
\let\saverefname=\refname%
\def\refname{Літаратура}%
\def\figurename{Іл.}%
}
\def\switchlangen{
\let\saverefname=\refname%
\def\refname{References}%
\def\figurename{Fig.}%
}
\def\switchlangru{
\let\saverefname=\refname%
\let\savefigurename=\figurename%
\def\refname{Литература}%
\def\figurename{Рис.}%
}

\hyphenation{admi-ni-stra-tive}
\hyphenation{ex-pe-ri-ence}
\hyphenation{fle-xi-bi-li-ty}
\hyphenation{Py-thon}
\hyphenation{ma-the-ma-ti-cal}
\hyphenation{re-ported}
\hyphenation{imp-le-menta-tions}
\hyphenation{pro-vides}
\hyphenation{en-gi-neering}
\hyphenation{com-pa-ti-bi-li-ty}
\hyphenation{im-pos-sible}
\hyphenation{desk-top}
\hyphenation{elec-tro-nic}
\hyphenation{com-pa-ny}
\hyphenation{de-ve-lop-ment}
\hyphenation{de-ve-loping}
\hyphenation{de-ve-lop}
\hyphenation{da-ta-ba-se}
\hyphenation{plat-forms}
\hyphenation{or-ga-ni-za-tion}
\hyphenation{pro-gramming}
\hyphenation{in-stru-ments}
\hyphenation{Li-nux}
\hyphenation{en-vi-ron-ment}
\hyphenation{Te-le-pathy}
\hyphenation{Li-nux-ov-ka}

\def\progref!#1!{\texttt{#1}}
\renewcommand{\arraystretch}{2} %Іначай формулы ў матрыцы зліпаюцца з лініямі
\usepackage{array}

\def\interview #1 (#2), #3, #4, #5\par{

\section[#1, #3, #4]{#1, #5}
\def\qname{LVEE}
\def\aname{#1}
\def\q ##1\par{{\noindent \bf \qname: ##1 }\par}
\def\a{{\noindent \bf \aname: } \def\qname{L}\def\aname{#2}}
}

\switchlang{ru}
\begin{document}
\title{Опыт преподавания языка Ruby в рамках дисциплины <<Cовременные технологии разработки программного обеспечения>>}
\author{Максим Стержанов, Минск, Belarus\footnote{\url{msterjanov@gmail.com}, \url {https://lvee.org/ru/abstracts/320}}}
\maketitle
\begin{abstract}
Experience of teaching master degree students Ruby/RoR is detailed. We explain basic course structure, list main difficulties students face and way to resolve them. 
\end{abstract}
Кафедра Информатики БГУИР ведет подготовку бакалавров и магистров по специальности <<Информатика и технологии программирования>>. Одной из основных специальных дисциплин, читаемых при подготовке магистрантов является <<Современные технологии разработки программного обеспечения>>(СТРПО). Целью преподавания данной дисциплины является предоставление обучаемым знаний и умений в области проектирования, разработки, тестирования, отладки и внедрения программного обеспечения (ПО) вычислительной техники с использованием современных технологий.

В данной работе описывается перечень лабораторных задач,\linebreak предлагаемых студентам  для проработки и закрепления материала по предмету СТРПО в 2017 /2018 учебном году.

Устный опрос показывает, что основную группу магистрантов составляют программисты-практики с профильным высшим образованием (БГУИР или БГУ) и опытом работы в софтверных компаниях от 2 до 4 лет. Следовательно, данная аудитория должна иметь глубокое понимание теоретических основ информатики, опыт практического использования одного или нескольких языков и технологий. Исходя из этого, образовательный процесс фокусировался на ключевых особенностях языка Ruby.

В рамках первой лабораторной работы магистрантам предлагается познакомиться с основами написания скриптов на динамическом объектно-ориентированном языке Ruby и проработать применение базовых конструкций языка. В качестве среды разработки предлагается тестовый редактор Sublime Text или специализированная среда RubyMine. В рамках данной работы предлагается реализация простейшего алгоритма шифрования.

Вторая лабораторная работа посвящена изучению функционального стиля программирования в Ruby. Все функции в Ruby являются методами, то есть свойственны объектам. Цель выполнения работы --- изучение итераторов, блоков и замыканий. Также магистратам предлагается провести сравнительный анализ объектов, которые можно вызывать (proc, lambda, method).

В рамках третьей лабораторной работы магистрантам предлагается применить на практике знания об объектной модели Ruby. Мы предполагаем, что большая часть аудитории знакома с понятиями ООП на примере других языков. Ruby является полностью объектно-ориентированным языком: числа, строки, регулярные выражения, массивы --- это все объекты определенных классов.  Магистрантам предлагается изучить концепцию модуля и примеси, инкапсуляцию. Результатом выполнения работы является реализация взаимодействия объектов в соответствии с индивидуальным заданием.

Четвертая и пятая лабораторная работы посвящены метапрограммированию в объектной модели Ruby. Под метапрограммированием понимается расширение и изменение абстракций языка \cite{bib1}. Магистранты изучают способы динамического определения и вызова методов, применение method\_missing, синглетон-методы, синг\-летон-классы, отрабатывают техники динамического изменения \linebreak классов и методов.

На шестой, заключительной работе, магистрантам предлагается обобщить полученные знания при построении серверной части веб-приложения на платформе Ruby on Rails. Rails представляет собой среду, облегчающую разработку, развертывание и обслуживание веб-приложений \cite{bib3}. Магистранты создают REST ориентированные сервисы в соответствии с предложенными вариантами заданий (библиотека, ресторан, больница и т.д.). Задачей является продемонстрировать умение пользоваться фреймворком объектно-реляционного отображения ActiveRecord и основами ресурсного роутинга Rails. Реализация клиентской части (HTML представления) не требуется. Тестирование осуществляется при помощи программы POSTMAN (либо аналогичной).

Содержание лабораторных работ построено в единой логике и позволяет эффективно обучить магистрантов приемам программирования на современном скриптовом языке Ruby.

Опыт преподавания языка Ruby для магистрантов выявил некоторые проблемы:

\begin{itemize}
  \item недостаточная подготовка в области программирования (отсутствие умений и навыков разработки, отсутствие понятийного аппарата ООП) после окончания ВУЗа;
  \item нехватка времени для самостоятельной работы в связи с загруженностью по основному месту работы;
  \item выполнение работ на поверхностном уровне, нежелание переучиваться и погружаться в детали  новой и незнакомой технологии.
\end{itemize}

Для решения данных проблем отстающим магистрантам были предложены упрощенные версии индивидуальных заданий.

Не смотря на указанные сложности, изучение языка Ruby и платформы Ruby on Rails дает магистрантам уникальные возможности для расширения собственного багажа знаний и  опыта, которыми нельзя не воспользоваться.



\begin{thebibliography}{9}
\bibitem{bib1} A. Hunt. Programming Ruby./ A. Hunt, D. Thomas --- М.: Финансы и статистика, 2004. --- 864 p.
\bibitem{bib2} Perrotta P. Metaprogramming Ruby 2: Program Like the Ruby Pros. -- The Pragmatic Programmers, 2004. --- 262 p.
\bibitem{bib3} Руби С. Rails 4. Гибкая разработка веб-приложений. С. Руби, Д. Томас, Д. Хэнссон --- СПб.: Питер, 2014. --- 448 с.
\end{thebibliography}
\end{document}
