\documentclass[10pt, a5paper]{article}
\usepackage{pdfpages}
\usepackage{parallel}
\usepackage[T2A]{fontenc}
\usepackage{ucs}
\usepackage[utf8x]{inputenc}
\usepackage[polish,english,russian]{babel}
\usepackage{hyperref}
\usepackage{rotating}
\usepackage[inner=2cm,top=1.8cm,outer=2cm,bottom=2.3cm,nohead]{geometry}
\usepackage{listings}
\usepackage{graphicx}
\usepackage{wrapfig}
\usepackage{longtable}
\usepackage{indentfirst}
\usepackage{array}
\newcolumntype{P}[1]{>{\raggedright\arraybackslash}p{#1}}
\frenchspacing
\usepackage{fixltx2e} %text sub- and superscripts
\usepackage{icomma} % коскі ў матэматычным рэжыме
\PreloadUnicodePage{4}

\newcommand{\longpage}{\enlargethispage{\baselineskip}}
\newcommand{\shortpage}{\enlargethispage{-\baselineskip}}

\def\switchlang#1{\expandafter\csname switchlang#1\endcsname}
\def\switchlangbe{
\let\saverefname=\refname%
\def\refname{Літаратура}%
\def\figurename{Іл.}%
}
\def\switchlangen{
\let\saverefname=\refname%
\def\refname{References}%
\def\figurename{Fig.}%
}
\def\switchlangru{
\let\saverefname=\refname%
\let\savefigurename=\figurename%
\def\refname{Литература}%
\def\figurename{Рис.}%
}

\hyphenation{admi-ni-stra-tive}
\hyphenation{ex-pe-ri-ence}
\hyphenation{fle-xi-bi-li-ty}
\hyphenation{Py-thon}
\hyphenation{ma-the-ma-ti-cal}
\hyphenation{re-ported}
\hyphenation{imp-le-menta-tions}
\hyphenation{pro-vides}
\hyphenation{en-gi-neering}
\hyphenation{com-pa-ti-bi-li-ty}
\hyphenation{im-pos-sible}
\hyphenation{desk-top}
\hyphenation{elec-tro-nic}
\hyphenation{com-pa-ny}
\hyphenation{de-ve-lop-ment}
\hyphenation{de-ve-loping}
\hyphenation{de-ve-lop}
\hyphenation{da-ta-ba-se}
\hyphenation{plat-forms}
\hyphenation{or-ga-ni-za-tion}
\hyphenation{pro-gramming}
\hyphenation{in-stru-ments}
\hyphenation{Li-nux}
\hyphenation{sour-ce}
\hyphenation{en-vi-ron-ment}
\hyphenation{Te-le-pathy}
\hyphenation{Li-nux-ov-ka}
\hyphenation{Open-BSD}
\hyphenation{Free-BSD}
\hyphenation{men-ti-on-ed}
\hyphenation{app-li-ca-tion}

\def\progref!#1!{\texttt{#1}}
\renewcommand{\arraystretch}{2} %Іначай формулы ў матрыцы зліпаюцца з лініямі
\usepackage{array}

\def\interview #1 (#2), #3, #4, #5\par{

\section[#1, #3, #4]{#1 -- #3, #4}
\def\qname{LVEE}
\def\aname{#1}
\def\q ##1\par{{\noindent \bf \qname: ##1 }\par}
\def\a{{\noindent \bf \aname: } \def\qname{L}\def\aname{#2}}
}

\def\interview* #1 (#2), #3, #4, #5\par{

\section*{#1\\{\small\rm #3, #4. #5}}

\def\qname{LVEE}
\def\aname{#1}
\def\q ##1\par{{\noindent \bf \qname: ##1 }\par}
\def\a{{\noindent \bf \aname: } \def\qname{L}\def\aname{#2}}
}


\begin{document}

\title{Обзор одноплатного микрокомпьютера BeagleBone}%\footnote{Текст данных и последующих тезисов, кроме специально оговоренных случаев, доступен под лицензией Creative Commons Attribution-ShareAlike 3.0}

\author{Дмитрий Горох\footnote{Минск, Беларусь}}
\maketitle

\begin{abstract}
Over the last years the single-board computers based on ARM CPU core such as Raspberry PI, PandaBoard and SheevaPlug have rapidly gained popularity. Volunteer driven start-up BeagleBoard.org recently added the new device to this list: a single-board hardware hacker oriented computer for only \$89.
\end{abstract}

Последнее время стремительно набирают популярность недорогие одноплатные микрокомпьютеры на базе ARM процессоров, такие как Raspberry PI, PandaBoard, SheevaPlug и другие. Некоммерческий стартап BeagleBoard.org недавно пополнил этот список своей новой разработкой: одноплатным компьютером BeagleBone ориентированным на DIY энтузиастов ценой всего \$89.

\subsection*{Краткое описание}



Среди большинства других одноплатных микрокомпьютеров BeagleBone выделяется своей ориентированностью на энтузиастов, желающих иметь расширяемую аппаратную платформу на базе производительного и мало потребляющего процессора под управлением ОС Linux. Вокруг BeagleBone уже образовалось активное сообщество, публикующее новые отчёты об оригинальном применении BeagleBone. На момент написания статьи на сайте beagleboard.org было зарегистрировано 244 проекта, среди которых можно найти:

\begin{itemize}
  \item погодная станция
  \item сетевая камера слежения
  \item игровая консоль
  \item осциллограф
  \item сетевое хранилище файлов
\end{itemize}

Кроме того ничто не мешает использовать BeagleBone как традиционный Linux сервер. В пакетном репозитарии поставляемого Linux дистрибутива можно найти большое множество серверов: lighthttpd (HTTP Server), rtorrent (Bittorrent Client), postfix (SMTP Mail Server), git (distributed revision control) и многое другое.

\subsection*{Разработка ПО}

Для разработчиков с платой поставляется полный набор программного обеспечения, позволяющего собрать из исходников ядро Линукса, загрузчик и корневую файловую систему дистрибутива \AA{}ngstr\"{o}m. На плате имеется USB-to-UART конвертер, подключённый к UART и JTAG портам процессора. Таким образом вам понадобится только USB кабель чтобы иметь полный контроль над платой в процессе разработки.
В качестве IDE поставляется Eclipse и Cloud9. Последняя IDE предлагает интересный способ разработки простых приложений в web браузере. Приложения пишутся на jScript в асинхронном стиле и запускаются из-под сервера Node.js опираясь на библиотеку bonescript. Пример программы, мигающей светодиодом:

\begin{verbatim}
var bb = require('./bonescript');

var ledPin = bone.P8_3;
var ledPin2 = bone.USR3;

setup = function() {
    pinMode(ledPin, OUTPUT);
    pinMode(ledPin2, OUTPUT);
};

loop = function() {
    digitalWrite(ledPin, HIGH);
    digitalWrite(ledPin2, HIGH);
    delay(1000);
    digitalWrite(ledPin, LOW);
    digitalWrite(ledPin2, LOW);
    delay(1000);
};

bb.run();
\end{verbatim}

Разработчики библиотеки bonescript ставят своей целью создание простой для освоения и использования платформы, аналогичной Wiring, существующей для Arduino.

\subsection*{Платы расширения}

BeagleBoard имеет два 46-пиновых разъёма, дающие доступ практически ко всем портам процессора. Форм-фактор платы устроен таким образом, что платы расширения можно стыковать друг на друга «бутербродом». На настоящий момент существует около десятка плат расширения, и этот список растёт за счёт активности со стороны сообщества. Имеются платы реализующие DVI видео выход, LCD дисплей, аудио входы и выходы, батарейное питание. Форм-фактор платы также располагает к быстрому прототипированию простых схем навесным монтажом или на макетной плате.

\subsection*{Заключение}

BeagleBoard является мощной основой как для любительского творчества, так и для профессиональной разработки Embedded Linux приложений. Современные Linux дистрибутивы, ориентированные на ARM платформы, позволяют работать с одноплатными компьютерами практически с тем же комфортом, что и на x86 машинах. При этом ARM системы (в особенности Cortex-A8) обеспечивают беспрецедентно низкое удельное энергопотребление на единицу производительности (Watt / MIPS) и лидируют по компактности, позволяя строить широкий спектр портативных устройств с автономным питанием.

\begin{thebibliography}{9}
  \bibitem {G11} Домашняя страница BeagleBone \url{http://beagleboard.org/bone}
  \bibitem {G12} Руководство по быстрому запуску BeagleBoard \url{http://beagleboard.org/static/beaglebone/a3/README.htm}
  \bibitem {G13} Библиотека bonescript \url{https://github.com/jadonk/bonescript}
  \bibitem {G14} Использование Eclipse для BeagleBone \url{http://elinux.org/BeagleBoardEclipse}
  \bibitem {G15} Дистрибутив \AA{}ngstr\"{o}m \url{http://www.angstrom-distribution.org/}
  \bibitem {G16} Пример программирования на jScript под Cloud9 и на Python \url{http://www.gigamegablog.com/2012/01/05/beaglebone-coding-101-blinking-an-led/}
  \bibitem {G17} Инструкция по установке Android \url{http://processors.wiki.ti.com/index.php/BeagleBone-Android-DevKit_Guide}
  \bibitem {G18} Платы расширения для BeagleBone \url{http://beagleboardtoys.com/wiki/index.php5?title=Main_Page}
  \bibitem {G19} BeagleBone на Farnell \url{http://ru.farnell.com/circuitco/bb-bone-000/kit-dev-beaglebone-cortex-a8/dp/2063627}
  \bibitem {G20} BeagleBone на DigiKey \url{http://search.digikey.com/us/en/cat/programmers-development-systems/general-embedded-dev-boards-and-kits-mcu-dsp-fpga-cpld/2621773?k=beaglebone}
\end{thebibliography}



\end{document}




