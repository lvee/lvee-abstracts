\documentclass[10pt, a5paper]{article}
\usepackage{pdfpages}
\usepackage{parallel}
\usepackage[T2A]{fontenc}
\usepackage{ucs}
\usepackage[utf8x]{inputenc}
\usepackage[polish,english,russian]{babel}
\usepackage{hyperref}
\usepackage{rotating}
\usepackage[inner=2cm,top=1.8cm,outer=2cm,bottom=2.3cm,nohead]{geometry}
\usepackage{listings}
\usepackage{graphicx}
\usepackage{wrapfig}
\usepackage{longtable}
\usepackage{indentfirst}
\usepackage{array}
\newcolumntype{P}[1]{>{\raggedright\arraybackslash}p{#1}}
\frenchspacing
\usepackage{fixltx2e} %text sub- and superscripts
\usepackage{icomma} % коскі ў матэматычным рэжыме
\PreloadUnicodePage{4}

\newcommand{\longpage}{\enlargethispage{\baselineskip}}
\newcommand{\shortpage}{\enlargethispage{-\baselineskip}}

\def\switchlang#1{\expandafter\csname switchlang#1\endcsname}
\def\switchlangbe{
\let\saverefname=\refname%
\def\refname{Літаратура}%
\def\figurename{Іл.}%
}
\def\switchlangen{
\let\saverefname=\refname%
\def\refname{References}%
\def\figurename{Fig.}%
}
\def\switchlangru{
\let\saverefname=\refname%
\let\savefigurename=\figurename%
\def\refname{Литература}%
\def\figurename{Рис.}%
}

\hyphenation{admi-ni-stra-tive}
\hyphenation{ex-pe-ri-ence}
\hyphenation{fle-xi-bi-li-ty}
\hyphenation{Py-thon}
\hyphenation{ma-the-ma-ti-cal}
\hyphenation{re-ported}
\hyphenation{imp-le-menta-tions}
\hyphenation{pro-vides}
\hyphenation{en-gi-neering}
\hyphenation{com-pa-ti-bi-li-ty}
\hyphenation{im-pos-sible}
\hyphenation{desk-top}
\hyphenation{elec-tro-nic}
\hyphenation{com-pa-ny}
\hyphenation{de-ve-lop-ment}
\hyphenation{de-ve-loping}
\hyphenation{de-ve-lop}
\hyphenation{da-ta-ba-se}
\hyphenation{plat-forms}
\hyphenation{or-ga-ni-za-tion}
\hyphenation{pro-gramming}
\hyphenation{in-stru-ments}
\hyphenation{Li-nux}
\hyphenation{sour-ce}
\hyphenation{en-vi-ron-ment}
\hyphenation{Te-le-pathy}
\hyphenation{Li-nux-ov-ka}
\hyphenation{Open-BSD}
\hyphenation{Free-BSD}
\hyphenation{men-ti-on-ed}
\hyphenation{app-li-ca-tion}

\def\progref!#1!{\texttt{#1}}
\renewcommand{\arraystretch}{2} %Іначай формулы ў матрыцы зліпаюцца з лініямі
\usepackage{array}

\def\interview #1 (#2), #3, #4, #5\par{

\section[#1, #3, #4]{#1 -- #3, #4}
\def\qname{LVEE}
\def\aname{#1}
\def\q ##1\par{{\noindent \bf \qname: ##1 }\par}
\def\a{{\noindent \bf \aname: } \def\qname{L}\def\aname{#2}}
}

\def\interview* #1 (#2), #3, #4, #5\par{

\section*{#1\\{\small\rm #3, #4. #5}}

\def\qname{LVEE}
\def\aname{#1}
\def\q ##1\par{{\noindent \bf \qname: ##1 }\par}
\def\a{{\noindent \bf \aname: } \def\qname{L}\def\aname{#2}}
}

\begin{document}
\title{Свободное производство информации в постиндустриальном обществе}
\author{Елена Иванова\footnote{Москва, Россия, \url{http://coofeed.com}}, Дмитрий Костюк\footnote{Брест, Беларусь, \url{dmitriykostiuk@gmail.com}}}
\date{}
\maketitle
\begin{abstract}
The concept of postindustrial society is discussed in relation to free / open source software community-driven production. Its better correlation with Bell principles than classic information-as-a-good approach is shown. 
\end{abstract}

Концепция постиндустриализма в качестве следующей стадии социального развития приобрела широкую известность в 1970-х годах, когда Д. Беллом были сформулированы следующие положения \cite{ivanova1}, лежащие в его основе: главенствующая роль знаний в обществе (1), главенствующая роль производства не товаров, но услуг (2), университет как очаг производства новых знаний и инноваций (3). Исходно реализацию перечисленных принципов затрудняла дороговизна и малая эффективность коммуникации между специалистами. Ставшая популярной в 90-х концепция информационного общества \cite{ivanova2} включила в себя построение глобального информационного пространства с использованием новых компьютерных технологий, и таким образом в значительной степени снизила коммуникационный барьер за счет на порядок более эффективного удаленного взаимодействия.

Популярное толкование положения о информации и знаниях, как главном продукте производства, рассматривает поточное производство информации как товара в рамках сервисной (т. е. ориентированной на рынок услуг) экономики. В определенном смысле первый и третий из сформулированных Беллом признаков постиндустриального общества рассматриваются в качестве сателлитов положения о примате производства услуг. 

Производство информации более прибыльно в сравнении с материальным производством, так как достаточно изготовить первоначальный образец, а затраты на копирование несущественны. Но эти же свойства информации делают ее неудобным товаром с точки зрения традиционных экономических отношений. Когда под предоставлением услуг понимается торговля информацией или сдача информации в аренду, приходится изобретать искусственные ограничительные механизмы. Необходима развитая юридическая защита прав интеллектуальной собственности. Права на информацию, которые подлежат юридической защите, должны носить монопольный характер (это является не только необходимым условием для превращения информации в товар, но и позволяет извлекать монопольную прибыль, увеличивая рентабельность постиндустриальной экономики). Необходимо огромное количество потребителей, которые готовы платить за информацию, предоставляемую в данной ограниченной форме.

Однако возрастание числа людей, занятых информационными технологиями, коммуникациями и производством информационных продуктов и услуг "--- людей, основным «средством производства» которых является не оборудование работодателя, а их собственная квалификация \cite{ivanova2} "--- вносит в данный подход коррективы. В результате, существенная часть членов общества оказывается неудовлетворенной накладываемыми на информационный продукт искусственными ограничениями, делающими его использование менее удобным и осознает, что сверхприбыли поставщиков информационного продукта противоречат интересам их, как потребителей. Такая группа имеет средства производства для коллективного воссоздания аналога и сетевую инфраструктуру, позволяющую добиться предсказанной апологетами постиндустриализма быстрой самоорганизации творческих коллективов для решения конкретных задач, а ее представители готовы воспринимать в качестве опосредованного источника прибыли дополнительные факторы, такие как рост личного профессионализма и профессиональную репутацию.

Наиболее ярко на сегодняшний день данное явление проявляется в сфере свободного программного обеспечения. Группы специалистов, неудовлетворенные ситуацией на рынке, объединяются, выделяя часть своего свободного времени (являющегося в постиндустриальном обществе оцениваемым ресурсом) для создания равноценного программного продукта, свободного от искусственных ограничений, связанных с интеллектуальной монополией \cite{ivanova3} какого-либо производителя. В качестве страховки интересов потребителей такой продукт часто оснащается собственными лицензионными ограничениями, направленными против его возможной монополизации.
В рамках сформулированной концепции информационного общества данное явление можно рассматривать как массовое коллективное инвестирование, базирующееся на совладении материальными и нематериальными активами (crowd funding). Характерно также, что первоначальными источниками появления свободных программ являлись университеты, в полном соответствии с одним из сформулированных Беллом положений.

Модели бизнеса, построенные вокруг таких продуктов, хорошо укладываются в постиндустриальную модель, предоставляя коммерческие услуги по сопровождению и/или доработке продукта, а также доступ к продукту (консолидированному и обслуживаемому централизованно на мощных вычислительных станциях) в рамках облачных вычислений и технологии Software-As-Service (приложение как услуга).

Явление коллективного производства свободно распространяемых информационных продуктов претерпевает некоторые эволюционные изменения, связанные с большей доступностью для потребителя и проникновением в смежные области, примером чему служат коллективное создание высококачественного энциклопедического контента википедии и родственных проектов, а также активизация на рынке средств мобильной связи. В частности, идеи объединения функциональности сотового телефона и карманного персонального компьютера появились практически сразу после появления первых карманных персональных компьютеров в начале 90-х. Наличие полнофункциональной операционной системы делает смартфоны и коммуникаторы более привлекательными в глазах большинства пользователей. Современные телефоны прекрасно справляются со многими задачами, выходящими за рамки телефонных: работа с электронной почтой, просмотр текстовых документов и электронных таблиц, работа с планировщиком задач и многими другими. Однако резкий рост популярности смартфонов произошел одновременно с развитием инфраструктуры открытой операционной системы Android \cite{ivanova4}.
Аналогичным по своей сути направлением эволюции в компьютерных технологиях является децентрализация, примером которой могут служить файловый обмен (торенты), электронные системы платежей, такие как BitCoin \cite{ivanova5}, распределенные социальные сети.

\begin{thebibliography}{9}
	\bibitem{ivanova1} Д. Белл. Грядущее постиндустриальное общество. М., Академия, 1999.
	\bibitem{ivanova2} Постиндустриальное общество. \url{http://ru.wikipedia.org}  
	\bibitem{ivanova3} Л. Лессиг. Свободная культура. \url{http://www.gumer.info/bibliotek_Buks/Culture/lessig/index.php}
	\bibitem{ivanova4} Source: Gartner (May 2011) \url{http://www.gartner.com/it/page.jsp?id=1689814}
	\bibitem{ivanova5} BitCoin --- анархическая пиринговая криптовалюта. \url{http://blogerator.ru/page/bitcoin}
\end{thebibliography}
\end{document}


