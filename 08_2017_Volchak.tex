\documentclass[10pt, a5paper]{article}
\usepackage{pdfpages}
\usepackage{parallel}
\usepackage[T2A]{fontenc}
\usepackage{ucs}
\usepackage[utf8x]{inputenc}
\usepackage[polish,english,russian]{babel}
\usepackage{hyperref}
\usepackage{rotating}
\usepackage[inner=2cm,top=1.8cm,outer=2cm,bottom=2.3cm,nohead]{geometry}
\usepackage{listings}
\usepackage{graphicx}
\usepackage{wrapfig}
\usepackage{longtable}
\usepackage{indentfirst}
\usepackage{array}
\newcolumntype{P}[1]{>{\raggedright\arraybackslash}p{#1}}
\frenchspacing
\usepackage{fixltx2e} %text sub- and superscripts
\usepackage{icomma} % коскі ў матэматычным рэжыме
\PreloadUnicodePage{4}

\newcommand{\longpage}{\enlargethispage{\baselineskip}}
\newcommand{\shortpage}{\enlargethispage{-\baselineskip}}

\def\switchlang#1{\expandafter\csname switchlang#1\endcsname}
\def\switchlangbe{
\let\saverefname=\refname%
\def\refname{Літаратура}%
\def\figurename{Іл.}%
}
\def\switchlangen{
\let\saverefname=\refname%
\def\refname{References}%
\def\figurename{Fig.}%
}
\def\switchlangru{
\let\saverefname=\refname%
\let\savefigurename=\figurename%
\def\refname{Литература}%
\def\figurename{Рис.}%
}

\hyphenation{admi-ni-stra-tive}
\hyphenation{ex-pe-ri-ence}
\hyphenation{fle-xi-bi-li-ty}
\hyphenation{Py-thon}
\hyphenation{ma-the-ma-ti-cal}
\hyphenation{re-ported}
\hyphenation{imp-le-menta-tions}
\hyphenation{pro-vides}
\hyphenation{en-gi-neering}
\hyphenation{com-pa-ti-bi-li-ty}
\hyphenation{im-pos-sible}
\hyphenation{desk-top}
\hyphenation{elec-tro-nic}
\hyphenation{com-pa-ny}
\hyphenation{de-ve-lop-ment}
\hyphenation{de-ve-loping}
\hyphenation{de-ve-lop}
\hyphenation{da-ta-ba-se}
\hyphenation{plat-forms}
\hyphenation{or-ga-ni-za-tion}
\hyphenation{pro-gramming}
\hyphenation{in-stru-ments}
\hyphenation{Li-nux}
\hyphenation{sour-ce}
\hyphenation{en-vi-ron-ment}
\hyphenation{Te-le-pathy}
\hyphenation{Li-nux-ov-ka}
\hyphenation{Open-BSD}
\hyphenation{Free-BSD}
\hyphenation{men-ti-on-ed}
\hyphenation{app-li-ca-tion}

\def\progref!#1!{\texttt{#1}}
\renewcommand{\arraystretch}{2} %Іначай формулы ў матрыцы зліпаюцца з лініямі
\usepackage{array}

\def\interview #1 (#2), #3, #4, #5\par{

\section[#1, #3, #4]{#1 -- #3, #4}
\def\qname{LVEE}
\def\aname{#1}
\def\q ##1\par{{\noindent \bf \qname: ##1 }\par}
\def\a{{\noindent \bf \aname: } \def\qname{L}\def\aname{#2}}
}

\def\interview* #1 (#2), #3, #4, #5\par{

\section*{#1\\{\small\rm #3, #4. #5}}

\def\qname{LVEE}
\def\aname{#1}
\def\q ##1\par{{\noindent \bf \qname: ##1 }\par}
\def\a{{\noindent \bf \aname: } \def\qname{L}\def\aname{#2}}
}

\switchlang{be}
\begin{document}
\title{Бізнэс-мадэлі для супольнай уласнасці або як прадаваць атамы і не біты.\footnote{\url{fannrm@gmail.com}, \url{http://lvee.org/ru/abstracts/253}}}
\author{Mikhail Volchak, Minsk, Belarus}
\maketitle
\begin{abstract}
In the presentation different business models based on the digital commons were summarized. The first point is a consideration who controls a modern pool of resources and which characteristics are critical for the resource management. In the second point we discuss how the digital revolution change price-creation of a digital content. Further we elaborate how to make money and foster development based on Creative Commons principles. The last section describes how businesses can operate based on non-market premises and develop the third way in the content creation, distribution and consumption.
\end{abstract}
Бізнэс мадэлі для супольнай уласнасці або як прадаваць атамы і не біты.

<<Кожны збытак стварае новы дыфіцыт>> 
(Свабодны, Андэрсан)

Найбольш частае пытанне, якое задаюць слухачы падчас лекцый і выступаў па тэме распаўсюду ідэй Creative Commons: як творчы чалавек можа зарабіць у Беларусі сабе на жыццё або стаць багатым выкарыстощваючы Creative Commons ліцэнзіі?

Перад тым як пачаць пералічваць што значыць быць зробленым творчым агульным, звернемся да асноўваў сучаснай палітэканоміі. Тут будзе палягаць частка адказу мета ўзроўню, а менавіта хто кіруе рэсурсамі.

\subsubsection*{Агульнае супольнае}

Гістарычна склалася 3 асноўных падыхода у кіраванні рэсурамі:

\begin{enumerate}
  \item Дзяржаўная манаполія, значыць, што урад кантралюе вытворчасць, размеркаванне і спажыванне рэсурсаў,
  \item Падыход рынкавых адносінаў кантралюе рэсурсы інстытутам прыватнай ўласнасці.
  \item Супольныя рэсурсы ~--- гэта значыць, што рэсурсы кантралююцца калектыўна, грамадзянамі.
\end{enumerate}

Доўгі час у Беларусі мы мелі сітуацыю, калі рынкавыя і грамадскія механізмы былі ўлучаны ў адзін ~--- дзяржаўны, і кантраляваліся урадам (або партыяй).

Апошнія 25 год незалежнасці Беларусь абрала рынкавы \linebreak механізм. У разуменні шляхоў развіцця замацавалася два асноўных падыхода. Першы кажа, пра тое, што дзяржава павінна кантраляваць рэсурсы і эканоміку, а другі прасоўвае свабодны рынак, які павінен выканаць функцыю мадэрнізатара эканомікі і жыцця грамадзян.

Але ў гэтай схеме ёсць недаацэненая кампанента ~--- грамадства. Якое прапаноўвае іншыя варыянт кіравання рэсурсамі у адрозненні ад дзяржаўнага або неаліберальнага капіталізма. Гэты падыход базуецца на ~--- супольным, або супольнай уласнасці.

Эліанор Острам выдзяляе асноўныя 4 кампаненты ў кіраванні рэсурсамі:

\begin{itemize}
  \item Характарыстыка рэсурсаў. Яны натуральныя або створаны чалавекам, іх дэфіцыт, або лішак, маюць фізічную або лічбавую аснову?
  \item Нормы і правілы. Што кантралюе рэсурсы: нефармальныя нормы, або фармальныя правілы ~--- законы?
  \item Людзі і працэсы. Хто мае доступ і можа карыстацца імі, хто мае ўладу над рэсурсамі, прамую або ўскосную?
  \item Мэты. Як карыстаецца здабытае, рэсурс пастаянна дадаецца (як вікіпедыя) або выймаецца (як нафта)?
\end{itemize}

У 1980-90х адбылася лічбавая рэвалюцыя і Столман з паплечнікамі аб’явілі 4 свабоды для эканомікі супольнага (або грамадскага набытку на падставе праватных дамоваў):

\begin{itemize}
  \item карыстацца праграмай;
  \item змяняць і вывучаць праграму;
  \item распаўсюджваць яе;
  \item распаўсюджваць у тым ліку змененыя копіі.
\end{itemize}

Гэты рух надалей стварыў перадумавы для лічбавага супольнага, у які была уключаны не толькі код, але і кантэнт.

Інтэрнет даў тры асноўныя магчымасці:

\begin{itemize}
  \item Прыблізіць кошто копіі амаль да нуля, што дае магчымасць неабмежаванага тыражу.
  \item І аўтарства стала ўніверсальнай з’явай дзе амаль кожны можа быць аўтарам і распаўсбджваць свой кантэнт, у тым ліку бясплатна.
  \item Што ў сваю чаргу стварыла новыя ўмовы для кіравання рэсурсамі. Дзе стваральнікі кантэнту канкуруюць не толькі з тымі, хто прадае кантэнт, але з аўтарамі, які распаўсюджвае яго бясплатна.
\end{itemize}

Такім чынам, аўтараў, якія жадаюць прытрымлівацца супольнага падыходу да кіравання рэсурсамі стаіць няпростае пытанне: як захоўваючы каштоўнасць адкрытага доступа, да сваёй творчасці зарабіць грошы?

\subsubsection*{Што рабіць з нулявым коштам?}

Ніжэй прапанаваны падыходы сабраныя Пола Стэйсці і Сарай Персан з 24 пачынанняў які выкарысталі наступныя прынцыпы у сваіх бізнэс мадэлях:

\begin{itemize}
  \item Больш аўдыторыі. Creative Commons не ствараюць вірусны эффект, але яны ствараюць ўмовы для свабоднага распаўсюду. Тым самым правакуючы эффект далучэння да большасці.
  \item Прызнанне і рэпутацыя. Creative Commons механізмы патрабуюць пазначэнне аўтарства твораў, што з’яўляецца нормай для шматлікіх грамадстваў і супольнасцяў. Таким чынам \linebreak аўтарства як значны атрыбут для дадання вартасци творчаму прадукту.
  \item Раскрутка і прасоўванне. Электронная версія прадукт для прасоўвання, а фізічная як тавар, такім чынам эканомяцца грошы на рэкламу.
  \item Гуляць па сваіх правілах. Creative Commons дае магчымасць вызначаць свае правілы гульні і ўцягнення аўдыторыі, на рынку ў тым ліку.
  \item Сустварэнне. Creative Commons дае магчымасць перапрацоўваць творы і гэта ўключае спажыўцоў (напрыклад, мастакоў і музыкаў) у сумесную творчасць, тым самым павялічваючы сувязі з творамі.
\end{itemize}

\subsubsection*{Як зарабляць грошы?}

Галоўнае пытанне ў гэтай часцы, што ж cтварае дадатковую вартасць, за якую людзі гатовы плаціць?

Выкарыстоўваючы стандартныя рынкавыя мезанізмы:

Прадастаўленне кастамізаванага сэрвіса. Кантэнту бясплатнага шмат, але, каб атрымаць патрэбны толькі сабе кантэнт, неабходна заплаціць.
Прадастаўленне фізічнай копіі. Электроная копія бясплатна, але фізічная рэалізацыя за грошы. Напрыклад, дызайн мэблі або электронікі.

\begin{itemize}
  \item Аплата за персанізаваную версію. Персанальны досвед творчасці. Напрыклад, канцэрт, або лекцыя.
  \item Продаж свэга і мерчэндайза. Цішоткі і т.п.
  \item Сбор грошай з рэкламадаўца і спонсараў. Традыцыйны падыход.
  \item Стваральнікі кантэнта аплочваюць месца. Напрыклад, платформа навуковых артыкулаў.
  \item Аплата транзакцый. Напрыклад, платформа, на якой узаемадзенічаюць кліенты і творцы.
  \item Аплата прыладаў. Платформы збіраюць грошы за выкарыстанне прасунутых інструментаў.
  \item Выкарыстанне брэнда. Напрыклад, брэнда, які асацыюецца з пэўнай якасцю, накірункам, даверам.
\end{itemize}

Але ёсць накірунак, які больш факусуецца на не рынкавых механізмах, а на ўзаемнасці:

\begin{itemize}
  \item Сяброўскія ўнёскі або інтывідуальныя ахвяраванні. Чым \linebreak больш  тых, хто атрымлівае дадатковую вартасць, тым больш тых, хто верагодна можа падтрымаць праект.
  \item Свабодны кошт. Карыстальнікі плоцяць столькі колькі пажадаюць, як акт удзячнасці.
  \item Краўдфандынг. Вырошчванне супольнасці для таго, каб яна не толькі падтрымала, але і хацела поспеха праекту.
\end{itemize}

\subsubsection*{Злучаць людзей}

Акрамя выкарыстання Creative Commons, як прававога інструмента, з’яўляецца сродкам нагадвання, што за кожным творам стаіць чалавек, тым самым? ствараючы пачуццё абавязку. Далей пералічана шэраг ідэй, якія выкарыстоўваюцца ў новых бізнэс мадэлях:

\begin{itemize}
  \item Расказаваць гісторыі творчасці і не баяцца крытыкі, каманда не лашчоны брэнд.
  \item Адкрытасць да камунікацыі і адказнасць. Адказнасць ~--- не значыць канфармізм, а магчымасць аўдыторыі ведаць рашэнні і падставы.
  \item Далёка не ўсё людзі прымаюць рашэнне на падставе эканамічнай выгады і эгаізма. Напрыклад, шмат хто цэніць магчымасць папрацаваць разам або давер адзін да аднаго.
  \item Разглядаць людзей як індывідаў. Гэта значыць прамыя кантакты стваральніка і карыстальніка/фаната.
  \item Публічна заяўляўленне сваіх прынцыпаў. Выражэнне каштоўнасцяў павінна быць яўным, не хаваючы іх.
  \item Будаваць супольнасць. Ствараць практыку патрэбнасці і прыналежнасці да супольнай працы, і стварэння агульнага адзінадумцамі. Супольнасць забірае сілы, але аб’ядноўвае высілкі.
  \item Шэрынг ~--- значыць, даваць больш, чым браць. Адкрыты кантэнт ~--- гэта не поле халяўных рэсурсаў, і не проста постынг свайго кантэнта, але дадатковая вартасць.Напрыклад, пэўная культура зносінаў або падтрымкі ўдзельнікаў.
  \item Уключаць людзей у тое, што робіш. Малыя ўнёскі найбольш значныя.
\end{itemize}

Статыстыка паказвае, што ў свеце існуе больш за 1.4 міл’ярда твораў пад СС. Гэта насамрэч ураджваючая з’ява сведчыць пра наяўнасць сусветнай супольнасці, якая папаўняе менавіта “лічбавае супольнае”, як альтэрнатыўны шлях эканамічнага, культурнага і сацыяльнага развіцця двум крайнасцям: дзяржаўнай або рынкавай манаполіі ў кіраванні рэсурсамі.

\begin{thebibliography}{99}

\bibitem{Volchak1} Paul Stacey, The new world of digital commons, 2017 (Book. Made Wi creative Commons)
\bibitem{Volchak2} Sarah Pearson, How to be made with creative commons, 2017 (Book. Made Wh creative Commons)
 \bibitem{Volchak3} Report. State of the commons, 2016 ~--- \url{https://stateof.creativecommons.org/}
\end{thebibliography}

\end{document}
