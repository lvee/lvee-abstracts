\documentclass[10pt, a5paper]{article}
\usepackage{pdfpages}
\usepackage{parallel}
\usepackage[T2A]{fontenc}
\usepackage{ucs}
\usepackage[utf8x]{inputenc}
\usepackage[polish,english,russian]{babel}
\usepackage{hyperref}
\usepackage{rotating}
\usepackage[inner=2cm,top=1.8cm,outer=2cm,bottom=2.3cm,nohead]{geometry}
\usepackage{listings}
\usepackage{graphicx}
\usepackage{wrapfig}
\usepackage{longtable}
\usepackage{indentfirst}
\usepackage{array}
\newcolumntype{P}[1]{>{\raggedright\arraybackslash}p{#1}}
\frenchspacing
\usepackage{fixltx2e} %text sub- and superscripts
\usepackage{icomma} % коскі ў матэматычным рэжыме
\PreloadUnicodePage{4}

\newcommand{\longpage}{\enlargethispage{\baselineskip}}
\newcommand{\shortpage}{\enlargethispage{-\baselineskip}}

\def\switchlang#1{\expandafter\csname switchlang#1\endcsname}
\def\switchlangbe{
\let\saverefname=\refname%
\def\refname{Літаратура}%
\def\figurename{Іл.}%
}
\def\switchlangen{
\let\saverefname=\refname%
\def\refname{References}%
\def\figurename{Fig.}%
}
\def\switchlangru{
\let\saverefname=\refname%
\let\savefigurename=\figurename%
\def\refname{Литература}%
\def\figurename{Рис.}%
}

\hyphenation{admi-ni-stra-tive}
\hyphenation{ex-pe-ri-ence}
\hyphenation{fle-xi-bi-li-ty}
\hyphenation{Py-thon}
\hyphenation{ma-the-ma-ti-cal}
\hyphenation{re-ported}
\hyphenation{imp-le-menta-tions}
\hyphenation{pro-vides}
\hyphenation{en-gi-neering}
\hyphenation{com-pa-ti-bi-li-ty}
\hyphenation{im-pos-sible}
\hyphenation{desk-top}
\hyphenation{elec-tro-nic}
\hyphenation{com-pa-ny}
\hyphenation{de-ve-lop-ment}
\hyphenation{de-ve-loping}
\hyphenation{de-ve-lop}
\hyphenation{da-ta-ba-se}
\hyphenation{plat-forms}
\hyphenation{or-ga-ni-za-tion}
\hyphenation{pro-gramming}
\hyphenation{in-stru-ments}
\hyphenation{Li-nux}
\hyphenation{sour-ce}
\hyphenation{en-vi-ron-ment}
\hyphenation{Te-le-pathy}
\hyphenation{Li-nux-ov-ka}
\hyphenation{Open-BSD}
\hyphenation{Free-BSD}
\hyphenation{men-ti-on-ed}
\hyphenation{app-li-ca-tion}

\def\progref!#1!{\texttt{#1}}
\renewcommand{\arraystretch}{2} %Іначай формулы ў матрыцы зліпаюцца з лініямі
\usepackage{array}

\def\interview #1 (#2), #3, #4, #5\par{

\section[#1, #3, #4]{#1 -- #3, #4}
\def\qname{LVEE}
\def\aname{#1}
\def\q ##1\par{{\noindent \bf \qname: ##1 }\par}
\def\a{{\noindent \bf \aname: } \def\qname{L}\def\aname{#2}}
}

\def\interview* #1 (#2), #3, #4, #5\par{

\section*{#1\\{\small\rm #3, #4. #5}}

\def\qname{LVEE}
\def\aname{#1}
\def\q ##1\par{{\noindent \bf \qname: ##1 }\par}
\def\a{{\noindent \bf \aname: } \def\qname{L}\def\aname{#2}}
}

\begin{document}
\title{strace\footnote{\url{ldv@altlinux.org}, \url{http://lvee.org/ru/abstracts/227}}}
\author{Dmitry Levin, Moscow, Russian Federation}
\maketitle
\begin{abstract}
strace is a diagnostic, debugging and instructional userspace utility for monitoring interactions between processes and the Linux kernel, which include system calls, signal deliveries, and changes of process state. This paper gives an overview of strace development, lists the most noteworthy changes made in recent releases and the new features currently being worked on.
\end{abstract}

\section*{Немного цифр}

На LVEE 2013 три года назад был представлен обзор проекта strace с 1991 года
до середины 2013 года\cite{lvee2013}. Нынешний обзор охватывает значительно более короткий период с середины 2013 года по настоящее время, что позволяет рассказать о некоторых значимых событиях более подробно.

За минувшие три года было выпущено пять версий strace.
Периодичность выпусков постепенно сокращалась с одного раза в год
до одного раза в два-три месяца, одновременно с выпусками ядра Linux.

Интенсивность разработки за эти три года по сравнению с предыдущим
трехлетним периодом выросла в три раза, число авторов коммитов выросло в
полтора раза, а число авторов 80\% коммитов уменьшилось в два раза. Более
подробно статистическая информация о коммитах и их авторах представлена в
таблице.

\begin{tabular}{clrrrrrc}
\hline
\multicolumn{8}{c}{Число коммитов и их авторов по версиям} \\ \hline
& & \multicolumn{2}{c}{число} & \multicolumn{3}{c}{число} & число \\
дата & номер & \multicolumn{2}{c}{коммитов} & \multicolumn{3}{c}{авторов} & авторов 80\% \\
выпуска & версии & всего & в год & прежних & всего & в год & коммитов \\ \hline
13.04.2010 & 4.5.20 & & & & & & \\ \hline
15.03.2011 & 4.6 & 112 & 122 & 5 & 12 & 13 & 4 \\ \hline
02.05.2012 & 4.7 & 400 & 353 & 3 & 11 & 10 & 2 \\ \hline
03.06.2013 & 4.8 & 237 & 218 & 4 & 17 & 16 & 3 \\ \hline
\multicolumn{2}{l}{всего за период} & 749 & 238 & & 33 & 11 & 2 \\ \hline
\hline
15.08.2014 & 4.9 & 247 & 206 & 4 & 22 & 18 & 3 \\ \hline
06.03.2015 & 4.10 & 400 & 719 & 4 & 15 & 27 & 1 \\ \hline
21.12.2015 & 4.11 & 586 & 737 & 5 & 15 & 19 & 1 \\ \hline
31.05.2016 & 4.12 & 799 & 1804 & 5 & 16 & 36 & 1 \\ \hline
26.07.2016 & 4.13 & 182 & 1182 & 4 & 6 & 39 & 1 \\ \hline
\multicolumn{2}{l}{всего за период} & 2214 & 703 & 8 & 51 & 16 & 1 \\ \hline
\end{tabular} \bigskip

\section*{Что нового}

Из наиболее заметных изменений можно отметить следующие нововведения и
улучшения.

Добавлены новые параметры:
\begin{description}
\item[-k] (экспериментальный): печать стека вызовов функций трассируемых
процессов после каждого трассируемого системного вызова;
\item[-w] сбор статистики по времени, проведенному трассируемыми процессами
в системных вызовах;
\item[-yy] печать протокола и адреса, связанных с трассируемыми сокетами.
\end{description}

Реализована надежная трассировка процессов, использующих различные ABI ядра
Linux, в частности, процессов x86 и x32 на x86\_64.

Значительно расширена поддержка системного вызова ioctl.

Реализована поддержка всех системных вызовов, поддерживаемых ядром Linux 4.7.

Обнаружены и исправлены все известные на данный момент случаи не вполне
корректного декодирования системных вызовов.

Система тестирования strace, в которой три года назад было 5 тестов,
значительно доработана и на данный момент состоит из 314 тестов, покрывающих
более 72\% строк исходного кода.

\section*{В работе}
В рамках GSoC 2016 завершается работа над следующими проектами:
\begin{description}
\item[Structured output]: поддержка вывода в машиночитаемом виде в формате
JSON;
\item [Netlink socket parsers]: расширяемый парсер данных, передаваемых через
сокеты семейства netlink;
\item [Fault injection]: многофункциональный инжектор ошибок в системных
вызовах.
\end{description}

Подробнее об этих проектах можно узнать на сайте GSoC 2016 strace
projects\cite{gsoc2016}.

\section*{Не strace'ом единым}
В процессе доработки системы тестирования strace были обнаружены и
исправлены ошибки в других свободных проектах, в частности,
\begin{description}
\item [Linux kernel]:
	\begin{itemize}
	\item неправильная работа модуля unix\_diag\cite{diag};
	\item неправильный номер системного вызова fgetxattr на архитектуре
	sh64\cite{sh64};
	\item неправильный код возврата системного вызова personality на
	архитектуре sparc64\cite{sparc64};
	\item неправильный номер системного вызова restart\_syscall для x32
	ABI на архитектуре x86\_64\cite{x32};
	\item неправильная реализация системных вызовов preadv2 и pwritev2
	для x86 ABI на архитектуре x86\_64\cite{x86};
	\item kernel panic на архитектуре parisc\cite{parisc};
	\item kernel panic на ядрах RHEL5.
	\end{itemize}
\item [GNU libc]: некорректная обработка кода возврата системного вызова
personality\cite{glibc};
\item [musl libc]: некорректное поведение модификатора формата \%o в семействе
функций printf\cite{musl}.
\end{description}

\begin{thebibliography}{9}

\bibitem{lvee2013} LVEE 2013. strace from upstream point of view \\
\url{https://lvee.org/ru/abstracts/94}

\bibitem{gsoc2016} GSoC 2016 strace projects \\
\url{https://summerofcode.withgoogle.com/organizations/5106770607341568/#projects}

\bibitem{diag} unix\_diag: fix incorrect sign extension in
unix\_lookup\_by\_ino \\
\url{https://git.kernel.org/cgit/linux/kernel/git/torvalds/linux.git/commit/?id=b5f0549231ffb025337be5a625b0ff9f52b016f0}

\bibitem{sh64} sh64: fix \_\_NR\_fgetxattr \\
\url{https://git.kernel.org/cgit/linux/kernel/git/torvalds/linux.git/commit/?id=2d33fa1059da4c8e816627a688d950b613ec0474}

\bibitem{sparc64} sparc64: fix incorrect sign extension in
sys\_sparc64\_personality \\
\url{https://git.kernel.org/cgit/linux/kernel/git/torvalds/linux.git/commit/?id=525fd5a94e1be0776fa652df5c687697db508c91}

\bibitem{x32} x86/signal: Fix restart\_syscall number for x32 tasks \\
\url{https://git.kernel.org/cgit/linux/kernel/git/torvalds/linux.git/commit/?id=22eab1108781eff09961ae7001704f7bd8fb1dce}

\bibitem{x86} x86: Use compat version for preadv2 and pwritev2 \\
\url{https://git.kernel.org/cgit/linux/kernel/git/torvalds/linux.git/commit/?id=9a7a076e8e4ffcfec05e3cafe4c4e31d41ddbaa0}

\bibitem{parisc} parisc: fix a bug when syscall number of tracee is \_\_NR\_Linux\_syscalls \\
\url{https://git.kernel.org/cgit/linux/kernel/git/torvalds/linux.git/commit/?id=9a7a076e8e4ffcfec05e3cafe4c4e31d41ddbaa0}

\bibitem{glibc} Fix linux personality syscall wrapper \\
\url{https://sourceware.org/git/?p=glibc.git;a=commitdiff;h=e0043e17dfc52fe1702746543127cb4a87232bcd}

\bibitem{musl} fix printf regression with alt-form octal, default precision \\
\url{http://www.openwall.com/lists/musl/2016/08/04/1}

\end{thebibliography}