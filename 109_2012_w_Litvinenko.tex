\documentclass[10pt, a5paper]{article}
\usepackage[T2A]{fontenc}
\usepackage{ucs}
\usepackage[utf8x]{inputenc}
\usepackage[polish,english,russian]{babel}
\usepackage{hyperref}
\usepackage[inner=2cm,top=1.8cm,outer=2cm,bottom=2.3cm,nohead]{geometry}
\usepackage{listings}
\usepackage{graphicx}
\usepackage{wrapfig}
\usepackage{longtable}
\usepackage{indentfirst}
\frenchspacing
\usepackage{fixltx2e} %text sub- and superscripts
\usepackage{icomma} % коскі ў матэматычным рэжыме
\PreloadUnicodePage{4}

\newcommand{\longpage}{\enlargethispage{\baselineskip}}
\newcommand{\shortpage}{\enlargethispage{-\baselineskip}}

\def\switchlang#1{\expandafter\csname switchlang#1\endcsname}
\def\switchlangbe{
\let\saverefname=\refname%
\def\refname{Літаратура}%
\def\figurename{Іл.}%
}
\def\switchlangen{
\let\saverefname=\refname%
\def\refname{References}%
\def\figurename{Fig.}%
}
\def\switchlangru{
\let\saverefname=\refname%
\let\savefigurename=\figurename%
\def\refname{Литература}%
\def\figurename{Рис.}%
}

\hyphenation{admi-ni-stra-tive}
\hyphenation{ex-pe-ri-ence}
\hyphenation{fle-xi-bi-li-ty}
\hyphenation{Py-thon}
\hyphenation{ma-the-ma-ti-cal}
\hyphenation{re-ported}
\hyphenation{imp-le-menta-tions}
\hyphenation{pro-vides}
\hyphenation{en-gi-neering}
\hyphenation{com-pa-ti-bi-li-ty}
\hyphenation{im-pos-sible}
\hyphenation{desk-top}
\hyphenation{elec-tro-nic}
\hyphenation{com-pa-ny}
\hyphenation{de-ve-lop-ment}
\hyphenation{de-ve-loping}
\hyphenation{de-ve-lop}
\hyphenation{da-ta-ba-se}
\hyphenation{plat-forms}
\hyphenation{or-ga-ni-za-tion}
\hyphenation{pro-gramming}
\hyphenation{in-stru-ments}
\hyphenation{Li-nux}
\hyphenation{en-vi-ron-ment}
\hyphenation{Te-le-pathy}
\hyphenation{Li-nux-ov-ka}

\def\progref!#1!{\texttt{#1}}
\renewcommand{\arraystretch}{2} %Іначай формулы ў матрыцы зліпаюцца з лініямі
\usepackage{array}

\def\interview #1 (#2), #3, #4, #5\par{

\section[#1, #3, #4]{#1, #5}
\def\qname{LVEE}
\def\aname{#1}
\def\q ##1\par{{\noindent \bf \qname: ##1 }\par}
\def\a{{\noindent \bf \aname: } \def\qname{L}\def\aname{#2}}
}


\begin{document}

\switchlang{be}
\title{Асаблівасці выкарыстання сістэм кантролю версій для падрыхтоўкі навуковых публікацый}%\footnote{Текст данных и последующих тезисов, кроме специально оговоренных случаев, доступен под лицензией Creative Commons Attribution-ShareAlike 3.0}

\author{Антон Літвіненка\footnote{Кіеў, Украiна}}
\maketitle

\begin{abstract}
The article discusses version control systems (VCS) usage to overcome typical issues at preparing scientific publications, and the peculiarities of this type of VCS usage. Author emphasizes the advantages and disadvantages of centeralised and distributed version control systems with respect to scientific publications \linebreak preparation features, and gives some proposals on exact VCS types, VCS hosting services, front-end programs etc.
\end{abstract}

Падрыхтоўка навуковых прац да публікацыі з'яўляецца абавязковым этапам навуковага даследавання. Навуковая праца патрабуе арганізаванай супольнай працы ўсіх суаўтараў (для прац у галіне хіміі характэрна актыўнае выкарыстанне супольных даследаванняў між рознымі аддзеламі, арганізацыямі і краінамі, праз што ў падрыхтоўцы працы часта бярэ ўдзел 5--10 суаўтараў). Для дакладных навук тыповым з'яўляецца выкарыстанне вялікай колькасці дадатковых дадзеных: рысункаў, файлаў з лічбавымі формамі апісання спектраў, крысталічных структур, табліц з вынікамі эксперыментаў, файлаў з вынікамі праграмнай апрацоўкі эксперыментальных дадзеных альбо з вынікамі разлікаў і пад. Мэтазгодным з'яўляецца таксама захоўванне побач калекцый спасылак (напрыклад, базаў BibTeX), некаторых арыгінальных публікацый і пад. Навуковая праца праходзіць праз вялікую колькасць рэдагаванняў цягам пэўнага часу (ад некалькіх тыдняў да гадоў), некаторыя з якіх могуць радыкальна яе змяняць.

Такім чынам, перад аўтарамі паўстаюць наступныя праблемы:

\begin{enumerate}
  \item Неабходна выконваць рэзервовае капіраванне ўсяго працоўнага каталога. З улікам колькасці рэдагаванняў ды аб'ёмаў дадзеных поўная гісторыя публікацыі можа займаць заўважную колькасць месца на дыску.
  \item Рэзервовае капіраванне элементарнымі сродкамі не спрыяе напісанню дакладных каментарыяў наконт сутнасці зробленых зменаў, што робіць складаным пошук па гісторыі рэдагаванняў.
  \item Бязладнасць працэсу абмену дадзенымі, парушэнне аднастайнай сістэмы нумарацыі рэдагаванняў вядзе да рассінхранізацыі гісторыі рэдагаванняў на працоўных ды хатніх \linebreak камп'ютарах розных суаўтараў "--- такім чынам, частка гісторыі становіцца недаступнай, а гісторыя ўвогуле "--- хаатычнай і неарганізаванай.
  \item Змены файлаў з дадатковымі дадзенымі не заўсёды звязаныя са зменамі асноўнага тэксту публікацыі, і адказ на пытанне <<якому рэдагаванню аднаго файла адпавядае гэтая версія другога файла>> можа быць нетрывіяльным.
\end{enumerate}

Для рашэння аналагічных праблем у галіне распрацоўкі праграмнага забеспячэння актыўна ўжываюцца сістэмы кантролю версій (СКВ, version control systems, VCS). Мэтазгодным падаецца выкарыстаць аналагічны падыход і для падрыхтоўкі навуковых публікацый. Гэта дазваляе не дубляваць інфармацыю (захоўваючы толькі розніцу між версіямі), зрабіць больш арганізаванай гісторыю рэдагаванняў, цэнтралізаваць абмен рэдагаваннямі.

Разам з тым, у задачы падрыхтоўкі публікацыі ёсць некаторыя асаблівасці ў параўнанні з задачай распрацоўкі ПЗ:

\begin{enumerate}
  \item Рэдагуюцца пераважна двайковыя дадзеныя (асабліва ў выпадку, калі тэкст публікацыі рыхтуецца ў рэдактарах \linebreak WYSIWYG).
  \item Навуковая праца не прадугледжвае сталага развіцця, яна канчаецца апублікаваннем.
  \item Навуковай працы не ўласцівае актыўнае галінаванне, таму большасць магчымасцяў СКВ, звязаных з галінаваннем, не з'яўляюцца істотнымі для гэтай задачы.
  \item Роля часткі суаўтараў часта зводзіцца выключна да <<плённай дыскусіі>> і выказвання заўваг, актыўна рэдагуюць звычайна ўсяго некалькі людзей.
  \item Навуковая праца мусіць заставацца закрытай да моманту яе апублікавання.
\end{enumerate}

Практычныя спробы рэалізаваць падрыхтоўку навуковае працы да публікацыі былі выкананыя (і выконваюцца цяпер) з выкарыстаннем СКВ SVN і Mercurial (як прыклады цэнтралізаванай і дэцэнтралізаванай СКВ, адпаведна).

Галоўныя перавагі цэнтралізаванай СКВ у працэсе падрыхтоўкі навуковых прац:

\begin{enumerate}
  \item Вялікая колькасць двайковых дадзеных схіляе да мадэлі супольнай працы, якая прадугледжвае блакіраванне файлаў \linebreak (locking) "--- гэта магчыма толькі ў цэнтралізаванай СКВ.
  \item Мадэль працы СКВ прасцей для разумення неадмыслоўцамі.
\end{enumerate}

Галоўныя перавагі дэцэнтралізаванай СКВ у працэсе падрыхтоўкі навуковых прац:

\begin{enumerate}
  \item Незалежнасць ад наяўнасці і якасці канала сувязі.
  \item Большасць аперацый адбываецца лакальна, і, праз тое, хутка.
  \item Наяўнасць Інтэрнэт-сэрвісаў, што надаюць паслугі хостынгу дэцэнтралізаваных СКВ для закрытых прац малых груп суаўтараў бясплатна (напрыклад, bitbucket.org) "--- для SVN знайсці такія сэрвісы цяжэй.
\end{enumerate}

Асаблівасці задачы цалкам дапускаюць працэс выкарыстання СКВ выключна з дапамогай графічнага інтэрфейсу (напрыклад, RapidSVN, TortoiseSVN, TortoiseHg). Для стварэння бясплатных рэпазіторыяў могуць быць выкарыстаныя сэрвісы bitbucket.org, xp-dev.com і г.д.

Такім чынам, сістэмы кантролю версій, выкарыстанне якіх \linebreak з'яўляецца тыповым для задач распрацоўкі праграмнага забеспячэння, могуць быць плённа выкарыстаныя для задач падрыхтоўкі навуковых прац да публікацыі, хаця такое выкарыстанне СКВ мае шэраг асаблівасцяў, якія ўплываюць як на выбар СКВ, так і на працэс працы з ёй.



\end{document}




