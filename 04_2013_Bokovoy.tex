\documentclass[10pt, a5paper]{article}
\usepackage{pdfpages}
\usepackage{parallel}
\usepackage[T2A]{fontenc}
\usepackage{ucs}
\usepackage[utf8x]{inputenc}
\usepackage[polish,english,russian]{babel}
\usepackage{hyperref}
\usepackage{rotating}
\usepackage[inner=2cm,top=1.8cm,outer=2cm,bottom=2.3cm,nohead]{geometry}
\usepackage{listings}
\usepackage{graphicx}
\usepackage{wrapfig}
\usepackage{longtable}
\usepackage{indentfirst}
\usepackage{array}
\newcolumntype{P}[1]{>{\raggedright\arraybackslash}p{#1}}
\frenchspacing
\usepackage{fixltx2e} %text sub- and superscripts
\usepackage{icomma} % коскі ў матэматычным рэжыме
\PreloadUnicodePage{4}

\newcommand{\longpage}{\enlargethispage{\baselineskip}}
\newcommand{\shortpage}{\enlargethispage{-\baselineskip}}

\def\switchlang#1{\expandafter\csname switchlang#1\endcsname}
\def\switchlangbe{
\let\saverefname=\refname%
\def\refname{Літаратура}%
\def\figurename{Іл.}%
}
\def\switchlangen{
\let\saverefname=\refname%
\def\refname{References}%
\def\figurename{Fig.}%
}
\def\switchlangru{
\let\saverefname=\refname%
\let\savefigurename=\figurename%
\def\refname{Литература}%
\def\figurename{Рис.}%
}

\hyphenation{admi-ni-stra-tive}
\hyphenation{ex-pe-ri-ence}
\hyphenation{fle-xi-bi-li-ty}
\hyphenation{Py-thon}
\hyphenation{ma-the-ma-ti-cal}
\hyphenation{re-ported}
\hyphenation{imp-le-menta-tions}
\hyphenation{pro-vides}
\hyphenation{en-gi-neering}
\hyphenation{com-pa-ti-bi-li-ty}
\hyphenation{im-pos-sible}
\hyphenation{desk-top}
\hyphenation{elec-tro-nic}
\hyphenation{com-pa-ny}
\hyphenation{de-ve-lop-ment}
\hyphenation{de-ve-loping}
\hyphenation{de-ve-lop}
\hyphenation{da-ta-ba-se}
\hyphenation{plat-forms}
\hyphenation{or-ga-ni-za-tion}
\hyphenation{pro-gramming}
\hyphenation{in-stru-ments}
\hyphenation{Li-nux}
\hyphenation{sour-ce}
\hyphenation{en-vi-ron-ment}
\hyphenation{Te-le-pathy}
\hyphenation{Li-nux-ov-ka}
\hyphenation{Open-BSD}
\hyphenation{Free-BSD}
\hyphenation{men-ti-on-ed}
\hyphenation{app-li-ca-tion}

\def\progref!#1!{\texttt{#1}}
\renewcommand{\arraystretch}{2} %Іначай формулы ў матрыцы зліпаюцца з лініямі
\usepackage{array}

\def\interview #1 (#2), #3, #4, #5\par{

\section[#1, #3, #4]{#1 -- #3, #4}
\def\qname{LVEE}
\def\aname{#1}
\def\q ##1\par{{\noindent \bf \qname: ##1 }\par}
\def\a{{\noindent \bf \aname: } \def\qname{L}\def\aname{#2}}
}

\def\interview* #1 (#2), #3, #4, #5\par{

\section*{#1\\{\small\rm #3, #4. #5}}

\def\qname{LVEE}
\def\aname{#1}
\def\q ##1\par{{\noindent \bf \qname: ##1 }\par}
\def\a{{\noindent \bf \aname: } \def\qname{L}\def\aname{#2}}
}


\begin{document}

\title{FreeIPA: как распутать клубок безопасности?}%\footnote{Текст данных и последующих тезисов, кроме специально оговоренных случаев, доступен под лицензией Creative Commons Attribution-ShareAlike 3.0}

\author{Александр Боковой\footnote{Эспоо, Финляндия; \url{ab@samba.org}}}
\maketitle

\begin{abstract}
FreeIPA is an open source project that eases use of a secure and reliable GNU/Linux infrastructure. FreeIPA provides central \linebreak management of Kerberos credentials, LDAP data store, DNS management, Certificate Authority, and resource use control with  convenient and easy Web-based and command line interfaces.
\end{abstract}

Проект FreeIPA (\url{http://freeipa.org}) позволяет развернуть и централизовано управлять защищенной инфраструктурой для предприятия на основе GNU/Linux. FreeIPA интегрирует Kerberos, \linebreak LDAP, DNS и собственный центр сертификации, что дает дает возможность  простым и наглядным способом через веб- и интерфейс командной строки управлять доступными ресурсами.

В отличие от традиционных подходов, направленных на упрощение администрирования одного сервера, FreeIPA решает комплексную задачу. Репликация данных между серверами делает систему более устойчивой к повышенным нагрузкам и сложным конфигурациям. Интеграция на клиентской стороне позволяет обеспечить полный спектр возможностей "--- от оффлайновой авторизации до централизованного контроля ресурсов посредством гибких правил.

В Fedora 19 проект FreeIPA расширил свои возможности,  добавив двухфакторную авторизацию пользователей и плотную интеграцию с существующими доменами Active Directory.

Двухфакторная авторизация на основе протокола Kerberos стала доступной благодаря усовершенствованиям, вошедшим в MIT Kerberos 1.12
после двух лет работы "--- вначале над протоколом, описанным в RFC 6560 (\url{http://tools.ietf.org/html/rfc6560}), а затем и непосредственно над инфраструктурой Kerberos FAST \linebreak (\url{http://k5wiki.kerberos.org/wiki/Projects/FAST}). FreeIPA представляет два вида двухфакторной авторизации:

\begin{itemize}
  \item внутренний OTP, реализованный средствами FreeIPA;
  \item прокси к внешнему серверу RADIUS.
\end{itemize}

В обоих случаях введенные пользователем PIN и переменный код передаются по сети в зашифрованном и подписанном канале соединения Kerberos. FreeIPA является первой в мире системой, представляющей такую возможность. Взаимодействие с внешним сервером RADIUS позволяет прозрачно интегрироваться с любыми источниками авторизации "--- для пользователя все выглядит как стандартная двухфакторная схема.

Для внутреннего OTP поддерживаются токены TOTP, совместимые, например, с Google Authenticator.

Интеграция с доменами Active Directory позволяет прозрачным образом расширять используемые в организации ресурсы системами на основе
GNU/Linux. Для Active Directory в этом случае FreeIPA выглядит как отдельный корневой домен, с которым могут быть установлены доверительные отношения. Контроллеры доменов Active Directory могут выдавать билеты Kerberos своим пользователям для доступа к ресурсам в домене FreeIPA, а на стороне FreeIPA корректно обрабатывается информация о членстве в группах пользователей из Active Directory.

Для реализации этого проекта потребовалась активная интеграция между
FreeIPA и Samba. Ведется также работа над расширением Samba AD DC с тем, чтобы получить полностью открытую инфраструктуру организации.


\end{document}




