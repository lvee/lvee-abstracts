\documentclass[10pt, a5paper]{article}
\usepackage{pdfpages}
\usepackage{parallel}
\usepackage[T2A]{fontenc}
\usepackage{ucs}
\usepackage[utf8x]{inputenc}
\usepackage[polish,english,russian]{babel}
\usepackage{hyperref}
\usepackage{rotating}
\usepackage[inner=2cm,top=1.8cm,outer=2cm,bottom=2.3cm,nohead]{geometry}
\usepackage{listings}
\usepackage{graphicx}
\usepackage{wrapfig}
\usepackage{longtable}
\usepackage{indentfirst}
\usepackage{array}
\newcolumntype{P}[1]{>{\raggedright\arraybackslash}p{#1}}
\frenchspacing
\usepackage{fixltx2e} %text sub- and superscripts
\usepackage{icomma} % коскі ў матэматычным рэжыме
\PreloadUnicodePage{4}

\newcommand{\longpage}{\enlargethispage{\baselineskip}}
\newcommand{\shortpage}{\enlargethispage{-\baselineskip}}

\def\switchlang#1{\expandafter\csname switchlang#1\endcsname}
\def\switchlangbe{
\let\saverefname=\refname%
\def\refname{Літаратура}%
\def\figurename{Іл.}%
}
\def\switchlangen{
\let\saverefname=\refname%
\def\refname{References}%
\def\figurename{Fig.}%
}
\def\switchlangru{
\let\saverefname=\refname%
\let\savefigurename=\figurename%
\def\refname{Литература}%
\def\figurename{Рис.}%
}

\hyphenation{admi-ni-stra-tive}
\hyphenation{ex-pe-ri-ence}
\hyphenation{fle-xi-bi-li-ty}
\hyphenation{Py-thon}
\hyphenation{ma-the-ma-ti-cal}
\hyphenation{re-ported}
\hyphenation{imp-le-menta-tions}
\hyphenation{pro-vides}
\hyphenation{en-gi-neering}
\hyphenation{com-pa-ti-bi-li-ty}
\hyphenation{im-pos-sible}
\hyphenation{desk-top}
\hyphenation{elec-tro-nic}
\hyphenation{com-pa-ny}
\hyphenation{de-ve-lop-ment}
\hyphenation{de-ve-loping}
\hyphenation{de-ve-lop}
\hyphenation{da-ta-ba-se}
\hyphenation{plat-forms}
\hyphenation{or-ga-ni-za-tion}
\hyphenation{pro-gramming}
\hyphenation{in-stru-ments}
\hyphenation{Li-nux}
\hyphenation{sour-ce}
\hyphenation{en-vi-ron-ment}
\hyphenation{Te-le-pathy}
\hyphenation{Li-nux-ov-ka}
\hyphenation{Open-BSD}
\hyphenation{Free-BSD}
\hyphenation{men-ti-on-ed}
\hyphenation{app-li-ca-tion}

\def\progref!#1!{\texttt{#1}}
\renewcommand{\arraystretch}{2} %Іначай формулы ў матрыцы зліпаюцца з лініямі
\usepackage{array}

\def\interview #1 (#2), #3, #4, #5\par{

\section[#1, #3, #4]{#1 -- #3, #4}
\def\qname{LVEE}
\def\aname{#1}
\def\q ##1\par{{\noindent \bf \qname: ##1 }\par}
\def\a{{\noindent \bf \aname: } \def\qname{L}\def\aname{#2}}
}

\def\interview* #1 (#2), #3, #4, #5\par{

\section*{#1\\{\small\rm #3, #4. #5}}

\def\qname{LVEE}
\def\aname{#1}
\def\q ##1\par{{\noindent \bf \qname: ##1 }\par}
\def\a{{\noindent \bf \aname: } \def\qname{L}\def\aname{#2}}
}

\begin{document}
\title{О том как маленький opensource"=проект меняет жизнь большой компании}
\author{Павел Емельянов, Москва, РФ\footnote{\url{xemul@openvz.org}, \url{http://lvee.org/en/abstracts/149}}}
\maketitle
\begin{abstract}
CRIU is the open source checkpoint"=restore project of the Odin (former Parallels) company. It provides basis for containers live migration, seamless kernel update and a set of other features. In this talk I will present the current state of the project, describe the community that has grown around it and show how the open development model of a small project affected the life of the whole company.
\end{abstract}
OpenVZ "--- это открытый проект компании Odin, дающий пользователям простую, но надёжную контейнерную платформу. Неотъемлемой частью проекта с самого его начала является возможность живой миграции контейнеров, для чего используется технология <<снятия контрольных точек>> (checkpoint) и восстановления из них (restore). В процессе продвижения контейнерной технологии в массы инженеры компании были вынуждены переписать C/R подсистему практически с нуля и в другой парадигме "--- вместо ядерного модуля checkpoint"=restore теперь делается силами процесса с использованием открытых ядерных интерфейсов. Вместе со сменой <<адресного пространства>> кода был изменён и подход к разработке "--- CRIU это 100\% открытый проект без скрытых компонент и без диктатуры инженеров Odin при принятии архитектурных и технических решений.

За 4 года своего существования CRIU разросся до 100 тысяч строк кода и, что ещё важнее, достиг определённых успехов в социальной сфере.

Во"=первых, проект завоевал признание в сообществе Linux kernel, куда изначально предлагалась реализация технологии, и теперь достаточным поводом для начала обсуждения ядерных патчей может служить простая фраза: <<это надо для CRIU>>.

Во"=вторых, CRIU <<подружился>> с другими проектами, например Docker и LXC, так что теперь идеи и улучшения мы получаем не только от клиентов Odin"=а.

В"=третьих, CRIU оброс небольшим сообществом, которое уже принесло свои плоды "--- портирование на архитектуры AArch64 и Power, интеграция с LXC и Docker и много другого было сделано не нами, но и для нас в том числе.

И, наконец, проект оказал сильное влияние на весь процесс разработки компании Odin. Недавно начатая новая жизнь OpenVZ планировалась с учетом приобретённого в CRIU опыта ведения открытых проектов.

\end{document}
