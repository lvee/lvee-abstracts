\documentclass[10pt, a5paper]{article}
\usepackage{pdfpages}
\usepackage{parallel}
\usepackage[T2A]{fontenc}
\usepackage{ucs}
\usepackage[utf8x]{inputenc}
\usepackage[polish,english,russian]{babel}
\usepackage{hyperref}
\usepackage{rotating}
\usepackage[inner=2cm,top=1.8cm,outer=2cm,bottom=2.3cm,nohead]{geometry}
\usepackage{listings}
\usepackage{graphicx}
\usepackage{wrapfig}
\usepackage{longtable}
\usepackage{indentfirst}
\usepackage{array}
\newcolumntype{P}[1]{>{\raggedright\arraybackslash}p{#1}}
\frenchspacing
\usepackage{fixltx2e} %text sub- and superscripts
\usepackage{icomma} % коскі ў матэматычным рэжыме
\PreloadUnicodePage{4}

\newcommand{\longpage}{\enlargethispage{\baselineskip}}
\newcommand{\shortpage}{\enlargethispage{-\baselineskip}}

\def\switchlang#1{\expandafter\csname switchlang#1\endcsname}
\def\switchlangbe{
\let\saverefname=\refname%
\def\refname{Літаратура}%
\def\figurename{Іл.}%
}
\def\switchlangen{
\let\saverefname=\refname%
\def\refname{References}%
\def\figurename{Fig.}%
}
\def\switchlangru{
\let\saverefname=\refname%
\let\savefigurename=\figurename%
\def\refname{Литература}%
\def\figurename{Рис.}%
}

\hyphenation{admi-ni-stra-tive}
\hyphenation{ex-pe-ri-ence}
\hyphenation{fle-xi-bi-li-ty}
\hyphenation{Py-thon}
\hyphenation{ma-the-ma-ti-cal}
\hyphenation{re-ported}
\hyphenation{imp-le-menta-tions}
\hyphenation{pro-vides}
\hyphenation{en-gi-neering}
\hyphenation{com-pa-ti-bi-li-ty}
\hyphenation{im-pos-sible}
\hyphenation{desk-top}
\hyphenation{elec-tro-nic}
\hyphenation{com-pa-ny}
\hyphenation{de-ve-lop-ment}
\hyphenation{de-ve-loping}
\hyphenation{de-ve-lop}
\hyphenation{da-ta-ba-se}
\hyphenation{plat-forms}
\hyphenation{or-ga-ni-za-tion}
\hyphenation{pro-gramming}
\hyphenation{in-stru-ments}
\hyphenation{Li-nux}
\hyphenation{sour-ce}
\hyphenation{en-vi-ron-ment}
\hyphenation{Te-le-pathy}
\hyphenation{Li-nux-ov-ka}
\hyphenation{Open-BSD}
\hyphenation{Free-BSD}
\hyphenation{men-ti-on-ed}
\hyphenation{app-li-ca-tion}

\def\progref!#1!{\texttt{#1}}
\renewcommand{\arraystretch}{2} %Іначай формулы ў матрыцы зліпаюцца з лініямі
\usepackage{array}

\def\interview #1 (#2), #3, #4, #5\par{

\section[#1, #3, #4]{#1 -- #3, #4}
\def\qname{LVEE}
\def\aname{#1}
\def\q ##1\par{{\noindent \bf \qname: ##1 }\par}
\def\a{{\noindent \bf \aname: } \def\qname{L}\def\aname{#2}}
}

\def\interview* #1 (#2), #3, #4, #5\par{

\section*{#1\\{\small\rm #3, #4. #5}}

\def\qname{LVEE}
\def\aname{#1}
\def\q ##1\par{{\noindent \bf \qname: ##1 }\par}
\def\a{{\noindent \bf \aname: } \def\qname{L}\def\aname{#2}}
}


\begin{document}

\title{Применение свободного ПО для исследования поведения нелинейных динамических систем}%\footnote{Текст данных и последующих тезисов, кроме специально оговоренных случаев, доступен под лицензией Creative Commons Attribution-ShareAlike 3.0}

\author{Александра Кононова, Алексей Городилов\footnote{Москва, Зеленоград, РФ}}
\maketitle

\begin{abstract}
The article discusses a possibility of using free software tools for research of non-linear dynamic systems, arising problems and their resolutions.
\end{abstract}

В настоящее время существует множество свободных систем, которые могут использоваться при исследовании нелинейных динамических систем. Для выполнения математических вычислений наиболее известны: пакет GNU Octave, созданный как свободная альтернатива известной системы MATLAB, так что язык Octave в основном совместим с MATLAB; пакет Scilab, имеющий схожий, но несовместимый с MATLAB язык программирования; существуют узкоспециализированные программы, такие, например, как Xppaut; а также множество интересных, но плохо документированных или малоизвестных пока пакетов, таких как Euler, model-builder, freemat.

Динамика нелинейной системы чаще всего описывается в виде системы обыкновенных дифференциальных уравнений (ОДУ):

$$
\left\{ 
\begin{array}{lll}
\dot x_1 &=& F_1(x_1, x_2,   \ldots  x_n )  \\
\dot x_2 &=& F_2(x_1, x_2,   \ldots  x_n )  \\
\ldots \\
\dot x_n &=& F_n(x_1, x_2,  \ldots  x_n )  \\
\end{array}
\right.
$$

\subsection*{GNU Octave}

В Octave существует множество функций для решения систем обыкновенных дифференциальных уравнений \cite{Kon1}. Две из них встроенные "--- это lsode для решения системы ОДУ и lsode\_options для задания опций работы lsode. Здесь, как и в MATLAB, существует семейство функций ode* для решения систем ОДУ различными методами. В Debian они доступны после установки пакета octave-odepkg. Наиболее часто используемые среди них "--- функции ode23 и ode45 (методы Рунге-Кутты). Для решения жёстких ОДУ, в отличие от MATLAB, используются функции ode5r и ode2r, а не семейство ode*s.

Для визуализации фазовой траектории на плоскости используется функция plot, в трёхмерном случае "--- plot3. Для того, чтобы получившийся график можно было вращать мышкой, перед запуском Octave необходимо выполнить в используемой оболочке команду \verb@GNUTERM=wxt@

Рассмотрим пример "--- исследование модели развития вуза. Результатом является траектория системы "--- график её развития.
Для расчёта поведения динамической модели развития вуза \cite{Kon2} были заданы уравнения:
\begin{verbatim}
function xdot=F(t,x);
    Ue = 0.62;
    tR = .1;
    AR = 0.5;
    tS = 1;
    tU = .2;
    AU = 1;
    AS = 2 / (Ue*AR);

xdot=[

(  -x(1) + AR*x(2)  ) / tR;

(  -x(2) + AS*x(1)*x(3)  ) / tS;

(  Ue--x(3)--AU*x(1)*x(2)  ) / tU

];

endfunction;
\end{verbatim}
расчёт траектории при интересующих нас начальных условиях (точка $(0.5, 0.1, 0.1)$) на отрезке времени от $0$ до $12$:

\verb![t,x] = ode45(@F,[0 12],[0.5 0.1 0.1]);!

визуализация:

\verb@plot3(x(:,1), x(:,2), x(:,3) );@

На экране появится трёхмерный график "--- траектория развития системы.

\subsection*{Scilab}

В Scilab для решения системы ОДУ предусмотрена функция ode. Метод решения задаётся её первым необязательным параметром \cite{Kon3}.

Для визуализации рассчитанной фазовой траектории используются функции plot/plot2d для двумерного случая и param3d/param3d1 для трёхмерного.

Для расчёта вышеупомянутой динамической модели вуза задаются уравнения (совпадение с GNU Octave вследствие отсутствия матричных операций):
\begin{verbatim}
function xdot=F(t,x);
    Ue = 0.62;
    tR = .1;
    AR = 0.5;
    tS = 1;
    tU = .2;
    AU = 1;
    AS = 2 / (Ue*AR);

xdot=[

(  -x(1) + AR*x(2)  ) / tR;

(  -x(2) + AS*x(1)*x(3)  ) / tS;

(  Ue--x(3)--AU*x(1)*x(2)  ) / tU

];

endfunction;
\end{verbatim}

Для расчёта траектории необходимо:
задать вектор"=столбец $\vect{x_0}$ начальных условий, начальный момент времени $t_0$ и вектор моментов времени $\vect t$, в которых требуется рассчитать траекторию (может быть и столбцом, и строкой):

\begin{verbatim}
x0 = [0.5; 0.1; 0.1];
t0 = 0;
t=t0:0.1:12;
\end{verbatim}

рассчитать траекторию функцией ode:
\verb@x=ode(x0, t0, t, F);@
количество строк вектора $\vect{x}$ совпадает с количеством строк $\vect{x_0}$, а количество столбцов "--- с длиной $\vect t$, т.\,е., для построения траектории $\vect{x}$ делится на строки:
\verb@param3d(x(:,1), x(:,2), x(:,3) );@

На экране появится рассчитанная траектория.

\subsection*{Особенности экспорта в текстовый документ}

Возможности Scilab и GNU Octave в построении графиков весьма обширны, но не всегда достаточны. Поэтому часто возникает необходимость сохранить траекторию в текстовый файл, который затем может быть вставлен в рисунок TikZ/PGF командой:
\verb@\draw plot[smooth] file {plots/sample.trj};@

В GNU Octave сохранить матрицу координат в файл можно командой save:
\verb@save sample.trj x;@
в Scilab "--- функцией write:\linebreak
\verb@write('sample.trj',x);@

\subsection*{Xppaut}

Xppaut является специализированной программой для исследования нелинейных динамических систем. Она распространяется по лицензии GNU GPL и доступна в репозиториях Debian, начиная с Debian Wheezy (testing). Предназначена для численного исследования различных видов разностных и дифференциальных уравнений.
При старте программы необходимо задать исследуемую систему \cite{Kon4}. Это делается с помощью текстового файла, формат которого описан в документации программы по адресу \url{http://www.math.pitt.edu/bard/xpp/help/xppodes.html}.

Рассмотрим для примера классическую модель Лотки "--- Вольтерра. Она представляется файлом LotkaVolterra.ode:


\begin{verbatim}
@# LotkaVolterra.ode@
x'= a*x - b*x*y
y'= d*x*y - g*y
par a=1,b=2,d=3.7,g=1
init x=1,y=0.1
@ total=200
@ xp=x,yp=y,xlo=-2,xhi=5,ylo=-4,yhi=5
done
\end{verbatim}

Кроме вида уравнений, необходимо задать значения параметров, начальные условия и параметры отображения "--- в данном случае указано, что по оси абсцисс указывается переменная $x$, по оси ординат "--- $y$; заданы границы для каждой из осей. Команда @xppaut LotkaVolterra.ode@ позволит исследовать поведение системы с заданными параметрами.
При запуске xppaut без указания системы будет открыт диалог выбора файла.

На основании личного опыта работы в области исследования нелинейных динамических систем как с проприетарным, так и свободным ПО, можно сделать вывод, что возможности СПО в данной области нисколько не уступают возможностям проприетарных программ.


\begin{thebibliography}{9}

\bibitem{Kon1} Алексеев Е. Р., Чеснокова О. В. GNU Octave для студентов и преподавателей. Донецк.: ДонНТУ, Технопарк ДонНТУ УНИТЕХ, 2011. 332 с.

\bibitem{Kon2} Акулёнок М. В., Кононова А. И., Трояновский В. М. Исследование динамики сложной организационной структуры на примере вуза // Изв. вузов. Электроника (ВАК). 2011. № 1(87). С. 70–77.

\bibitem{Kon3} Алексеев Е. Р., Чеснокова О. В., Рудченко Е. А. Scilab. Решение инженерных и математических задач М.: ALT Linux; БИНОМ. Лаборатория знаний, 2008. 269 с.

\bibitem{Kon4} Xppaut online documentation  \url{http://www.math.pitt.edu/ bard/xpp/help/xpphelp.html}

\end{thebibliography}

\end{document}




