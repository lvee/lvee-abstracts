\documentclass[10pt, a5paper]{article}
\usepackage[T2A]{fontenc}
\usepackage{ucs}
\usepackage[utf8x]{inputenc}
\usepackage[polish,english,russian]{babel}
\usepackage{hyperref}
\usepackage[inner=2cm,top=1.8cm,outer=2cm,bottom=2.3cm,nohead]{geometry}
\usepackage{listings}
\usepackage{graphicx}
\usepackage{wrapfig}
\usepackage{longtable}
\usepackage{indentfirst}
\frenchspacing
\usepackage{fixltx2e} %text sub- and superscripts
\usepackage{icomma} % коскі ў матэматычным рэжыме
\PreloadUnicodePage{4}

\newcommand{\longpage}{\enlargethispage{\baselineskip}}
\newcommand{\shortpage}{\enlargethispage{-\baselineskip}}

\def\switchlang#1{\expandafter\csname switchlang#1\endcsname}
\def\switchlangbe{
\let\saverefname=\refname%
\def\refname{Літаратура}%
\def\figurename{Іл.}%
}
\def\switchlangen{
\let\saverefname=\refname%
\def\refname{References}%
\def\figurename{Fig.}%
}
\def\switchlangru{
\let\saverefname=\refname%
\let\savefigurename=\figurename%
\def\refname{Литература}%
\def\figurename{Рис.}%
}

\hyphenation{admi-ni-stra-tive}
\hyphenation{ex-pe-ri-ence}
\hyphenation{fle-xi-bi-li-ty}
\hyphenation{Py-thon}
\hyphenation{ma-the-ma-ti-cal}
\hyphenation{re-ported}
\hyphenation{imp-le-menta-tions}
\hyphenation{pro-vides}
\hyphenation{en-gi-neering}
\hyphenation{com-pa-ti-bi-li-ty}
\hyphenation{im-pos-sible}
\hyphenation{desk-top}
\hyphenation{elec-tro-nic}
\hyphenation{com-pa-ny}
\hyphenation{de-ve-lop-ment}
\hyphenation{de-ve-loping}
\hyphenation{de-ve-lop}
\hyphenation{da-ta-ba-se}
\hyphenation{plat-forms}
\hyphenation{or-ga-ni-za-tion}
\hyphenation{pro-gramming}
\hyphenation{in-stru-ments}
\hyphenation{Li-nux}
\hyphenation{en-vi-ron-ment}
\hyphenation{Te-le-pathy}
\hyphenation{Li-nux-ov-ka}

\def\progref!#1!{\texttt{#1}}
\renewcommand{\arraystretch}{2} %Іначай формулы ў матрыцы зліпаюцца з лініямі
\usepackage{array}

\def\interview #1 (#2), #3, #4, #5\par{

\section[#1, #3, #4]{#1, #5}
\def\qname{LVEE}
\def\aname{#1}
\def\q ##1\par{{\noindent \bf \qname: ##1 }\par}
\def\a{{\noindent \bf \aname: } \def\qname{L}\def\aname{#2}}
}

\begin{document}
\title{Альт на <<Эльбрусе>>: путь к дистрибутиву\footnote{\url{mike@altlinux.org}, \url{https://lvee.org/en/abstracts/269}}}
\author{Михаил Шигорин, Москва, Россия}
\maketitle
\begin{abstract}
We learned to install our OS onto Elbrus systems in an almost user-friendly manner, not only to just boot it, over this year. Quite a feat given that ALT is the third known operating system to run on e2k!
\end{abstract}
Сейчас это кажется странным~--- но год назад, в феврале 2017, мы ещё не умели загружать свою операционку на единственной имеющейся рабочей станции <<Эльбрус-401>>.  Научились только в марте.

С тех пор собраны не только пакеты в достаточном для многих прикладных задач (как вот сесть и написать тезисы для LVEE), но и ядра для всех актуальных процессоров (4С, 8С, 1С+), проверенные на серверных и настольных системах, и инфраструктура сборки образов ОС.

При этом первые установки делались по сути вручную: загруженный спасательный образ клонировался rsync'ом на свежесозданные /boot и корень, подправлялись /boot/boot.conf и /etc/fstab, перегенерировался initrd.

Несколько позже оказалось практичней просто заливать образ сразу на SSD, подключенный через USB-адаптер, вместо промежуточной флэшки и растягивать разделы при помощи gparted~--- такую <<инструкцию по установке>> уже смогли без особых проблем выполнить и другие люди.

Ну а на прошлой неделе мы поставили первую систему при помощи livecd-install почти без рукоприкладства :)

Разумеется, впереди ещё много работы~--- уборка и расширение репозитория, переход на транзакционную сборочницу girar, более глубокая интеграция с альтовыми технологиями, обычный installer, в конце концов.

Но эта работа чем дальше, тем больше переходит в обычную планомерную.  А начиналось всё два года назад с прорывов, тогда без них было никак\ldots{}

%\begin{thebibliography}{20}

%\bibitem{Shigorin-1} \url{https://lvee.org/ru/abstracts/251}
%\bibitem{Shigorin-2} \url{https://lvee.org/ru/abstracts/180}
%\bibitem{Shigorin-3} \url{https://altlinux.org/ports/e2k}
%\bibitem{Shigorin-4} \url{https://www.basealt.ru/partners/}
%\bibitem{Shigorin-5} \url{https://sdelanounas.ru/blog/shigorin/}
%\end{thebibliography}

\end{document}
