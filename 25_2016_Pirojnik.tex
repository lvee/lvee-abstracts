\documentclass[10pt, a5paper]{article}
\usepackage{pdfpages}
\usepackage{parallel}
\usepackage[T2A]{fontenc}
\usepackage{ucs}
\usepackage[utf8x]{inputenc}
\usepackage[polish,english,russian]{babel}
\usepackage{hyperref}
\usepackage{rotating}
\usepackage[inner=2cm,top=1.8cm,outer=2cm,bottom=2.3cm,nohead]{geometry}
\usepackage{listings}
\usepackage{graphicx}
\usepackage{wrapfig}
\usepackage{longtable}
\usepackage{indentfirst}
\usepackage{array}
\newcolumntype{P}[1]{>{\raggedright\arraybackslash}p{#1}}
\frenchspacing
\usepackage{fixltx2e} %text sub- and superscripts
\usepackage{icomma} % коскі ў матэматычным рэжыме
\PreloadUnicodePage{4}

\newcommand{\longpage}{\enlargethispage{\baselineskip}}
\newcommand{\shortpage}{\enlargethispage{-\baselineskip}}

\def\switchlang#1{\expandafter\csname switchlang#1\endcsname}
\def\switchlangbe{
\let\saverefname=\refname%
\def\refname{Літаратура}%
\def\figurename{Іл.}%
}
\def\switchlangen{
\let\saverefname=\refname%
\def\refname{References}%
\def\figurename{Fig.}%
}
\def\switchlangru{
\let\saverefname=\refname%
\let\savefigurename=\figurename%
\def\refname{Литература}%
\def\figurename{Рис.}%
}

\hyphenation{admi-ni-stra-tive}
\hyphenation{ex-pe-ri-ence}
\hyphenation{fle-xi-bi-li-ty}
\hyphenation{Py-thon}
\hyphenation{ma-the-ma-ti-cal}
\hyphenation{re-ported}
\hyphenation{imp-le-menta-tions}
\hyphenation{pro-vides}
\hyphenation{en-gi-neering}
\hyphenation{com-pa-ti-bi-li-ty}
\hyphenation{im-pos-sible}
\hyphenation{desk-top}
\hyphenation{elec-tro-nic}
\hyphenation{com-pa-ny}
\hyphenation{de-ve-lop-ment}
\hyphenation{de-ve-loping}
\hyphenation{de-ve-lop}
\hyphenation{da-ta-ba-se}
\hyphenation{plat-forms}
\hyphenation{or-ga-ni-za-tion}
\hyphenation{pro-gramming}
\hyphenation{in-stru-ments}
\hyphenation{Li-nux}
\hyphenation{sour-ce}
\hyphenation{en-vi-ron-ment}
\hyphenation{Te-le-pathy}
\hyphenation{Li-nux-ov-ka}
\hyphenation{Open-BSD}
\hyphenation{Free-BSD}
\hyphenation{men-ti-on-ed}
\hyphenation{app-li-ca-tion}

\def\progref!#1!{\texttt{#1}}
\renewcommand{\arraystretch}{2} %Іначай формулы ў матрыцы зліпаюцца з лініямі
\usepackage{array}

\def\interview #1 (#2), #3, #4, #5\par{

\section[#1, #3, #4]{#1 -- #3, #4}
\def\qname{LVEE}
\def\aname{#1}
\def\q ##1\par{{\noindent \bf \qname: ##1 }\par}
\def\a{{\noindent \bf \aname: } \def\qname{L}\def\aname{#2}}
}

\def\interview* #1 (#2), #3, #4, #5\par{

\section*{#1\\{\small\rm #3, #4. #5}}

\def\qname{LVEE}
\def\aname{#1}
\def\q ##1\par{{\noindent \bf \qname: ##1 }\par}
\def\a{{\noindent \bf \aname: } \def\qname{L}\def\aname{#2}}
}

\begin{document}
\title{Zabbix "--- контроль ситуации везде и всегда\footnote{\url{ghrono@gmail.com}, \url{http://lvee.org/ru/abstracts/222}}}
\author{Тимофей Пирожник, Minsk, Belarus}
\maketitle
\begin{abstract}
With the development of infrastructure and the addition of services (third party products) it becomes too difficult or even impossible to keep track of the current state of systems and services. The most correct response on such challenge is to use the monitoring system. Zabbix project is a free/libre solution for such a task.
\end{abstract}
\subsection*{Введение}

С развитием инфраструктуры и добавлением серверов и сервисов (сторонних продуктов) уследить за текущим состоянием  информационной системы становится череcчур тяжело либо скорее всего невозможно. Наиболее правильный выход из сложившейся ситуации "--- ввод в эксплуатацию системы мониторинга.

Когда мы столкнулись с подобной идеей, из представленных программных продуктов была сделана простейшая выборка, которая исключала коммерческие продукты (в т.ч. таковые с бесплатным пробным периодом); также были исключены системы мониторинга, использующие в своей непосредственной работе Java либо Javascript; исключены системы, работающие исключительно с *nix (на момент внедрения в организации использовались различные ОС). В результате выбор пал на Zabbix.

\subsection*{Обзор возможностей/ Технический обзор}

Согласно документации, Zabbix следит за «узлами» сети. Под  узлами подразумевается все, что можно опросить и получить удобочитаемый/внятный ответ об их состоянии. Очевидный объект для опроса "--- физические (гипервизоры) и виртуальные сервера. Приложив определённые усилия, можно наблюдать за рабочими станциями (это актуально для тех, кто работает на аутсорсе). Также доступно наблюдение за состоянием определенных сервисов (приложения, службы, ресурсы,  веб-мониторинг). И конечно же, доступен мониторинг для сети и сетевого оборудования.

Минимальные требования по аппаратной части для сервера мониторинга вариативны из-за разного количества узлов и наблюдаемых параметров.  Например, два разных сервера Zabbix с приблизительно одинаковым количеством узлов и одинаковой архитектурой (на одном сервере установлен сам Zabbix и база данных), работающие через агент и SNMP, потребовали совершенно разные ресурсы.

Способы мониторинга также разнообразны. <<Из коробки>> предлагаются такие варианты, как web-мониторинг "--- проверка наличия в доступе станицы (сайта) плюс имитация действий пользователя, статистические данные (время ответа). Протокол SNMP "--- один из основных способов мониторинга сетевого оборудования, и его поддерживает подавляющее количество устройств. Также используется Zabbix-агент — клиентская часть для серверов и рабочих станций.

Набор действий при появлении событий на первый взгляд скромен: можно либо отправить сообщение о событии, либо выполнить некое действие. Варианты оповещения достаточно вариативны, и выбор протокола остается за администратором. Действия, в контексте Zabbix, это выполнение скриптов, и таки образом способы реагировать на события расширяются до границ фантазии администратора.

\subsection*{Опыт использования в реальном проекте}

Zabbix был использован нами для построения системы мониторинга за сервисом <<клиент-банк>> крупной  финансовой компании. Он показал себя как отличный пример  решения open source, которое работает <<из коробки>>, хотя и требует определенной грамотности и здравого смысла при развёртывании. По нашему опыту можно сказать, что в большинстве ситуаций внедрение не вызывает сложностей, но есть незадокументированые особенности, выявление которых происходит только во время эксплуатации.

\subsection*{Нестандартное применение Zabbix}

С завидным постоянством (вероятно, в результате развития информационных технологий) появляются задачи, которые по сути не относятся к мониторингу, но могут быть реализованы средствами системы. Система мониторинга как таковая ограничивается только  фантазией администратора, и потому при необходимости легко обеспечивает выход за рамки стандартных проверок серверов и сервисов.

\subsection*{Заключение}

После ввода системы мониторинга в нашем случае сократилось время простоя основных сервисов, предоставляемых организацией (соответственно начался спад недовольных клиентов), появилось больше времени на оптимизацию сервисов и улучшение их качества.

\end{document}
