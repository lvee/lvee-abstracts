\documentclass[10pt, a5paper]{article}
\usepackage{pdfpages}
\usepackage{parallel}
\usepackage[T2A]{fontenc}
\usepackage{ucs}
\usepackage[utf8x]{inputenc}
\usepackage[polish,english,russian]{babel}
\usepackage{hyperref}
\usepackage{rotating}
\usepackage[inner=2cm,top=1.8cm,outer=2cm,bottom=2.3cm,nohead]{geometry}
\usepackage{listings}
\usepackage{graphicx}
\usepackage{wrapfig}
\usepackage{longtable}
\usepackage{indentfirst}
\usepackage{array}
\newcolumntype{P}[1]{>{\raggedright\arraybackslash}p{#1}}
\frenchspacing
\usepackage{fixltx2e} %text sub- and superscripts
\usepackage{icomma} % коскі ў матэматычным рэжыме
\PreloadUnicodePage{4}

\newcommand{\longpage}{\enlargethispage{\baselineskip}}
\newcommand{\shortpage}{\enlargethispage{-\baselineskip}}

\def\switchlang#1{\expandafter\csname switchlang#1\endcsname}
\def\switchlangbe{
\let\saverefname=\refname%
\def\refname{Літаратура}%
\def\figurename{Іл.}%
}
\def\switchlangen{
\let\saverefname=\refname%
\def\refname{References}%
\def\figurename{Fig.}%
}
\def\switchlangru{
\let\saverefname=\refname%
\let\savefigurename=\figurename%
\def\refname{Литература}%
\def\figurename{Рис.}%
}

\hyphenation{admi-ni-stra-tive}
\hyphenation{ex-pe-ri-ence}
\hyphenation{fle-xi-bi-li-ty}
\hyphenation{Py-thon}
\hyphenation{ma-the-ma-ti-cal}
\hyphenation{re-ported}
\hyphenation{imp-le-menta-tions}
\hyphenation{pro-vides}
\hyphenation{en-gi-neering}
\hyphenation{com-pa-ti-bi-li-ty}
\hyphenation{im-pos-sible}
\hyphenation{desk-top}
\hyphenation{elec-tro-nic}
\hyphenation{com-pa-ny}
\hyphenation{de-ve-lop-ment}
\hyphenation{de-ve-loping}
\hyphenation{de-ve-lop}
\hyphenation{da-ta-ba-se}
\hyphenation{plat-forms}
\hyphenation{or-ga-ni-za-tion}
\hyphenation{pro-gramming}
\hyphenation{in-stru-ments}
\hyphenation{Li-nux}
\hyphenation{sour-ce}
\hyphenation{en-vi-ron-ment}
\hyphenation{Te-le-pathy}
\hyphenation{Li-nux-ov-ka}
\hyphenation{Open-BSD}
\hyphenation{Free-BSD}
\hyphenation{men-ti-on-ed}
\hyphenation{app-li-ca-tion}

\def\progref!#1!{\texttt{#1}}
\renewcommand{\arraystretch}{2} %Іначай формулы ў матрыцы зліпаюцца з лініямі
\usepackage{array}

\def\interview #1 (#2), #3, #4, #5\par{

\section[#1, #3, #4]{#1 -- #3, #4}
\def\qname{LVEE}
\def\aname{#1}
\def\q ##1\par{{\noindent \bf \qname: ##1 }\par}
\def\a{{\noindent \bf \aname: } \def\qname{L}\def\aname{#2}}
}

\def\interview* #1 (#2), #3, #4, #5\par{

\section*{#1\\{\small\rm #3, #4. #5}}

\def\qname{LVEE}
\def\aname{#1}
\def\q ##1\par{{\noindent \bf \qname: ##1 }\par}
\def\a{{\noindent \bf \aname: } \def\qname{L}\def\aname{#2}}
}

\begin{document}
\title{Коммерциализация СПО под GPL лицензией}
\author{Александр Рябиков, Сергей Середа, Москва, РФ\footnote{\url{rsashka@mail.ru}, \url{http://lvee.org/en/abstracts/121}}}
\maketitle
\begin{abstract}
In accordance to main goals of ADempiere Foundation our task was to find a way to get return on investments in Free Software modification/enhancement for a SMB developer company without dual licensing and without accompanying the pro\-duct with proprietary add"=ons/plugins or additional services or merchandise. Also there were a need to provide a mechanism of costs' sharing between several interested SMB companies.

After initial economic analysis we've concluded that the only way to get return on investments in free software development is to create a time lag between the moment of software product provision to a user and the moment of rights provision for this software according to GPL. Otherwise, demand and supply law just will not work with free software.

Also we have studied the legislation and FSF Comprehensive FAQ about the GNU Licenses. As a result, there were constructed two schemes able to create such time lag without GPL violation. 

First is based mostly on GNU FAQ, second is based on the provisions of contract law.
\end{abstract}
%\begin{center}\textbf{Фонд поддержки и развития делового свободного программного обеспечения ``Адемпиере''}\end{center}

%\begin{flushright}Авторы: 
%Александр Рябиков, 
%Сергей Середа, к.э.н.,\end{flushright}

\subsection*{СПО под GPL лицензией}

Как известно, «свободные» лицензии дают конечным пользователям полную свободу, в том числе и свободу распространять конечный программный продукт наравне с разработчиком. А это означает, что каждый раз, когда разработчик продаёт копию ПО под «свободной» лицензией, он создает себе конкурента. Принято считать, что невозможно зарабатывать деньги непосредственно на доработке и распространении свободного программного обеспечения, распространяемого на условиях GPL или совместимых с нею лицензий. Описываемые в литературе и применяемые на практике бизнес"=модели практически безусловно предполагают бесплатное (или за символическую плату) распространение таких программ. В противоположность этому, наличие «традиционной» (на жаргоне называемой «проприетарной») лицензии позволяет зарабатывать непосредственно на распространении программного обеспечения.

Обладателю исключительных прав на программу ещё доступна модель двойного лицензирования, подразумевающая «несвободную» коммерческую лицензию на ПО для бизнес"=заказчиков и GPL"=совместимую лицензию  для представителей сообщества. Для всех же вторичных проектов (так называемых «форков») подобная возможность исключена. Любые доработки кода для ПО, выпущенного под GPL, могут быть выпущены только под этой же или совместимой «вирусной» лицензией, а замена GPL на несовместимую лицензию допускается только с согласия всех обладателей исключительных прав на программу.

С учетом этого, считается, что зарабатывать на Free Software, не являясь обладателем прав на оригинальный продукт, можно либо оказывая сервисные услуги (сопровождение, документирование, обучение и т.п.), либо компонуя его с собственными разработками с закрытым исходным кодом, способом, который допускается используемой свободной лицензией.

Описываемый ниже способ коммерциализации доработок свободного ПО позволит в определенной мере изменить сложившуюся ситуацию. Предлагаемая бизнес"=модель предназначена, в основном для b2b"=разработчиков, развивающих уже существующие свободные программные продукты, но не владеющих исключительным правом на сам продукт (т.е. наиболее часто встречающаяся ситуация).

Суть бизнес"=модели сводится к созданию условий, аналогичных «традиционному» модели распространения программного обеспечения, т.е. к «продаже» доработанного продукта сразу нескольким покупателям, что даст возможность окупить понесенные затраты на доработку СПО за счет их многократной продажи. Такой эффект достигается за счет создания временного лага между началом продаж доработанного программного продукта и моментом его размещения в свободном доступе для бесплатной загрузки, что позволяет разработчику стать, на какое"=то время, единственным продавцом, у кого эти доработки можно будет приобрести.

Особенность предлагаемой модели заключается в том, что при этом не происходит ни изменения, ни нарушения условий «вирусного» лицензионного соглашения "--- эффективное ограничение распространения продукта определяется договором между хозяйствующими субъектами и вытекает из процесса выполнения заказных программных разработок.

\subsection*{1. Решение, «подсказанное» Free Software \linebreak Foundation (FSF)}

Временной лаг создаётся за счёт превращения пользователя в (со)разработчика, подписания с ним соглашения о нераспространении экземпляров продукта и предоставления ему программного обеспечения для тестирования на продуктивных данных.

\subsubsection*{Часто забываемые факты о GPL}

\begin{itemize}
  \item GPL допускает платное распространение программ \cite{ryabikov1}.
  \item GPL не требует от разработчика делать исходные коды своих разработок доступными широкой публике \cite{ryabikov2}.
  \item Разработчик не вправе запрещать Заказчику дальнейшее распространение ПО \cite{ryabikov3, ryabikov4}.
  \item GPL позволяет Заказчику запретить разработчику распространять результаты заказной разработки  \cite{ryabikov5}.
\end{itemize}

\subsection*{Сценарий по «подсказке» FSF}

Исходная ситуация: Разработчиком доработан GPL"=продукт, \linebreak есть компания"=пользователь, желающая внедрить доработанный \linebreak продукт, но не готовая оплатить полную стоимость доработки, Разработчику необходимо обеспечить возврат инвестиций в доработку, но есть риск, что компания"=пользователь опубликует доработки, как только получит к ним доступ.

\subsubsection*{Предлагаемая схема взаимодействия к компанией"=пользователем:}

\begin{itemize}
  \item Разработчик продает компании"=пользователю исходный продукт без своих доработок.
  \item Разработчик нанимает компанию"=пользователя для выполнения работ, связанных с новым кодом, в частности, для тестирования.
  \item В соответствии с комментариями FSF \cite{ryabikov5} существует законная возможность ограничить распространение тестируемого кода на время действия соответствующего договора.
  \item Время действия этого договора и будет определять временной лаг между началом распространения доработок GPL продукта и моментом его появления в свободном доступе.
  \item Разработчик может одновременно работать по такой схеме сразу с несколькими заинтересованными компаниями.
\end{itemize}

\subsection*{2. Решение на основе договорного права}

Временной лаг создаётся за счёт того, что права на созданный по заказу код и возможность его дальнейшего распространения возникают только после полного исполнения обязательств сторонами договора на доработку ПО.

\subsubsection*{Сценарий на основе договорного права}

Исходная ситуация: Разработчиком доработан GPL"=продукт \linebreak  (разработчик готов его доработать), есть компания"=пользователь, желающая внедрить доработанный продукт, но не готовая оплатить полную стоимость доработки, Разработчику необходимо обеспечить возврат инвестиций в доработку, но есть риск, что компания"=пользователь опубликует доработки, как только получит к ним доступ.

\subsubsection*{Предлагаемая схема взаимодействия к компанией"=пользователем:}

\begin{itemize}
  \item Разработчик заключает с компанией"=пользователем договор на доработку исходного GPL продукта.
  \item В соответствии с действующим законодательством РФ (Статья 712. Право подрядчика на удержание, Статья 1296. Программы для ЭВМ и базы данных, созданные по заказу), в этом случае существует законная возможность ограничить распространение тестируемого кода на время действия соответствующего договора.
  \item Время действия этого договора и будет определять временной лаг между началом распространения доработок продукта под GPL лицензией и моментом его появления в свободном доступе.
  \item Разработчик может одновременно работать по такой схеме сразу с несколькими заинтересованными компаниями.
\end{itemize}

Описанная конструкция не является каким"=то искусственным построением. Наоборот, она является описанием вполне реального процесса заказной разработки корпоративного ПО, который завершается только после проведения опытной или опытно"=промышленной эксплуатации разработок (с их полным сопровождением все это время) и передачи их в промышленную эксплуатацию.

По мнению авторов, регулируя длительность временного лага между началом продаж доработанного программного продукта и моментом его размещения в свободном доступе для бесплатной загрузки, в зависимости от типа, сложности, популярности и иных характеристик дорабатываемого программного продукта, можно обеспечить возврат инвестиций в его развитие, достаточный для поддержания коммерческого предприятия. При этом, за счет заключения подобных договоров с несколькими (в идеале "--- со многими) пользователями"=заказчиками будет также обеспечено и разделение затрат между ними (в отличие от схемы bounty source, в рамках которой доработка целиком оплачивается одним заказчиком).

Конечно, наложение любых ограничений на распространение СПО может быть воспринято «в штыки» сообществом и, возможно, потребуются определенные усилия, что бы сторонники двух диаметрально противоположных точек зрения на способы распространения ПО смогли оценить положительные и отрицательные стороны от такого «нововведения». Тем не менее, описанная бизнес"=модель полностью основана на законе и не нарушает ни одного положения о защите интеллектуальной собственности. Более того, она, по сути, является просто специализированной модификацией широко применяемой в корпоративном секторе стандартной модели заказной разработки ПО.

Мы обсудили с Ричардом Столлманом эти способы и он сделал важные замечания к описываемым моделям. Второй способ на основе контрактного права соответствует GPLv2, но нарушает GPLv3, где в явном виде прописано, что лицензия имеет приоритет над контрактными обязательствами. С этим мнением не согласны наши юристы, но кто из них прав может решить только судебная практика.
Насчет первого способа, Ричард конечно не в восторге, но нарушение GPL будет в том случае, если работа пользователя будет фиктивной. Другими словами, пользователь должен по настоящему работать в соответствии с контрактными обязательствами и эта работа должна реально оплачиваться.

Конечно, используя описанные выше принципы, первое время код будет отсутствовать в свободном доступе, но через какое то время он все равно станет доступен для всех. И в результате, и инвесторы вернут вложенные средства и сообщество получит ощутимый вклад в развитие СПО.

Данная методика коммерциализации СПО с лицензией GPL была представлена на Russian Open Source Summit 2014. Статья по материалам доклада доступна на сайте PCWEEK \cite{ryabikov6}.

\begin{thebibliography}{9}
\bibitem{ryabikov1} \url{http://www.gnu.org/licenses/gpl-faq.en.html#DoesTheGPLAllowMoney}
\bibitem{ryabikov2} \url{http://www.gnu.org/licenses/gpl-faq.en.html#DoesTheGPLRequireAvailabilityToPublic}
\bibitem{ryabikov3} \url{http://www.gnu.org/licenses/gpl-faq.en.html#DoesTheGPLAllowNDA}
\bibitem{ryabikov4} \url{http://www.gnu.org/licenses/gpl-faq.en.html#DoesTheGPLAllowModNDA}
\bibitem{ryabikov5} \url{http://www.gnu.org/licenses/gpl-faq.en.html#DevelopChangesUnderNDA}
\bibitem{ryabikov6} \url{http://www.pcweek.ru/foss/article/detail.php?ID=164583}
\end{thebibliography}
\end{document}
