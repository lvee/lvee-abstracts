\documentclass[10pt, a5paper]{article}
\usepackage{pdfpages}
\usepackage{parallel}
\usepackage[T2A]{fontenc}
\usepackage{ucs}
\usepackage[utf8x]{inputenc}
\usepackage[polish,english,russian]{babel}
\usepackage{hyperref}
\usepackage{rotating}
\usepackage[inner=2cm,top=1.8cm,outer=2cm,bottom=2.3cm,nohead]{geometry}
\usepackage{listings}
\usepackage{graphicx}
\usepackage{wrapfig}
\usepackage{longtable}
\usepackage{indentfirst}
\usepackage{array}
\newcolumntype{P}[1]{>{\raggedright\arraybackslash}p{#1}}
\frenchspacing
\usepackage{fixltx2e} %text sub- and superscripts
\usepackage{icomma} % коскі ў матэматычным рэжыме
\PreloadUnicodePage{4}

\newcommand{\longpage}{\enlargethispage{\baselineskip}}
\newcommand{\shortpage}{\enlargethispage{-\baselineskip}}

\def\switchlang#1{\expandafter\csname switchlang#1\endcsname}
\def\switchlangbe{
\let\saverefname=\refname%
\def\refname{Літаратура}%
\def\figurename{Іл.}%
}
\def\switchlangen{
\let\saverefname=\refname%
\def\refname{References}%
\def\figurename{Fig.}%
}
\def\switchlangru{
\let\saverefname=\refname%
\let\savefigurename=\figurename%
\def\refname{Литература}%
\def\figurename{Рис.}%
}

\hyphenation{admi-ni-stra-tive}
\hyphenation{ex-pe-ri-ence}
\hyphenation{fle-xi-bi-li-ty}
\hyphenation{Py-thon}
\hyphenation{ma-the-ma-ti-cal}
\hyphenation{re-ported}
\hyphenation{imp-le-menta-tions}
\hyphenation{pro-vides}
\hyphenation{en-gi-neering}
\hyphenation{com-pa-ti-bi-li-ty}
\hyphenation{im-pos-sible}
\hyphenation{desk-top}
\hyphenation{elec-tro-nic}
\hyphenation{com-pa-ny}
\hyphenation{de-ve-lop-ment}
\hyphenation{de-ve-loping}
\hyphenation{de-ve-lop}
\hyphenation{da-ta-ba-se}
\hyphenation{plat-forms}
\hyphenation{or-ga-ni-za-tion}
\hyphenation{pro-gramming}
\hyphenation{in-stru-ments}
\hyphenation{Li-nux}
\hyphenation{sour-ce}
\hyphenation{en-vi-ron-ment}
\hyphenation{Te-le-pathy}
\hyphenation{Li-nux-ov-ka}
\hyphenation{Open-BSD}
\hyphenation{Free-BSD}
\hyphenation{men-ti-on-ed}
\hyphenation{app-li-ca-tion}

\def\progref!#1!{\texttt{#1}}
\renewcommand{\arraystretch}{2} %Іначай формулы ў матрыцы зліпаюцца з лініямі
\usepackage{array}

\def\interview #1 (#2), #3, #4, #5\par{

\section[#1, #3, #4]{#1 -- #3, #4}
\def\qname{LVEE}
\def\aname{#1}
\def\q ##1\par{{\noindent \bf \qname: ##1 }\par}
\def\a{{\noindent \bf \aname: } \def\qname{L}\def\aname{#2}}
}

\def\interview* #1 (#2), #3, #4, #5\par{

\section*{#1\\{\small\rm #3, #4. #5}}

\def\qname{LVEE}
\def\aname{#1}
\def\q ##1\par{{\noindent \bf \qname: ##1 }\par}
\def\a{{\noindent \bf \aname: } \def\qname{L}\def\aname{#2}}
}

\begin{document}
\title{История одного маленького дистрибутива Linux}
\author{Денис Пынькин, Минск, Беларусь\footnote{\url{denis_pynkin@epam.com}, \url{http://lvee.org/en/abstracts/167}}}
\maketitle
\begin{abstract}
Article presents the creation and support history of the Linux-based infrastructure in ECM department of BSUIR. Evolution and current state of technical decisions used tp create Linux-based infrastructure are described.
\end{abstract}
\section*{Введение}

Инфраструктура в образовательных заведениях играет первоочередную роль в обучении студентов самым разным технологиям. Если нет системы, на которой можно потренироваться, понять, что работает как описано, а где возникают <<нюансы работы>>, то никакие видео-обзоры и тонны литературы не заменят реального опыта. Именно поэтому в учебных лабораториях необходимо иметь возможность поработать с различными операционными системами, включая ОС Linux.

\section*{Историческая справка}

Первые эксперименты по внедрению ОС Linux в обучение на кафедре ЭВМ БГУИР начались примерно в 2002--2003 годах с классического подхода к установке системы параллельно с основной ОС. К счастью, этот подход не прижился --- нехватка места на диске, легкость <<поломать все>>, неопытность внедряющих сделали свое дело.

Следующим шагом стала попытка минимизировать простои систем в случае поломок. Хотя задача и была решена <<в лоб>> с помощью банального восстановления соответствующих разделов, это позволило добиться более-менее стабильной работы обеих используемых ОС, поскольку появилась возможность восстанавливать работоспособную систему в течении одного перерыва.

Интересным оказался эффект такого подхода. Во-первых, появилась возможность <<доверить>> студентам права администратора на студенческих машинах, а во-вторых, после введения соответствующего пункта в меню загрузчика, задача восстановления работоспособности машины перекочевала на плечи студентов.

Примерно с 2003--2004 годов начались попытки <<поиграться>> с сетевой загрузкой  --- начиная с загрузки самописной системы восстановления и заканчивая полноценной загрузкой систем с развернутым корнем на сервере \cite{SP01}. Тем не менее, возникли те же проблемы, что и с локально  установленной системой и, в дополнение к этому, добавились проблемы с общим разделяемым ресурсом, добавилась необходимость отслеживания состояния этих развернутых систем и невозможность студентам починить систему самостоятельно. А ломались и <<ломались>> такие системы довольно часто --- как случайно, так и намеренно.

Интересным опытом оказалось использование рабочих машин в качестве тонких клиентов вплоть до исключительно ssh-клиента на серверную систему. Это было настоящее раздолье для любопытствующих личностей --- начиная от fork-бомб и получения привелегий рута, парализующих работу целой лаборатории, и заканчивая файлами <<посмотри-меня-xxx-в-sd-качестве.avi>> с установленными executable битами.

Примерно в 2005-2006 годах стало понятно, что от идеальной учебной системы требуется, чтобы она:

\begin{itemize}
  \item загружалась по сети;
  \item работала локально на студенческих системах;
  \item не использовала НЖМД;
  \item работала в режиме R/O;
  \item содержала максимально полный набор ПО для разработки;
  \item имела графическое окружение, не вызывающее желания тут же убежать.
\end{itemize}

К тому моменту такие системы уже существовали в виде Live-CD и initrd-only систем.
Несмотря на все достоинства использования initrd в качестве базовой системы, у нее есть один недостаток, перечеркивающий все плюсы такого подхода --- вся система находится в памяти, отъедая этот дефицитный ресурс у полезных программ. Кроме того, появляется жесткое ограничение на размер образа, что уменьшает количество полезного ПО и требует зачастую нетривиальных оптимизаций.
Отличие же Live-CD от такой же системы, но загружающейся по сети, минимально и, фактически, заключается в начальной инициализации системы\cite{P01}.

Примерно в это же время удалось добиться полноценной работы ОС семейства Windows, что дало возможность отказаться от дисковых систем при создании компьютерных лабораторий на кафедре ЭВМ БГУИР начиная с 2007 года\cite{PGO01}.

\section*{Дистрибутив}

На данный момент архитектура построения stateless-систем, загружающихся по сети \cite{P01},  уже считается классикой и используется во всех распространенных дистрибутивах.

Разница по большей части заключается в наборе ПО, которое содержится на stateless-системе, и его изначальной конфигурации. Видимо, максимально близко к нужному составу ПО стоит Knoppix, разрабатываемый Клаусом Кноппером, однако все равно необходимо проводить адаптацию к целевой инфраструктуре.

Кроме того, некоторое ПО сугубо специфично для некоторых курсов, читаемых на кафедре ЭВМ, например MPI и Cuda.

Отдельным пунктом хотелось бы упомянуть, что зачастую недостаточно просто установить нужный пакет из репозитория --- ему требуется дополнительная настройка для корректной работы: например, OpenMPI требует корректной настройки ssh, а также корректировки настроек для работы с несколькими сетевыми картами.

В дополнение к вышесказанному необходимо отметить, что хотелось бы иметь более-менее современные версии ПО, что приводит к еще одной проблеме --- весь состав ПО, все правки постоянно изменяются во времени.

Все вышесказанное приводит к тому, что фактически приходится разрабатывать свой собственный дистрибутив, предназначенный для stateless-работы в учебных заведениях. Естественно, что создание такого дистрибутива с чистого листа крайне ресурсоемкая задача, а так как универсального решения, как и универсального <<эталонного>> дистрибутива, пока не существует, то необходимо делать выбор в пользу одного из существующих решений.

Чтобы не вступать в дискуссии о том, какой дистрибутив брать в качестве базового для решения этой задачи, просто отмечу, что до начала 10-х годов XXI века далеко не все дистрибутивы задавались вопросом разработки целостной платформы для создания своих собственных <<кастомных>> решений. Одним из счастливых исключений является ALT Linux, который используется в качестве базовой системы для построения нужного дистрибутива.

Сейчас в дистрибутивах семейства ALT Linux используется система сборки дистрибутива <<mkimage-profiles>> \cite{S01}; по сути она --- набор Makefile’ов, формат которых знаком каждому программисту.

\section*{Человеческий фактор}

В связи с тем, что у Министерства Образования нет четкой политики по использованию открытого ПО в целом и ОС Linux в частности, в ВУЗах используют, если вообще используют, тот дистрибутив, который нравится местному системному администратору \cite{DKP01}. Поэтому при смене Linux-администратора на кафедре сборка текущего дистрибутива прекратила свое развитие, а создание новой сборки, на новой базе --- достаточно долгий, тонкий и кропотливый процесс, отнимающий много времени и сил.

Немаловажным фактором являлась фактически непубличная разработка предыдущей версии дистрибутива. Причин масса, однако главными из них являются стыд и лень: стыдно за те <<хаки>>, которые использовались при кастомизации дистрибутива, а с учетом достаточно большого количества мелких изменений всегда возникает соблазн отложить <<генеральную уборку>> еще <<на чуть-чуть>> --- до тех пор, пока не станет слишком поздно.

Масла в огонь подлило открытие совместной лаборатории \linebreak БГУИР и Epam, в которой проводятся занятия по изучению ОС Linux \cite{PS01}.  Внезапно выяснилось, что, помимо ожидаемой разницы между различными дистрибутивами ОС Linux в структуре и работе различных утилит, в рамках даже одного дистрибутива изменений уже достаточно, чтобы сорвать часть занятия.

Таким образом, под влиянием внешних факторов воля была собрана в кулак и началось приведение помойки почти трехлетней давности в порядок, с учетом накопленного опыта.

\section*{Новое время --- новые вызовы}

Первый этап по генеральной чистке и приведению правил сборки в порядок был успешно преодолен с помощью разработчика <<mkimage-profiles>> Михаила Шигорина.

Но, к сожалению, одной только чистки и облагораживания недостаточно: какое-то ПО развилось до неузнаваемого, какое-то исчезло, появились новые интересные приложения и популярные языки программирования --- все это необходимо осознать и интегрировать. Добавились современные системы виртуализации и управления контейнерами. При этом все же желательно уложиться в лимит 4GB, чтобы иметь возможность создавать DVD-образ.

И, наконец, ведется работа по адаптации получившейся системы в сетях ОС Windows, чтобы была возможность загружать ОС Linux в учебных заведениях (и не только) без изменения их инфраструктуры.

\begin{thebibliography}{9}

\bibitem{SP01} Р.Х. Садыхов, Д.А. Пынькин. Технология построения кластерных систем для образо­вательных целей. Доклады Международной научной конференции SSA'2004. Минск, 26--28 октября 2004.

\bibitem{P01} Пынькин Д.А, Бездисковые рабочие станции на базе технологий ALT Linux, \url{http://lvee.org/media/presentations/lvee2008_03-1.pdf}

\bibitem{PGO01} Д.А. Пынькин, И.И. Глецевич, А.В. Отвагин. Использование бездисковых рабочих станций в образовательном процессе \linebreak БГУИР // Материалы 4 международной конференции <<Информационные системы и технологии>> IST’2008.  Минск, 2008.  С.~310--315.

\bibitem{S01} М.А.Шигорин. Макраме из дистрибутивов: mkimage-profiles. Девятая конференция разработчиков свободных программ: Тезисы докладов / Обнинск, 23–24 июля 2012 года. С.48--49, \url{http://www.altlinux.ru/media/protva-2012.pdf}

\bibitem{DKP01} S.S. Derechennik, D.A. Kostiuk, D.A. Pynkin. Free/libre software usage in the belarusian system of higher educational institutions // Друга міжнародна науково-практична конференція FOSS Lviv-2012: Збірник наукових праць/ Львів, 26--28 квітня 2012 р.

\bibitem{PS01} Д.А. Пынькин, В.В. Шахов. Обучение Linux в корпоративном секторе. Зимняя международная конференция LVEE’2013. Тезисы докладов. \url{http://lvee.org/en/abstracts/57}
\end{thebibliography}
\end{document}
