\documentclass[10pt, a5paper]{article}
\usepackage{pdfpages}
\usepackage{parallel}
\usepackage[T2A]{fontenc}
\usepackage{ucs}
\usepackage[utf8x]{inputenc}
\usepackage[polish,english,russian]{babel}
\usepackage{hyperref}
\usepackage{rotating}
\usepackage[inner=2cm,top=1.8cm,outer=2cm,bottom=2.3cm,nohead]{geometry}
\usepackage{listings}
\usepackage{graphicx}
\usepackage{wrapfig}
\usepackage{longtable}
\usepackage{indentfirst}
\usepackage{array}
\newcolumntype{P}[1]{>{\raggedright\arraybackslash}p{#1}}
\frenchspacing
\usepackage{fixltx2e} %text sub- and superscripts
\usepackage{icomma} % коскі ў матэматычным рэжыме
\PreloadUnicodePage{4}

\newcommand{\longpage}{\enlargethispage{\baselineskip}}
\newcommand{\shortpage}{\enlargethispage{-\baselineskip}}

\def\switchlang#1{\expandafter\csname switchlang#1\endcsname}
\def\switchlangbe{
\let\saverefname=\refname%
\def\refname{Літаратура}%
\def\figurename{Іл.}%
}
\def\switchlangen{
\let\saverefname=\refname%
\def\refname{References}%
\def\figurename{Fig.}%
}
\def\switchlangru{
\let\saverefname=\refname%
\let\savefigurename=\figurename%
\def\refname{Литература}%
\def\figurename{Рис.}%
}

\hyphenation{admi-ni-stra-tive}
\hyphenation{ex-pe-ri-ence}
\hyphenation{fle-xi-bi-li-ty}
\hyphenation{Py-thon}
\hyphenation{ma-the-ma-ti-cal}
\hyphenation{re-ported}
\hyphenation{imp-le-menta-tions}
\hyphenation{pro-vides}
\hyphenation{en-gi-neering}
\hyphenation{com-pa-ti-bi-li-ty}
\hyphenation{im-pos-sible}
\hyphenation{desk-top}
\hyphenation{elec-tro-nic}
\hyphenation{com-pa-ny}
\hyphenation{de-ve-lop-ment}
\hyphenation{de-ve-loping}
\hyphenation{de-ve-lop}
\hyphenation{da-ta-ba-se}
\hyphenation{plat-forms}
\hyphenation{or-ga-ni-za-tion}
\hyphenation{pro-gramming}
\hyphenation{in-stru-ments}
\hyphenation{Li-nux}
\hyphenation{sour-ce}
\hyphenation{en-vi-ron-ment}
\hyphenation{Te-le-pathy}
\hyphenation{Li-nux-ov-ka}
\hyphenation{Open-BSD}
\hyphenation{Free-BSD}
\hyphenation{men-ti-on-ed}
\hyphenation{app-li-ca-tion}

\def\progref!#1!{\texttt{#1}}
\renewcommand{\arraystretch}{2} %Іначай формулы ў матрыцы зліпаюцца з лініямі
\usepackage{array}

\def\interview #1 (#2), #3, #4, #5\par{

\section[#1, #3, #4]{#1 -- #3, #4}
\def\qname{LVEE}
\def\aname{#1}
\def\q ##1\par{{\noindent \bf \qname: ##1 }\par}
\def\a{{\noindent \bf \aname: } \def\qname{L}\def\aname{#2}}
}

\def\interview* #1 (#2), #3, #4, #5\par{

\section*{#1\\{\small\rm #3, #4. #5}}

\def\qname{LVEE}
\def\aname{#1}
\def\q ##1\par{{\noindent \bf \qname: ##1 }\par}
\def\a{{\noindent \bf \aname: } \def\qname{L}\def\aname{#2}}
}


\begin{document}

\title{Применение iSCSI RAID LVM для создания частного облачного хранилища данных}%\footnote{Текст данных и последующих тезисов, кроме специально оговоренных случаев, доступен под лицензией Creative Commons Attribution-ShareAlike 3.0}

\author{Вячеслав Бочаров\footnote{Минск, Беларусь}}
\maketitle

\begin{abstract}
Creation of a cloud storage system is presented, nodes of which are workstations based on existing technologies (RAID, iSCSI, LVM). Lowcost file storage from nothing and its evolution \linebreak perspectives are discussed.
\end{abstract}

Потребность пользователей информационных систем в дисковом пространстве для надежного хранения данных приводит к необходимости развертывания систем хранения данных с достаточно высокой стоимостью приобретения и эксплуатации.
В тоже время дисковое пространство рабочих мест пользователей информационной системы остается незадействованным, и не используется в инфраструктуре предприятия.
Фактически, такое нужное дисковое пространство <<рассыпано>> у нас под ногами, и его можно заставить работать на благо нашей инфраструктуры --- все нужные технологии у нас уже есть.

Для этого необходимо выполнить следующие шаги:

\begin{enumerate}
  \item Предоставить часть дискового пространства рабочей станции в общее пользование --- для этого мы используем iSCSI taraget. При среднем размере диска рабочей станции в 300 ГБ реально используется не более 150, поэтому еще столько же можно отдать в облако. Если на предприятии 100 рабочих станций --- результат будет 15 Тб <<сырого>> пространства на 100 дисках.
  \item Собрать предоставленное пространство в единый массив на сервере.
\end{enumerate}

Реализация перечисленных действий требует решения нескольких проблем. 
В частности, используя рабочие станции пользователей как поставщики дискового пространства, мы должны учитывать следующее: то, что для СХД является нештатной ситуацией (выход из строя нескольких дисков),  для нашего облака --- обычное явление. Поэтому избыточность должна быть большой. Я использую конфигурацию RAID-10 через mdadm на CentOS. Это позволяет контролировать состояние RAID-массива, менять вышедшие из эксплуатации узлы ISCSI, оперативно настраивать массив.

Также необходимо большое количество дисков HOT SPARE, и нужна система их активации. Это повышает процент потери пространства (в опробованной конфигурации до 40\% уходит на резервирование), но зато это пространство практически взято из воздуха.

Еще один проблемный момент--окончание рабочего дня, когда большая часть узлов выключается. Можно было бы реализовать схему со спящим режимом, но тогда возрастает энергопотребление организации. Проще рассматривать нерабочий период как прерывание работы контроллера системы и выключение контроллера по расписанию либо по достижению определенного порога выхода из сети iSCSI-дисков, что легко реализуется скриптами bash.

\begin{enumerate}
  \item Вполне естественно будет упомянуть необходимость использования в сети 1Gb и Jambo Frame.
    В отношении скорости такой системы можно заметить, что данная конфигурация дает показатели IOPS 75/80--значения, сравнимые с показателями SATA 7200 HDD.
  \item Оперативное изменение размера облака решаемо штатными средствами--LVM. Благодаря использованию LVM мы имеем возможность расширять существующее дисковое пространство просто добавлением еще одной группы RAID.
\end{enumerate}

Хотя предлагаемое решение не предназначено быть заменой высоконагруженным решениям, оно может вполне успешно использоваться для системы файлового хранения --- как известно, самой прожорливой.
 Стоимость данного решения чрезвычайно низка. Оно позволяет эксплуатировать ресурсы предприятия полностью.
 Развитие представленного направления в облачных СХД имеет большое будущее. Благодаря таким подходам надежные системы хранения станут не уделом дорогих систем корпоративного уровня, а будут доступны рядовым пользователям. При разработке специализированной системы с резервированием контроллеров, оптимизацией распределение ресурсов между дисками, повышением отказоустойчивости и уменьшением потерь дискового пространства, возможно даже появление сообществ, позволяющих создавать дисковое пространство, разделяемое не только между пользователями одной организации, но и между членами самоорганизующихся сообществ.




\end{document}




