\documentclass[10pt, a5paper]{article}
\usepackage{pdfpages}
\usepackage{parallel}
\usepackage[T2A]{fontenc}
\usepackage{ucs}
\usepackage[utf8x]{inputenc}
\usepackage[polish,english,russian]{babel}
\usepackage{hyperref}
\usepackage{rotating}
\usepackage[inner=2cm,top=1.8cm,outer=2cm,bottom=2.3cm,nohead]{geometry}
\usepackage{listings}
\usepackage{graphicx}
\usepackage{wrapfig}
\usepackage{longtable}
\usepackage{indentfirst}
\usepackage{array}
\newcolumntype{P}[1]{>{\raggedright\arraybackslash}p{#1}}
\frenchspacing
\usepackage{fixltx2e} %text sub- and superscripts
\usepackage{icomma} % коскі ў матэматычным рэжыме
\PreloadUnicodePage{4}

\newcommand{\longpage}{\enlargethispage{\baselineskip}}
\newcommand{\shortpage}{\enlargethispage{-\baselineskip}}

\def\switchlang#1{\expandafter\csname switchlang#1\endcsname}
\def\switchlangbe{
\let\saverefname=\refname%
\def\refname{Літаратура}%
\def\figurename{Іл.}%
}
\def\switchlangen{
\let\saverefname=\refname%
\def\refname{References}%
\def\figurename{Fig.}%
}
\def\switchlangru{
\let\saverefname=\refname%
\let\savefigurename=\figurename%
\def\refname{Литература}%
\def\figurename{Рис.}%
}

\hyphenation{admi-ni-stra-tive}
\hyphenation{ex-pe-ri-ence}
\hyphenation{fle-xi-bi-li-ty}
\hyphenation{Py-thon}
\hyphenation{ma-the-ma-ti-cal}
\hyphenation{re-ported}
\hyphenation{imp-le-menta-tions}
\hyphenation{pro-vides}
\hyphenation{en-gi-neering}
\hyphenation{com-pa-ti-bi-li-ty}
\hyphenation{im-pos-sible}
\hyphenation{desk-top}
\hyphenation{elec-tro-nic}
\hyphenation{com-pa-ny}
\hyphenation{de-ve-lop-ment}
\hyphenation{de-ve-loping}
\hyphenation{de-ve-lop}
\hyphenation{da-ta-ba-se}
\hyphenation{plat-forms}
\hyphenation{or-ga-ni-za-tion}
\hyphenation{pro-gramming}
\hyphenation{in-stru-ments}
\hyphenation{Li-nux}
\hyphenation{sour-ce}
\hyphenation{en-vi-ron-ment}
\hyphenation{Te-le-pathy}
\hyphenation{Li-nux-ov-ka}
\hyphenation{Open-BSD}
\hyphenation{Free-BSD}
\hyphenation{men-ti-on-ed}
\hyphenation{app-li-ca-tion}

\def\progref!#1!{\texttt{#1}}
\renewcommand{\arraystretch}{2} %Іначай формулы ў матрыцы зліпаюцца з лініямі
\usepackage{array}

\def\interview #1 (#2), #3, #4, #5\par{

\section[#1, #3, #4]{#1 -- #3, #4}
\def\qname{LVEE}
\def\aname{#1}
\def\q ##1\par{{\noindent \bf \qname: ##1 }\par}
\def\a{{\noindent \bf \aname: } \def\qname{L}\def\aname{#2}}
}

\def\interview* #1 (#2), #3, #4, #5\par{

\section*{#1\\{\small\rm #3, #4. #5}}

\def\qname{LVEE}
\def\aname{#1}
\def\q ##1\par{{\noindent \bf \qname: ##1 }\par}
\def\a{{\noindent \bf \aname: } \def\qname{L}\def\aname{#2}}
}

\switchlang{ru}
\begin{document}
\title{Автоматизированный поиск багов в C/C++\footnote{\url{zamazan4ik@tut.by}, \url{https://lvee.org/en/abstracts/279}}}
\author{Александр Зайцев, Минск, Belarus}
\maketitle
\begin{abstract}
Nowadays programs are too complex for verification and any developer can't guarantee that program is valid in every situation. Inthis talk I'll try to introduce modern ways for automatic testing C/C++ programs. I will talk about proper compiler settings, static and dynamic analyzers, sanitizers and fuzzing tecniques.
\end{abstract}

Сегодня хотелось бы рассказать о такой вещи как фаззинг\linebreak (fuzzing) тестирование и чем оно будет полезно как для разработчиков программ, так и для их мейнтейнеров в различных дистрибутивах. Так как я в основном пользуюсь в своей работе языком программирования C++, то я буду рассказывать про фаззинг в контексте разработки на данном языке программирования (Си тоже касается :)

\subsection*{Почему эти языки находятся в зоне риска?}

Языки программирования C/C++ заточены на производительность. За это приходится платить тем, что в ходе работы программы отсутствуют многие элементарные проверки валидности программы. Даже банальный выход за пределы массива или разыменование нулевого указателя ведёт не к исключению, а к так называемому <<неопределённому поведению>>, которое позволяет компилятору творить с вашей программой всё что угодно (читать как <<программа становится автоматически невалидной>>).

\subsection*{Типичные примеры неопределённого поведения:}
\begin{itemize}
\item Разыменование нулевого указателя
\item Использование невалидного указателя\textbackslash{}итератора\textbackslash{}ссылки
\item Переполнение знаковых переменных
\item Нарушение контрактов стандартных алгоритмов
\item Многократное освобождение памяти
\item Использование памяти после освобождения
\end{itemize}
Как будем бороться с этим?

\subsection*{Настройка компилятора}
\begin{itemize}
\item Включение как можно большего числа предупреждений (аккуратнее на грязной кодовой базе, а то есть шанс утонуть)
\item Обновляйте компиляторы по возможности~---новые компиляторы~---новые предупреждения и диагностики
\item Для поддержания чистоты~---pedantic, чтобы неповадно было игнорировать предупреждения
\end{itemize}
\subsection*{Статический анализ кода}

Когда нам мало диагностик компилятора, на помощь нам могут прийти такие средства как статические анализаторы кода. В них обычно реализовано гораздо больше проверок на типичные ошибки разработчика, нежели в компиляторах (на то они и есть специализированные средства). Рекомендуемые к использованию статические анализаторы кода:
\begin{itemize}
\item cppcheck (GPL-3.0)
\item Clang-Tidy (BSD)
\item (не уверен, что стоит называть проприетарные средства тут)
\end{itemize}
Вкупе с компиляторами большУю часть ошибок можно отсеять ещё до запуска самой программы только путём анализа исходных файлов.

\subsection*{Санитайзеры}

Санитайзеры (sanitizers)~---утилиты из уже поставки компиляторов, которые помогают найти проблемы в вашем коде.
Санитайзеры бывают:
\begin{itemize}
\item Address~---помогает найти проблемы с памятью (утечка, двойное разыменование и т.д.). Работает как перехват всех операций с памятью и проверка \textbf{во время работы}. Значительное замедление работы
\item Thread~---помогает находить состояния гонки и deadlocks в вашем коде
\item Undefined~---помогает диагностировать неопределённое поведение в коде. Само собой разумеется, что он не может гарантированно найти все ошибки, потому что это C++.
\item CFE
\end{itemize}

Разработчики поопытнее ходят между граблей аккуратнее и с использованием специальных инструментов~---санитайзеров, но даже это их не спасает, так как с ростом размера программы сложно гарантировать её валидность в целом. Тестами всю систему в целом тоже очень и очень сложно покрыть. Можно к этому стремиться, но не всегда получается.

\subsection*{Альтернатива санитайзерам}

Можно для обнаружения проблем во время исполнения программы использовать Valgrind и построенные на его базе программы. Тут нам помогут программы DRD, Helgrind, Memcheck. Если Вы пользователь Windows, то для Вас альтернативой будет являться программа Dr. Memory. Лицензия Valgrind~---GPL-2.0, Dr. Memory~---LGPL.

\subsection*{Что такое фаззинг-тестирование?}

Грубо говоря, это вид тестирования, при котором на вход нашей программе подаётся огромное количество различных входных данных, после чего мы смотрим, как наша программа себя поведёт на них. В идеале, наша программа на корректных данных должна выдавать какой-либо корректный результат, а на некорректных завершаться каким-либо образом, который задумал автор. К сожалению, не всегда так происходит.

Рекомендуемые к использованию фаззеры: 
\begin{itemize}
\item AFL (Apache)
\item Libfuzzer (BSD)
\end{itemize}

\subsection*{Фаззинг и санитайзеры}

Так как фаззеры запускают нашу программу огромное количество раз на абсолютно разных данных, то очень хорошей идеей является запуск программы под фаззером и под санитайзером одновременно, так как таким образом увеличивается шанс нахождения какой-либо ошибки (утечки памяти, например), а то всё тоже является невалидным поведением.

\subsection*{Фаззинг для Open-Source}

Для тестирования Open-Source проектов Google сделал проект Oss-Fuzz. Это проект, который автоматически запускает программы с открытым исходным кодом под фаззерами на мощностях\linebreak Google. От проекта всего лишь требуется добавить свой проект в репозиторий, следовать довольно простым требованиям и дождаться одобрения от ребят из Google. И всё~---потом можно расслабиться и ждать отчётов с ошибками.

В целом это всё, что я хотел рассказать. Спасибо за внимание!

\end{document}
