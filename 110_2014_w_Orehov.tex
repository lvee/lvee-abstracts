\documentclass[10pt, a5paper]{article}
\usepackage{pdfpages}
\usepackage{parallel}
\usepackage[T2A]{fontenc}
\usepackage{ucs}
\usepackage[utf8x]{inputenc}
\usepackage[polish,english,russian]{babel}
\usepackage{hyperref}
\usepackage{rotating}
\usepackage[inner=2cm,top=1.8cm,outer=2cm,bottom=2.3cm,nohead]{geometry}
\usepackage{listings}
\usepackage{graphicx}
\usepackage{wrapfig}
\usepackage{longtable}
\usepackage{indentfirst}
\usepackage{array}
\newcolumntype{P}[1]{>{\raggedright\arraybackslash}p{#1}}
\frenchspacing
\usepackage{fixltx2e} %text sub- and superscripts
\usepackage{icomma} % коскі ў матэматычным рэжыме
\PreloadUnicodePage{4}

\newcommand{\longpage}{\enlargethispage{\baselineskip}}
\newcommand{\shortpage}{\enlargethispage{-\baselineskip}}

\def\switchlang#1{\expandafter\csname switchlang#1\endcsname}
\def\switchlangbe{
\let\saverefname=\refname%
\def\refname{Літаратура}%
\def\figurename{Іл.}%
}
\def\switchlangen{
\let\saverefname=\refname%
\def\refname{References}%
\def\figurename{Fig.}%
}
\def\switchlangru{
\let\saverefname=\refname%
\let\savefigurename=\figurename%
\def\refname{Литература}%
\def\figurename{Рис.}%
}

\hyphenation{admi-ni-stra-tive}
\hyphenation{ex-pe-ri-ence}
\hyphenation{fle-xi-bi-li-ty}
\hyphenation{Py-thon}
\hyphenation{ma-the-ma-ti-cal}
\hyphenation{re-ported}
\hyphenation{imp-le-menta-tions}
\hyphenation{pro-vides}
\hyphenation{en-gi-neering}
\hyphenation{com-pa-ti-bi-li-ty}
\hyphenation{im-pos-sible}
\hyphenation{desk-top}
\hyphenation{elec-tro-nic}
\hyphenation{com-pa-ny}
\hyphenation{de-ve-lop-ment}
\hyphenation{de-ve-loping}
\hyphenation{de-ve-lop}
\hyphenation{da-ta-ba-se}
\hyphenation{plat-forms}
\hyphenation{or-ga-ni-za-tion}
\hyphenation{pro-gramming}
\hyphenation{in-stru-ments}
\hyphenation{Li-nux}
\hyphenation{sour-ce}
\hyphenation{en-vi-ron-ment}
\hyphenation{Te-le-pathy}
\hyphenation{Li-nux-ov-ka}
\hyphenation{Open-BSD}
\hyphenation{Free-BSD}
\hyphenation{men-ti-on-ed}
\hyphenation{app-li-ca-tion}

\def\progref!#1!{\texttt{#1}}
\renewcommand{\arraystretch}{2} %Іначай формулы ў матрыцы зліпаюцца з лініямі
\usepackage{array}

\def\interview #1 (#2), #3, #4, #5\par{

\section[#1, #3, #4]{#1 -- #3, #4}
\def\qname{LVEE}
\def\aname{#1}
\def\q ##1\par{{\noindent \bf \qname: ##1 }\par}
\def\a{{\noindent \bf \aname: } \def\qname{L}\def\aname{#2}}
}

\def\interview* #1 (#2), #3, #4, #5\par{

\section*{#1\\{\small\rm #3, #4. #5}}

\def\qname{LVEE}
\def\aname{#1}
\def\q ##1\par{{\noindent \bf \qname: ##1 }\par}
\def\a{{\noindent \bf \aname: } \def\qname{L}\def\aname{#2}}
}

\begin{document}
\title{SDN сегодня}
\author{Дмитрий Орехов, Минск, Беларусь\footnote{\url{Dmitry_Orekhov@epam.com}, \url{http://lvee.org/ru/abstracts/113}}}
\maketitle
\begin{abstract}
Today Software-Defined Networking is still a cutting-edge rather than a common production technology. In spite of this, SDN technologies are evolving actively, as well as it's enthusiasts com\-mu\-ni\-ty and engineer's culture. It holds out hope that common usage of SDN in world-wide networks is a matter of the closest future. And you're able to be a part of this!
We made a review of current SDN state and most valuable solutions which are available right now to build and manage SDN, it's platforms and technology stacks. Wherein we paid a special attention on Open Source software which enables SDN, so everyone can contribute in the future of networking.
\end{abstract}
\subsection*{Программно-управляемые сети}

Концепция SDN (Software-defined networking) основана на идее разделения уровня передачи данных и уровня управления правилами, по которым данные передаются. Обычно говорят, что на уровне управления действует Контроллер, а на уровне передачи данных "--- Свич. Контроллер при этом не только устанавливает правила для Свича, но и принимает от него сообщения о событиях, происходящих в сети, обеспечивая обратную связь.

\subsection*{OpenFlow}

Ясно, что одним из ключевых элементов SDN является протокол взаимодействия между Свичом и Контроллером. В настоящий момент таким протоколом в основном является OpenFlow. Ключевое понятие этого протокола "--- Flow или правило, по которому обрабатываются пактеы внутри Свича. Flows объединены в таблицу Flow Table, которые, в свою очередь, объединены в конвейер.

Фактически, Flows представляют собой микрокоманды для программирования сети. Как средство управления сетью, OpenFlow прошел большой путь от примитивной концепции, включающей в себя одну Flow Table и весьма ограниченный набор критериев, в версии 1.0, до версии 1.3 (готовится к выпуску версия 1.4), с конвейером таблиц, улучшенной обратной связью между Свичом и Контроллером, поддержкой множества критериев сравнения как то MPLS, IPv6 и т.\,д., и возможностью писать весьма сложные «программы» для управления сетью.

\subsection*{Открытые лицензии для открытого стандарта}

Уже сейчас наиболее интересные и полные решения, позволяющие строить OpenFlow топологии, опубликованы под различными Open Source лицензиями. Интересно то, что большая часть из них поддерживается крупными корпорациями "--- поставщиками сетевого обурудования и программного обеспечения. Решения эти используют очень разнообразные стеки технологий. Ниже представлен краткий список наиболее интересных, на мой взгляд, сообществ и их решений.

\begin{itemize}
  \item Project Floodlight (бывший OpenFlowHub.org)
Представляют контроллер OpenFlow Floodlight, базирующиеся на нём решения для управления сетью, Java API, REST API, а также средства для тестирования OpenFlow устройств. Основные языки программирования "--- Java, Python, C
  \item CPqD
Представляют OpenFlow свич и контроллер, OpenFlow драйвер с API для создания кастомных контроллеров, поддерживают дистрибутив OpenWRT с поддержкой OpenFlow. Основные языки "--- С/С++ и Python
  \item Ryu SDN framework
Очень функциональный фреймворк для создания контроллеров, написанный на Питоне. Оттестирован с большим числом свичей, поддерживается OpenStack'ом
  \item FlowForwarding.org
Представляет решения с использованием двух различных стеков:\begin{itemize}
  \item Erlang VM: LINC Switch, Loom controller, Tapestry "--- анализатор сети
  \item Java VM: Warp OpenFlow драйвер с API для создания кастомных контроллеров и Warp Controller на базе фреймворка Akka
\end{itemize}


\end{itemize}

\subsection*{Высокоуровневое программирование}

И все же OpenFlow остается низкоуровневым «языком программирования»; энтузаистами SDN был запущен проект Frenetiс, ставящий своей целью создание и развитие высокоуровневого языка программирования для сетей. С запуском же проекта Pyretic (Python + Frenetic) эта инициатива приобрела черты вполне реального domain"=specific language.

\subsection*{Прогулки с монстрами}

Усилиями SDN-сообщества, The Linux Foundation и корпораций-доноров был запущен весьма амбициозный проект OpenDaylight, представляющий собой SDN-стек и фреймворк для создания SDN-сетей.

Поскольку технологии SDN прекрасно подходят для создания виртуальных сетей, они широко используются в проекте OpenStack

\subsection*{Оптимизм}

Будущее SDN, еще недавно весьма туманное, теперь вызывает сдержанный оптимизм. Множество решений опубликованных под свободными лицензиями позволяют привлечь энтузиастов и стимулируют активное развитие области.

\end{document}
