\documentclass[10pt, a5paper]{article}
\usepackage{pdfpages}
\usepackage{parallel}
\usepackage[T2A]{fontenc}
\usepackage{ucs}
\usepackage[utf8x]{inputenc}
\usepackage[polish,english,russian]{babel}
\usepackage{hyperref}
\usepackage{rotating}
\usepackage[inner=2cm,top=1.8cm,outer=2cm,bottom=2.3cm,nohead]{geometry}
\usepackage{listings}
\usepackage{graphicx}
\usepackage{wrapfig}
\usepackage{longtable}
\usepackage{indentfirst}
\usepackage{array}
\newcolumntype{P}[1]{>{\raggedright\arraybackslash}p{#1}}
\frenchspacing
\usepackage{fixltx2e} %text sub- and superscripts
\usepackage{icomma} % коскі ў матэматычным рэжыме
\PreloadUnicodePage{4}

\newcommand{\longpage}{\enlargethispage{\baselineskip}}
\newcommand{\shortpage}{\enlargethispage{-\baselineskip}}

\def\switchlang#1{\expandafter\csname switchlang#1\endcsname}
\def\switchlangbe{
\let\saverefname=\refname%
\def\refname{Літаратура}%
\def\figurename{Іл.}%
}
\def\switchlangen{
\let\saverefname=\refname%
\def\refname{References}%
\def\figurename{Fig.}%
}
\def\switchlangru{
\let\saverefname=\refname%
\let\savefigurename=\figurename%
\def\refname{Литература}%
\def\figurename{Рис.}%
}

\hyphenation{admi-ni-stra-tive}
\hyphenation{ex-pe-ri-ence}
\hyphenation{fle-xi-bi-li-ty}
\hyphenation{Py-thon}
\hyphenation{ma-the-ma-ti-cal}
\hyphenation{re-ported}
\hyphenation{imp-le-menta-tions}
\hyphenation{pro-vides}
\hyphenation{en-gi-neering}
\hyphenation{com-pa-ti-bi-li-ty}
\hyphenation{im-pos-sible}
\hyphenation{desk-top}
\hyphenation{elec-tro-nic}
\hyphenation{com-pa-ny}
\hyphenation{de-ve-lop-ment}
\hyphenation{de-ve-loping}
\hyphenation{de-ve-lop}
\hyphenation{da-ta-ba-se}
\hyphenation{plat-forms}
\hyphenation{or-ga-ni-za-tion}
\hyphenation{pro-gramming}
\hyphenation{in-stru-ments}
\hyphenation{Li-nux}
\hyphenation{sour-ce}
\hyphenation{en-vi-ron-ment}
\hyphenation{Te-le-pathy}
\hyphenation{Li-nux-ov-ka}
\hyphenation{Open-BSD}
\hyphenation{Free-BSD}
\hyphenation{men-ti-on-ed}
\hyphenation{app-li-ca-tion}

\def\progref!#1!{\texttt{#1}}
\renewcommand{\arraystretch}{2} %Іначай формулы ў матрыцы зліпаюцца з лініямі
\usepackage{array}

\def\interview #1 (#2), #3, #4, #5\par{

\section[#1, #3, #4]{#1 -- #3, #4}
\def\qname{LVEE}
\def\aname{#1}
\def\q ##1\par{{\noindent \bf \qname: ##1 }\par}
\def\a{{\noindent \bf \aname: } \def\qname{L}\def\aname{#2}}
}

\def\interview* #1 (#2), #3, #4, #5\par{

\section*{#1\\{\small\rm #3, #4. #5}}

\def\qname{LVEE}
\def\aname{#1}
\def\q ##1\par{{\noindent \bf \qname: ##1 }\par}
\def\a{{\noindent \bf \aname: } \def\qname{L}\def\aname{#2}}
}

\begin{document}
\title{Надёжные ссылки}
\author{Андреев Юрий, Санкт-Петербург, РФ\footnote{\url{andreev.yurij@gmail.com}}}
\maketitle
\begin{abstract}
The rights to study and modify free software imply access to its source code.
The rights to verify scientific result imply access to its background and
previous results. That could be a problem if an access was given by a (url-)link.
The problem will be considered in the present article.
\end{abstract}

\subsection*{Обзор}
 
Свободное Программное Обеспечение (СПО) предполагает доступность исходного 
кода. Это отражает научный подход разработки ПО, при 
котором любой результат должен быть верифицируемым и открытым для 
критики. В этом смысле программный код является аналогом
научной публикации. Что же касается самих 
научных публикаций и публикации на тему СПО, то они предполагают 
доступность своих источников a priori.   
Однако, с появлением сети Интернет, источники всё чаще указываются
в виде url-ссылок, которые могут стать недоступными чисто технически.
И было бы обидно "засекретить" результат работы таким образом. 

Каждый день в глобальной сети появляются миллионы новых  ссылок.
И  каждый  день  миллионы  ссылок  становятся  битыми  –   то   есть
указывающими на недоступный или на неверный ресурс. Эти изменения  –
простое следствие динамичности  сети.  И  наличие  таких   временных
окон доступности ресурсов нужно  учитывать  при  проектировании  ПО,
подборке используемых  библиотек,  внедрении  удаленных  сервисов  и
контента; также тема битых ссылок постоянно обсуждается  в  связи  с
вопросами  поисковой  оптимизации  (SEO).   Учитываться  это   может
совершенно по разному.  К  примеру,  появился  «социальный»  сервис,
\texttt{snapchat.com}, принципиально  не  сохраняющий  контент  и  
ссылок  на него  – и  сразу  нашел  себе  пользователя.  
Нас  будет  интересовать
противоположная  ситуация,  и  ситуация  эта  будет  рассмотрена   в
контексте научных и научно-технических публикаций. Действительно,  в
этом случае желательно, чтобы ссылки указывали  на  верный  материал
когда бы статья ни была прочитана. Потеря же  ссылки  на  предыдущий
результат  в  научной  публикации  ставит  под   сомнение   верность
текущего  результата.  (Тут  можно  вспомнить  про  Великую  Теорему
Ферма,  которая  более  трёхсот  лет  оставалась  гипотезой,   из-за
«недоступности» оригинального доказательства /если оно было/.)

А  ссылки  появляются  везде.  Автор  лично  присутствовал   на
серьезной математической конференции,  где  один  из  докладчиков  в
своем выступлении приводил переписку из  ночного  чата.

Доступность материалов  в  библиотеке,  вообще  говоря,  кажется
более стабильной,  чем  в  интернете.  В  библиотеке найдется всё,
если... к примеру, не сгорит (однажды в Библиотеке Академии  Наук 
\cite{AY2}
я читал статью из сборника, который, судя  по  надписям  и  штампам,
сначала тонул в наводнении 1824 года, а потом горел  в  пожаре  1988
года). Более того,  существуют  системы  обязательного  хранения. 
С момента своего основания, Национальная Библиотека Беларуси
стала получать два обязательных бесплатных экземпляра белорусских 
изданий и один обязательный бесплатный экземпляр изданий, 
выходивших в СССР (РФ) до марта 1995 г, библиотека также является 
депозитарием материалов ООН и других 
международных организаций \cite{AY3}. Тоже можно сказать про 
Российскую    Государственную    Библиотеку. \cite{AY4}. 
Важно  понять,  что
найдя нужную книгу, мы сразу может начать читать, то  есть  получаем
доступ к её содержанию.
    С появлением компьютеров возник вопрос физической доступности  —
нельзя  считывать  информацию  напрямую.   А   носители   информации
меняются очень быстро,  и  с  магнитной  пленки,  да  и  с  дискеты,
информацию  теперь  прочитать  будет   крайне   затруднительно
\footnote{Однако, развитие идёт по спирали – утрату большой части
античного письменного наследия связывают также со сменой
технологий, так как множество текстов никогда не переписывалось с
папирусов на пергаментные кодексы \cite{AY5}.}. 
С появлением  интернета  вопрос  сохранения  копированием  стал  менее
актуален. Теперь вопрос сохранности информации  предстает  в  другом
разрезе -- в проблеме контроля за её доступностью.

\subsection*{Терминология}

Уточним терминологию. Мы будем рассматривать url-\textit{ссылки}, но в  более
общем смысле – как  ссылки  на  внешние,  по  отношению  к  материалу,
ресурсы       (ссылки       в       тексте,       из        разделов
<<ссылки>>,<<библиография>>,<<литература>>). Ссылка указывает  на  ресурс.
\textit{Ресурс} – это может быть какой-либо  контент,  сервис,  или  печатная
статья, книга. 
\textit{Надёжность ссылки} – способность указывать на верный, доступный 
ресурс.  

\subsection*{Классификация}

Классифицируем  ссылки  по  степени
надёжности. (С  допуском  на  то,  что  это  достаточно  
индивидуальный  вопрос, особенно для внутренних ссылок.)
Перечислим возможные варианты по степени убывания надёжности.
\begin{itemize}
\item Ссылки на части в оригинальном материале кажутся надёжными по 
определению, если это не документ, страницы из которого могут "выпасть".  
\item Разумеется, стабильнее всего ссылки,  принадлежащие  организациям,
отвечающим за их распределение и обслуживающим интернет:  
\texttt{icann.org,  w3.org,  example.com,...}
\item Сайты сообществ, таких как \texttt{gnu.org,  ctan.org}, специальные
архивы \texttt{arxiv.org}, университетские сайты, \texttt{<site>.edu}
\item Правительственные  сайты,   сайты   больших   компаний.   (Правда
бывает... В 1999  году  компания  Microsoft  забыла  продлить  домен
\texttt{passport.com},  что  привело  к  недоступности  Hotmail  для   многих
пользователей. Видимо, этот случай  не  послужил  Microsoft  уроком,
поскольку в 2003 году компания снова забыла продлить  важный  домен,
на этот раз \texttt{hotmail.co.uk}.\cite{AY1})
\item Нужно  учитывать  различные  характеристики:  есть  ли  архив   у
новостного сайта, сколько подписчиков и форков у проекта  на  github
и т.п.
    Вообще говоря, доступность  материала  по  ссылке  —  это  также
вопрос доступности сервиса, на  котором  этот  материал  расположен.
Одно дело, проект расположен на github, другое дело – на  каком-либо
частном ресурсе,  который  может  закрыться.  Пример  –  закрывшаяся
почта  @london.com  (кстати,  там  был  и  платный  сервис).  Автор,
указавший ее в контактах станет недоступен.
\item При ссылке  на  Википедию  нужно  учитывать,  что  материал  может
измениться. Понятно, что это не всегда  желательно.  При  ссылке  на
github, обычно, важнее  само  наличие  контента,  и,  в  большинстве
случаев, он, наоборот, должен изменяться.
\item Блоги.  Иногда  достаточно   постоянные,   зависит   от   автора.
Социальные сервисы в  этом  плане  намного динамичнее,  доступность
гораздо  хуже  (не  говоря  о  том,  что  доступность  может  просто
регулироваться закрытостью для незарегистрированных  пользователей).
На отдельном авторском блоге материалы  могут  представлять  большую
ценность для автора,  он  заинтересован  в  сохранении  доступности.
Хотя всегда есть вероятность, что имя не будет продлено.
\end{itemize}

Графически степень надёжности ссылок в пределах 
одного сайта можно изобразить в виде треугольной диаграммы (рис. 1).
Обычно, над корнем находятся основные ссылки (1), 
выше -- контент сайта (2),
еще выше -- пользовательский контент (3).   

\begin{figure}[h!]
  \centering
  \includegraphics[scale=0.8]{diagram.pdf}
\end{figure}

Выделим "центр тяжести" и составим общую диаграмму надёжности, 
соответствующую нашей классификации (рис. 2). 
Самые надёжные ссылки располагаются внизу. Ссылки с верхних слоёв 
легко сдуваются ветром перемен.

\subsection*{Структура ссылок и алгоритм поиска}

При выборе ссылки важно учитывать её структуру. Вкратце остановимся на этом.
Рассмотрим ссылку вида:

\texttt{http://<site-name>.tld/<section>/<page-name>?<query>}

Если ссылка становится недоступной, то куда мы попадём,  зависит  от
организации  редиректа   (перенаправления)   на   сайте.   Возможные
варианты: это  будет  страница с ошибкой,  главная  станица  или
страница  вышележащего  раздела;  лучший  вариант  –   редирект   на
актуальный адрес. Обычный алгоритм поиска таков:

\noindent
1. Ищем информацию на доступных ссылках сайта (двигаясь вниз по рис.1)
\begin{itemize}
\item \texttt{http://<site-name>.tld/<section>/<page-name>}
\item \texttt{http://<site-name>.tld/<section>/}
\item \texttt{http://<site-name>.tld}
\end{itemize}
2. Через «поиск»
\begin{itemize}
\item поиск \texttt{<section>, <page-name>} по сайту если есть возможность
\item google \texttt{<site-name>, <section>, <page-name>}
\item поиск исходя из контекста, в котором ссылка была указана.
\end{itemize}
3. Поиск "следов" информации в сети: сервис 
\texttt{archive.org}, кэши поисковых систем. 
\footnote{Отдельная тема. Наличие в кэше зависит 
от частоты индексации, настроек robots.txt, самой поисковой системы и др.
Время снимков \texttt{archive.org} не связано с изменениями на сайте, о чём они 
сразу честно предупреждают.}

Плохо, если  \texttt{<site-name>, <section>, <page-name>} 
не  несут  информации  (это, кстати, правило SEO).

К  примеру, \url{http://www.red-bean.com/kfogel/948.html}  
\footnote{на самом деле, это контрпример, там написано как раз об этом}
(заметьте, данный пример остается актуальным в рамках  этой  статьи  даже  если
ссылка  станет  недоступной)  Поэтому  следует приводить такие ссылки с
поясняющей информацией.

\subsection*{Примеры ссылок из сборника}
Приведем несколько интересных ссылок из сборника \cite{AY5}
\begin{itemize}
\item классическая ссылка: 
The GNU Project Debugger, \url{http://www.gnu.org/software/gdb/}
\item ссылка с датой просмотра: 
\url{http://www.blendernation.com/} Дата просмотра: 20.04.2014
\item сокращенная ссылка: \url{http://bit.ly/1sk1bZt}

Сервис bit.ly позволяет сокращать ссылки, делая из длинной
непонятной короткую непонятную. Хранит он ссылки вечно, но не
стоит забывать, что это всего лишь ссылка на ссылку. Она увеличивает
читабельность, но не надёжность.
\end{itemize}

\subsection*{Правила подбора ссылок}
Сформулируем правила подбора надёжных ссылок.
\begin{itemize}
\item Иногда следует добавлять информацию к ссылке, она может помочь, как 
в поиске информации, так и в понимании её актуальности.
\item Иногда следует включить материал в саму статью. К примеру, когда
материал – это чья-та цитата, размещенная на блоге. (В 
истории про Microsoft, описанной выше, непонятно, какую ссылку
давать, но гуглится хорошо.)
\item Из различных представлений одной ссылки выбираем более читабельное.
Для нечитабельных можно воспользоваться сервисами преобразования ссылок, 
при условии надёжности последних. Иногда лучше давать \textit{структурную} ссылку 
(см. пример -- \cite{AY3}).
\item Если есть выбор из нескольких ссылок, указывающих на один ресурс,
выбираем наиболее надёжную.
\item Если есть выбор между ссылкой на печатное издание или электронное,
выбираем обе.
\item Нужно исходить из актуальности основного материала и 
здравого смысла. Однако, значимость материала может недооцениваться 
автором, поэтому, если есть возможность, лучше давать более надёжную 
ссылку. 
\item Хорошо само существование раздела ссылок. Сам по себе, 
этот раздел помогает понимать общий контекст материала
в предметной области.  

\end{itemize}


\begin{thebibliography}{9}
\bibitem{AY1} \url{http://google.com}
\bibitem{AY2} \url{https://ru.wikipedia.org/wiki/}\\ \texttt{Библиотека\_Российской\_ академии\_наук}
\bibitem{AY3} Национальная Библиотека Беларуси, \url{http://nlb.by}, \texttt{главная $\to$ информационные ресурсы $\to$ фонды и коллекции библиотеки}
\bibitem{AY4} \url{https://ru.wikipedia.org/wiki/}\\ \texttt{Российская\_государственная\_библиотека},\\ 
см. также федеральный закон РФ <<Об обязательном экземпляре документов>> от 29.12.1994 № 77-ФЗ
\bibitem{AY5} \url{https://ru.wikipedia.org/wiki/}\\ \texttt{Книжные\_утраты\_в\_Поздней\_античности\_и\_«Тёмных\_веках»}
\bibitem{AY6} Материалы международной конференции Linux Vacation / Eastern
Europe (LVEE 2014), \url{http://lvee.org/ru/reports/materials_lvee_2014}
\end{thebibliography}

\end{document}
