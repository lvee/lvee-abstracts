\documentclass[10pt, a5paper]{article}
\usepackage{pdfpages}
\usepackage{parallel}
\usepackage[T2A]{fontenc}
\usepackage{ucs}
\usepackage[utf8x]{inputenc}
\usepackage[polish,english,russian]{babel}
\usepackage{hyperref}
\usepackage{rotating}
\usepackage[inner=2cm,top=1.8cm,outer=2cm,bottom=2.3cm,nohead]{geometry}
\usepackage{listings}
\usepackage{graphicx}
\usepackage{wrapfig}
\usepackage{longtable}
\usepackage{indentfirst}
\usepackage{array}
\newcolumntype{P}[1]{>{\raggedright\arraybackslash}p{#1}}
\frenchspacing
\usepackage{fixltx2e} %text sub- and superscripts
\usepackage{icomma} % коскі ў матэматычным рэжыме
\PreloadUnicodePage{4}

\newcommand{\longpage}{\enlargethispage{\baselineskip}}
\newcommand{\shortpage}{\enlargethispage{-\baselineskip}}

\def\switchlang#1{\expandafter\csname switchlang#1\endcsname}
\def\switchlangbe{
\let\saverefname=\refname%
\def\refname{Літаратура}%
\def\figurename{Іл.}%
}
\def\switchlangen{
\let\saverefname=\refname%
\def\refname{References}%
\def\figurename{Fig.}%
}
\def\switchlangru{
\let\saverefname=\refname%
\let\savefigurename=\figurename%
\def\refname{Литература}%
\def\figurename{Рис.}%
}

\hyphenation{admi-ni-stra-tive}
\hyphenation{ex-pe-ri-ence}
\hyphenation{fle-xi-bi-li-ty}
\hyphenation{Py-thon}
\hyphenation{ma-the-ma-ti-cal}
\hyphenation{re-ported}
\hyphenation{imp-le-menta-tions}
\hyphenation{pro-vides}
\hyphenation{en-gi-neering}
\hyphenation{com-pa-ti-bi-li-ty}
\hyphenation{im-pos-sible}
\hyphenation{desk-top}
\hyphenation{elec-tro-nic}
\hyphenation{com-pa-ny}
\hyphenation{de-ve-lop-ment}
\hyphenation{de-ve-loping}
\hyphenation{de-ve-lop}
\hyphenation{da-ta-ba-se}
\hyphenation{plat-forms}
\hyphenation{or-ga-ni-za-tion}
\hyphenation{pro-gramming}
\hyphenation{in-stru-ments}
\hyphenation{Li-nux}
\hyphenation{sour-ce}
\hyphenation{en-vi-ron-ment}
\hyphenation{Te-le-pathy}
\hyphenation{Li-nux-ov-ka}
\hyphenation{Open-BSD}
\hyphenation{Free-BSD}
\hyphenation{men-ti-on-ed}
\hyphenation{app-li-ca-tion}

\def\progref!#1!{\texttt{#1}}
\renewcommand{\arraystretch}{2} %Іначай формулы ў матрыцы зліпаюцца з лініямі
\usepackage{array}

\def\interview #1 (#2), #3, #4, #5\par{

\section[#1, #3, #4]{#1 -- #3, #4}
\def\qname{LVEE}
\def\aname{#1}
\def\q ##1\par{{\noindent \bf \qname: ##1 }\par}
\def\a{{\noindent \bf \aname: } \def\qname{L}\def\aname{#2}}
}

\def\interview* #1 (#2), #3, #4, #5\par{

\section*{#1\\{\small\rm #3, #4. #5}}

\def\qname{LVEE}
\def\aname{#1}
\def\q ##1\par{{\noindent \bf \qname: ##1 }\par}
\def\a{{\noindent \bf \aname: } \def\qname{L}\def\aname{#2}}
}


\begin{document}

\title{Writeat : доступные книги для читателей и писателей}%\footnote{Текст данных и последующих тезисов, кроме специально оговоренных случаев, доступен под лицензией Creative Commons Attribution-ShareAlike 3.0}

\author{Александр Загацкий\footnote{Витебск, Беларусь}}
\maketitle

\begin{abstract}
Writeat is free and simple tool for creating electronic and printed books. Books are written in a pod6 format. Pod6 --- a simple and concise markup language. To create a file in this format you can use any text editor. Writeat is opensource startup.
\end{abstract}


Современная экономика направлена на воспитание в нас потребителей. Поэтому с 
понятием <<доступный>>  свзяывают в первую очередь низкую стоимость какого-либо 
товара или услуги. Например: доступное жилье, доступные продукты питания, 
доступные книги.

Однако насколько легко построить дом или написать книгу? Ответ на данный вопрос
зависит от доступности орудий производтства и технологий. Если построить дом просто для строителя, то и квартиры в этом доме будут доступны большему количеству жильцов.

Проект Writeat \cite{zag1} является бесплатным и простым инструментом для создания книг в 
электронном и печатном виде.

Пишутся книги в формате pod6. Pod6--простой и лаконичный язык разметки.
Для создания файлов в этом формате подойдет любой текстовый редактор.

Шаблон книги выглядит следующим образом:

\begin{verbatim}
=TITLE Моя очередная книга
=SUBTITLE или как просто делиться знаниями
=AUTHOR
firstname:Вася
surname:Пупкин
=DESCRIPTION
В этой книге описаны основные правила, которые позволят 
наиболее просто поделиться своими знаниями и опытом.
=CHAPTER Вступление
Начало книги !
\end{verbatim}

Специальные директивы начинаются со знака <<=>> и следующим за ним названия блока.
Используются следующие блоки:

\begin{itemize}
  \item[~] =TITLE   заголовок книги
  \item[~] =SUBTITLE подзаголовок
  \item[~] =AUTHOR автор
  \item[~] =DESCRIPTION краткое описание книги
  \item[~] =CHAPTER название главы
\end{itemize}

Основной чертой формата pod6 является его расширяемость. Благодаря этому 
стала возможной вставка изображений и разбиение книг на части.

Иногда бывает удобно, чтобы главы располагались в отдельных файлах. Для этих целей используется блок =Include :

\begin{verbatim}
=Include src/preface.pod6
=Include src/basics.pod6
=Include src/operators.pod6
=Include src/subs-n-sigs.pod6
\end{verbatim}

Такой прием облегчает совместную работу над книгой нескольких авторов.

Для вставки изображений используется следующий блок:

\begin{verbatim}
=Image img/bold1.jpg
\end{verbatim}

Writeat позволяет облегчить поддержку технической документации, особенно публичную. Дополнительный блок CHANGES позволяет вести журнал изменений документа. Он 
располагается в самом начале документа и необходим для описания основных 
изменений. Например, изменения в API сервиса.

\begin{verbatim}
=begin CHANGES
Jun 6th 2012(v0.2)[zag]   Предисловие
May 27th 2012(v0.1)[zag]   Начальная версия
=end CHANGES
\end{verbatim}

Для установки последней версии пакета writeat для Ubuntu необходимо выполнить команды:

\begin{verbatim}
sudo add-apt-repository ppa:zahatski/ppa
sudo apt-get install writeat
man writeat
\end{verbatim}

Writeat на данный момент находится в начале развития. Он бесплатен и открыт \cite{zag2}. В его основе лежат открытые технологии и форматы.
Наличие бесплатного и простого способа создать книгу позволяет решить задачу
доступности книг как для читателей, так и 
для писателей. Это значит, что будущее, в котором учебники 
бесплатны, вполне реально.
\begin{thebibliography}{9}
\bibitem{zag1} Сайт проекта с образцами книг.\url{http://writeat.com}
\bibitem{zag2} Репозиторий проекта writeat. \url{https://github.com/zag/writeat}
\end{thebibliography}


\end{document}




