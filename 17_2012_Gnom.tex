\documentclass[10pt, a5paper]{article}
\usepackage{pdfpages}
\usepackage{parallel}
\usepackage[T2A]{fontenc}
\usepackage{ucs}
\usepackage[utf8x]{inputenc}
\usepackage[polish,english,russian]{babel}
\usepackage{hyperref}
\usepackage{rotating}
\usepackage[inner=2cm,top=1.8cm,outer=2cm,bottom=2.3cm,nohead]{geometry}
\usepackage{listings}
\usepackage{graphicx}
\usepackage{wrapfig}
\usepackage{longtable}
\usepackage{indentfirst}
\usepackage{array}
\newcolumntype{P}[1]{>{\raggedright\arraybackslash}p{#1}}
\frenchspacing
\usepackage{fixltx2e} %text sub- and superscripts
\usepackage{icomma} % коскі ў матэматычным рэжыме
\PreloadUnicodePage{4}

\newcommand{\longpage}{\enlargethispage{\baselineskip}}
\newcommand{\shortpage}{\enlargethispage{-\baselineskip}}

\def\switchlang#1{\expandafter\csname switchlang#1\endcsname}
\def\switchlangbe{
\let\saverefname=\refname%
\def\refname{Літаратура}%
\def\figurename{Іл.}%
}
\def\switchlangen{
\let\saverefname=\refname%
\def\refname{References}%
\def\figurename{Fig.}%
}
\def\switchlangru{
\let\saverefname=\refname%
\let\savefigurename=\figurename%
\def\refname{Литература}%
\def\figurename{Рис.}%
}

\hyphenation{admi-ni-stra-tive}
\hyphenation{ex-pe-ri-ence}
\hyphenation{fle-xi-bi-li-ty}
\hyphenation{Py-thon}
\hyphenation{ma-the-ma-ti-cal}
\hyphenation{re-ported}
\hyphenation{imp-le-menta-tions}
\hyphenation{pro-vides}
\hyphenation{en-gi-neering}
\hyphenation{com-pa-ti-bi-li-ty}
\hyphenation{im-pos-sible}
\hyphenation{desk-top}
\hyphenation{elec-tro-nic}
\hyphenation{com-pa-ny}
\hyphenation{de-ve-lop-ment}
\hyphenation{de-ve-loping}
\hyphenation{de-ve-lop}
\hyphenation{da-ta-ba-se}
\hyphenation{plat-forms}
\hyphenation{or-ga-ni-za-tion}
\hyphenation{pro-gramming}
\hyphenation{in-stru-ments}
\hyphenation{Li-nux}
\hyphenation{sour-ce}
\hyphenation{en-vi-ron-ment}
\hyphenation{Te-le-pathy}
\hyphenation{Li-nux-ov-ka}
\hyphenation{Open-BSD}
\hyphenation{Free-BSD}
\hyphenation{men-ti-on-ed}
\hyphenation{app-li-ca-tion}

\def\progref!#1!{\texttt{#1}}
\renewcommand{\arraystretch}{2} %Іначай формулы ў матрыцы зліпаюцца з лініямі
\usepackage{array}

\def\interview #1 (#2), #3, #4, #5\par{

\section[#1, #3, #4]{#1 -- #3, #4}
\def\qname{LVEE}
\def\aname{#1}
\def\q ##1\par{{\noindent \bf \qname: ##1 }\par}
\def\a{{\noindent \bf \aname: } \def\qname{L}\def\aname{#2}}
}

\def\interview* #1 (#2), #3, #4, #5\par{

\section*{#1\\{\small\rm #3, #4. #5}}

\def\qname{LVEE}
\def\aname{#1}
\def\q ##1\par{{\noindent \bf \qname: ##1 }\par}
\def\a{{\noindent \bf \aname: } \def\qname{L}\def\aname{#2}}
}


\begin{document}

\title{Использование клиент-серверной архитектуры MythTV 0.25}%\footnote{Текст данных и последующих тезисов, кроме специально оговоренных случаев, доступен под лицензией Creative Commons Attribution-ShareAlike 3.0}

\author{Алексей Бутько\footnote{Минск, Беларусь}}
\maketitle

\begin{abstract}
The article discusses specifics of MythTV usage. MythTV turns a computer into a network streaming digital video recorder, a digital multimedia home entertainment system, or home theater personal computer. It has two main logical elements: the backend, which contains the TV capture cards, and stores the recorded video, and the frontend, which is connected to user's TV screen and lets him watch LiveTV and recorded shows. The simplest configuration puts both frontend and backend in the same physical box, while advanced setup might separate backend and frontend hardware.
\end{abstract}


Слияние технологий сегодня играет немаловажную роль в цифровых устройствах, которые мы используем. В целом, пакет MythTV можно рассматривать как центр управления всеми приложениями и устройствами, которые отвечают за наш цифровой досуг.

MythTV в первом приближении состоит из двух логических компонентов:

\begin{itemize}
  \item backend --- работает с картами захвата, хранит и записывает видео.
  \item frontend --- пользовательский интерфейс, получающий данные от backend'а и выводящий изображение непосредственно на экран телевизора.
\end{itemize}

Обычно под построение HTPC выделяется один физический системный блок, вследствие чего backend и frontend делят его между собой.  Однако все функции такого мощного инструмента как MythTV раскрываются как раз в сетевом (клиент-серверном) исполнении. Первое, и главное преимущество --- это возможность использования нескольких frontend'ов по всему дому или даже за его пределами. Также появляется возможность управления конкретным frontend'ом с мобильного устройства по сети. Такой метод управления с расширением использования технологий WiFi приобретает всё большую популярность и вскоре обещает вытеснить традиционные инфракрасные пульты. Не так давно возможность использовать MythTV (иметь полнофункциональный frontend)  появилась на платформах Android и iPhone, что делает медиа-контент еще более доступным.

Возможности MythTV и впрямь широки: система умеет работать с большинством карт захвата (аналоговых и цифровых), имеется поддержка IPTV (хотя по мнению автора ещё недостаточно зрелая), мощный планировщик может вести запись телепрограмм по расписанию, имеется возможность удаления рекламы.

Еще больше возможностей содержат плагины. Из офицальных можно перечислить:

\begin{itemize}
  \item MythBrowser --- минибраузер, не полнофйункциональный, но может оказаться полезным.
  \item MythArchive --- средство создания DVD из имеющихся ТВ-записей и любых других видеофайлов.
  \item MythGallery --- инструмент для просмотра изображений.
  \item MythGame --- интерфейс для эмуляторов игровых консолей
  \item MythVideo --- в текущей версии не являестя плагином, а входит в состав frontend'а. Позволяет обозревать и просматривать домашнюю коллекцию фильмов. Примечателен тем, что выискивает подробную информацию о конкретной картине в онлайн-сервисах.
  \item MythMusic --- прослушивание музыки, составление playlist по различным критериям. До последней версии интерфейс обладал низким функционалом и слабой эргономикой
  \item MythNews --- отслеживание лент RSS и отображение новостей на экране.
  \item MythWeather --- онлайновый прогноз погоды с весьма обширным функционалом. Работает с различными источниками данных, может отображать живую карту метеообстановки.
  \item MythWeb --- плагин для удалённого web-конфигурирования backend'а. В условиях вынесения последнего на отдельный сервер может оказаться крайне полезным.
\end{itemize}

Последние полтора года активно шла работа над версией 0.25, и, наконец, в апреле состоялся релиз.  В новую версии вошло более 5200 коммитов. Из ключевых улучшений можно отметить:

\begin{itemize}
  \item Долгожданная поддержка аппаратного ускорения декодирования видео с использованием VAAPI и поддержка архитектуры акселерации DirectX Video Acceleration 2;
  \item Поддержка аудио-кодеков E-AC3, TrueHD и DTS-HD;
  \item Улучшены средства для управления метаданными для записываемых видеоматериалов. Удалена поддержка утилиты для работы с метаданными jamu, вместо которой теперь используется компонент MythMetadataLookup;
  \item Представлен полнофункциональный сервисный API для обеспечения взаимодействия внешних приложений c MythTV, как с бэкендом, так и с фронтэндом. Новый API можно использовать в том числе для организации потоковой доставки контента поверх HTTP (HTTP Live Streaming). Ранее используемый API MythXML объявлен устаревшим;
  \item Полностью переписан модуль MythMusic, используемый для обеспечения проигрывания музыки и управления музыкальной коллекцией. Переработана архитектура видеоплеера \linebreak MythVideo. Функции MythMusic и MythVideo теперь непосредственно интегрированы в MythTV, а не распространяются в виде плагинов;
  \item Коллекция визуальных тем MythThemes более не рассматривается как внешний репозиторий, все визуальные темы, включая темы от сторонних разработчиков,  могут быть загружены непосредственно через интерфейс выбора тем, интегрированный во фронтэнд;
  \item Проигрыванием контента при помощи MythNetvision, например, при просмотре роликов из YouTube, можно управлять через пульт ДУ. В MythNetvision по возможности используется встроенный базовый плеер MythTV;
  \item Поддержка 3D-эффектов при выводе горизнтального и вертикального меню; 
Начальная поддержка анимации в MythUI;
  \item Прекращена поддержка механизма акселерации XvMC и удалена поддержка libmpeg2 для проигрывания видео;
  \item Переписана система ведения логов;
  \item Прекращена поддержка Python 2.5, в качестве минимальной версии рекомендуется Python 2.6. Также для работы требуется Taglib 1.6+ и Qt 4.6+. Из списка зависимостей исключены libvisual, libsdl, libcdaudio, libcdda\_paranoia и wget.
\end{itemize}

Резюмируя, можно сказать, что MythTV развился в серьёзный и универсальный инструмент для построения разветвлённой домашней медиасистемы. Пожалуй,  единственное, что мешает массовому его использованию и отпугивает пользователей --- черезмерная  сложность в установке и настройке. Рекомендации по его конфигурированию имеют объем отдельной книгу.



\end{document}




