\documentclass[10pt, a5paper]{article}
\usepackage{pdfpages}
\usepackage{parallel}
\usepackage[T2A]{fontenc}
\usepackage{ucs}
\usepackage[utf8x]{inputenc}
\usepackage[polish,english,russian]{babel}
\usepackage{hyperref}
\usepackage{rotating}
\usepackage[inner=2cm,top=1.8cm,outer=2cm,bottom=2.3cm,nohead]{geometry}
\usepackage{listings}
\usepackage{graphicx}
\usepackage{wrapfig}
\usepackage{longtable}
\usepackage{indentfirst}
\usepackage{array}
\newcolumntype{P}[1]{>{\raggedright\arraybackslash}p{#1}}
\frenchspacing
\usepackage{fixltx2e} %text sub- and superscripts
\usepackage{icomma} % коскі ў матэматычным рэжыме
\PreloadUnicodePage{4}

\newcommand{\longpage}{\enlargethispage{\baselineskip}}
\newcommand{\shortpage}{\enlargethispage{-\baselineskip}}

\def\switchlang#1{\expandafter\csname switchlang#1\endcsname}
\def\switchlangbe{
\let\saverefname=\refname%
\def\refname{Літаратура}%
\def\figurename{Іл.}%
}
\def\switchlangen{
\let\saverefname=\refname%
\def\refname{References}%
\def\figurename{Fig.}%
}
\def\switchlangru{
\let\saverefname=\refname%
\let\savefigurename=\figurename%
\def\refname{Литература}%
\def\figurename{Рис.}%
}

\hyphenation{admi-ni-stra-tive}
\hyphenation{ex-pe-ri-ence}
\hyphenation{fle-xi-bi-li-ty}
\hyphenation{Py-thon}
\hyphenation{ma-the-ma-ti-cal}
\hyphenation{re-ported}
\hyphenation{imp-le-menta-tions}
\hyphenation{pro-vides}
\hyphenation{en-gi-neering}
\hyphenation{com-pa-ti-bi-li-ty}
\hyphenation{im-pos-sible}
\hyphenation{desk-top}
\hyphenation{elec-tro-nic}
\hyphenation{com-pa-ny}
\hyphenation{de-ve-lop-ment}
\hyphenation{de-ve-loping}
\hyphenation{de-ve-lop}
\hyphenation{da-ta-ba-se}
\hyphenation{plat-forms}
\hyphenation{or-ga-ni-za-tion}
\hyphenation{pro-gramming}
\hyphenation{in-stru-ments}
\hyphenation{Li-nux}
\hyphenation{sour-ce}
\hyphenation{en-vi-ron-ment}
\hyphenation{Te-le-pathy}
\hyphenation{Li-nux-ov-ka}
\hyphenation{Open-BSD}
\hyphenation{Free-BSD}
\hyphenation{men-ti-on-ed}
\hyphenation{app-li-ca-tion}

\def\progref!#1!{\texttt{#1}}
\renewcommand{\arraystretch}{2} %Іначай формулы ў матрыцы зліпаюцца з лініямі
\usepackage{array}

\def\interview #1 (#2), #3, #4, #5\par{

\section[#1, #3, #4]{#1 -- #3, #4}
\def\qname{LVEE}
\def\aname{#1}
\def\q ##1\par{{\noindent \bf \qname: ##1 }\par}
\def\a{{\noindent \bf \aname: } \def\qname{L}\def\aname{#2}}
}

\def\interview* #1 (#2), #3, #4, #5\par{

\section*{#1\\{\small\rm #3, #4. #5}}

\def\qname{LVEE}
\def\aname{#1}
\def\q ##1\par{{\noindent \bf \qname: ##1 }\par}
\def\a{{\noindent \bf \aname: } \def\qname{L}\def\aname{#2}}
}

\begin{document}
\title{Обобщенные деревья поиска: \\[2mm] от теории к практике\footnote{\url{andreev_yurij@mail.ru}, \url{http://lvee.org/ru/abstracts/232}}}
\author{Андреев Юрий, Санкт-Петербург, РФ}
\maketitle
\begin{abstract}

This article describes the Generalized Search Tree (GiST), an index structure, 
which allows new data types to be indexed in a manner supporting queries natural to the types.

В статье дается краткий обзор обобщенных деревьев поиска (GiST),
которые облегчают и унифицируют разработку методов доступа, оптимальных для конкретного типа данных.
\end{abstract}

\subsection*{Постановка задачи}
За эффективный доступ к данным в современных Системах Управления Базами Данных (СУБД) 
отвечают специальные структуры, \textit{\textbf{индексы}, реализующие алгоритмы поиска
и обеспечивающие поддержку надежности, конкурентности, восстановления}.
Используя особенности конкретного типа данных, обычно удается найти алгоритмы,
работающие эффективнее общих схем.
Однако, реализация таких алгоритмов в индексе сопряжена с проблемами обеспечения должной поддержки.
Решение заключается в том, чтобы реализовать в СУБД структуру индекса на более высоком уровне абстракции (GiST),
оставив пользователю свободу в описании алгоритма через функциональный интерфейс. 

Классической задачей, требующий специальный подход к индексированию, являются
задачи с пространственными данными, \textit{spatial searching}. Примером такой задачи
может быть нахождение всех географических объектов, расположенных в определенной близости к заданному.
Осложнением в данном случае являет то, что такие объекты часто не могут быть представлены лишь парой координат,
а занимают некоторую площадь.  

\subsection*{Реализация}
В качестве алгоритма для решения описанной задачи, задачи индексации многомерной информации, 
было предложено построение R-Tree дерева (\textit{Rectangle} Tree).
Хорошее представление об алгоритмах, работающих с различными типами данных, можно получить,
познакомившись с такими структурами, как R-Tree, RD-Tree (\textit{Russian Doll} Tree), KD-Tree (\textit{k-Dimensional} Tree) 
(\cite{DB}, \cite{R-Tree}, \cite{RD-Tree}, Wikipedia).

Обобщенное дерево поиска, \textit{\textbf{Generalized Search Tree (GiST)}}, 
обеспечивает унифицированный подход для реализации алгоритмов \cite{GiST}. 
(В частности, посредством GiST может быть реализовано R-Tree дерево.)  

Для детального практического изучения индексирования, важно иметь свободную СУБД,
с реализацией GiST. Также желательна хорошая расширяемость и документация.    
Примером такой СУБД может служить СУБД PostgreSQL, обладающая всеми перечисленными свойствами.
PostgreSQL предоставляет семь интерфейсных функций, для работы со структурой GiST. 
Они, в свою очередь, манипулируют тремя базовыми методами дерева: search, insert, delete.

\textit{Пользователь может контролировать само построение дерева поиска}
с помощью функций \texttt{union} и \texttt{picksplit}, отвечающих за объединение и разделение узлов дерева. 
Функцией \texttt{penalty} определяется метод оценки сложности вставки новых данных.
Минимизация <<пенальти>> является показателем эффективности индексации и учитывается при перестроении дерева.

\textit{Пользователь может контролировать поиск по дереву}. Решение о переходе в один из дочерних узлов 
принимается с помощью функции \texttt{consistent}. 

Реализация и примеры использования GiST в PostgreSQL подробно изложены в \cite{Pg_GiST}. 
Наряду с GIST, в PostgreSQL реализованы и другие обобщенные структуры, такие как SP-GIST (\textit{Space-Partitioned} GIST).
Техническая документация также является хорошим и легко читаемым источником информации \cite{Pg_Docs}. 

\begin{thebibliography}{9}
\bibitem{DB} Гектор Гарсиа-Молина, Джеффри Ульман, Дженнифер Уидом, <<Системы Баз Данных>>, изд. <<Вильямс>> 2002; 
в частности, гл.14 <<Многомерные и точечные индексы>> 
\bibitem{R-Tree} Antonin Guttman, <<R-Trees: A Dynamic Index Structure for Spatial Searching>>
\bibitem{RD-Tree} Joseph M. Hellerstein, Avi Pfeffer, <<The RD-Tree: An Index Structure for Sets>>
\bibitem{GiST} Joseph M. Hellerstein, Jeffrey F. Naughton, Avi Pfeffer, <<Generalized Search Trees for Database Systems>>,
In Proceedings of the 21st International Conference on Very Large Data Bases, Zurich, Switzerland, 1995.
\bibitem{Pg_GiST} Олег Бартунов, Федор Сигаев, <<Написание расширений для PostgreSQL с использованием GiST>>, \url{http://citforum.ru/database/postgres/gist/}
\bibitem{Pg_Docs} PostgreSQL Documentation, \url{https://www.postgresql.org/docs/}
\end{thebibliography}

\end{document}
