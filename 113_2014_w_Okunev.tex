\documentclass[10pt, a5paper]{article}
\usepackage{pdfpages}
\usepackage{parallel}
\usepackage[T2A]{fontenc}
\usepackage{ucs}
\usepackage[utf8x]{inputenc}
\usepackage[polish,english,russian]{babel}
\usepackage{hyperref}
\usepackage{rotating}
\usepackage[inner=2cm,top=1.8cm,outer=2cm,bottom=2.3cm,nohead]{geometry}
\usepackage{listings}
\usepackage{graphicx}
\usepackage{wrapfig}
\usepackage{longtable}
\usepackage{indentfirst}
\usepackage{array}
\newcolumntype{P}[1]{>{\raggedright\arraybackslash}p{#1}}
\frenchspacing
\usepackage{fixltx2e} %text sub- and superscripts
\usepackage{icomma} % коскі ў матэматычным рэжыме
\PreloadUnicodePage{4}

\newcommand{\longpage}{\enlargethispage{\baselineskip}}
\newcommand{\shortpage}{\enlargethispage{-\baselineskip}}

\def\switchlang#1{\expandafter\csname switchlang#1\endcsname}
\def\switchlangbe{
\let\saverefname=\refname%
\def\refname{Літаратура}%
\def\figurename{Іл.}%
}
\def\switchlangen{
\let\saverefname=\refname%
\def\refname{References}%
\def\figurename{Fig.}%
}
\def\switchlangru{
\let\saverefname=\refname%
\let\savefigurename=\figurename%
\def\refname{Литература}%
\def\figurename{Рис.}%
}

\hyphenation{admi-ni-stra-tive}
\hyphenation{ex-pe-ri-ence}
\hyphenation{fle-xi-bi-li-ty}
\hyphenation{Py-thon}
\hyphenation{ma-the-ma-ti-cal}
\hyphenation{re-ported}
\hyphenation{imp-le-menta-tions}
\hyphenation{pro-vides}
\hyphenation{en-gi-neering}
\hyphenation{com-pa-ti-bi-li-ty}
\hyphenation{im-pos-sible}
\hyphenation{desk-top}
\hyphenation{elec-tro-nic}
\hyphenation{com-pa-ny}
\hyphenation{de-ve-lop-ment}
\hyphenation{de-ve-loping}
\hyphenation{de-ve-lop}
\hyphenation{da-ta-ba-se}
\hyphenation{plat-forms}
\hyphenation{or-ga-ni-za-tion}
\hyphenation{pro-gramming}
\hyphenation{in-stru-ments}
\hyphenation{Li-nux}
\hyphenation{sour-ce}
\hyphenation{en-vi-ron-ment}
\hyphenation{Te-le-pathy}
\hyphenation{Li-nux-ov-ka}
\hyphenation{Open-BSD}
\hyphenation{Free-BSD}
\hyphenation{men-ti-on-ed}
\hyphenation{app-li-ca-tion}

\def\progref!#1!{\texttt{#1}}
\renewcommand{\arraystretch}{2} %Іначай формулы ў матрыцы зліпаюцца з лініямі
\usepackage{array}

\def\interview #1 (#2), #3, #4, #5\par{

\section[#1, #3, #4]{#1 -- #3, #4}
\def\qname{LVEE}
\def\aname{#1}
\def\q ##1\par{{\noindent \bf \qname: ##1 }\par}
\def\a{{\noindent \bf \aname: } \def\qname{L}\def\aname{#2}}
}

\def\interview* #1 (#2), #3, #4, #5\par{

\section*{#1\\{\small\rm #3, #4. #5}}

\def\qname{LVEE}
\def\aname{#1}
\def\q ##1\par{{\noindent \bf \qname: ##1 }\par}
\def\a{{\noindent \bf \aname: } \def\qname{L}\def\aname{#2}}
}

\begin{document}
\title{clsync "--- live sync utility}
\author{Dmitry Okunev, NRNU MEPhI, Moscow, Russia}
\maketitle
\begin{abstract}
The report focuses on a live syncing utility "clsync" developed by UNIX-tech department of NRNU MEPhI for an LXC-based infrastructure. "clsync" is free and open solution, that appears as replacement of "lsyncd" in fine-tuned systems. A practical experience of applying the utility to setup an LXC HA cluster, a backuping system and a configuration files syncing through an HPC cluster is given.
\end{abstract}
Поддержка синхронности файлов между узлами — это типовая задача при реализации большинства кластеров. Для решеиня разных задач формулируются разные требования к синхронности файлов между узлами, но в рамках данной работы в первую очередь рассматриваются системы высокой доступности (от англ. «high availability», далее — «HA»). В HA-системах типовыми требованиями являются:

\begin{itemize}
  \item высокая производительность (сравнимая с производительностью локальной файловой системы на локальной дисковой подсистеме);
  \item высокая доступность (отказ сервиса не более нескольких секунд);
  \item высокая надёжность (не создавать дополнительных отказов сервиса за счёт использования данного решения);
  \item универсальность (применимость для широкого спектра разных решений, разворачиваемых на данной системе).
\end{itemize}

В общем случае, данная задача на данный момент остаётся нерешённой. Однако создано большое множество решений, которые в разных приближениях решают эту задачу, определяя свои порядоки приоритетов данных требований. И основными подходами на данный момент являются:

\begin{itemize}
  \item \textbf{файловые системы только для чтения} — это простое и надёжное решение, однако имеет очень узкий спект применимости;
  \item \textbf{единые общие файловые хранилища} — простое и универсальное решение, но создаёт единую точку отказа;
  \item \textbf{блочная репликация} — надёжное и универсальное решение, но создаёт большие потери производительности и очень чувствительно к качеству интерконнекта;
  \item \textbf{файловая репликация} — компромиссное решение, которое не выполняет ни одно требование полностью;
  \item комбинации вышеперечисленных подходов.
\end{itemize}

В НИЯУ МИФИ кроме прочих задач необходимо поддерживать HA-инфраструктуру c резервным копированием для быстрого создания VPS, в которых допускается работать малоопытным web-программистам \footnotemark[1]. Опыт показал, что использование блочной репликации для в данной ситуации приводит к недопустимой потере производительности. Поэтому был использован lsyncd \footnotemark[2] для синхронизации LXC-контейнеров между узлами кластера и их резервного копирования.

Однако lsyncd оказался сложным в настройке для ряда экзотических ситуаций, малопроизводительным, а также недостаточно надёжным, простым в отладке, гибким и переносимым. В результате был написана альтернативная реализация под внутренние нужды отдела UNIX-технологий НИЯУ МИФИ — clsync \footnotemark[3]. Далее clsync был документирован и опубликован через репозитории пакетов Debian.

Clsync написан на GNU C99 с использованием inotify \footnotemark[4] и адаптирован для реализации LXC HA-кластера (в связке с rsync \footnotemark[5]), создания системы резервного копирования (аналогично) и синхронизации конфигурационных файлов на HPC-кластере (в связке с pdcp\footnotemark[6]).

Для реализации LXC HA-кластера в clsync начата реализация подсистемы уведомления других инстанций посредством multicast. Однако данная подсистема перестала быть актуальной после перенастройки кластера для запуска отдельной инстанции для каждого LXC контейнера. На данный момент в процессе реализации API и менеджер для централизованного управления инстанциями clsync.

\textbf{Список литературы:}
1. «Опыт внедрения отказоустойчивого web-кластера для портала приёмной комиссии НИЯУ МИФИ», Окунев Д.Ю., научная сессия НИЯУ МИФИ, 2012, http://www.pandia.ru/text/78/343/297.php
2. «Manual to Lsyncd 2.1.x», https://github.com/axkibe/lsyncd/wiki/Manual-to-Lsyncd-2.1.x
3. «file live sync daemon based on inotify, written in GNU C», https://github.com/xaionaro/clsync
4. «inotify -- monitoring file system events», http://linux.die.net/man/7/inotify
5. «rsync --- a fast, versatile, remote (and local) file-copying tool», http://linux.die.net/man/1/rsync
6. «pdcp(1) -- Linux man page», http://linux.die.net/man/1/pdcp

\end{document}
