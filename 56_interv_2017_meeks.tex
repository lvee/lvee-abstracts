\documentclass[10pt, a5paper]{article}
\usepackage{pdfpages}
\usepackage{parallel}
\usepackage[T2A]{fontenc}
\usepackage{ucs}
\usepackage[utf8x]{inputenc}
\usepackage[polish,english,russian]{babel}
\usepackage{hyperref}
\usepackage{rotating}
\usepackage[inner=2cm,top=1.8cm,outer=2cm,bottom=2.3cm,nohead]{geometry}
\usepackage{listings}
\usepackage{graphicx}
\usepackage{wrapfig}
\usepackage{longtable}
\usepackage{indentfirst}
\usepackage{array}
\newcolumntype{P}[1]{>{\raggedright\arraybackslash}p{#1}}
\frenchspacing
\usepackage{fixltx2e} %text sub- and superscripts
\usepackage{icomma} % коскі ў матэматычным рэжыме
\PreloadUnicodePage{4}

\newcommand{\longpage}{\enlargethispage{\baselineskip}}
\newcommand{\shortpage}{\enlargethispage{-\baselineskip}}

\def\switchlang#1{\expandafter\csname switchlang#1\endcsname}
\def\switchlangbe{
\let\saverefname=\refname%
\def\refname{Літаратура}%
\def\figurename{Іл.}%
}
\def\switchlangen{
\let\saverefname=\refname%
\def\refname{References}%
\def\figurename{Fig.}%
}
\def\switchlangru{
\let\saverefname=\refname%
\let\savefigurename=\figurename%
\def\refname{Литература}%
\def\figurename{Рис.}%
}

\hyphenation{admi-ni-stra-tive}
\hyphenation{ex-pe-ri-ence}
\hyphenation{fle-xi-bi-li-ty}
\hyphenation{Py-thon}
\hyphenation{ma-the-ma-ti-cal}
\hyphenation{re-ported}
\hyphenation{imp-le-menta-tions}
\hyphenation{pro-vides}
\hyphenation{en-gi-neering}
\hyphenation{com-pa-ti-bi-li-ty}
\hyphenation{im-pos-sible}
\hyphenation{desk-top}
\hyphenation{elec-tro-nic}
\hyphenation{com-pa-ny}
\hyphenation{de-ve-lop-ment}
\hyphenation{de-ve-loping}
\hyphenation{de-ve-lop}
\hyphenation{da-ta-ba-se}
\hyphenation{plat-forms}
\hyphenation{or-ga-ni-za-tion}
\hyphenation{pro-gramming}
\hyphenation{in-stru-ments}
\hyphenation{Li-nux}
\hyphenation{sour-ce}
\hyphenation{en-vi-ron-ment}
\hyphenation{Te-le-pathy}
\hyphenation{Li-nux-ov-ka}
\hyphenation{Open-BSD}
\hyphenation{Free-BSD}
\hyphenation{men-ti-on-ed}
\hyphenation{app-li-ca-tion}

\def\progref!#1!{\texttt{#1}}
\renewcommand{\arraystretch}{2} %Іначай формулы ў матрыцы зліпаюцца з лініямі
\usepackage{array}

\def\interview #1 (#2), #3, #4, #5\par{

\section[#1, #3, #4]{#1 -- #3, #4}
\def\qname{LVEE}
\def\aname{#1}
\def\q ##1\par{{\noindent \bf \qname: ##1 }\par}
\def\a{{\noindent \bf \aname: } \def\qname{L}\def\aname{#2}}
}

\def\interview* #1 (#2), #3, #4, #5\par{

\section*{#1\\{\small\rm #3, #4. #5}}

\def\qname{LVEE}
\def\aname{#1}
\def\q ##1\par{{\noindent \bf \qname: ##1 }\par}
\def\a{{\noindent \bf \aname: } \def\qname{L}\def\aname{#2}}
}

%\switchlang{be}
%\usepackage{color}
\begin{document}
%\title{Интервью с участниками}
%\author{}
\date{}
%\maketitle


\begin{Parallel}[p]{}{}

     \ParallelLText{%
\interview* Michael Meeks (M.), Cambridge, UK, General Manager, Collabora Productivity

\q Hi, Michael. Can you briefly introduce yourself?

\a I'm Michael Meeks: Christian, Husband, Hacker — I live near Cambridge in the UK, and these days I work primarily on LibreOffice.

\q Tell us something about your first experience with the open source software.

\a Heh; so — I played with Linux a fair bit quite early on, when I became a Christian.

\q Your interest to GNU/Linux was driven by a Christian ethics? Can you recall (approximately), which age was it? Some more details?

\a When I was around eighteen, as a somewhat aggressive and inquisitive, agnostic skeptic, lodging in my gap year with a very loving and patient Christian family. I guess I had enough of my questions, and fears addressed that I was confident that the rest could be. I've never regretted building my life on this rock, though I often fail in so many ways He is so satisfyingly good.

I became rather unwillingly convicted that my stolen copy of Windows, compiler tools etc. were wrong. So I installed this (at the time unutterably terrible) GNU/Linux thing, and got hooked, after replacing my hard-disk (presumed bricked by some kernel bug). Anyhow — in the last 20 years, things have improved — and that choice turned out to launch me on an amazing career working with some extraordinarily sharp and interesting people on GNOME, openSUSE, and LibreOffice.

\q And, which GNU/Linux distro was your first one? This one, unutterably terrible.

\a I started with some ancient version of Slackware, I really can't remember which version; I recall the upgrade to RedHat Linux 4.2 with some pleasure. Now I use openSUSE exclusively across the family's estate; its a great distribution, quite apart from my familiarity gained working on it. My preferred DE is (sadly) these days XFCE — due to its reliable 2D virtual workspaces — but my less expert family use vanilla GNOME. For development I use a mix of emacs, occasionally vim, Visual Studio, and XCode (depending on platform).

\q How did you become the developer of FLOSS?

\a So, after some small hacks I found I was missing a warning for undefined behaviour in gcc; so I grabbed the code, read it, added a warning for this behavior (-Wsequence-point). Unfortunately this lingered for years on the gcc mailing list with no response; so I moved on to GNOME, making mahjongg solvable, then lots of work on Gnumeric around file formats, before LibreOffice, openSUSE and more.

\q Patching a compiler is rather serious start. Our readers would be interested to know, how much development experience you had up to this moment?'' 

\a Oh; yes — a particularly serious start when you read the gcc code for a bit: not a good place to learn to program. Having said that — it really is just software I suppose, not some kind of magic — I'm convinced that a reasonable, intelligent person can become the world's expert on any one particular (well scoped) issue in FLOSS simply by focused research and code reading in a reasonable time even without a programming background. Having said that, I had a lot of preparation: writing games in BBC Basic, 6502 assembler then x86 assembler, C, and a year's internship writing Pascal at Quantel for real-time embedded video editing systems (68040 based, compiled on a VAX — using a real VT420 green-screen terminal). That certainly helped a lot.

\q Something about your experience with the community?

\a My experience of the FLOSS community is of amazing diversity of experience and background, a huge richness of ideas and cultures which is exciting, interesting, and challenging. I love the experience of intellectual freedom — of discussing ideas and perspectives intensely without people taking things personally. I have good friends with whom I disagree really profoundly on all sorts of topics, but really enjoy working alongside in various projects.

\q And what about your impressions from contributing to GNOME and OpenOffice/LibreOffice? How is it to be the part of such a large FLOSS project?

\a Its great. Most newbies to FLOSS think they want to start their own project, and grow it to being huge and successful — so they can take the credit. Inevitably this dooms them to lonely futility. A FLOSS project has a significant fixed-cost for releasing, infrastructure, website, translation, marketing, brand building etc. Even with the great new services springing up around git — its still not trivial to build your own project — and to scale it. Working with others is really stimulating, and a great learning experience. I'd strongly recommend contributing to a large existing project where you can gain deep skills before trying to start anything of your own — as it turns out LibreOffice has a great Easy Hacks list of simple first tasks to get involved with.

\q What do you think about serious changes in the vision of the project, how they occur? Something like transition from GNOME 2.x to 3.x.

\a Software is a very human thing; it is largely a reflection of the people and personalities that are involved and how well they work together. I think bad things happen to projects when they don't listen carefully particularly to their contributors, but also users. I was involved quite heavily in the development and transition from GNOME 0.x to 1.0, 1.2, 1.4, 2.0 and beyond. Over that time there was lots of change eg. the team who decided that 'sawfish' was written in a hard-to-debug scripting language and needed re-writing in C (as Metacity) to make it maintainable to some heart-ache; then some years later the same team decide it needed re-writing into harder to debug and maintain JavaScript (Mutter). Personally I thought the GNOME 3.0 transition pathologies were refreshingly different to anything I'd seen before; so I disagree with the orthodox group-think on that topic that this is somehow normal. The project had different personalities in charge at this transition, but I'm no expert on GNOME anymore sadly. There are some really great guys who started their life in GNOME who now fight at the cutting edge of other big problems in the FLOSS world.

\q It looks like GNOME 3.0 transition was rather uncomfortable for you. What do you think about MATE as a continuation of GNOME 2.х?

\a Really uncomfortable. With regard to MATE — I was responsible for trying to redeem some of the worse design decisions in GNOME 2.x around the ‘Object Model Environment’ piece of GNOME, and I'd be rather happy if that code didn't exist anymore. Each to their own though — if maintaining that floats someone's boat. Then again, if there is lots of spare man-power floating around, then we have tons of really valuable work that can be done on LibreOffice and a great team to work with.

\q Let's turn to LibreOffice then :) What do you think about the attempts to supply LibreOffice with an improved tab-based UI, like ‘ribbons’ appeared in MS Office after 2007?

\a Personally I really like a tab-based palette of tools to use, LibreOffice now has a Notebookbar which may be more familiar to some newer users. I can't see us ever removing the classic toolbars though, and of course — we have a great sidebar panel which is a far better use of screen real-estate on 16×9 screens I think. Through good design and re-use, they are reasonably easy to maintain (I hope).

\a Again — I like the power of possibility — of seeing people who firmly disagree on what is the right way to do toolbars, menus (or whatever) work towards a common goal of thrilling each of their constituents with a choice of the functionality they like best. I really don't buy the rationales for: ‘We will choose everything for you’, and luckily the LibreOffice UX team is interested both in making thing much easier to use (there is lots of scope for that), and also in allowing choice.

\newpage
\q Finally, it would be interesting to know how would you compare your impressions on FLOSS with one on proprietary software. Something about good and bad features you are noticing…

\a Good and bad features?

I guess the bad feature I see in the world of FLOSS is that proprietary software appears extremely easy to sell; whereas services and support around great FLOSS like LibreOffice is hard to close. I regularly suffer customers who refuse to pay even 10\% of the price of MS Office for their deployment in return for great services, support and maintenance — which in turn Collabora can re-invest into LibreOffice development. Apparently it is a common hope that someone else can pay to make LibreOffice better, and to contribute about the third of commits we write. While the license of course allows this, that's a really bad feature of my experience at least.

On the good side — I'm a huge fan of LibreOffice Online — which Collabora has invested a fortune into creating. We've put a huge amount of time and effort to allow people to host their own collaborative productivity software in conjunction with our many partners in ways which respect their privacy, as well as letting people exchange open file formats without needing to install software at the recipients' end, that's a great ‘good’ and it excites me.

\vfill
     }
     \ParallelRText{%
       \selectlanguage{russian}
\interview* Michael Meeks (М.), Кембридж, Великобритания, General Manager, Collabora Productivity

\q Привет, Майкл. Представься, пожалуйста.

\a Меня зовут Майкл Микс, я христианин, муж, хакер, живу недалеко от Кембриджа в Великобритании, и сейчас в основном работаю над LibreOffice.

\q Расскажи нам, пожалуйста, о своём первом опыте со свободным ПО.

\a Ну… я начал интересоваться Линуксом достаточно давно, когда я стал христианином.

\q Твой интерес к GNU/Linux был вызван христианской этикой? Помнишь (примерно) в каком возрасте это было? Интересны подробности.

\a Когда мне было около восемнадцати, я был достаточно агрессивным и пытливым агостиком-скептиком. В то время я снимал комнату в доме любящей и терпеливой христианской семьи. Я нашёл ответы на большинство своих вопросов и опасений, и пришёл к убедеждению, что всё это имеет смысл. Я никогда не жалел о том, что связал свою жизнь с христианством, хотя я часто терплю неудачи в вещах, в которых Он настолько хорош.

В какой-то момент я нехотя пришёл к выводу, что пользоваться пиратскими Windows, компилятором и т.~д. неправильно. Так что я установил эту штуковину под названием GNU/Linux, невероятно ужасную в то время, и «подсел» — после замены жёсткого диска (который я «убил» каким-то багом в ядре). В любом случае, за последние 20 лет всё сильно улучшилось, и в результате этого решения мне открылись двери к замечательной карьере и возможности работать с невероятно умными и интересно людьми в проектах GNOME, openSUSE и LibreOffice.

\q И какой же был твой первый дистрибутив GNU/Linux? Ну тот, невероятно ужасный.

\a Я начинал с какой-то старой версии Slackware, не помню уже даже какой; что я помню, так это то, что апгрейд на RedHat Linux 4.2 принёс много радости. Сейчас у меня дома везде openSUSE; кроме того, что я к нему привык, я считаю, что это замечательный дистрибутив. Я предпочитаю XFCE (к сожалению) — в основном, из-за надёжно работающих 2D виртуальных рабочих столов, но члены моей семьи пользуются «ванильным» GNOME. Для разработки я пользуюсь emacs, иногда vim, Visual Studio и XCode — в зависимости от платформы.

\q Как ты стал разработчиком свободного ПО?

\a Однажды я выяснил, что в gcc не хватает предупреждения о неопределённом поведении. Я скачал код, изучил его, добавил предупреждение (-Wsequence-point). К сожалению, патч слишком долго пробыл без ответа в рассылке gcc, так что я устал ждать и занялся GNOME, починил Mahjongg, много работал над поддержкой файловых форматов в Gnumeric, ну а потом LibreOffice, openSUSE и так далее.

\q Патчить компилятор — это серьёзное начало. Нашим читателям было бы интересно узнать, каков был твой опыт разработки на тот момент?

\a А, ну да, чтение кода gcc — довольно серьёзное начало, там можно многому научиться. С другой стороны, это просто код, никакой магии. Я уверен, что любой разумный человек может стать экспертом в любой области свободного ПО просто усиленно изучая код какое-то продолжительное время, даже если у него не было серьёзного опыта программирования до этого. У меня, правда, опыта было много: написание игр на Бейсике для микрокомпьютера BBC, потом ассемблер 6502 и x86, Си, год практики в Quantel, программируя системы видеоредактирования в реальном времени на Паскале (процессор был 68040, но компилировали мы на VAXе — с зелёным терминалом VT420!) Всё это, конечно, помогло.

\q А как насчёт опыта взаимодействия с сообществом?

\a Сообщество удивительно разнообразно опытом и интересами отдельных его членов, люди принадлежат к разным культурам, и у них всех разные мнения и идеи. Это одновременно интересно, захватывающе, но и сложно. Мне нравится этот опыт интеллектуальной свободы: горячо обсуждать идеи и точки зрения, при этом не принимая вещи на свой счёт. У меня есть хорошие друзья, с которыми я совершенно несогласен во многом, но вместе с которыми мне в то же время приятно работать.

\q Что скажешь насчёт опыта работы с проектами GNOME и OpenOffice/LibreOffice? Каково это, быть частью такого огромного проекта?

\a Это круто. Большинство новичков в свободном ПО хотят начать свой собственный проект, который бы стал известным и успешным, и прославил бы их, но с таким подходом проект совершенно бесперспективен. В проектах по разработке свободного ПО всё имеет определённую фиксированную цену: инфраструктура, веб-сайт, переводы, маркетинг, создание бренда, релизы… Даже несмотря на то, сколько сейчас возникает классных сервисов вокрут git, создать и поддерживать новый проект не легко. Работа с другими людьми очень полезна, приносит новый опыт и по-настоящему стимулирует. Я рекомендую каждому попробовать быть частью большого имеющегося проекта, в котором можно набраться опыта и чему-то научиться, прежде чем начинать что-то своё. Например, в проекте LibreOffice есть список Easy Hacks, простых задач, на которых можно потренироваться, принося пользу проекту.

\q Что ты думаешь о том, когда проект резко меняет курс, как это получается? Например, такие ситуации, как с переходом от GNOME 2.x к 3.x.

\a Программное обеспечение пишется людьми и с большего отражает характеры и особенности людей, которые его пишут, а также их взаимоотношения. Как мне кажется, изменения к худшему случаются, когда проекты не слушают не только своих контрибуторов, но и пользователей. Я был серьёзно вовлечён в разработку GNOME версий 0.x, 1.0, 1.2, 1.4, 2.0 и далее. В то время в проекте происходило много радикальных изменений. Например, одна команда решила, что т.~к. sawfish написан на скриптовом языке, который нелегко отлаживать, его необходимо переписать на Си (результат известен как Metacity), чтобы его хоть как-то можно было поддерживать. Несколько лет спустя та же самая команда решила, что его нужно переписать на Javascript, который тоже нелегко отлаживать и поддерживать (Mutter). Лично я думаю, что процесс перехода на GNOME 3.0 сильно отличались [в худшую сторону] от чего-либо, что я видел ранее, и я несогласен с распространённым мнением, что это нормально. В GNOME за переход были ответственны разные люди, но, к сожалению, я более не слежу за проектом. Есть несколько классных разработчиков, которые начали свою профессиональную жизнь в GNOME, и теперь они на переднем крае, сражаются с другими важными проблемами мира FLOSS.

\q Похоже, что переход к GNOME 3.0 для тебя не был очень приятным. А что ты думаешь насчёт MATE как продолжения идей GNOME 2.x?

\a Да, этот переход был для меня очень неприятным. А насчёт MATE скажу, что именно я был ответственным за попытки исправления некоторых плохих решений в области «объектной модели» GNOME, и я был бы более чем счастлив, если бы этим кодом больше никто не пользовался. Но о вкусах не спорят — если этот проект кому-то нужен, то пусть будет. С другой стороны, если кому-то хочется чем-то заняться, у нас в LibreOffice есть много работы и приятный коллектив.

\q Давай поговорим про LibreOffice :) Что ты думаешь насчёт попыток внедрить в LibreOffice интерфейс на основе «вкладок», подобный тому из MS Office 2007?

\a Мне лично нравятся панели инструментов с вкладками, и в LibreOffice теперь есть есть Notebookbar, к которому, вероятно, привыкли новые пользователи. С другой стороны, я не могу представить, что мы когда-нибудь отойдём от классических панелей инструментов. Ну и потом, у нас есть замечательная боковая панель, которая, на мой взгляд, гораздо эффективнее расходует экранное пространство на экранах с соотношением сторон 16:9. Through good design and re-use, they are reasonably easy to maintain (I hope).

Ну а вообще, мне нравится сама возможность этого, это замечательно, что есть люди, у которых совершенно разные мнения насчёт того, как правильно делать панели инструментов, меню и так далее, и они все работают вместе, чтобы предложить пользователям выбор. Я не согласен с подходом: «Мы всё решим за вас». К счастью, команда UX LibreOffice заинтересована и в том, чтобы сделать LibreOffice проще в использовании (у нас есть что улучшать!), и в то же время предоставлять возможность выбора.

\newpage
\q Наконец, было бы интересно узнать твои впечатления о свободном ПО в сравнении в проприетарным. И хорошие, и плохие особенности, которые ты замечаешь… 

\a Хорошие и плохие особенности?

Наверное, плохо то, что судя по всему, что проприетарный софт гораздо легче продать, чем службы и поддержку для свободного ПО (например LibreOffice). Часто случается, что клиенты отказываются платить даже 10\% того, что они платят за MS Office, за поддержку LibreOffice — деньги, которые Collabora может пустить на дальнейшую разработку. Видимо, все надеются, что \emph{кто-то другой} может заплатить за улучшение LibreOffice, и внести около трети тех коммитов, которые пишем мы. Хотя, конечно же, лицензия это позволяет, из моего опыта это достаточно неприятно.

Из хорошего — LibreOffice Online. Collabora вложила огромные средства в его создание. Мы потратили много времени и сил, чтобы позволить пользователям размещать на их собственной инфраструктуре офисное ПО с функцией коллективного доступа, общими усилиями с нашими партнёрами, так, чтобы это не нарушало их конфиденциальности, дали людям возможность обмениваться файлами в открытых форматах без необходимости установки клиентского ПО --- и это \emph{хорошее} приводит меня в восторг.

\vfill
     }
   \end{Parallel}









 
\end{document}


