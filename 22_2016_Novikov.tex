\documentclass[10pt, a5paper]{article}
\usepackage[T2A]{fontenc}
\usepackage{ucs}
\usepackage[utf8x]{inputenc}
\usepackage[polish,english,russian]{babel}
\usepackage{hyperref}
\usepackage[inner=2cm,top=1.8cm,outer=2cm,bottom=2.3cm,nohead]{geometry}
\usepackage{listings}
\usepackage{graphicx}
\usepackage{wrapfig}
\usepackage{longtable}
\usepackage{indentfirst}
\frenchspacing
\usepackage{fixltx2e} %text sub- and superscripts
\usepackage{icomma} % коскі ў матэматычным рэжыме
\PreloadUnicodePage{4}

\newcommand{\longpage}{\enlargethispage{\baselineskip}}
\newcommand{\shortpage}{\enlargethispage{-\baselineskip}}

\def\switchlang#1{\expandafter\csname switchlang#1\endcsname}
\def\switchlangbe{
\let\saverefname=\refname%
\def\refname{Літаратура}%
\def\figurename{Іл.}%
}
\def\switchlangen{
\let\saverefname=\refname%
\def\refname{References}%
\def\figurename{Fig.}%
}
\def\switchlangru{
\let\saverefname=\refname%
\let\savefigurename=\figurename%
\def\refname{Литература}%
\def\figurename{Рис.}%
}

\hyphenation{admi-ni-stra-tive}
\hyphenation{ex-pe-ri-ence}
\hyphenation{fle-xi-bi-li-ty}
\hyphenation{Py-thon}
\hyphenation{ma-the-ma-ti-cal}
\hyphenation{re-ported}
\hyphenation{imp-le-menta-tions}
\hyphenation{pro-vides}
\hyphenation{en-gi-neering}
\hyphenation{com-pa-ti-bi-li-ty}
\hyphenation{im-pos-sible}
\hyphenation{desk-top}
\hyphenation{elec-tro-nic}
\hyphenation{com-pa-ny}
\hyphenation{de-ve-lop-ment}
\hyphenation{de-ve-loping}
\hyphenation{de-ve-lop}
\hyphenation{da-ta-ba-se}
\hyphenation{plat-forms}
\hyphenation{or-ga-ni-za-tion}
\hyphenation{pro-gramming}
\hyphenation{in-stru-ments}
\hyphenation{Li-nux}
\hyphenation{en-vi-ron-ment}
\hyphenation{Te-le-pathy}
\hyphenation{Li-nux-ov-ka}

\def\progref!#1!{\texttt{#1}}
\renewcommand{\arraystretch}{2} %Іначай формулы ў матрыцы зліпаюцца з лініямі
\usepackage{array}

\def\interview #1 (#2), #3, #4, #5\par{

\section[#1, #3, #4]{#1, #5}
\def\qname{LVEE}
\def\aname{#1}
\def\q ##1\par{{\noindent \bf \qname: ##1 }\par}
\def\a{{\noindent \bf \aname: } \def\qname{L}\def\aname{#2}}
}

\begin{document}
\title{Миграция виртуальной инфраструктуры на свободное ПО XenServer\footnote{\url{arol90@gmail.com}, \url{http://lvee.org/ru/abstracts/220}}}
\author{Антон Новиков, Минск, Беларусь}
\maketitle
\begin{abstract}
Migrate virtual infrastructure to open source XenServer. Real, complete project.
\end{abstract}
\subsection*{Введение}
Виртуализация плотно вошла в «IT  мир», закрепилась в ЦОДах крупного и среднего бизнеса, зачастую внедрение таких систем сопровождается множеством подсчетов и сравнений. Систем с каждым годом, если не сказать месяцем становится больше, времени на реализацию проектов все меньше.
Данная статья о крупный игроке на рынке который отдал свою разработку в заботливые руки Open Source. Это — XenServer. Именно о нем как о представителе СПО этот доклад.

\subsection*{О системах Виртуализации}
Хороших а главное гибких систем не так уж и много. Как правило выделяются всего пара таких проектов. В качестве ярчайших представителей можно уверенно рассматривать XenServer(СПО) и VmWare ESXi. Да существует огромное количество СПО систем виртуализации, которые бесспорно готовы во многом дать преимущество выше указанным гигантам, но в купе всех качеств в итоге проигрывают.

\subsection*{Почему был выбран именно XenServer}
Помимо очевидного — «цены», простота, гибкость, открытость, функционал, сильное комьюнити.
Это далеко не весь перечень, но хватает и минусов и к сожалению весьма досадных, вещи которые в других системах воспринимаются как должное в XenServer отсутствуют вовсе.
Чего только, стоит вопрос с пробросом USB порта в виртуальную систему.

После выбора системы виртуализации предстоял процесс внедрения,  трудоемкий процесс и самое главное в данном мероприятии -- время простоя бизнеса, и к сожалению в финансовой сфере оно равно нулю. Финансовые вложения тоже ограничены и нельзя просто построить рядом еще один ЦОД и мигрировать всё на него, процесс становится многослойным. Перейдем к нему немного детальнее.

Реальный проект перехода системы вирутализации и инфраструктуры крупной финансовой компании на СПО. Была проведена детальная проработка плана перехода, подготовка оборудования, и многие многие иные вопросы упавшие на плечи нашего IT подразделения, и самое главное все это подразумевало непрерывность бизнеса во время и после интеграции.

Был построен HA кластер на основе Blade серверов HP BL-460c
Каким бы свободным не было ПО а серверное оборудования не получится скачать из репозитариев. Вложения в оборудование зачастую в разы выше вложений в персонал или ПО.
Описываемый проект реализован на серверных шасси HP c7000 с использованием FiberChannel в качестве сети хранения данных и HP 3Par 7200 в качестве системы хранения.

Была произведена миграция с проприетарного программного обеспечения на СПО Xenserver.

\subsection*{По итогам проекта} 
Можно сделать вывод, что система справляется с возложенными на неё функциями, отлично управляется, не доставляет хлопот в обслуживании обновлении и администрировании, легко масштабируема. А главное пример доказывает, что СПО может и обязательно будет занимать своё заслуженное место в виртуализации не только мелкого\textbackslash{}среднего но и крупного бизнеса.

\end{document}
