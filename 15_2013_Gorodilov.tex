\documentclass[10pt, a5paper]{article}
\usepackage{pdfpages}
\usepackage{parallel}
\usepackage[T2A]{fontenc}
\usepackage{ucs}
\usepackage[utf8x]{inputenc}
\usepackage[polish,english,russian]{babel}
\usepackage{hyperref}
\usepackage{rotating}
\usepackage[inner=2cm,top=1.8cm,outer=2cm,bottom=2.3cm,nohead]{geometry}
\usepackage{listings}
\usepackage{graphicx}
\usepackage{wrapfig}
\usepackage{longtable}
\usepackage{indentfirst}
\usepackage{array}
\newcolumntype{P}[1]{>{\raggedright\arraybackslash}p{#1}}
\frenchspacing
\usepackage{fixltx2e} %text sub- and superscripts
\usepackage{icomma} % коскі ў матэматычным рэжыме
\PreloadUnicodePage{4}

\newcommand{\longpage}{\enlargethispage{\baselineskip}}
\newcommand{\shortpage}{\enlargethispage{-\baselineskip}}

\def\switchlang#1{\expandafter\csname switchlang#1\endcsname}
\def\switchlangbe{
\let\saverefname=\refname%
\def\refname{Літаратура}%
\def\figurename{Іл.}%
}
\def\switchlangen{
\let\saverefname=\refname%
\def\refname{References}%
\def\figurename{Fig.}%
}
\def\switchlangru{
\let\saverefname=\refname%
\let\savefigurename=\figurename%
\def\refname{Литература}%
\def\figurename{Рис.}%
}

\hyphenation{admi-ni-stra-tive}
\hyphenation{ex-pe-ri-ence}
\hyphenation{fle-xi-bi-li-ty}
\hyphenation{Py-thon}
\hyphenation{ma-the-ma-ti-cal}
\hyphenation{re-ported}
\hyphenation{imp-le-menta-tions}
\hyphenation{pro-vides}
\hyphenation{en-gi-neering}
\hyphenation{com-pa-ti-bi-li-ty}
\hyphenation{im-pos-sible}
\hyphenation{desk-top}
\hyphenation{elec-tro-nic}
\hyphenation{com-pa-ny}
\hyphenation{de-ve-lop-ment}
\hyphenation{de-ve-loping}
\hyphenation{de-ve-lop}
\hyphenation{da-ta-ba-se}
\hyphenation{plat-forms}
\hyphenation{or-ga-ni-za-tion}
\hyphenation{pro-gramming}
\hyphenation{in-stru-ments}
\hyphenation{Li-nux}
\hyphenation{sour-ce}
\hyphenation{en-vi-ron-ment}
\hyphenation{Te-le-pathy}
\hyphenation{Li-nux-ov-ka}
\hyphenation{Open-BSD}
\hyphenation{Free-BSD}
\hyphenation{men-ti-on-ed}
\hyphenation{app-li-ca-tion}

\def\progref!#1!{\texttt{#1}}
\renewcommand{\arraystretch}{2} %Іначай формулы ў матрыцы зліпаюцца з лініямі
\usepackage{array}

\def\interview #1 (#2), #3, #4, #5\par{

\section[#1, #3, #4]{#1 -- #3, #4}
\def\qname{LVEE}
\def\aname{#1}
\def\q ##1\par{{\noindent \bf \qname: ##1 }\par}
\def\a{{\noindent \bf \aname: } \def\qname{L}\def\aname{#2}}
}

\def\interview* #1 (#2), #3, #4, #5\par{

\section*{#1\\{\small\rm #3, #4. #5}}

\def\qname{LVEE}
\def\aname{#1}
\def\q ##1\par{{\noindent \bf \qname: ##1 }\par}
\def\a{{\noindent \bf \aname: } \def\qname{L}\def\aname{#2}}
}


\begin{document}

\title{Подход к преподаванию десктопных и сетевых возможностей GNU/Linux в университете}%\footnote{Текст данных и последующих тезисов, кроме специально оговоренных случаев, доступен под лицензией Creative Commons Attribution"=ShareAlike 3.0}

\author{Алексей Городилов, Александра Кононова\footnote{Москва, г. Зеленоград, Россия; \url{kaverina@mail.ru}, \url{illinc@bk.ru}}}
\maketitle

\begin{abstract}
An approach to teaching GNU/Linux basics in university is proposed. This program is the result of refactoring of earlier courses taught at MIET. This course's aim is to provide a working knowledge of the operating system, so students could start to use GNU/Linux. This is in contrast with previous course, which lost connection to practice and which was too difficult. The new version proved to be successful as a popular elective course.
\end{abstract}
Курс, посвящённый GNU/Linux, существовал в МИЭТ в течение многих лет. Но в определённый момент сложилась ситуация, при которой этот курс стал крайне непопулярен у студентов. Кроме того, прослушавшие курс студенты оказывались плохо подготовленными к использованию GNU/Linux на практике. Стало очевидно, что данный курс требует существенной переработки и модернизации.

В состав старого курса входило изучение Bash и утилит Unix, таких, как grep, awk, vi. Кроме того, в курс входила разработка программ для Unix с использованием fork-exec.

Хотя теоретически это давало студентам глубокое понимание операционной системы, но на практике студенты, не знакомые с Unix ранее, не могли освоить работу в этой системе. Кроме того, курс совершенно не давал представления о практическом применении Unix или GNU/Linux.

Студентам приходилось изучать три новых языка программирования, а также несколько новых библиотек.  Студентам было непонятно, зачем нужно изучать этот инструментарий. Ответить на вопрос «зачем» можно было только теоретически. Такая сложность  старого курса приводила к тому, что более половины студентов, изучающих его, начинали к концу семестра списывать лабораторные работы. Этот признак непонятного и переусложнённого курса проявляется и в других дисциплинах.

Два года назад было решено полностью модернизировать данный курс. В разработанном нами варианте основная часть курса была посвящена изучению графического интерфейса и прикладных программ.

В состав курса были включены 4 лекции по 2 академических часа и 8 практических занятий по 4 академических часа. 
В курс вошли следующие практические занятия:

\begin{enumerate}
  \item Работа в графических средах KDE и GNOME, на котором студенты получают базовые навыки по работе с системой, такие как доступ и поиск файлов, запуск программ и настройка параметров системы.
  \item Восстановление системы после сбоев и зависаний программ, на котором изучаются способы восстановить работоспособность системы после сбоев различной степени тяжести, от зависания отдельной программы до отключения X-сервера.
  \item Офисные приложения "--- на этом занятии изучается работа в офисных приложениях, которые являются аналогами собственнических офисных приложений, изучаемых студентами в курсе базовой компьютерной подготовки.
  \item Графика: растровые, векторные и 3D-редакторы "--- на этом занятии студенты пробуют работать в gimp, inkscape и blender. Целью этого занятия является знакомство с возможностями редакторов, чтобы студенты могли решить, следует ли изучать эти программы далее.
  \item Работа со звуком: настройка звуковой подсистемы и использование звуковых и midi- редакторов "--- студентам предлагается самостоятельно подключить к компьютеру и настроить микрофон, динамики (наушники), а также midi-клавиатуру, отредактировать записанный звук в audacity, а также запустить jack и перенастроить вывод на него, что позволяет глубже понять работу звуковой подсистемы в ОС, и диагностировать проблемы, связанные с ней.
  \item Работа с интерфейсом командной строки "--- на этом занятии изучаются основы работы с bash, полезность чего уже не вызывает у студентов сомнений.
  \item Настройка сети в GNU/Linux, в том числе маршрутизации и брандмауэра "--- на этом занятии студенты настраивают подключение компьютера к сети, а также маршрутизацию для раздачи интернета другим компьютерам с помощью networkmanager, а также с помощью традиционных средств (ifconfig, iptables), и базовую настройку прокси-сервера squid.
  \item Установка ОС "--- студенты могут установить на виртуальную машину ОС Ubuntu, Fedora или любой другой дистрибутив по их выбору.
\end{enumerate}

Такой курс оказался более успешным. Изначально это подтверждалось только отзывами студентов, но, когда в МИЭТ была введена возможность выбирать курсы, обновлённый курс «Основы GNU/Linux» выбрали более 80\% студентов соответствующих специальностей.


\end{document}




