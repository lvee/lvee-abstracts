\documentclass[10pt, a5paper]{article}
\usepackage{pdfpages}
\usepackage{parallel}
\usepackage[T2A]{fontenc}
\usepackage{ucs}
\usepackage[utf8x]{inputenc}
\usepackage[polish,english,russian]{babel}
\usepackage{hyperref}
\usepackage{rotating}
\usepackage[inner=2cm,top=1.8cm,outer=2cm,bottom=2.3cm,nohead]{geometry}
\usepackage{listings}
\usepackage{graphicx}
\usepackage{wrapfig}
\usepackage{longtable}
\usepackage{indentfirst}
\usepackage{array}
\newcolumntype{P}[1]{>{\raggedright\arraybackslash}p{#1}}
\frenchspacing
\usepackage{fixltx2e} %text sub- and superscripts
\usepackage{icomma} % коскі ў матэматычным рэжыме
\PreloadUnicodePage{4}

\newcommand{\longpage}{\enlargethispage{\baselineskip}}
\newcommand{\shortpage}{\enlargethispage{-\baselineskip}}

\def\switchlang#1{\expandafter\csname switchlang#1\endcsname}
\def\switchlangbe{
\let\saverefname=\refname%
\def\refname{Літаратура}%
\def\figurename{Іл.}%
}
\def\switchlangen{
\let\saverefname=\refname%
\def\refname{References}%
\def\figurename{Fig.}%
}
\def\switchlangru{
\let\saverefname=\refname%
\let\savefigurename=\figurename%
\def\refname{Литература}%
\def\figurename{Рис.}%
}

\hyphenation{admi-ni-stra-tive}
\hyphenation{ex-pe-ri-ence}
\hyphenation{fle-xi-bi-li-ty}
\hyphenation{Py-thon}
\hyphenation{ma-the-ma-ti-cal}
\hyphenation{re-ported}
\hyphenation{imp-le-menta-tions}
\hyphenation{pro-vides}
\hyphenation{en-gi-neering}
\hyphenation{com-pa-ti-bi-li-ty}
\hyphenation{im-pos-sible}
\hyphenation{desk-top}
\hyphenation{elec-tro-nic}
\hyphenation{com-pa-ny}
\hyphenation{de-ve-lop-ment}
\hyphenation{de-ve-loping}
\hyphenation{de-ve-lop}
\hyphenation{da-ta-ba-se}
\hyphenation{plat-forms}
\hyphenation{or-ga-ni-za-tion}
\hyphenation{pro-gramming}
\hyphenation{in-stru-ments}
\hyphenation{Li-nux}
\hyphenation{sour-ce}
\hyphenation{en-vi-ron-ment}
\hyphenation{Te-le-pathy}
\hyphenation{Li-nux-ov-ka}
\hyphenation{Open-BSD}
\hyphenation{Free-BSD}
\hyphenation{men-ti-on-ed}
\hyphenation{app-li-ca-tion}

\def\progref!#1!{\texttt{#1}}
\renewcommand{\arraystretch}{2} %Іначай формулы ў матрыцы зліпаюцца з лініямі
\usepackage{array}

\def\interview #1 (#2), #3, #4, #5\par{

\section[#1, #3, #4]{#1 -- #3, #4}
\def\qname{LVEE}
\def\aname{#1}
\def\q ##1\par{{\noindent \bf \qname: ##1 }\par}
\def\a{{\noindent \bf \aname: } \def\qname{L}\def\aname{#2}}
}

\def\interview* #1 (#2), #3, #4, #5\par{

\section*{#1\\{\small\rm #3, #4. #5}}

\def\qname{LVEE}
\def\aname{#1}
\def\q ##1\par{{\noindent \bf \qname: ##1 }\par}
\def\a{{\noindent \bf \aname: } \def\qname{L}\def\aname{#2}}
}


\begin{document}

\title{Основы работы в Украинском Национальном ГРИД}%\footnote{Текст данных и последующих тезисов, кроме специально оговоренных случаев, доступен под лицензией Creative Commons Attribution-ShareAlike 3.0}

\author{Дмитрий Сподарец, Григорий Драган\footnote{Одесский национальный университет имени И.И.Мечникова, Одесса, Украина}}
\maketitle

\begin{abstract}
The paper introduces basic concepts and principles of GRID functioning illustrated by the Ukrainian National GRID, \linebreak describes goals and tasks of High-Performance Computing \& Free /Open Source Centre of I.I. Mechinkov Odessa National University, and explains steps required to register and work in GRID.
\end{abstract}


ГРИД-технологии позволяют объединять информационные и \linebreak вычислительные ресурсы путём создания единой компьютерной инфраструктуры нового типа, которая обеспечивает глобальную интеграцию этих ресурсов на базе сетевых технологий и специализированного программного обеспечения промежуточного уровня, а также набора стандартизированных служб для обеспечения доступа к географически распределённых информационных и вычислительных ресурсов: компьютеров, кластеров, хранилищ данных.

Для развития и популяризации Украинского национального \linebreak ГРИД (УНГ) на юге Украины, а также для увеличения собственных вычислительных ресурсов при Одесском национальном университете имени И.И. Мечникова был создан Центра суперкомпьютерных вычислений и свободного программного обеспечения. Сегодня Центр активно участвует в популяризации ГРИД-технологий, на его базе проводятся научные исследования по самоорганизации упорядоченных структур в дымовой плазме, он выполняет роль регионального регистратора УНГ, проводит различные семинары и конференции, в частности, осенью на конференции FOSS Sea 2012 будет сформирована отдельная секция, посвящённая HPC и GRID-технологиям. Кроме того Центр активно поддерживает OpenSource-сообщество, открыт для сотрудничества и воплощения в жизнь новых интересных совместных проектов.

Подключиться в УНГ и начать работать в нём достаточно просто. Рассмотрим основные шаги по регистрации и началу работы.

Первым делом необходимо получить сертификат пользователя. Полная процедура описана по адресу \url{http://ung.in.ua/ua/certification/}.

Если вы желаете подключить свои вычислительные ресурсы в УНГ, то вам необходимо пройти ещё регистрацию ГРИД-сайта, которая описана здесь: \url{http://ung.in.ua/ua/join/}. На кластерах, подключаемых в УНГ, должно быть настроено программное обеспечение ARC. Одно из описаний, которое может помочь в настройке данного ПО, можно найти здесь: \url{http://clusterui.bitp.kiev.ua/howto/}.

Если вы желаете создать Виртуальную организацию со своими коллегами, то с инструкцией по её регистрации в УНГ можно ознакомиться здесь: \url{http://ung.in.ua/ua/vo_registration/}.

После того, как получен сертификат и настроено рабочее место, через которое вы будете входить в ГРИД, можно приступить к работе.

Первым делом проверяем наличия всех ключей в директории .globus:

\begin{itemize}
  \item usercert.pem --- цифровой сертификат пользователя (часть с открытым ключом). Ключ обязательно должен быть подписан Центром сертификации и быть добавлен в какую-то виртуальную организацию;
  \item userkey.pem --- секретный ключ сертификата пользователя.
\end{itemize}

Доступ в среду GRID происходит под именем, содержащемся в сертификате, и контролируется  с помощью специальной программы-посредника (электронной «доверенности» – proxy), которая создается на определенный ограниченный срок с помощью персонального ключа (userkey.pem) пользователя. Сервисные службы GRID могут выполнять любые действия, только если располагают копией такой доверенности.

Данная доверенность создаётся при помощи команды grid-proxy-init и действительна на протяжении  12 часов (для увеличения срока можно использовать параметр параметра -hours). Уничтожить доверенность до истечения ее срока можно с помощью команды grid-proxy-destroy. Для получения информации о выданной доверенности используйте команду grid-proxy-info с параметром –all, которая выдает полную информацию о доверенности.
Система отправки заданий в среду GRID представляет собой набор команд для направления заданий, проверки их статуса и получения результатов. В отличие от локальных систем управления заданиями (таких как PBS, LSF и др.), система отправки заданий GRID:

\begin{itemize}
  \item обеспечивает единообразный доступ к ресурсам на различных узлах сети;
  \item автоматически согласовывает требования, необходимые для выполнения задания, с имеющимися ресурсами.
\end{itemize}

Как и в кластерных системах, пользовательская команда запуска содержит имя скрипта, запрос ресурсов в котором специфицируется в виде строки XRSL.

Команды управления заданиями имеют следующий вид: ngsub <job.jdl> -- команда отправки файла с описанием задания; ngstat <jobID> --- запрос статуса задания по его идентификационному номеру jobID (PREPARING --- подготовка к выполнению, INLRMS:Q --- Ожидание освобождения ресурса в очереди LRMS, INLRMS:R --- Выполнение задачи, FINISHING – Завершение задачи, FINISHED -- Задание завершено, PURGED -- Удалено, FAILED); ngkill <jobID> --- отмена задания.
Файл с описанием задания создается с помощью языка описания заданий (Extended Resource Specification Language, XRSL) и содержит необходимые входные данные, требования к ресурсам и сведения о том, куда должны быть записаны результаты обработки задания. Типичный xrsl-файл имеет вид:

\begin{verbatim}
&
(* this is comment *)
(executable=ex.sh)
(executables=example1)
(inputFiles=(example1 ""))
(arguments=”100000000” “13” “0.324”)
(stdout="out.txt")
(stderr="err.txt")
(outputFiles=("out.txt" "")("err.txt" "")("sol.ps" "")
("err.ps" "")("data.txt" ""))
(gmlog="gridlog")
(jobname="Example")
(cputime=20)
(middleware>="nordugrid-arc-0.3.24")
\end{verbatim}

Больше информации об УНГ, учебные материалы и анонсы можно найти на сайтах \url{http://ung.in.ua},  \url{http://infrastructure.kiev.ua}, \url{http://grid.nas.gov.ua/}, \url{http://grid.bitp.kiev.ua}, \linebreak \url{http://hpcandfosscenter.od.ua}.


\end{document}




