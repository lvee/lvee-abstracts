\documentclass[10pt, a5paper]{article}
\usepackage[T2A]{fontenc}
\usepackage{ucs}
\usepackage[utf8x]{inputenc}
\usepackage[polish,english,russian]{babel}
\usepackage{hyperref}
\usepackage[inner=2cm,top=1.8cm,outer=2cm,bottom=2.3cm,nohead]{geometry}
\usepackage{listings}
\usepackage{graphicx}
\usepackage{wrapfig}
\usepackage{longtable}
\usepackage{indentfirst}
\frenchspacing
\usepackage{fixltx2e} %text sub- and superscripts
\usepackage{icomma} % коскі ў матэматычным рэжыме
\PreloadUnicodePage{4}

\newcommand{\longpage}{\enlargethispage{\baselineskip}}
\newcommand{\shortpage}{\enlargethispage{-\baselineskip}}

\def\switchlang#1{\expandafter\csname switchlang#1\endcsname}
\def\switchlangbe{
\let\saverefname=\refname%
\def\refname{Літаратура}%
\def\figurename{Іл.}%
}
\def\switchlangen{
\let\saverefname=\refname%
\def\refname{References}%
\def\figurename{Fig.}%
}
\def\switchlangru{
\let\saverefname=\refname%
\let\savefigurename=\figurename%
\def\refname{Литература}%
\def\figurename{Рис.}%
}

\hyphenation{admi-ni-stra-tive}
\hyphenation{ex-pe-ri-ence}
\hyphenation{fle-xi-bi-li-ty}
\hyphenation{Py-thon}
\hyphenation{ma-the-ma-ti-cal}
\hyphenation{re-ported}
\hyphenation{imp-le-menta-tions}
\hyphenation{pro-vides}
\hyphenation{en-gi-neering}
\hyphenation{com-pa-ti-bi-li-ty}
\hyphenation{im-pos-sible}
\hyphenation{desk-top}
\hyphenation{elec-tro-nic}
\hyphenation{com-pa-ny}
\hyphenation{de-ve-lop-ment}
\hyphenation{de-ve-loping}
\hyphenation{de-ve-lop}
\hyphenation{da-ta-ba-se}
\hyphenation{plat-forms}
\hyphenation{or-ga-ni-za-tion}
\hyphenation{pro-gramming}
\hyphenation{in-stru-ments}
\hyphenation{Li-nux}
\hyphenation{en-vi-ron-ment}
\hyphenation{Te-le-pathy}
\hyphenation{Li-nux-ov-ka}

\def\progref!#1!{\texttt{#1}}
\renewcommand{\arraystretch}{2} %Іначай формулы ў матрыцы зліпаюцца з лініямі
\usepackage{array}

\def\interview #1 (#2), #3, #4, #5\par{

\section[#1, #3, #4]{#1, #5}
\def\qname{LVEE}
\def\aname{#1}
\def\q ##1\par{{\noindent \bf \qname: ##1 }\par}
\def\a{{\noindent \bf \aname: } \def\qname{L}\def\aname{#2}}
}


\begin{document}

\title{Тонкий клиент на базе процессора Marvell Kirkwood}%\footnote{Текст данных и последующих тезисов, кроме специально оговоренных случаев, доступен под лицензией Creative Commons Attribution-ShareAlike 3.0}

\author{Сергей Зенькевич\footnote{Минск, Беларусь, Promwad Engineering, sergei.zenkevich@promwad.com}}
\maketitle

\begin{abstract}
Hardware and software parts of the thin client device based on Marvell Kirkwood processor and Linux Debian 6.0 are described.
\end{abstract}


Представленное устройство предназначено для организации терминального доступа к серверам, построенным на решениях фирм Microsoft (на основе протокола RDP), VMware (VMware View), Citrix (Citrix XenDesktop). Устройство в основном ориентировано на доставку к рабочим столам Windows 7, а также для RDP соединений поддерживается доставка отдельных приложений.

Аппаратная платформа построена на базе процессора Marvell Kirkwood 88F6282. Данный процессор реализован на ядре Sheeva, работающем на частотах до 2ГГц (в устройстве используется процессор 1,6 Ггц). Процессор имеет два гигабитных Ethernet, которые подключены к внешнему PHY 88E1121R от Marvell. В процессоре имеется также два порта PCIe: первый порт используется для подключения GPU устройства, а второй выведен на внутренний разъём mini-PCI, к которому возможно подключение дополнительных внешних устройств или модулей WI-FI.

В качестве графического контроллера используется микросхема VOLARI-Z11 от компании SiS. Видеосигнал выведен на DVI-разъем. Устройство поддерживает разрешение до $1600\times1200\times32$. Звуковая подсистема реализована на отдельном аудио-кодеке, который имеет выход для наушников и микрофонный вход. Устройство оснащено четырьмя портами USB. Опционально на панель устройства может быть выведен разъем COM-порта. В устройство устанавливается 1 Гбайт оперативной памяти и 1 Гбайт NAND Flash (используется файловая система UBIFS). Питание системы осуществляется от внешнего блока питания.

Основой ПО тонкого клиента является дистрибутив Debian 6 (Squeeze). Так же, на начальном этапе, делались попытки построения дистрибутива на базе Open Embedded. Используется ядро Linux версии 2.6.39. В качестве среды рабочего стола выбран XFCE 4.4. Для организации терминального доступа используются следующие клиенты: rdesktop 1.6, FreeRDP 1.0, VMware View Open Client 4.5.0, Citrix Receiver for Linux 11.100.

Также нами было разработано приложение «Менеджер Тонкого Клиента» (на базе Qt) для управления тонким клиентом. Это приложение позволяет создавать, запускать, изменять параметры терминальных соединений различного типа (RDP, Citrix, VMware), редактировать и разграничивать права доступа к терминальным соединениям, обновлять ПО тонкого клиента, получать информацию о текущем состоянии тонкого клиента. Также на тонкий клиент устанавливается дополнительное ПО для работы в автономном режиме, а именно веб-браузер и мультимедиа-проигрыватель.



\end{document}




