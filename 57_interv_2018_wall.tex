\documentclass[10pt, a5paper]{article}
\usepackage{pdfpages}
\usepackage{parallel}
\usepackage[T2A]{fontenc}
\usepackage{ucs}
\usepackage[utf8x]{inputenc}
\usepackage[polish,english,russian]{babel}
\usepackage{hyperref}
\usepackage{rotating}
\usepackage[inner=2cm,top=1.8cm,outer=2cm,bottom=2.3cm,nohead]{geometry}
\usepackage{listings}
\usepackage{graphicx}
\usepackage{wrapfig}
\usepackage{longtable}
\usepackage{indentfirst}
\usepackage{array}
\newcolumntype{P}[1]{>{\raggedright\arraybackslash}p{#1}}
\frenchspacing
\usepackage{fixltx2e} %text sub- and superscripts
\usepackage{icomma} % коскі ў матэматычным рэжыме
\PreloadUnicodePage{4}

\newcommand{\longpage}{\enlargethispage{\baselineskip}}
\newcommand{\shortpage}{\enlargethispage{-\baselineskip}}

\def\switchlang#1{\expandafter\csname switchlang#1\endcsname}
\def\switchlangbe{
\let\saverefname=\refname%
\def\refname{Літаратура}%
\def\figurename{Іл.}%
}
\def\switchlangen{
\let\saverefname=\refname%
\def\refname{References}%
\def\figurename{Fig.}%
}
\def\switchlangru{
\let\saverefname=\refname%
\let\savefigurename=\figurename%
\def\refname{Литература}%
\def\figurename{Рис.}%
}

\hyphenation{admi-ni-stra-tive}
\hyphenation{ex-pe-ri-ence}
\hyphenation{fle-xi-bi-li-ty}
\hyphenation{Py-thon}
\hyphenation{ma-the-ma-ti-cal}
\hyphenation{re-ported}
\hyphenation{imp-le-menta-tions}
\hyphenation{pro-vides}
\hyphenation{en-gi-neering}
\hyphenation{com-pa-ti-bi-li-ty}
\hyphenation{im-pos-sible}
\hyphenation{desk-top}
\hyphenation{elec-tro-nic}
\hyphenation{com-pa-ny}
\hyphenation{de-ve-lop-ment}
\hyphenation{de-ve-loping}
\hyphenation{de-ve-lop}
\hyphenation{da-ta-ba-se}
\hyphenation{plat-forms}
\hyphenation{or-ga-ni-za-tion}
\hyphenation{pro-gramming}
\hyphenation{in-stru-ments}
\hyphenation{Li-nux}
\hyphenation{sour-ce}
\hyphenation{en-vi-ron-ment}
\hyphenation{Te-le-pathy}
\hyphenation{Li-nux-ov-ka}
\hyphenation{Open-BSD}
\hyphenation{Free-BSD}
\hyphenation{men-ti-on-ed}
\hyphenation{app-li-ca-tion}

\def\progref!#1!{\texttt{#1}}
\renewcommand{\arraystretch}{2} %Іначай формулы ў матрыцы зліпаюцца з лініямі
\usepackage{array}

\def\interview #1 (#2), #3, #4, #5\par{

\section[#1, #3, #4]{#1 -- #3, #4}
\def\qname{LVEE}
\def\aname{#1}
\def\q ##1\par{{\noindent \bf \qname: ##1 }\par}
\def\a{{\noindent \bf \aname: } \def\qname{L}\def\aname{#2}}
}

\def\interview* #1 (#2), #3, #4, #5\par{

\section*{#1\\{\small\rm #3, #4. #5}}

\def\qname{LVEE}
\def\aname{#1}
\def\q ##1\par{{\noindent \bf \qname: ##1 }\par}
\def\a{{\noindent \bf \aname: } \def\qname{L}\def\aname{#2}}
}

%\switchlang{be}
%\usepackage{color}
\begin{document}
\title{Интервью с участниками}
%\author{}
\date{}
%\maketitle

\begin{Parallel}[p]{}{}

     \ParallelLText{%
      \selectlanguage{english}
\interview* Gustav Wall (GW.), Oldenburg, Germany, 

{\noindent \bf LVEE: First of all, tell a few words about yourself.}

{\noindent \bf Gustav Wall:} My name is Gustav, I live in Germany, in the city of Oldenburg~--- it's somewhere between Bremen and Holland. I moved to Germany in the turbulent nineties, when much was destroyed, and something was created, and I lived in a small regional center in Kazakhstan\ldots My relocation coincided with the time when the young family was getting on its feet, and there were all these worries, starting with the shopping lines for a cabbage\ldots Some people of German origin where moving to Germany, and we in the family conferred and moved too.

I am currently engaged in bringing technical support for the users and responsible for the hardware in a large organization with branches all over Germany. Since this organization is very large, I am in charge for a narrow area of work, being the link between the technicians (including users).

{\noindent \bf L: Please, remember how did you encounter free/libre software for the first time? }

{\noindent \bf GW:} It was  in time when there was a wave of popularity of the Pirate Party, and I took part in it for ideological reasons at first, and then (as is known, pirates are very tightly engaged in IT) this software became a spark that lighted this political movement. But earlier there was a Netscape web browser, even before Firefox, and I used it. It turns out that was the first time I ran into free/libre software~--- in the second half of the nineties somewhere.

{\noindent \bf L: Did you already know then that ideology of free/libre software is something more than free-as-a-beer, or did this information come to you later?} 

{\noindent \bf GW:}  In its purest form, I did not think about such a statement of the question, it just turned out by itself. At that time I started to participate in the community quite early, without even thinking, because it was just nice to be a part of something. You do something useful or necessary for yourself, and at the same time you communicate with people.

{\noindent \bf L: And how did it become for you to become an active member of the community? }

{\noindent \bf GW:} There were periods of time when I actively participated in support, through a platform where users ask questions and create tickets. I volunteered to give answers for the new users.

And then I offered my boss a regain. It is a search engine for your own resource, in Java Server Pages, and my employer at that time was a large state institution at the level of a separate state of Germany. Then I began to contribute to the project, developed things I needed, and gave back to the community, as it is typically done in the open source software.

{\noindent \bf L: lear. Can you tell a little more about your participation in the Pirate Party? This party have also shared interests related to free/libre content, software\ldots}

{\noindent \bf GW:} When I came to Germany, I actually avoided politics at the beginning and bypassed large structures on the political stage, because they have reminded me the big and non-dialogue-promoting Communist Party. I can't say everything was bad in Soviet Union, I got a wonderful education there, but something really was~--- I mean despotism, lack of opportunity to express oneself, ones abilities, strong limitations in the boundaries\ldots And I looked hard, avoiding of getting into where I left\ldots It was about 20 years ago, big parties, like CDU and SPD, had more influence on the politic scene than they have now. And when the pirates appeared, it was like a fresh wind, like in song of Scorpions, the <<Wind of Changes>>. Moreover, the topic of IT~--- something that was relevant and to date is no less relevant in many spheres of life~--- had a very high level of importance for pirates. And I was always a public activist, from the time of Soviet Union, and since I do not change my nature, I decided to smell this powder.

Now I have not taken active part in the Pirate Party for several years, and even it had almost disappeared from political life here.
 
{\noindent \bf L: And by the way, why?}

{\noindent \bf GW:} I think there are several reasons. On the one hand, it influenced the development of digitalization, the introduction of digital technologies. Thanks to pirates, other parties discovered this side for themselves. And they somehow <<took away>> this topic (I'm not an expert, it's like looking from the outside, but from the inside too).

And yet, inconvenient people from many other parties gathered in the Pirate Party. Very different ones: on the one hand they already had the experience of political work, and on the other, you can imagine that when many people gather, each with his own opinion and everyone knows better way of doing things~--- it takes a very long time to grind-in. And this occupation burns a lot of nerves, it takes a lot of effort to find compromises. This I think to be the second reason.

Finally the third reason is related to the media. They should be free and independent, but they are under the influence of large political parties, such as CDU or SPD. And even if they are not biased, as the yellow press, so the one which is less yellow, follow things which can attract the reader's or spectator's attention~--- I mean scandals\ldots Including those related to the new party.

{\noindent \bf L: How would you rate the results of participation in the Pirate Party for yourself?}

{\noindent \bf GW:} I would say that one of the lessons I learned from this period is that now I respect politicians of any color. Being engaged in this, you get ingratitude, as from your party comrades, so from the voters. I would not want to be in the place of some minister: to be very connected, to have certain dates, meetings, regardless of how I feel and what my situation in the family is, I must, at least die, to be there. Of course they get their privileges, but this work is not for everyone. In addition, all dirty clothes will be picked up very often, either by the mass media, or by some kind of squabbler competitors. This is a very difficult occupation, which requires a lot of time and nerves.

One of the other lessons for me personally was that thanks to this activity I worked very close with a free office package, which at that time was OpenOffice. And I have understood that the easiest way to learn some free software for the user is to avoid predetermined expectations about how it should work. One of the natural properties of free software is lack of standards. This is certainly the freedom of creativity of developers, I am a programmer myself and can understand what is being done and why.
But the first thing which user faces is the navigation through the interface. The free software developer is free and says: <<I think it's more convenient>>. And he does so, and there are a lot of users who already got acquainted with Microsoft software, who absorbed, to say, with the mother's milk, what and how should be there\ldots When I decided to ditch Microsoft office suite (and since then I manage doing it), I had psychological difficulties, I was angry at first: what an incorrect program is it, the menu is constructed differently, and it reacts not in the right way\ldots

{\noindent \bf L:  Perhaps, the decline in interest to the Pirate Party can be also related to the fact that in terms of free technologies, issues related to a free/libre content are not in trend so much now, changed to something based on blockchain, crypto-currencies\ldots}

{\noindent \bf GW:} There is a filter bubble (the so-called gated communities). And all the same, this is something community-based or just general IT~--- one should find out in which community this topic is in trend. Many of us began to get acquainted with the topic of blockchain from the bitcoin, which in fact is just one of the blockchain applications. It was exotic for me five years ago or earlier. Media in Europe presented it for a long time that if it is a currency that can be paid~--- then it is used to buy weapons or drugs through the Internet. I confess, that I had also fell for this bait: thought it to be something slippery, incomprehensible, the Darknet.

Later, when the blockchain technologies were already quite developed and other applications have appeared, I encountered a project of the distributed computing resources (there are already dozens of such projects now). I have installed this application on my computer, but I still was not understanding the principle at that time, it still seemed to be somewhat dubious.

Finally, at some point, I got an electronic brochure by a futurist, where he talked about how the society would develop, with an example of a blockchain based decentralized autonomous organization (DAO).


{\noindent \bf L: This is closer to the Ethereum project. }

{\noindent \bf GW:} On the one hand quite exotic predictions where expressed there. The author described the future transport communication in cities with parking problems, when moving will be cheaper than paying for parking, and self-governing taxi cabs without a driver will be free of charge, but will privide additional paid services to be used during the trip. And on the other hand, there is the theme of a decentralized autonomous organization, the DAO. When I understood the technical principle and began to wonder what could be the practical application~--- then I finally understood.

I was actively posting in my blog at the time when Snowden had disclosed that fatal surveillance was taking place at the state level, and I wrote one article on this topic that Eric Schmidt started to appeal  publicly to the state institutions not to break the Internet. Because when each step of everyone is under supervision, then those people who are interesting to Google as potential customers will hide in their filter bubbles, will become opaque, and then Google will be in trouble as a commercial concept. I wrote in my blog post that, hmmm, yes, you are afraid, but development can not be stopped, and cited that there is another development that makes a serious threat to financial institutions.

I see great potential in blockchain technology, but there are certainly risks also. For the Western society, in which information technologies play a big role, I would compare its potential to the invention of printing or to the development of atomic energy.

{\noindent \bf L:  So great?}

{\noindent \bf GW:} I will touch the theme of the Solaris movie here. The Internet is an infrastructure. And this infrastructure is at the same time a capacity, where the <<brain>> is technically located, the environment. It is like our own brain resides in a clot of protoplasm, in which some kind of exchange between the neurons takes place. Similarly, whole this information structure, which can even be divided, just as the division is present in our brain\ldots

The concept of the Noosphere was invented by Vernadsky, not by me, and these ideas have also seemed abstract to me earlier. So the Solaris movie initially seemed very boring, and only after I got the understanding of the blockchain, I rethought it. It is impressive how people could foresee the development.

Both infrastructure and solutions are already represented in the film, as far as unexpected side effects.


{\noindent \bf L: You mean the intelligent ocean of Solaris\ldots }

{\noindent \bf GW:}  On the one hand, there is a structure where some smart processes can occur, and on the other hand, data can be stored in nodes, including millions of different applications: not just dead data but ones processed by algorithms. To date, a huge leap has occurred, I mean neural networks, about which the press likes to say that the developers do not really know exactly how they work because of the effect of self-learning.

And then there was a massacre in Munich, which was classified as amok~--- for me it was also an example.

{\noindent \bf L: Somehow connected with such general decentralized <<smart>> system? }

{\noindent \bf GW:}  The security forces thought attacks were done by Islamists. Munich was practically paralyzed for the whole day. It would be better for the population if the signals were still processed on a real-time basis by twitter or telephone, and the quality of the information would be assessed~--- and the assessment can be done if you know what kind of person this was written by. I understand that this is also related to surveillance, but my understanding is that this technology could still be applied, provided that it is not in the hands of law enforcement agencies, does not depend on the state, so that there is less risk of wrong use.



\interviewfooter{Questions and Russian translation by Dmitriy Kostiuk.}
\vfill
     }
     \ParallelRText{%
       \selectlanguage{russian}
\interview* Густав Валь (LP.), Ольденбург, Германия,
       
{\noindent \bf LVEE: Для начала расскажи пару слов о себе.}

{\noindent \bf Gustav Wall:} Меня зовут Густав, я живу в Германии, в городе Ольденбург~--- это где-то между Бременом и Голландией. В Германию переехал в бурные девяностые годы, когда много чего разрушалось, и кое-что создавалось, тогда я жил в небольшом районном центре в Казахстане\ldots И у меня переезд совпал с тем временем, когда молодая семья становилась на ноги, и всё это переживала, начиная с очередей за капустой\ldots Соотечественники немецкого происхождения уезжали в Германию, и мы в семье посовещались и переехали.

Занимаюсь я в настоящее время тем, что оказываю поддержку пользователям и отвечаю за аппаратное обеспечение в большой организации, которая имеет филиалы по всей Германии. Поскольку организация очень большая, отвечаю за узкий участок работы. Я~--- связующее звено между техниками (и в том числе пользователями).

{\noindent \bf L: Помнишь, как ты впервые столкнулся со свободным ПО? }

{\noindent \bf GW:} Конкретно~--- когда была волна популярности Партии Пиратов, и я сначала участвовал по идейным соображениям, а потом (как известно, пираты темой IT очень плотно занимаются)~--- оно стало как бы искрой, зажегшей это политическое движение. Хотя еще раньше был браузер Netscape, еще до Firefox, и им я пользовался. Получается, что тогда в первый раз и столкнулся~--- во второй половине девяностых годов где-то. 

{\noindent \bf L: А ты уже тогда знал, что за идеологией свободного ПО стоит нечто большее, чем бесплатность, или эта информация стала до тебя доходить позже?} 

{\noindent \bf GW:}  В чистом виде я над такой постановкой вопроса не задумывался, просто само собой получалось. Я в то время довольно рано начал участвовать в работе комьюнити, даже не задумываясь, потому что просто приятно быть частью чего-то. Делаешь что-то полезное или нужное для себя,  и в то же время общаешься с людьми.

{\noindent \bf L: А как получилось для тебя стать активным участником комьюнити? }

{\noindent \bf GW:} Были отрезки времени, когда я достаточно активно участвовал в поддержке, через платформу, где пользователи задают вопросы, создают тикеты. Я на общественных началах брал на себя ответы начинающим пользователям.

И потом я предложил своему шефу \verb!regain!. Это такая система поисковая для для собственного ресурса, на Java Server Pages, а мой работодатель на тот момент~--- большое государственное учреждение на уровне отдельной федеральной земли. Тогда я стал вносить в проект свою лепту, разрабатывал то, что мне было надо, и возвращал сообществу, как это принято в открытом ПО.

{\noindent \bf L: Понятно. Про участие в Партии Пиратов можешь немного подробнее? Эта партия ведь выражала интересы в том числе по части свободного контента, программного обеспечения\ldots}

{\noindent \bf GW:} Когда приехал в Германию, я вообще-то сторонился политики в начале и обходил стороной большие структуры на политической сцене, потому что они напоминали тоже большую и не способствующую диалогу Коммунистическую Партию. Я не могу сказать, что все плохо было в Союзе, я получил замечательное образование, но кое что было~--- я имею в виду произвол, или отсутствие возможности проявить себя, свои способности, сильную ограниченность рамками\ldots И чтобы не оказаться там, откуда ушел, я приглядывался\ldots Это было лет 20 назад, тогда большие партии, ХДС и СПГ, имели больше влияния на политической сцене. И вот когда пираты появились, это было как свежий ветер, как у Scorpions, Wind of Changes. Тем более, что очень большое место занимала у пиратов и тема IT~--- то, что было актуально и на сегодняшний день не меньше актуально во многих сферах жизни. А общественником я был всегда, еще в Союзе, и поскольку натуру свою не изменишь~--- решил понюхать этого пороха. 

Сейчас уже несколько лет я активного участия в деятельности Партии Пиратов не принимаю, да и она из политической жизни почти исчезла здесь. 
 
{\noindent \bf L: А, кстати, почему?}

{\noindent \bf GW:} Я думаю, причин несколько. С одной стороны она оказала влияние на развитие цифровизации, внедрение цифровых технологий. Благодаря пиратам другие партии открыли эту сторону для себя. И они как бы <<отобрали>> эту тему (я не эксперт, это как взгляд со стороны, но и изнутри тоже). 

А ещё, в Партии Пиратов собрались неудобные люди из многих других партий. Очень разные: с одной стороны у них уже был опыт политической работы, а с другой~--- можешь себе представить, что когда собирается много личностей, каждая со своим мнением и каждая знает, как правильно~--- очень много времени уходит на притирание. А это занятие сжигает много нервов, требует много сил  на поиск компромиссов. Это я думаю вторая причина. 

Третья причина~--- СМИ, которые хотя и должны быть свободные и независимые, но в них тоже есть влияние больших политических партий, таких как ХДС или СПГ. И потом они даже не то, чтобы преподносили предвзятое мнение, а просто что желтая пресса, что менее желтая, они живут тем, чем можно завоевать внимание читателя или зрителя~--- скандалами. В том числе и связанными с новой партией. 


{\noindent \bf L: Как бы ты оценил результаты участия в Партии Пиратов лично для себя?}

{\noindent \bf GW:} Я бы сказал, один из уроков, которые я получил благодаря этому периоду~--- у меня есть уважение к политикам любой окраски. Занимаясь этим, получаешь неблагодарность, что от своих соратников по партии, что от избирателей.  Я не хотел бы быть на месте какого-то министра: он очень связан, у него есть определенные сроки, встречи, независимо от того, как он себя чувствует и какая у него ситуация в семье, он должен, хоть умри, быть там. Конечно они получают свои привилегии, но это работа не для каждого. Кроме того, что сплошь и рядом всё грязное бельё поднимут или средства массовой информации, или какие-то склочники~---конкуренты. Это очень тяжелое занятие, требующее больших затрат времени и нервов.

Один из других уроков лично для меня~--- благодаря этой активности я очень конкретно поработал со свободным офисным пакетом, на тот момент OpenOffice. И усвоил, пользователю проще всего освоить какое-то свободное ПО, если не подходить с заранее определенными ожиданиями, как оно должно работать. Одно из естественных свойств свободного ПО~--- то, что там нет стандартов. Это конечно свобода творчества разработчиков, я сам программист и могу понять что как делается и почему. 
Но первое, с чем сталкивается пользователь~--- навигация по интерфейсу. Разработчик свободного ПО свободен и говорит: <<Я считаю, что так удобнее>>. И он делает так, а очень многие пользователи начинают знакомство с программным обеспечением Microsoft, впитывают, образно говоря, с молоком матери, что и как должно быть\ldots Я когда решил обходиться без офисного пакета Microsoft (и с тех пор обхожусь)~--- были психологические трудности, я злился поначалу, что за такая неправильная программа, меню не так построено, реагирует не так\ldots

{\noindent \bf L:  Пожалуй, спад интереса к тематике Партии Пиратов может быть ещё связан с тем, что в последнее время по части свободных технологий в моде не вопросы, связанные со свободным контентом, а скорее что-то на основе блокчейн, криптовалюты\ldots}

{\noindent \bf GW:} Тут действует пузырь фильтров (так называемые gated\linebreak communities, или огороженные сообщества). И все равно, это СПО-сообщество или просто IT~--- сплошь и рядом в чисто практическом аспекте надо смотреть, для какого именно сообщества эта тема модная.  Многие начали знакомство с темой блокчейн  с биткоинов, которые по сути лишь одно из приложений блокчейна. Лет пять еще назад или раньше для меня это была экзотика. СМИ в Европе долго преподносили, что если это валюта, которой можно платить~--- то этим пользуются там, где или оружие можно купить через Интернет, или наркотики. Я, признаюсь, тоже попал на эту удочку: это что-то скользкое, непонятное, Даркнет. 

Позже, когда блокчейн-технологии уже были довольно развиты и появились другие приложения, я столкнулся с проектом распределенных вычислительных ресурсов (сейчас таких проектов уже десятки есть). Установил это приложение на своем компьютере, но в принцип работы не вник, на тот момент это все еще казалось чем-то сомнительным.  

И, наконец, в какой-то момент мне попала в руки электронная брошюра одного футуролога, где он рассуждал, как общество будет развиваться, и пример привел  децентрализованной автономной организации (ДАО) на основе блокчейна. 

{\noindent \bf L: Это уже ближе к проекту Ethereum. }

{\noindent \bf GW:} С одной стороны там высказывались довольно экзотические прогнозы, автор говорил, как будет выглядеть транспортное сообщение в городах, где проблемы с парковкой, когда в будущем ездить станет дешевле, чем платить за парковку, и самоуправляющиеся машины такси без водителя будут бесплатными, а платными будут услуги, которыми ты будешь пользоваться во время поездки. А с другой~--- тема децентрализованной автономной организации, ДАО. Когда я разобрался с техническим принципом работы и начал задумываться, какое может быть практическое применение~--- торгда я наконец понял. 

В то время я активно публиковал блог, тогда как раз Сноуден разоблачил, что на государственном уровне фатальная слежка происходит, и я написал одну статью на эту тему, что Эрик Шмидт начал призывы на публику к государственным учреждениям, чтоб они не сломали Интернет. Потому что когда все на каждом шагу под наблюдением, тогда те люди, которые фирме Google интересны как потенциальные клиенты, спрячутся в свои пузыри фильтров, станут непрозрачны, и тогда Google как предпринимательский концепт окажется под угрозой. Я в своей статье написал, что хм, да, вы боитесь, но развитие не остановить, и процитировал, что есть и другое развитие, которое серьезную угрозу представляет финансовым институтам.

В блокчейн-технологии я вижу большой потенциал, но есть конечно и риски. Для развития западного общества, где информационные технологии большую роль играют, я сравнил бы это по потенциалу с изобретением печати или освоением атомной энергии. 

{\noindent \bf L:  Настолько большой?}

{\noindent \bf GW:} Я коснусь темы фильма Солярис. Интернет~--- это инфраструктура. И эта инфраструктура одновременно ёмкость, где мозг технически расположен, среда, как наш собственный мозг находится в комке протоплазмы, в которой между нейронами происходит какой-то обмен.  Точно так же и вся эта информационная структура, и может она даже разделена, точно так же как разделение присутствует и в нашем мозге.

Понятие Ноосферы изобретено не мной, это академик Вернадский, и мне эти идеи тоже раньше  абстрактными казались. Так же и фильм Солярис мне изначально показался очень скучным, и  только после того, как с блокчейном столкнулся, я его переосмыслил. Впечатляет, насколько люди смогли предвидеть развитие. 

И инфраструктура, и решения уже представлены в фильме, и то, какие могут быть неожиданные побочные явления.


{\noindent \bf L: То есть разумный океан Соляриса\ldots }

{\noindent \bf GW:}  С одной стороны есть структура, где какие-то умные процессы могут происходить, а с другой в узлах там могут храниться данные, и в том числе милионны разных приложений: данные, которые не просто мертвым грузом лежат, а обрабатываются алгоритмами. На сегодняшний день огромный скачок произошел, я имею в виду нейронные сети, про которые пресса любит говорить, что сами разработчики не знают толком, как именно они работают из-за эффекта самообучения.

И потом случилось массовое убийство в Мюнхене, которое классифицировали как амок~--- для меня это тоже был пример.

{\noindent \bf L: Как-то связанный с такой общей децентрализованной <<умной>> системой? }

{\noindent \bf GW:}  Органы безопасности думали, что это исламисты, нападения, Мюнхен был на целый день практически парализован. Населению было бы лучше, если бы сигналы, все равно по твиттеру или телефону, в режиме реального времени обрабатывались, и оценивалось бы качество информации~--- а оценить можно, если знать, что за человек это писал. Я понимаю, что это тоже со слежкой связано, но мое понимание – что эту технологию можно было бы применить при условии, что она не находится в руках силовых структур, не зависит от государства, чтобы был меньше риск злоупотребления.

\interviewfooter{Вопросы и русский перевод Дмитрия Костюка.}

\vfill

     }
   \end{Parallel}

 
\end{document}


