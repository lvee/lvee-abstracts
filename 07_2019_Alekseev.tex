\documentclass[10pt, a5paper]{article}
\usepackage[T2A]{fontenc}
\usepackage{ucs}
\usepackage[utf8x]{inputenc}
\usepackage[polish,english,russian]{babel}
\usepackage{hyperref}
\usepackage[inner=2cm,top=1.8cm,outer=2cm,bottom=2.3cm,nohead]{geometry}
\usepackage{listings}
\usepackage{graphicx}
\usepackage{wrapfig}
\usepackage{longtable}
\usepackage{indentfirst}
\frenchspacing
\usepackage{fixltx2e} %text sub- and superscripts
\usepackage{icomma} % коскі ў матэматычным рэжыме
\PreloadUnicodePage{4}

\newcommand{\longpage}{\enlargethispage{\baselineskip}}
\newcommand{\shortpage}{\enlargethispage{-\baselineskip}}

\def\switchlang#1{\expandafter\csname switchlang#1\endcsname}
\def\switchlangbe{
\let\saverefname=\refname%
\def\refname{Літаратура}%
\def\figurename{Іл.}%
}
\def\switchlangen{
\let\saverefname=\refname%
\def\refname{References}%
\def\figurename{Fig.}%
}
\def\switchlangru{
\let\saverefname=\refname%
\let\savefigurename=\figurename%
\def\refname{Литература}%
\def\figurename{Рис.}%
}

\hyphenation{admi-ni-stra-tive}
\hyphenation{ex-pe-ri-ence}
\hyphenation{fle-xi-bi-li-ty}
\hyphenation{Py-thon}
\hyphenation{ma-the-ma-ti-cal}
\hyphenation{re-ported}
\hyphenation{imp-le-menta-tions}
\hyphenation{pro-vides}
\hyphenation{en-gi-neering}
\hyphenation{com-pa-ti-bi-li-ty}
\hyphenation{im-pos-sible}
\hyphenation{desk-top}
\hyphenation{elec-tro-nic}
\hyphenation{com-pa-ny}
\hyphenation{de-ve-lop-ment}
\hyphenation{de-ve-loping}
\hyphenation{de-ve-lop}
\hyphenation{da-ta-ba-se}
\hyphenation{plat-forms}
\hyphenation{or-ga-ni-za-tion}
\hyphenation{pro-gramming}
\hyphenation{in-stru-ments}
\hyphenation{Li-nux}
\hyphenation{en-vi-ron-ment}
\hyphenation{Te-le-pathy}
\hyphenation{Li-nux-ov-ka}

\def\progref!#1!{\texttt{#1}}
\renewcommand{\arraystretch}{2} %Іначай формулы ў матрыцы зліпаюцца з лініямі
\usepackage{array}

\def\interview #1 (#2), #3, #4, #5\par{

\section[#1, #3, #4]{#1, #5}
\def\qname{LVEE}
\def\aname{#1}
\def\q ##1\par{{\noindent \bf \qname: ##1 }\par}
\def\a{{\noindent \bf \aname: } \def\qname{L}\def\aname{#2}}
}

\switchlang{ru}
\begin{document}
\title{Использование свободного программного обеспечения при подготовке бакалавров и магистров направления <<Прикладная математика и информатика>>}
\author{Е.Р. Алексеев, к.т.н., доцент, доцент кафедры ИТО \\ КубГУ, Краснодар, Россия\footnote{\url{ er.alekseev@yandex.ru}, \url {https://lvee.org/ru/abstracts/312}}}
\maketitle
\begin{abstract}
This article presents the experience of using open-source software in the training of bachelors of Applied Mathematics and \linebreak Informatics and masters of Technologies of Parallel Programming and High-performance Calculations.
\end{abstract}
Свободное программное обеспечение широко использовалось при подготовке бакалавров и магистров направления <<Прикладная математика и информатика>> в Вятском Государственном Университете \cite{bib1}.

Бакалавриат <<Прикладная математика и информатика>> являлся основной для обучения в магистратуре по направлению <<Технологии параллельного программирования и высокопроизводительные вычисления>>.

В связи с этим кроме основных курсов кафедрой были предложены дисциплины по выбору, которые нужны будущему специалисту в области параллельных вычислений. В качестве необходимых будущему выпускнику преподавались следующие курсы: численные методы, операционные системы, параллельные вычисления, практикум по администрированию вычислительных систем.

Была переработана серия классических курсов с ориентацией на использование свободного программного обеспечения.

\textbf{Операционные системы} --- классический курс для специалистов в области прикладной математики и информатики.

Лекционный курс включает в себя: такие разделы, как история развития операционных систем, алгоритмы управления процессами, памятью и файловой системой.

Автором разработаны лабораторные работы на базе свободного ПО:

\begin{itemize}
  \item файловая система unix-подобных операционных систем;
  \item установка ОС семейства Linux на компьютер, первоначальная настройка рабочего стола;
  \item команды терминала Linux;
  \item знакомство с репозиторием программного обеспечения (ПО), установка программ в ОС Linux;
  \item управление пользователями в Linux;
  \item средства создания загрузочной флешки;
  \item компиляция и отладка программ в ОС Linux (знакомство с компилятором gcc, отладчик gdb, сборка приложений с помощью утилиты make);
  \item разработка кроссплатформенных приложений, программирование алгоритмов управления процессами;
  \item утилиты сборки дистрибутива.
\end{itemize}

В качестве курсового проекта студентам предлагалось собрать специализированный дистрибутив семейства Linux (\url{distributiv.wordpress.com}, \url{distributiv.wordpress.com}). Курс заканчивался экзаменом.

\textbf{Параллельные вычисления}. Курс предназначен для знакомства студентов третьего курса с параллельным программированием. Дополнительно к языку С(С++) изучается современный язык Фортран --- язык разработки конвейерных, параллельных и высокоэффективных вычислений. Студенты знакомятся с технологиями параллельного программирования MPI и OpenMP (языки программирования С/C++ и Фортран). Разрабатывают параллельные MPI-программы для решения классических задач вычислительной математики. Лабораторные и практические занятия проводились в специализированных лабораториях на компьютерах под управлением ОС Lubuntu с использованием компиляторов gcc, g++, gfortran, библиотека MPICH. Лабораторные работы ориентированы как на использование на локальных компьютеров, так и на работу с вычислительным кластером.

Также на базе свободного ПО был разработан практикум для курса <<\textbf{Администрирование в вычислительных сетях}>>. Студенты практически изучали следующие темы: настройка и использование FTP-сервера; клиент и сервер ssh; настройка и использование samba.

По итогам обучения в бакалавриате студент получал довольно широкие знания, и становился довольно квалифицированным пользователем unix-подобных ОС (Debian, Ubuntu).

Практически все специальные дисциплины магистратуры проводятся на компьютерах с использованием ОС семейства Linux.

\textbf{Архитектура параллельных вычислительных систем} --- базовый курс для подготовки магистров (профиль <<Технологии параллельного программирования и высокопроизводительные вычисления>>). Кроме изучения теоретических вопросов архитектуры современных вычислительных систем, магистранты самостоятельно <<строят>> локальный вычислительный кластер, который и используется для проведения лабораторных работ по курсу.

\textbf{Технологии параллельных вычислений}. Магистранты подробно изучают параллельное программирование с использованием технологий MPICH, OpenMP и CUDA. Кроме этого магистранты изучают технологии автораспараллеливания и комассивы. Используются компиляторы g++, gfortran, icpc, ifort. Технической базой являлись кластер ВятГУ, компьютеры в лаборатории математического моделирования (процессоp I7, ОЗУ --- 16 Гб, поддержка CUDA), локальный кластер, построенный магистрантами в курсе <<Архитектура параллельных вычислительных систем>>. Завершается курс экзаменом и защитой курсового проекта. Во время выполнения которого магистранты решают сложную вычислительную задачу с использованием различных технологий параллельного программирования, оценивают быстродействие программ с использованием различных технологий и делают выводы о выборе оптимальной технологии для решения данной задачи.

\textbf{Параллельные численные методы}. Завершающий курс по параллельным вычислениям в магистратуре, в котором магистранты с использованием технологии параллельного программирования разрабатывают параллельные приложения решения задач вычислительной математики с использованием технологий MPI, OpenMP, Co-array на языках С и Фортран. Используются свободно-распро\-страняемые средства разработки и компиляторы компании Intel.

Обучение параллельным вычислениям ориентированно на локальные компьютеры лаборатории математического моделирования и вычислительный кластер. На вычислительном кластере студентами и магистрантами используется 16 вычислительных узлов, на которых развёрнута операционная система Ubuntu 14.04 c необходимым ПО. Свободные компиляторы являются основным инструментом для разработки параллельных приложений при подготовке дипломов бакалаврами и написании магистерских диссертаций.

Опыт использования свободного ПО и построения учебных курсов на их основе позволяет сделать следующие выводы \cite{bib2}:

\begin{enumerate}
  \item Использование свободного программного обеспечение позволяет очень гибко строить учебные курсы.
  \item Студенты используют легальное программное обеспечение, выполняют требования российского и международного законов.
  \item Многолетний опыт использования автором свободного программного при подготовке студентов позволяет говорить о глубоком уровне компьютерной подготовки выпускников.
  \item Изучение ОС семейства Linux наряду с Windows, делает выпускника более подготовленным к практической деятельности.
\end{enumerate}

Однако преподавание на базе свободного программного обеспечения требует более квалифицированных преподавателей, профессионалов, а не модных сейчас <<менеджеров от образования>>.


\begin{thebibliography}{9}
\bibitem{bib1} {Евгений Алексеев. Свободное программное обеспечение при подготовке бакалавров и магистров на кафедре фундаментальной информатики и прикладной математики Вятского Государственного Университета// Двенадцатая конференция <<Свободное программное обеспечение в высшей школе>>: Материалы конференции / Переславль, 27–29 января 2017 года. М.: Basealt, 2017. --- C. 110-112.}
\bibitem{bib2} {Алексеев Е.Р., Лутошкин Д.А., Стародумов В.В. Опыт использования свободного программного обеспечения при подготовке специалистов высшей квалификации в университетах бывшего СССР //  Информационные системы и технологии в моделировании и управлении : сборник материалов IV Всероссийской научно-практической конференции с международным участием (21-23 мая 2019 г.). --- Симферополь, ИТ <<АРИАЛ>>, 2019. ---  C. 255-261.}\end{thebibliography}
\end{document}
