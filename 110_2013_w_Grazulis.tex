\documentclass[10pt, a5paper]{article}
\usepackage{pdfpages}
\usepackage{parallel}
\usepackage[T2A]{fontenc}
\usepackage{ucs}
\usepackage[utf8x]{inputenc}
\usepackage[polish,english,russian]{babel}
\usepackage{hyperref}
\usepackage{rotating}
\usepackage[inner=2cm,top=1.8cm,outer=2cm,bottom=2.3cm,nohead]{geometry}
\usepackage{listings}
\usepackage{graphicx}
\usepackage{wrapfig}
\usepackage{longtable}
\usepackage{indentfirst}
\usepackage{array}
\newcolumntype{P}[1]{>{\raggedright\arraybackslash}p{#1}}
\frenchspacing
\usepackage{fixltx2e} %text sub- and superscripts
\usepackage{icomma} % коскі ў матэматычным рэжыме
\PreloadUnicodePage{4}

\newcommand{\longpage}{\enlargethispage{\baselineskip}}
\newcommand{\shortpage}{\enlargethispage{-\baselineskip}}

\def\switchlang#1{\expandafter\csname switchlang#1\endcsname}
\def\switchlangbe{
\let\saverefname=\refname%
\def\refname{Літаратура}%
\def\figurename{Іл.}%
}
\def\switchlangen{
\let\saverefname=\refname%
\def\refname{References}%
\def\figurename{Fig.}%
}
\def\switchlangru{
\let\saverefname=\refname%
\let\savefigurename=\figurename%
\def\refname{Литература}%
\def\figurename{Рис.}%
}

\hyphenation{admi-ni-stra-tive}
\hyphenation{ex-pe-ri-ence}
\hyphenation{fle-xi-bi-li-ty}
\hyphenation{Py-thon}
\hyphenation{ma-the-ma-ti-cal}
\hyphenation{re-ported}
\hyphenation{imp-le-menta-tions}
\hyphenation{pro-vides}
\hyphenation{en-gi-neering}
\hyphenation{com-pa-ti-bi-li-ty}
\hyphenation{im-pos-sible}
\hyphenation{desk-top}
\hyphenation{elec-tro-nic}
\hyphenation{com-pa-ny}
\hyphenation{de-ve-lop-ment}
\hyphenation{de-ve-loping}
\hyphenation{de-ve-lop}
\hyphenation{da-ta-ba-se}
\hyphenation{plat-forms}
\hyphenation{or-ga-ni-za-tion}
\hyphenation{pro-gramming}
\hyphenation{in-stru-ments}
\hyphenation{Li-nux}
\hyphenation{sour-ce}
\hyphenation{en-vi-ron-ment}
\hyphenation{Te-le-pathy}
\hyphenation{Li-nux-ov-ka}
\hyphenation{Open-BSD}
\hyphenation{Free-BSD}
\hyphenation{men-ti-on-ed}
\hyphenation{app-li-ca-tion}

\def\progref!#1!{\texttt{#1}}
\renewcommand{\arraystretch}{2} %Іначай формулы ў матрыцы зліпаюцца з лініямі
\usepackage{array}

\def\interview #1 (#2), #3, #4, #5\par{

\section[#1, #3, #4]{#1 -- #3, #4}
\def\qname{LVEE}
\def\aname{#1}
\def\q ##1\par{{\noindent \bf \qname: ##1 }\par}
\def\a{{\noindent \bf \aname: } \def\qname{L}\def\aname{#2}}
}

\def\interview* #1 (#2), #3, #4, #5\par{

\section*{#1\\{\small\rm #3, #4. #5}}

\def\qname{LVEE}
\def\aname{#1}
\def\q ##1\par{{\noindent \bf \qname: ##1 }\par}
\def\a{{\noindent \bf \aname: } \def\qname{L}\def\aname{#2}}
}


\begin{document}
\title{F/LOSS for Open Science: Crystallography Open Database}
\author{Saulius Gražulis \footnote{Vilnius, Lithuania; \url{grazulis@ibt.lt}}}
\maketitle
\begin{abstract}
Free and Open Source software serves both as a model of development and as enabling methodology for many other fields of human enterprise. We have applied the Free software and Open source principles to create an open access scientific database in the field of chemical crystallography.
\end{abstract}
The COD project (abbreviated from the `Crystallography Open Database', \url{http://www.crystallography.net/}) aims at collecting in a single open access database all organic, inorganic and metal organic structures \cite{graz1} (except for the structures of biological macromolecules that are available at the PDB \cite{graz2}). The database was founded by Armel Le Bail, Lachlan Cranswick, Michael Berndt, Luca Lutterotti and Robert M. Downs in February 2003 as a response to Michael Berndt’s letter published in the Structure Determination by Powder Diffractometry (SDPD) mailing list \cite{graz3}. Since December 2007 the main database server is maintained and new software is developed in the Vilnius University Institute of Biotechnology by Saulius Gražulis and Andrius Merkys, and has now over 200 thousand records describing structures published in major crystallographic and chemical peer- \linebreak reviewed journals \cite{graz4}.

The COD database is implemented using F/LOSS software, on a LAMP platform, with addition of the home made Linux command line software licensed under GPL. The database itself is also governed using open principles: access to data is free (as in freedom) to all who would wish it, and deposition of data is open for all registered users, given the data they provide meet quality criteria accepted in science. In future, we plan to implement a Web based peer-review network, enabling new, open ways of doing science. We expect a considerable synergy from application of open source principles in combination with scientific merits.

\begin{thebibliography}{9}
\bibitem{graz1} Gražulis, S.; Chateigner, D.; Downs, R. T.; Yokochi, A. F. T.; Quirós, M.; Lutterotti, L.; Manakova, E.; Butkus, J.; Moeck, P. \& Le Bail, A. (2009). Crystallography Open Database – an open-access collection of crystal structures, Journal of Applied Crystallography 42: 726--729.
\bibitem{graz2} Berman, H.; Henrick, K. \& Nakamura, H. (2003). Announcing the worldwide Protein Data Bank, Nat Struct Mol Biol 10: 980--980.
\bibitem{graz3} Berndt, M. (2003). Open crystallographic database "--- a role for whom?, \url{http://tech.groups.yahoo.com/group/sdpd/message/1016} (retrieved 2013.01.31).
\bibitem{graz4} Gražulis, S.; Daškevič, A.; Merkys, A.; Chateigner, D.; Lutterotti, L.; Quirós, M.; Serebryanaya, N. R.; Moeck, P.; Downs, R. T. \& Le Bail, A. (2012). Crystallography Open Database (COD): an open-access collection of crystal structures and platform for world-wide collaboration, Nucleic Acids Research 40: D420--D427.
\end{thebibliography}

\end{document}
