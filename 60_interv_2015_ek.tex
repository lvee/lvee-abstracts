\documentclass[10pt, a5paper]{article}
\usepackage{pdfpages}
\usepackage{parallel}
\usepackage[T2A]{fontenc}
\usepackage{ucs}
\usepackage[utf8x]{inputenc}
\usepackage[polish,english,russian]{babel}
\usepackage{hyperref}
\usepackage{rotating}
\usepackage[inner=2cm,top=1.8cm,outer=2cm,bottom=2.3cm,nohead]{geometry}
\usepackage{listings}
\usepackage{graphicx}
\usepackage{wrapfig}
\usepackage{longtable}
\usepackage{indentfirst}
\usepackage{array}
\newcolumntype{P}[1]{>{\raggedright\arraybackslash}p{#1}}
\frenchspacing
\usepackage{fixltx2e} %text sub- and superscripts
\usepackage{icomma} % коскі ў матэматычным рэжыме
\PreloadUnicodePage{4}

\newcommand{\longpage}{\enlargethispage{\baselineskip}}
\newcommand{\shortpage}{\enlargethispage{-\baselineskip}}

\def\switchlang#1{\expandafter\csname switchlang#1\endcsname}
\def\switchlangbe{
\let\saverefname=\refname%
\def\refname{Літаратура}%
\def\figurename{Іл.}%
}
\def\switchlangen{
\let\saverefname=\refname%
\def\refname{References}%
\def\figurename{Fig.}%
}
\def\switchlangru{
\let\saverefname=\refname%
\let\savefigurename=\figurename%
\def\refname{Литература}%
\def\figurename{Рис.}%
}

\hyphenation{admi-ni-stra-tive}
\hyphenation{ex-pe-ri-ence}
\hyphenation{fle-xi-bi-li-ty}
\hyphenation{Py-thon}
\hyphenation{ma-the-ma-ti-cal}
\hyphenation{re-ported}
\hyphenation{imp-le-menta-tions}
\hyphenation{pro-vides}
\hyphenation{en-gi-neering}
\hyphenation{com-pa-ti-bi-li-ty}
\hyphenation{im-pos-sible}
\hyphenation{desk-top}
\hyphenation{elec-tro-nic}
\hyphenation{com-pa-ny}
\hyphenation{de-ve-lop-ment}
\hyphenation{de-ve-loping}
\hyphenation{de-ve-lop}
\hyphenation{da-ta-ba-se}
\hyphenation{plat-forms}
\hyphenation{or-ga-ni-za-tion}
\hyphenation{pro-gramming}
\hyphenation{in-stru-ments}
\hyphenation{Li-nux}
\hyphenation{sour-ce}
\hyphenation{en-vi-ron-ment}
\hyphenation{Te-le-pathy}
\hyphenation{Li-nux-ov-ka}
\hyphenation{Open-BSD}
\hyphenation{Free-BSD}
\hyphenation{men-ti-on-ed}
\hyphenation{app-li-ca-tion}

\def\progref!#1!{\texttt{#1}}
\renewcommand{\arraystretch}{2} %Іначай формулы ў матрыцы зліпаюцца з лініямі
\usepackage{array}

\def\interview #1 (#2), #3, #4, #5\par{

\section[#1, #3, #4]{#1 -- #3, #4}
\def\qname{LVEE}
\def\aname{#1}
\def\q ##1\par{{\noindent \bf \qname: ##1 }\par}
\def\a{{\noindent \bf \aname: } \def\qname{L}\def\aname{#2}}
}

\def\interview* #1 (#2), #3, #4, #5\par{

\section*{#1\\{\small\rm #3, #4. #5}}

\def\qname{LVEE}
\def\aname{#1}
\def\q ##1\par{{\noindent \bf \qname: ##1 }\par}
\def\a{{\noindent \bf \aname: } \def\qname{L}\def\aname{#2}}
}

\begin{document}
\title{Интервью с участниками}
%\author{}
\date{}
\maketitle

По традиции в сборник материалов входят интервью, в которых активные участники сообщества open source делятся своим мнением о свободном ПО, открытых технологиях, роли и месте GNU/Linux, рассказывают, как видят проблематику свободных проектов. В этот раз мы решили расспросить трёх участников конференции, какое-то время назад перебравшихся из Беларуси на территорию Европейского Союза.



\section[Евгений Калюта "--- developer, Ericsson, Хельсинки, Финляндия]{Евгений Калюта "--- developer,  Ericsson, \linebreak Хельсинки, Финляндия}

%\begin{figure}[ht]
%\centering{\includegraphics[width=4cm]{49_spons_altoros.jpg}}
%\end{figure}

{\noindent \bf L: Традиционный первый вопрос "--- твое первое знакомство с открытым ПО. Может быть первые впечатления, если они были?}

{\noindent \bf Евгений Калюта:} Я расскажу долгую историю :) 

{\noindent \bf L: Отлично :)}

{\noindent \bf Е:} Я из провинции. Доступность как информации, так и техники тогда была не на высоте. Учитель информатики у нас был молодой, активный, сразу после
института. Это был 7"=ой класс, когда нам поставили <<Корветы>>. 

{\noindent \bf L: Действительно, издалека :)}


{\noindent \bf Е:} Программы обучения толком не было, нас в класс пускали, но директор строго говорила
<<седьмому классу только игры>>. Однако учитель некоторым пытливым показал книжки по Basic и давал основы алгоритмизации (в моём классе нас таких пытливых было двое). Он же (учитель) как"=то рассказал, что для настоящего
программирования бывает ассемблер (что это я тогда представлял с трудом), и C.

{\noindent \bf L: Этого хотелось?}

{\noindent \bf Е:} Этого очень хотелось. Но книг в доступности не было (начало девяностых).

Однажды в книжном я таки увидел какую"=то брошюрку, то ли про C, то ли про что"=то ещё, но главное, что в предисловии было замечено, что вот такой вот он язык C, и на нём написали Unix, на котором работает Интернет.

Очень захотелось как C, так и Unix. При мысли о них в душе возникал некий трепет.

Заработать на первый PC мне удалось кажется на третьем курсе. Где"=то в это
время, кажется в <<Компьютерной газете>>, пробежала статься с заголовком <<Попробуйте Linux>>. Это был Unix, этого хотелось. Плюс мысль о том, что
можно посмотреть в исходный код настоящего ядра настоящей операционной
системы, вызывала ощущения на грани\ldots.

{\noindent \bf L: Напишем, что мысль вызывала катарсис.}

{\noindent \bf Е:} Хорошо :) Но этого негде было взять (из моего круга общения, ясное дело, который на
тот момент охватывал не очень много людей, приобщённых к IT). Первый диск, привезённый одной компьютерной фирмочкой, нёс на себе две безнадёжно испорченные версии дистрибутива <<Caldera>>, ни одна из них не могла поставиться
по объективным причинам.

{\noindent \bf L: Из"=за неумелой перепаковки?}

Ну как, если правильно подмонтировать распакованный tar.gz
одиного из них как umsdos, то может шансы и были бы, но я тогда я не имел
об этом ни малейшего представления. Я потрогал консоль инсталлятора,
смог даже перенести файл на досовый раздел, испытал\ldots катарсис, понятное
дело, ну и как бы на этом всё.

{\noindent \bf L: А твой первый работающий Linux?}

{\noindent \bf Е:} Первым работающим оказался русский клон Redhat 4.2 "--- он назывался <<Красная шапочка
5.0>>, он умел ставиться, он умел грузиться, на нём собиралось ядро и, если
мне не изменяет память, KDE 1.0 (к тому моменту у меня уже были контакты,
у кого это можно было взять).

Подытоживая, пришёл к открытому ПО я случайно (я о нём ничего не знал) из
желания приобщиться к великому, к Unix, и (за неимением)  других вариантов не искал.

{\noindent \bf L: Несколько слов про твой путь из пользователей свободного ПО в разработчики?}

{\noindent \bf Е:} Ну, вообще контрибуций у меня не очень много. В детстве я был <<хорошим
советским мальчиком>> и очень боялся публичного порицания. Поэтому долгое
время в <<серьёзные>> проекты было лезть страшновато\ldots Что очень зря. С
большего, по отношению к открытым проектам, это прошло только пару лет
назад. А с Debian в период большого желания просто случился неприятный
казус, который затормозил мой путь в Debian Developer.



{\noindent \bf L: У нас в этом году снова тематическое интервью. Поэтому еще группа вопросов, инспирированная отъездом интервьюируемого из Беларуси. Какие различия бросаются в глаза, если сравнивать опенсорс-комьюнити в
СНГ с англоязычным? Европейских и белорусских (и вообще русскоязычных, наверное) опенсорсников?}

{\noindent \bf Е:} Про англоязычные комьюнити особенно говорить бессмысленно, ибо белорусы "--- такие же полноправные участники этого коммунити. Сами и всё видят, и несут вклад в общую атмосферу.

{\noindent \bf L: Ну, речь скорее о локальных комьюнити. }

{\noindent \bf Е:} Локального, про Финляндию могу сказать чуть личного.

{\noindent \bf L: Очень хорошо. }

{\noindent \bf Е:} Я бы отметил, что различные открытые проекты тут занимают видимую часть 
общественной жизни "--- тут вообще модны всевозможные общественные обсуждения и
инициативы). Участие студентов в opensource очень естественно: помню, поразился количеством Debian Developers. Финляндия в топе. 

{\noindent \bf L: Ого! }

{\noindent \bf Е:} Могу ссылку дать\footnote{\url{https://lists.debian.org/debian-project/2013/08/msg00028.html}}. И потом, обсуждать
вопросы с местными ребятами мне лично очень приятно "--- это, как правило, спокойно, по делу, без излишнего давления.

Местные заинтересованные вполне чувствуют себя частью мирового движения, активно участвуют в проектах, конференциях, инициативах по всему миру, устраивают их у себя. Первый мой debconf был в Финляндии, пару раз был на
дебиановских bug\linebreak squashing party, они тоже как правило не совсем локальные. Можно, наверное, сказать, что открытости в сообществе порядочно поболе.

Это очень повлияло и на личную систему <<свой "--- чужой>>. Она слабо коррелирует с границами и языками.

Кроме того, на момент переезда (2006 год), проникновения IT в общественную жизнь было порядочно больше, чем в Беларуси, поэтому и восприятие околоайтишных движений более серьёзно.

В остальном точно так же: лобби коммерческих компаний имеет больший вес. Слышал истории об оспаривании некоторых государственных тендеров, на что банально не хватило денег.

{\noindent \bf L: Еще интересный вопрос "--- твой личный опыт использования СПО в корпоративном секторе. Понятно, всегда есть работодатель\ldots}

{\noindent \bf Е:} Используется активно, там где не противоречит коммерческим интересам. Но, как мы знаем, построить коммерческую систему на базе свободного ПО и поддержки
комьюнити у Нокии не получилось. Вклад она при этом внесла очень порядочный, надо отметить.

В Эриксоне, безусловно, оно тоже используется в разных местах, и даже что"=то выползает наружу. 

{\noindent \bf L: В смысле наработок, которые отдаются сообществу?}

{\noindent \bf Е:} На память сразу приходит TIPC\footnote{\url{http://en.wikipedia.org/wiki/TIPC}} и Eclipse.

Но в целом, отдавать из корпорации назад обычно сопряжено с трудностями. Как правило, это связано с законодательством США и нежеланием рисковать, но иногда и простая жадность.

\end{document}


