\documentclass[10pt, a5paper]{article}
\usepackage{pdfpages}
\usepackage{parallel}
\usepackage[T2A]{fontenc}
\usepackage{ucs}
\usepackage[utf8x]{inputenc}
\usepackage[polish,english,russian]{babel}
\usepackage{hyperref}
\usepackage{rotating}
\usepackage[inner=2cm,top=1.8cm,outer=2cm,bottom=2.3cm,nohead]{geometry}
\usepackage{listings}
\usepackage{graphicx}
\usepackage{wrapfig}
\usepackage{longtable}
\usepackage{indentfirst}
\usepackage{array}
\newcolumntype{P}[1]{>{\raggedright\arraybackslash}p{#1}}
\frenchspacing
\usepackage{fixltx2e} %text sub- and superscripts
\usepackage{icomma} % коскі ў матэматычным рэжыме
\PreloadUnicodePage{4}

\newcommand{\longpage}{\enlargethispage{\baselineskip}}
\newcommand{\shortpage}{\enlargethispage{-\baselineskip}}

\def\switchlang#1{\expandafter\csname switchlang#1\endcsname}
\def\switchlangbe{
\let\saverefname=\refname%
\def\refname{Літаратура}%
\def\figurename{Іл.}%
}
\def\switchlangen{
\let\saverefname=\refname%
\def\refname{References}%
\def\figurename{Fig.}%
}
\def\switchlangru{
\let\saverefname=\refname%
\let\savefigurename=\figurename%
\def\refname{Литература}%
\def\figurename{Рис.}%
}

\hyphenation{admi-ni-stra-tive}
\hyphenation{ex-pe-ri-ence}
\hyphenation{fle-xi-bi-li-ty}
\hyphenation{Py-thon}
\hyphenation{ma-the-ma-ti-cal}
\hyphenation{re-ported}
\hyphenation{imp-le-menta-tions}
\hyphenation{pro-vides}
\hyphenation{en-gi-neering}
\hyphenation{com-pa-ti-bi-li-ty}
\hyphenation{im-pos-sible}
\hyphenation{desk-top}
\hyphenation{elec-tro-nic}
\hyphenation{com-pa-ny}
\hyphenation{de-ve-lop-ment}
\hyphenation{de-ve-loping}
\hyphenation{de-ve-lop}
\hyphenation{da-ta-ba-se}
\hyphenation{plat-forms}
\hyphenation{or-ga-ni-za-tion}
\hyphenation{pro-gramming}
\hyphenation{in-stru-ments}
\hyphenation{Li-nux}
\hyphenation{sour-ce}
\hyphenation{en-vi-ron-ment}
\hyphenation{Te-le-pathy}
\hyphenation{Li-nux-ov-ka}
\hyphenation{Open-BSD}
\hyphenation{Free-BSD}
\hyphenation{men-ti-on-ed}
\hyphenation{app-li-ca-tion}

\def\progref!#1!{\texttt{#1}}
\renewcommand{\arraystretch}{2} %Іначай формулы ў матрыцы зліпаюцца з лініямі
\usepackage{array}

\def\interview #1 (#2), #3, #4, #5\par{

\section[#1, #3, #4]{#1 -- #3, #4}
\def\qname{LVEE}
\def\aname{#1}
\def\q ##1\par{{\noindent \bf \qname: ##1 }\par}
\def\a{{\noindent \bf \aname: } \def\qname{L}\def\aname{#2}}
}

\def\interview* #1 (#2), #3, #4, #5\par{

\section*{#1\\{\small\rm #3, #4. #5}}

\def\qname{LVEE}
\def\aname{#1}
\def\q ##1\par{{\noindent \bf \qname: ##1 }\par}
\def\a{{\noindent \bf \aname: } \def\qname{L}\def\aname{#2}}
}


\begin{document}
\title{Создание специализированного дистрибутива Linux для подключения к национальной GRID-сети}
\author{Денис Пынькин\footnote{Минск, Беларусь}}
\def\progref!#1!{\texttt{#1}}

\maketitle

\begin{abstract}
Unicore is ready"=to"=run system used as main platform for Be\-la\-ru\-sian national GRID network. 
There are no Linux distributions for bare-metal Unicore installation and user"=friendly admi\-ni\-stra\-tion. 
Article describes steps to produce the ALT~Linux based server with Unicore services working out of the box.
\end{abstract}

\section*{Сборка базового дистрибутива}
Современные дистрибутивы ALT Linux собираются с помощью инструмента mkimage,
который состоит из набора правил для утилиты make и вспомогательных скриптов 
и предоставляет базовый функционал для сборки образа Linux-системы. 
Фактически для создания своего собственного дистрибутива достаточно в 
правильном порядке и с правильными параметрами собрать из предоставляемых 
<<кирпичиков>> единое целое, что в общем-то является нетривиальной задачей.

Так было предпринято несколько попыток создать универсальные профили для mkimage,
но наиболее удачным и развитым оказался mkimage-profiles-desktop 
(в рассылках и на форумах часто сокращают до аббревиатуры M-P-D).

Далее, на примере создания дистрибутива Unicore для упрощения подключения к 
белорусской национальной научной GRID-сети, описывается как можно 
использовать M-P-D для создания собственного инсталляционного образа сервера.

Для начала необходимо получить правила сборки дистрибутива из git-репозитория:
\begin{verbatim}
git clone git://git.altlinux.org/people/boyarsh/packages/
 mkimage-profiles-desktop
\end{verbatim}

 Создать файл для пакетного менеджера apt с описанием репозитория и соответствующей 
 архитектурой, который необходимо расположить по пути {\tt \~/\$branch-\$arch.conf}, 
 в случае стабильного репозитория для Platform 6 и 64-битной архитектуры "--- \linebreak 
 {\tt \~/p6-x86\_64.conf}. В качестве минимума в этом файле должна содержаться 
 ссылка на файл с описаниями источников пакетов, для нашего примера "---
 {\tt \~/p6-sources-x86\_64.list}.

Например для белорусского зеркала ALT Linux эти файлы будут выглядеть следующим образом:

{\tt p6-x86\_64.conf}:
\begin{verbatim}
Dir::Etc::SourceList "/home/d4s/p6-sources-x86\_64.list";
\end{verbatim}

{\tt p6-sources-x86\_64.list}:
\begin{verbatim}
rpm [p6] ftp://ftp.mgts.by/pub/ALTLinux/p6/branch x86_64
 classic
rpm [p6] ftp://ftp.mgts.by/pub/ALTLinux/p6/branch noarch
 classic
\end{verbatim}

Для создания установочного диска своего сервера удобнее всего использовать 
профиль от Антона Фарыгина, который можно собрать в качестве теста, 
чтобы проверить работоспособность конфигурации. 
Например, таким образом можно собрать установочный диск для 64-битной 
архитектуры с дизайном Sisyphus:
\begin{verbatim}
archs=x86_64 ./make-distro server-light
 --with-branding=altlinux-sisyphus
\end{verbatim}

На консоль ничего не выводится, но это не страшно "--- <<все ходы записаны>>, 
а ход сборки протоколируется в файлы вида \linebreak <target>.<arch>.log или, в нашем 
случае, в файл \linebreak {\tt server-light.x86\_64.log}.

Получившийся образ диска сохраняется в домашнем каталоге {\tt ~/out/<target>} ({\tt ~/out/server-light/}).

Теперь можно приступать к созданию собственного сервера.

\section*{Модификация базового дистрибутива}
Модификация базового дистрибутива проводится с помощью дополнительных пакетов 
и изменения M-P-D для добавления своих правил сборки дистрибутива и 
списков программного обеспечения, которое будет доступно во время и после установки.

Все дополнительные пакеты условно можно разбить на две категории "--- функциональные и вспомогательные. 
К функциональным относятся пакеты, которые содержат все необходимое для 
полноценного выполнения сервером той задачи, для которой он создается.
Для создания управляющего сервера Unicore, технически готового для включения 
в белорусский национальный GRID, это пакеты, содержащие сами сервисы Unicore: 
gateway, unicorex, xuudb и TSI. 
Кроме того, для полноценного управления вычислительным кластером необходим сервис 
управления пакетными задачами Torque, а по требованию операционного центра 
добавлено программное обеспечение учета ресурсов и система мониторинга Ganglia.

Кроме того разработаны специально для дистрибутива два дополнительных пакета:

\begin{itemize}
	\item unicore-grid-by --- содержит конфигурационные скрипты и \linebreak файлы, специфичные для белорусской GRID-сети.
	\item alterator-unicore-servers --- минимальный интерфейс для \linebreak настройки Unicore, реализованный в виде модуля к системе управления Alterator.
\end{itemize}

Вспомогательные пакеты необходимы только для корректной установки функционала, создания интерфейса либо на подготовительном этапе создания своего установочного диска.

Так можно выделить следующие пакеты:
\begin{itemize}
	\item installer-distro-unicore-server --- этот пакет содержит правила, 
		которые будут выполняться во время установки на различных этапах. 
		Например тут можно указать в каком порядке и какие шаги выполнять 
		программе-установщику, какие сервисы будут запускаться. Сюда же можно 
		добавлять скрипты для более тонкой настройки системы в процессе инсталляции.
	\item installer-feature-vm-unicore-server-stage2 --- этот пакет позволяет 
		подсказать программе-установщику какие разделы и какого размера должны 
		быть созданы жестком диске по умолчанию, а также точки монтирования 
		и предпочтительные файловые системы.
	\item branding --- пакет содержит все графические части, дизайн и, 
		несущественные для сервера, первоначальные пользовательские настройки 
		для оконных сред. Именно здесь можно поменять внешний вид веб"=интерфейса, 
		загрузчика, добавить свои картинки для слайд-шоу во время инсталляции.
\end{itemize}

\end{document}


