\documentclass[10pt, a5paper]{article}
\usepackage{pdfpages}
\usepackage{parallel}
\usepackage[T2A]{fontenc}
\usepackage{ucs}
\usepackage[utf8x]{inputenc}
\usepackage[polish,english,russian]{babel}
\usepackage{hyperref}
\usepackage{rotating}
\usepackage[inner=2cm,top=1.8cm,outer=2cm,bottom=2.3cm,nohead]{geometry}
\usepackage{listings}
\usepackage{graphicx}
\usepackage{wrapfig}
\usepackage{longtable}
\usepackage{indentfirst}
\usepackage{array}
\newcolumntype{P}[1]{>{\raggedright\arraybackslash}p{#1}}
\frenchspacing
\usepackage{fixltx2e} %text sub- and superscripts
\usepackage{icomma} % коскі ў матэматычным рэжыме
\PreloadUnicodePage{4}

\newcommand{\longpage}{\enlargethispage{\baselineskip}}
\newcommand{\shortpage}{\enlargethispage{-\baselineskip}}

\def\switchlang#1{\expandafter\csname switchlang#1\endcsname}
\def\switchlangbe{
\let\saverefname=\refname%
\def\refname{Літаратура}%
\def\figurename{Іл.}%
}
\def\switchlangen{
\let\saverefname=\refname%
\def\refname{References}%
\def\figurename{Fig.}%
}
\def\switchlangru{
\let\saverefname=\refname%
\let\savefigurename=\figurename%
\def\refname{Литература}%
\def\figurename{Рис.}%
}

\hyphenation{admi-ni-stra-tive}
\hyphenation{ex-pe-ri-ence}
\hyphenation{fle-xi-bi-li-ty}
\hyphenation{Py-thon}
\hyphenation{ma-the-ma-ti-cal}
\hyphenation{re-ported}
\hyphenation{imp-le-menta-tions}
\hyphenation{pro-vides}
\hyphenation{en-gi-neering}
\hyphenation{com-pa-ti-bi-li-ty}
\hyphenation{im-pos-sible}
\hyphenation{desk-top}
\hyphenation{elec-tro-nic}
\hyphenation{com-pa-ny}
\hyphenation{de-ve-lop-ment}
\hyphenation{de-ve-loping}
\hyphenation{de-ve-lop}
\hyphenation{da-ta-ba-se}
\hyphenation{plat-forms}
\hyphenation{or-ga-ni-za-tion}
\hyphenation{pro-gramming}
\hyphenation{in-stru-ments}
\hyphenation{Li-nux}
\hyphenation{sour-ce}
\hyphenation{en-vi-ron-ment}
\hyphenation{Te-le-pathy}
\hyphenation{Li-nux-ov-ka}
\hyphenation{Open-BSD}
\hyphenation{Free-BSD}
\hyphenation{men-ti-on-ed}
\hyphenation{app-li-ca-tion}

\def\progref!#1!{\texttt{#1}}
\renewcommand{\arraystretch}{2} %Іначай формулы ў матрыцы зліпаюцца з лініямі
\usepackage{array}

\def\interview #1 (#2), #3, #4, #5\par{

\section[#1, #3, #4]{#1 -- #3, #4}
\def\qname{LVEE}
\def\aname{#1}
\def\q ##1\par{{\noindent \bf \qname: ##1 }\par}
\def\a{{\noindent \bf \aname: } \def\qname{L}\def\aname{#2}}
}

\def\interview* #1 (#2), #3, #4, #5\par{

\section*{#1\\{\small\rm #3, #4. #5}}

\def\qname{LVEE}
\def\aname{#1}
\def\q ##1\par{{\noindent \bf \qname: ##1 }\par}
\def\a{{\noindent \bf \aname: } \def\qname{L}\def\aname{#2}}
}

\begin{document}
\title{Опыт использования СПО в учебном процессе УРТК им. А.С. Попова\footnote{\url{au-mail@ya.ru}, \url{mirex08@ya.ru}, \url{http://lvee.org/ru/abstracts/173}}}
\author{Anton Uymin, Arseny Meshkov, Yekaterinburg, Russia}
\maketitle
\begin{abstract}
The article describes experience of adopting free software in a secondary vocational educational institution - a college. List of profession specialties and specialized classrooms that use free software is presented. Advantages and disadvantages of using free software in the educational process are discussed.
\end{abstract}
В настоящее время в Российском образовании складывается сложная ситуация. С одной стороны, идет активная реформа по преобразованию учебных заведений, их укрупнению и консолидации ресурсов, с другой стороны среднее профессиональное образование и высшее образование уже не дополняют друг друга, как это было ранее, а являются прямыми конкурентами на рынке образовательных услуг. При этом уровень финансирования СПО и ВПО отличается на порядки. Сегодня образовательное учреждение не может выжить, если будет предоставлять парты и доски, необходим качественно новый уровень технического и методического обеспечения. Каждый студент может получить полный объём информации в интернете, но он не может её качественно классифицировать и, зачастую, у него нет хорошей базы оборудования для изучения современных технологий. Поэтому ОУ сейчас не столько должны нести новые знания, сколько предоставлять платформу для образовательной деятельности.

<<СПО для нищебродов!>> "--- с этой фразы одного из представителей сообщества СПО в Екатеринбурге хотелось бы начать данную статью. Уральский радиотехнический колледж им. А.С. Попова находится в Екатеринбурге. У нас реализуется всего 12 специальностей. Из них 6 "--- по IT-направлению. До 2008 года в нашем учебном заведении из СПО были шлюз и сайт на FreeBSD. В учебном процессе СПО не использовалось никак. В 2008 году мы начали сотрудничать с отделом <<К>>, консультировались у них, отправляли к ним на практику студентов. В этом же году начался перевод учебных материалов на Linux.

В колледже 19 компьютерных лабораторий. Парк ПК насчитывает около 500 шт. Парк серверов 20 шт. Обслуживанием лабораторий занимаются заведующие лабораториями, которые являются и преподавателями спец. дисциплин.

Первой лабораторией, переведенной на Linux, стала лаборатория 104 <<Программно-аппаратной защиты объектов сетевой инфраструктуры>>. Выбор дистрибутива, который окажется штатным в лаборатории был долгим и сложным. Мы прошли путь от FreeBSD \textgreater{} PC-BSD \textgreater{} Debian \textgreater{} Fedora \textgreater{} Alt Linux. У Alt Linux подкупила простота начального вхождения, качество документирования и форум, на котором действительно помогают. Мы начали с Alt Linux Школьный Новый Легкий, сейчас работаем на Alt Linux Centaurus P7 и с нетерпением ждем выхода 8 платформы.

В таблице \ref{Uymin1} приведён перечень лабораторий и указан год перехода на СПО. Во всех лабораториях расположено по 12 рабочих мест студентов, кроме 304, где 27 раб. мест.

{\small
\begin{longtable}[h!]{|p{0.7cm}|p{2cm}|p{1cm}|p{2.5cm}|p{2.5cm}|}
\caption{Перечень лабораторий} \label{Uymin1} \\
\hline

\textbf{№} & \textbf{Название лаборатории} & \textbf{Год перехода} & \textbf{Преподаются дисциплины связанные с} & \textbf{На рабочем месте студента} \\
\hline
\endfirsthead % конец заголовка на первой странице

\hline
\textbf{№} (продолжение) & \textbf{Название лаборатории} (продолжение) & \textbf{Год перехода} (продолжение) & \textbf{Преподаются дисциплины связанные с} (продолжение) & \textbf{На рабочем месте студента} (продолжение) \\
\hline
\endhead % заголовок для других страниц 

 101 & Сборки, монтажа и эксплуатации средств вычислительной техники &  2012 & аппаратным обеспечением ПЭВМ. Установка, настройка системного ПО, развертывание служб и сервисов. Техническое обслуживание и ремонт СВТ. & ПЭВМ, принтер, сканер \\ 
\hline 
 104 & Программно-аппаратной защиты объектов сетевой инфраструктуры & 2009 & сетевыми операционными системами, информационной безопасностью. &  2 ПЭВМ, на которых студенты развертывают несколько сетей в системах виртуализации. \\ 
\hline 
 114 & Сетевая академия Cisco. & 2013 & сетевыми технологиями в рамках Cisco CCNA. &  ПЭВМ \\ 
\hline 
 117 & Лаборатория беспроводных технологий & 2014 & сетевыми технологиями: беспроводные сети. & ПЭВМ, с дискретными беспроводными и проводными адаптерами и видеокартой Radeon для анализа защищенности беспроводных соединений. В стенд входят два беспроводных маршрутизатора и ip-камера. \\
\hline 
 304 & Информати~\-ки. Программирования и баз данных &  2015 & информатикой и программированием. & ПЭВМ \\ 
\hline 
 305 & Программно~\-го обеспечения компьютерных сетей. Авторизованный центр Cisco Academy & 2015 & основами компьютерной грамотности. & ПЭВМ \\
\hline 

\end{longtable}
}

Есть участки, на которых нет возможности перейти на СПО и приходится использовать Windows. В связи с тем, что мы обучаем для конкретных предприятий и организаций, то прикладное ПО диктуют они. Используются такие программные пакеты как Altium Designer, Autodesk Autocad, MS Office, 1С и т.д.

Стандартным пакетом дополнительного ПО в лабораториях является: Remmina, Firefox, Geany, iTALC, OpenSSH, PuTTY, \linebreak VirtualBox.

К нестандартному относятся такие пакеты как Cisco Packet Tracer, LinSSID, Aircrack-ng, Wireshark, Metasploit Framework.

Сервисы на базе СПО, используемые для обслуживания образовательного процесса, приведены в таблице \ref{Uymin2}.
{\small
\begin{longtable}[h!]{|p{3cm}|p{3cm}|p{3cm}|}
\caption{Используемые сервисы} \label{Uymin2} \\
\hline

\textbf{Сервис} & \textbf{На чем настроен} & \textbf{Назначение} \\
\hline
\endfirsthead % конец заголовка на первой странице

\hline
\textbf{Сервис} (продолжение) & \textbf{На чем настроен} (продолжение) & \textbf{Назначение} (продолжение) \\
\hline
\endhead % заголовок для других страниц 

    Система видеонаблюдения &  Debian 7 + MotionEye & Обеспечение безопасности функционирования лабораторий, отслеживание инцидентов \\
\hline
    Локальное зеркало репозиториев Debian & Debian 7 + apt-mirror & Проведение лабораторных работ \\
\hline
    Локальное зеркало репозиториев Alt Linux & Alt Linux P7 + sisyphus-mirror & Обслуживание ПЭВМ \\
\hline
    Сервер сетевой загрузки & Debian 7 + tftpd-hpa + isc-dhcp-server + syslinux + nfs-kernel-server + smb + apache2 & Обслуживание ПЭВМ \\
\hline
    Вики-энциклопедия & Alt Linux P7 + Mediawiki & Создание и хранение документации \\
\hline
    Сервер SNMP-мониторинга & Debian 8 + Zabbix & Мониторинг сетевого оборудования \\
\hline
    Система дистанционного обучения & FreeBSD 7 + Moodle & СДО \\
\hline

\end{longtable}
}
Сервер сетевой загрузки позволяет производить загрузку следующих образов ОС – Debian 8, Alt Linux P7, Windows 7/8.1/2012R2, а так же системных и служебных утилит HDT, memtest86+, MS DaRT, MHDD. 
В 2016 году планируется развёртывание \linebreak LDAP-домена на базе Alt Linux с целью централизованного управления правами студентов в лабораториях и обеспечения непрерывной рабочей среды, так же мы начали сотрудничество с компанией РусБИТех, запланирована организация учебных мест на базе Astra Linux SE.

УРТК им. А.С. Попова пытается стать образовательной площадкой, на базе которой  студенты имеют возможность поработать на современном сетевом оборудовании "--- (учебные классы таких вендоров как Cisco, D-Link, TP-Link) с современным программным и аппаратным обеспечением "--- (ПЭВМ i7/16Gb/1Tb для развертывания виртуальных машин – в зависимости от лабораторной работы от 1 до 8 шт. для каждого студента). Кроме этого, мы стараемся держать инфраструктуру колледжа в актуальном состоянии, т.е. систематически добавлять новые интересные сервисы в процесс технического обслуживания образовательного процесса. СПО нам в этом неоценимо помогает, т.к. позволяет активно изучать интересные пакеты и сервисы, оптимально по денежным вложениям и нетребовательно по аппаратным ресурсам, а самое главное, доступно для различных дополнений и модификаций  под конкретные задачи.

\textbf{Вместо вывода:}

\begin{enumerate}
  \item Свободное программное обеспечение позволяет будущим специалистам более глубоко изучать технологии: если студент смог настроить DNS на Linux, то у него не возникнет проблем при аналогичных настройках на Windows или сетевом оборудовании.
  \item Кризис и санкции стимулировали внедрение СПО в СПО. Например, до 2015 года руководство особого интереса не проявляло, а в 2015 году мы выступали перед советом директоров ССУЗов Свердловской области с докладом о практике внедрения свободного программного обеспечения в нашем колледже.
  \item Свободное программное обеспечение сложнее в развёртывании, настройке и обслуживании. Руководство не готово оплачивать ни полноценную техническую поддержку ни доплачивать за обслуживание СПО, что влияет на мотивацию технического персонала.
  \item Так как у Linux отсутствует маркетинговая поддержка, то затруднительно агитировать и продвигать свободные технологии в массы. Например, колледж проводит много различных мероприятий: областные и международные олимпиады. С одной стороны, поддержку данным мероприятиям готовы оказать только люди, интересующиеся СПО в частном порядке. Компании и вендоры не готовы оказывать поддержку. С другой стороны, большое количество учебных заведений испытывают трудности при подготовке студентов к заданиям с использованием CПО.
  \item К сожалению, мы <<нищеброды>>, которые пытаются готовить профессионалов, в чём СПО нам активно помогает.
\end{enumerate}

\end{document}
