\documentclass[10pt, a5paper]{article}
\usepackage{pdfpages}
\usepackage{parallel}
\usepackage[T2A]{fontenc}
\usepackage{ucs}
\usepackage[utf8x]{inputenc}
\usepackage[polish,english,russian]{babel}
\usepackage{hyperref}
\usepackage{rotating}
\usepackage[inner=2cm,top=1.8cm,outer=2cm,bottom=2.3cm,nohead]{geometry}
\usepackage{listings}
\usepackage{graphicx}
\usepackage{wrapfig}
\usepackage{longtable}
\usepackage{indentfirst}
\usepackage{array}
\newcolumntype{P}[1]{>{\raggedright\arraybackslash}p{#1}}
\frenchspacing
\usepackage{fixltx2e} %text sub- and superscripts
\usepackage{icomma} % коскі ў матэматычным рэжыме
\PreloadUnicodePage{4}

\newcommand{\longpage}{\enlargethispage{\baselineskip}}
\newcommand{\shortpage}{\enlargethispage{-\baselineskip}}

\def\switchlang#1{\expandafter\csname switchlang#1\endcsname}
\def\switchlangbe{
\let\saverefname=\refname%
\def\refname{Літаратура}%
\def\figurename{Іл.}%
}
\def\switchlangen{
\let\saverefname=\refname%
\def\refname{References}%
\def\figurename{Fig.}%
}
\def\switchlangru{
\let\saverefname=\refname%
\let\savefigurename=\figurename%
\def\refname{Литература}%
\def\figurename{Рис.}%
}

\hyphenation{admi-ni-stra-tive}
\hyphenation{ex-pe-ri-ence}
\hyphenation{fle-xi-bi-li-ty}
\hyphenation{Py-thon}
\hyphenation{ma-the-ma-ti-cal}
\hyphenation{re-ported}
\hyphenation{imp-le-menta-tions}
\hyphenation{pro-vides}
\hyphenation{en-gi-neering}
\hyphenation{com-pa-ti-bi-li-ty}
\hyphenation{im-pos-sible}
\hyphenation{desk-top}
\hyphenation{elec-tro-nic}
\hyphenation{com-pa-ny}
\hyphenation{de-ve-lop-ment}
\hyphenation{de-ve-loping}
\hyphenation{de-ve-lop}
\hyphenation{da-ta-ba-se}
\hyphenation{plat-forms}
\hyphenation{or-ga-ni-za-tion}
\hyphenation{pro-gramming}
\hyphenation{in-stru-ments}
\hyphenation{Li-nux}
\hyphenation{sour-ce}
\hyphenation{en-vi-ron-ment}
\hyphenation{Te-le-pathy}
\hyphenation{Li-nux-ov-ka}
\hyphenation{Open-BSD}
\hyphenation{Free-BSD}
\hyphenation{men-ti-on-ed}
\hyphenation{app-li-ca-tion}

\def\progref!#1!{\texttt{#1}}
\renewcommand{\arraystretch}{2} %Іначай формулы ў матрыцы зліпаюцца з лініямі
\usepackage{array}

\def\interview #1 (#2), #3, #4, #5\par{

\section[#1, #3, #4]{#1 -- #3, #4}
\def\qname{LVEE}
\def\aname{#1}
\def\q ##1\par{{\noindent \bf \qname: ##1 }\par}
\def\a{{\noindent \bf \aname: } \def\qname{L}\def\aname{#2}}
}

\def\interview* #1 (#2), #3, #4, #5\par{

\section*{#1\\{\small\rm #3, #4. #5}}

\def\qname{LVEE}
\def\aname{#1}
\def\q ##1\par{{\noindent \bf \qname: ##1 }\par}
\def\a{{\noindent \bf \aname: } \def\qname{L}\def\aname{#2}}
}


\begin{document}

\title{On Digital Monies}

\author{Daniel Nagy\footnote{Budapest, Hungary, ePoint Systems Ltd., \url{nagydani@epointsystem.org}}}
\date{}
\maketitle
\renewcommand{\abstractname}{Abstract}
\begin{abstract}
Payment over digital communications networks has been a hot topic since
the introduction of the telegraph, the first such elec\-tro\-nic network;
originally, Western Union was a telegraph com\-pa\-ny. Money, however, is
more than a means of payment: it is also an instrument of saving and
credit, a measure of value and a unit of accounting. In my presentation,
I shall explore various attempts at creating digital money, both
proprietary and open-source, theoretical and practical.
\end{abstract}

The main difference between digitally represented information and
everything else that has been historically used as money is that unlike
the latter, digital information can be copied at no cost, with perfect
fidelity in essentially unlimited quantities. Very undesirable
properties for something that we intend to use as money. This is the
main technical challenge in using digital information as money (known as
`double spending'). How can we make sure that no more gets spent than
what has been earned?

There is, however, a different, no less interesting challenge that is
not purely technical: how to get actual people provide real value in
exchange for pieces of digital information. Why would people use pieces
of digital information as a means of exchange or to keep their savings
or to measure value?

The third challenge is a legal one. Since the first ever gold coins have
been minted at the behest of Croesus, King of Lydia in the sixth
century, B.C., money has been subject of government regulation and very
often exclusive monopoly. How can digital money exist within the current
legal framework?

Different solutions to these challenges have been proposed. We shall
explore the solutions to these challenges in the following systems in
some technical detail:

\subsubsection*{DigiCash}
David Chaum's company founded in 1990 on the basis of some clever
cryptography "--- blind digital signatures "--- he published in 1988. The
company declared bankruptcy in 1998, but in many ways it provided
inspiration to the many ventures that followed. I brief and simplified
introduction to the technology will be given as well as what I think
were the reasons for failure.

\subsubsection*{WebMoney}
Artiom Genkin went the other route: his doctoral thesis on the topic
followed business success. Founded in 1998, WebMoney made its name by
proving to be more resilient in the face of financial meltdown than the
Russian banking sector. To this day, WebMoney is one of the most
sophisticated and successful digital money, albeit mostly in the
Russian-speaking world.

\subsubsection*{BitCoin}
This is an open-source project started in 2009 by the mysterious Satoshi
Nakatomo, based on earlier research either done or popularized by Nick
Szabo (a former DigiCash employee). This year, BitCoin's popularity (and
valuation) exploded, making it one of the most im\-portant and interesting
innovations in the field.

\subsubsection*{ePoint}
This is my own open-source project, kicked off in 2005. However, I am
not going to talk much about how it works now; rather, I shall outline
where I would like to take it and what needs to be done to get there.
Maybe, like-minded developers of open-source software among LVEE 2011
participants will join me in creating something valuable and useful.


\end{document}




