\documentclass[10pt, a5paper]{article}
\usepackage[T2A]{fontenc}
\usepackage{ucs}
\usepackage[utf8x]{inputenc}
\usepackage[polish,english,russian]{babel}
\usepackage{hyperref}
\usepackage[inner=2cm,top=1.8cm,outer=2cm,bottom=2.3cm,nohead]{geometry}
\usepackage{listings}
\usepackage{graphicx}
\usepackage{wrapfig}
\usepackage{longtable}
\usepackage{indentfirst}
\frenchspacing
\usepackage{fixltx2e} %text sub- and superscripts
\usepackage{icomma} % коскі ў матэматычным рэжыме
\PreloadUnicodePage{4}

\newcommand{\longpage}{\enlargethispage{\baselineskip}}
\newcommand{\shortpage}{\enlargethispage{-\baselineskip}}

\def\switchlang#1{\expandafter\csname switchlang#1\endcsname}
\def\switchlangbe{
\let\saverefname=\refname%
\def\refname{Літаратура}%
\def\figurename{Іл.}%
}
\def\switchlangen{
\let\saverefname=\refname%
\def\refname{References}%
\def\figurename{Fig.}%
}
\def\switchlangru{
\let\saverefname=\refname%
\let\savefigurename=\figurename%
\def\refname{Литература}%
\def\figurename{Рис.}%
}

\hyphenation{admi-ni-stra-tive}
\hyphenation{ex-pe-ri-ence}
\hyphenation{fle-xi-bi-li-ty}
\hyphenation{Py-thon}
\hyphenation{ma-the-ma-ti-cal}
\hyphenation{re-ported}
\hyphenation{imp-le-menta-tions}
\hyphenation{pro-vides}
\hyphenation{en-gi-neering}
\hyphenation{com-pa-ti-bi-li-ty}
\hyphenation{im-pos-sible}
\hyphenation{desk-top}
\hyphenation{elec-tro-nic}
\hyphenation{com-pa-ny}
\hyphenation{de-ve-lop-ment}
\hyphenation{de-ve-loping}
\hyphenation{de-ve-lop}
\hyphenation{da-ta-ba-se}
\hyphenation{plat-forms}
\hyphenation{or-ga-ni-za-tion}
\hyphenation{pro-gramming}
\hyphenation{in-stru-ments}
\hyphenation{Li-nux}
\hyphenation{en-vi-ron-ment}
\hyphenation{Te-le-pathy}
\hyphenation{Li-nux-ov-ka}

\def\progref!#1!{\texttt{#1}}
\renewcommand{\arraystretch}{2} %Іначай формулы ў матрыцы зліпаюцца з лініямі
\usepackage{array}

\def\interview #1 (#2), #3, #4, #5\par{

\section[#1, #3, #4]{#1, #5}
\def\qname{LVEE}
\def\aname{#1}
\def\q ##1\par{{\noindent \bf \qname: ##1 }\par}
\def\a{{\noindent \bf \aname: } \def\qname{L}\def\aname{#2}}
}

\begin{document}
\title{О Модели Данных SQL/JSON}
\author{Андреев Юрий, Санкт-Петербург, РФ\footnote{\url{andreev.yurij@gmail.com}}}
\maketitle
\begin{abstract}
An SQL/JSON model is a part of SQL 2016 Standard\cite{AY1} which describes a JSON data model for SQL.
It describes a \textit{jsonpath} data type as a path language and nine functions to deal with JSON data.
A new type of queries will be shown on a special branch of PostgreSQL.
\end{abstract}

\subsection*{Введение}
В новою редакцию стандарта SQL\cite{AY1} была включена SQL/JSON модель.
Это означает, что стандарт теперь предполагает работу со слабо-структурированными
данными в JSON формате\cite{AY2}. СУБД PostgreSQL имеет долгую историю поддержки такого типа данных
и теперь идет работа по поддержке новых возможностей \linebreak стандарта\cite{AY4}.
Далее мы покажем, какие преимущества дает способ хранения в JSON формате и как 
предполагается делать запросы к таким данным.

\subsection*{Описание}
Для запросов в стандарте описаны девять новых функций, с префиксом 
\texttt{JSON\_ (JSON\_\{OBJECT, QUERY, ...\})},
а также \textit{jsonpath} --- язык для навигации по JSON. Это частичный аналог XPath для XML. 
Однако, в отличии от последнего, его синтаксис был продиктован уже имеющимися конструкциями в javascript. 
Выражение начинается со знака доллара, переход к дочерним элементам осуществляется оператором 'точка'.
(см. подробнее в \cite{AY3}). Путь передается во втором аргументе функций. 

\subsection*{Пример использования}
Хорошей иллюстрацией может служить список достопримечательностей. 
Допустим, мы хотели бы хранить дополнительную информацию об интересных местах. 
Если мы возьмем архитектурную достопримечательность, 
то в формате JSON информация могла бы быть записана следующим образом:
\begin{verbatim}
{
    "nearest_city": "city_name"
    "creator": "author_name"
    "style": "style_name"
    "date": {
        "start_centery": number
        "end_centery": number
    }
}
\end{verbatim}
В тоже время, для природных объектов бессмысленно указывать имя создателя или дату,
и могут понадобиться новые ключи: сведения о территории, высоте, глубине. 
С другой стороны, и те и другие могут содержать имя ближайшего населенного пункта, 
географическое расположение и прочее. 

Записать информацию в базу мы можем с помощью функций-конструкторов
\texttt{JSON\_OBJECT, JSON\_ARRAY}.

Если наши данные содержатся в колонке \texttt{details}, то запрос на наличие создателя (ключ \texttt{creator})
может выглядеть следующим образом:
\begin{verbatim}
SELECT 
    JSON_EXISTS(details, '$.creator') 
FROM table_name 
WHERE id = object_id;
\end{verbatim}
И вернет булево значение (\texttt{t/f}). 
Вывести время начала строительства можно функцией \texttt{JSON\_VALUE}:
\begin{verbatim}
SELECT 
    JSON_VALUE(details, '$.date.start_centery') 
FROM table_name 
WHERE id = object_id;
\end{verbatim}
Тут заметим, что вообще период строительства может описываться по разному, 
в зависимости от точности имеющейся информации. 
Поэтому могут понадобиться разные ключи и, соответственно, разная логика в запросе. 

\subsection*{Заключение}
Преимущество JSON/SQL модели, помимо прочего, заключается в том, 
что мы можем эффективно работать с данными, не имеющими четкой структуры. 
Компактно хранить и лаконично писать запросы к таким данным, 
добавлять новые ключи, без необходимости изменения схемы таблицы (что затратно). 

\begin{thebibliography}{9}
\bibitem{AY1} Стандарт SQL, ISO/IEC 9075-2:2016, \url{https://www.iso.org/standard/63556.html}
\bibitem{AY2} Стандарт JSON, RFC 7159, \url{https://www.rfc-editor.org/info/rfc7159}, 
\bibitem{AY3} Описание jsonpath, \url{http://goessner.net/articles/JsonPath/}
\bibitem{AY4} Исходный код PostgreSQL, sqljson branch, \url{https://github.com/postgrespro/sqljson/tree/sqljson/}
\end{thebibliography}

\end{document}
