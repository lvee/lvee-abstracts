\documentclass[10pt, a5paper]{article}
\usepackage{pdfpages}
\usepackage{parallel}
\usepackage[T2A]{fontenc}
\usepackage{ucs}
\usepackage[utf8x]{inputenc}
\usepackage[polish,english,russian]{babel}
\usepackage{hyperref}
\usepackage{rotating}
\usepackage[inner=2cm,top=1.8cm,outer=2cm,bottom=2.3cm,nohead]{geometry}
\usepackage{listings}
\usepackage{graphicx}
\usepackage{wrapfig}
\usepackage{longtable}
\usepackage{indentfirst}
\usepackage{array}
\newcolumntype{P}[1]{>{\raggedright\arraybackslash}p{#1}}
\frenchspacing
\usepackage{fixltx2e} %text sub- and superscripts
\usepackage{icomma} % коскі ў матэматычным рэжыме
\PreloadUnicodePage{4}

\newcommand{\longpage}{\enlargethispage{\baselineskip}}
\newcommand{\shortpage}{\enlargethispage{-\baselineskip}}

\def\switchlang#1{\expandafter\csname switchlang#1\endcsname}
\def\switchlangbe{
\let\saverefname=\refname%
\def\refname{Літаратура}%
\def\figurename{Іл.}%
}
\def\switchlangen{
\let\saverefname=\refname%
\def\refname{References}%
\def\figurename{Fig.}%
}
\def\switchlangru{
\let\saverefname=\refname%
\let\savefigurename=\figurename%
\def\refname{Литература}%
\def\figurename{Рис.}%
}

\hyphenation{admi-ni-stra-tive}
\hyphenation{ex-pe-ri-ence}
\hyphenation{fle-xi-bi-li-ty}
\hyphenation{Py-thon}
\hyphenation{ma-the-ma-ti-cal}
\hyphenation{re-ported}
\hyphenation{imp-le-menta-tions}
\hyphenation{pro-vides}
\hyphenation{en-gi-neering}
\hyphenation{com-pa-ti-bi-li-ty}
\hyphenation{im-pos-sible}
\hyphenation{desk-top}
\hyphenation{elec-tro-nic}
\hyphenation{com-pa-ny}
\hyphenation{de-ve-lop-ment}
\hyphenation{de-ve-loping}
\hyphenation{de-ve-lop}
\hyphenation{da-ta-ba-se}
\hyphenation{plat-forms}
\hyphenation{or-ga-ni-za-tion}
\hyphenation{pro-gramming}
\hyphenation{in-stru-ments}
\hyphenation{Li-nux}
\hyphenation{sour-ce}
\hyphenation{en-vi-ron-ment}
\hyphenation{Te-le-pathy}
\hyphenation{Li-nux-ov-ka}
\hyphenation{Open-BSD}
\hyphenation{Free-BSD}
\hyphenation{men-ti-on-ed}
\hyphenation{app-li-ca-tion}

\def\progref!#1!{\texttt{#1}}
\renewcommand{\arraystretch}{2} %Іначай формулы ў матрыцы зліпаюцца з лініямі
\usepackage{array}

\def\interview #1 (#2), #3, #4, #5\par{

\section[#1, #3, #4]{#1 -- #3, #4}
\def\qname{LVEE}
\def\aname{#1}
\def\q ##1\par{{\noindent \bf \qname: ##1 }\par}
\def\a{{\noindent \bf \aname: } \def\qname{L}\def\aname{#2}}
}

\def\interview* #1 (#2), #3, #4, #5\par{

\section*{#1\\{\small\rm #3, #4. #5}}

\def\qname{LVEE}
\def\aname{#1}
\def\q ##1\par{{\noindent \bf \qname: ##1 }\par}
\def\a{{\noindent \bf \aname: } \def\qname{L}\def\aname{#2}}
}

\begin{document}
\title{OSTree~--- атомарные обновления ОС в стиле git\footnote{\url{d4s@t-linux.by}, \url{https://lvee.org/en/abstracts/289}}}
\author{Denis Pynkin, Minsk, Belarus}
\maketitle
\begin{abstract}
LibOSTree (aka "OSTree") project is aimed to create git-like bootable filesystem trees.
This is shared library and set of utilities to manage content-addressed-object store
and local "checkouts" of filesystem trees allowing transactional upgrades and rollbacks of the system.
\end{abstract}

\section*{Что такое OSTree}

OSTree\footnote{\url{https://ostree.readthedocs.io}} предназначен для создания неизменяемых файловых систем в стиле git, обеспечивая атомарные обновления и откаты ОС. Проект предоставляет собой референсную библиотеку для работы с репозиториями <<ostree>>, а также утилиту командной строки, позволяющей производить все необходимые операции над локальным репозиторием.

В локальном <<ostree>> репозитории хранятся <<снимки>> файлов и директорий, позволяя быстро переключать корневую систему, ядро и конфигурацию загрузчика на любой из доступных вариантов.

Серверный репозиторий предназначен для использования в качестве источника обновлений поверх HTTP/HTTPS протокола. Возможность использования GPG-подписи отдельных коммитов, а также внутренняя архитектура репозитория, позволяют тиражировать серверный репозиторий и использовать его для обновлений даже из недоверенных источников.

\section*{Локальная архитектура OSTree}

OSTree предназначен для работы поверх любой POSIX-\linebreak совместимой файловой системы.

В репозитории <</ostree/repo>> хранятся файловые объекты, а также ссылки на них~--- в этом репозиторий <<ostree>> очень похож на репозиторий <<git>>. Имена всех объектов в системе представляют собой sha256 хэш от содержимого, таким образом обеспечивается автоматическая дедупликация данных на уровне репозитория.

В репозитории присутствуют 3 базовых типа файлов:

\begin{itemize}
  \item <sha256>.commit~--- описание коммита, а также <<имя>> корневой директории
  \item <sha256>.dirtree~--- список файловых объектов в директории
  \item <sha256>.file~--- файл
\end{itemize}

В отличие от git, при разворачивании (checkout, deploy) все файлы создаются в виде жёсткой ссылки на изначальный файл, находящийся в репозитории, что накладывает серьёзное ограничение~--- репозиторий и развёрнутый корень системы обязаны находится на одной ФС.

Директорий <</ostree/deploy>> хранит развернутые корневые файловые системы в поддиректории, соответствующей имени операционной системы. Да, OSTree позволяет устанавливать и переключаться между несколькими разными, не связанными между собой ОС!

В концепции OSTree предполагается, что только 2 системные директории <</etc>> и <</var>> остаются в режиме записи, причем <</etc>> копируется (3-way merge) при переключении, а <</var>> --- является общей в рамках одной ОС.

Таких развернутых версий разных ОС (stateroot) одновременно может быть несколько.

\section*{Загрузка системы}

При разворачивании корня ОС, как правило копируется ядро и initrd, поставляемые с этой версией, а также создается новая конфигурация загрузчика. До того момента, пока конфигурация загрузчика не <<переключится>> с помощью атомарной операции создания символической ссылки на обновленную конфигурацию, загрузка системы будет осуществляться в текущую версию ОС. Поэтому при обновлении системы данная операция осуществляется в последнюю очередь. В любой момент времени ОС будет в состоянии загрузиться либо в <<текущую>> версию, либо в <<обновленную>>.

Загрузка ОС, адаптированной для OSTree предполагает, что решение, какая ОС и её версия будут загружаться, принимается на этапе работы <<минисистемы>> в initrd, поскольку для создания загрузочной корневой системы необходимо <<правильно>> подмонтировать все её части. Из-за этого невозможно применять OSTree без initrd.

Версия ОС для загрузки, как правило передается с помощью опции командной строки для ядра <<ostree=>>, содржащей полный путь к развернутому корню операционной системы, например:

\begin{verbatim}
`ostree=/ostree/boot.1/apertis/0082844f3f7223ea5093f6c25
0f50a35c83a5bfe2e96799bc94c3e3be95a60a0/0`\end{verbatim}
Данный параметр генерируется, как часть конфигурации загрузчика на этапе разворачивания корневой ФС.

В отличие от <<классических>> ОС, которым достаточно правильно инициализировать блочное устройство, для OSTree необходимо дополнительно подмонтировать все корневые директории перед вызовом <<pivot\_root>>. Референсный пример <<switchroot.sh>> доступен в git-репозитории OSTree\footnote{\url{https://github.com/ostreedev/ostree/tree/master/src/switchroot}}.

\section*{Использование в Apertis}

Для адаптации ОС, основанной на Debian, к работе с OSTree потребовалось:

\begin{itemize}
  \item доработать OSTree для корректного взаимодействия с загрузчиком <<u-boot>>;
  \item обеспечить загрузку и работу ОС в режиме R/O для корневой файловой системы;
  \item фактически создать собственную сборочную систему на базе Debos;
  \item переписать тесты, требующие установки дополнительных пакетов.
\end{itemize}

Кроме того для внутренних нужд был разработан шаблон\footnote{\url{https://gitlab.apertis.org/infrastructure/apertis-image-recipes/tree/master/lxc}} для работы системы на основе OSTree в качестве контейнера LXC, что позволяет быстро и безопасно эксперементировать с технологией.

\end{document}
