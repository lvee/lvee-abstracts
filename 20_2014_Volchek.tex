\documentclass[10pt, a5paper]{article}
\usepackage{pdfpages}
\usepackage{parallel}
\usepackage[T2A]{fontenc}
\usepackage{ucs}
\usepackage[utf8x]{inputenc}
\usepackage[polish,english,russian]{babel}
\usepackage{hyperref}
\usepackage{rotating}
\usepackage[inner=2cm,top=1.8cm,outer=2cm,bottom=2.3cm,nohead]{geometry}
\usepackage{listings}
\usepackage{graphicx}
\usepackage{wrapfig}
\usepackage{longtable}
\usepackage{indentfirst}
\usepackage{array}
\newcolumntype{P}[1]{>{\raggedright\arraybackslash}p{#1}}
\frenchspacing
\usepackage{fixltx2e} %text sub- and superscripts
\usepackage{icomma} % коскі ў матэматычным рэжыме
\PreloadUnicodePage{4}

\newcommand{\longpage}{\enlargethispage{\baselineskip}}
\newcommand{\shortpage}{\enlargethispage{-\baselineskip}}

\def\switchlang#1{\expandafter\csname switchlang#1\endcsname}
\def\switchlangbe{
\let\saverefname=\refname%
\def\refname{Літаратура}%
\def\figurename{Іл.}%
}
\def\switchlangen{
\let\saverefname=\refname%
\def\refname{References}%
\def\figurename{Fig.}%
}
\def\switchlangru{
\let\saverefname=\refname%
\let\savefigurename=\figurename%
\def\refname{Литература}%
\def\figurename{Рис.}%
}

\hyphenation{admi-ni-stra-tive}
\hyphenation{ex-pe-ri-ence}
\hyphenation{fle-xi-bi-li-ty}
\hyphenation{Py-thon}
\hyphenation{ma-the-ma-ti-cal}
\hyphenation{re-ported}
\hyphenation{imp-le-menta-tions}
\hyphenation{pro-vides}
\hyphenation{en-gi-neering}
\hyphenation{com-pa-ti-bi-li-ty}
\hyphenation{im-pos-sible}
\hyphenation{desk-top}
\hyphenation{elec-tro-nic}
\hyphenation{com-pa-ny}
\hyphenation{de-ve-lop-ment}
\hyphenation{de-ve-loping}
\hyphenation{de-ve-lop}
\hyphenation{da-ta-ba-se}
\hyphenation{plat-forms}
\hyphenation{or-ga-ni-za-tion}
\hyphenation{pro-gramming}
\hyphenation{in-stru-ments}
\hyphenation{Li-nux}
\hyphenation{sour-ce}
\hyphenation{en-vi-ron-ment}
\hyphenation{Te-le-pathy}
\hyphenation{Li-nux-ov-ka}
\hyphenation{Open-BSD}
\hyphenation{Free-BSD}
\hyphenation{men-ti-on-ed}
\hyphenation{app-li-ca-tion}

\def\progref!#1!{\texttt{#1}}
\renewcommand{\arraystretch}{2} %Іначай формулы ў матрыцы зліпаюцца з лініямі
\usepackage{array}

\def\interview #1 (#2), #3, #4, #5\par{

\section[#1, #3, #4]{#1 -- #3, #4}
\def\qname{LVEE}
\def\aname{#1}
\def\q ##1\par{{\noindent \bf \qname: ##1 }\par}
\def\a{{\noindent \bf \aname: } \def\qname{L}\def\aname{#2}}
}

\def\interview* #1 (#2), #3, #4, #5\par{

\section*{#1\\{\small\rm #3, #4. #5}}

\def\qname{LVEE}
\def\aname{#1}
\def\q ##1\par{{\noindent \bf \qname: ##1 }\par}
\def\a{{\noindent \bf \aname: } \def\qname{L}\def\aname{#2}}
}

\begin{document}
\title{Что подтолкнуло белорусских пиратов создать филиал Creative Commons?}
\author{Михаил Волчек, Минск, Беларусь\footnote{\url{fannrm@gmail.com}, \url{http://lvee.org/en/abstracts/137}}}
\maketitle
\begin{abstract}
The initiative of Belarussian priates to open the Creative Com\-mons affiliate in Belarus is presented. Covered topics include the analysis of Belarussian copyright law and its incompartibilities with most of open source licenses, as well as proposed steps to initiate its improvement.
\end{abstract}
\subsection*{Введение}

Часто можно услышать фразу «Dura lex sed lex» "--- «Закон суров, но закон». Это древнеримское выражение предлагает нам подчиниться принуждению закона, который может быть неоправданно суров. Но те, кто знакомился с трактатом Цицерона «О законах», встречались с другой не менее интересной парадигмой римского права: «Salus populi suprema lex esto». Это значит, «благо народа "--- высший закон». В своём докладе автор видит необходимость рассказать, почему современный белорусский закон «Об авторском праве и смежных правах» (Закон «Об авторском праве», ЗоАП, копирайт) слишком «суров», почему он не действует на «salus populi», и каковы намерения белорусских пиратов в связи с созданием филиала Творческих Общин (Creative Commons).

\subsection*{Запретительная природа копирайта}

Копировать, изменять и распространять контент, программы запрещено, если это явно не разрешено правообладателем. Статья 16 ЗоАП гласит, что \emph{«…автору или иному правообладателю принадлежит право разрешать или запрещать другим лицам использовать произведение»}, согласно статье 16 пункту 2 "--- \emph{«переработку произведения для создания производного произведения»}. Теперь посмотрим, что такое «производное произведение». \emph{«Производное произведение "--- перевод или иная переработка произведения, являющиеся результатом творческого труда, в том числе обработка, обзор, пересказ, аннотация, резюме, реферат, инсценировка, музыкальная аранжировка»} (ст. 4 ЗоАП) по сути устанавливает запретительные ограничения на множество форм творчества.

\subsubsection*{Замедление культурного и интеллектуального обмена}

Страх автора опубликовать какие"=то данные в образовательных целях создаёт атмосферу, когда публичная циркуляция знаний перестаёт быть безопасной. Несколько примеров из белорусской Википедии. Например, автор статьи сомневается можно ли разместить автограф в статье \cite{Volchak1}. Или знаменитый пример, когда из материала удалялась фотография Национальной Библиотеки, т.к. «произведения архитектуры, градостроительства и садово"=паркового искусства» согласно Статья 6 ЗоАП тоже являются объектами авторского права. 

\subsubsection*{Экономика рантье}

Копирайт даёт монополию. Например, в Беларуси срок этой монополии составляет жизнь автора плюс 50 лет. Согласно статье 20 ЗоАП «Исключительное право на произведение действует в течение жизни автора и пятидесяти лет после его смерти\ldots{}»

На протяжении этого времени правообладатель не производит ничего нового, просто собирает прибыль из когда"=то созданного произведения. Живёт на проценты. Такой подход стимулирует создание перекупщиков «авторских прав», другими словами посредников.

В мире широко известны монополистическим беспределом организации по коллективному управлению авторскими правами \linebreak(ОКУП), которые пропускают через себя гигантские суммы денег, а также являются лоббистами крупных корпораций"=держателей целых портфелей «авторских прав» --- такие, как IFPI, RIAA, MPAA, BSA. В Беларуси функциями ОКУП занимается Национальный Центр Интеллектуальной Собственности, которому в принципе не чужды многие проблемы подобных организаций (известен пример 2013 года с ресурсом webmusic.by).

\subsubsection*{Массивный и непонятный большинству создателей контента}

Сферу авторского права регулируют различные документы. \linebreak Статья 3 ЗоАП отсылает читателя (это может быть, например, художник, или дизайнер) прочитать ещё пару (а их десятки) международных договоров: «если международным договором Республики Беларусь установлены иные правила, чем те, которые содержатся в настоящем Законе, то применяются правила международного договора». Расплывчатые и оценочные формулировки (например, «объективная форма» (ст.1), «оправданный объёмом, целью цитирования, информационной целью» (ст. 32, 33)) ЗоАП запутывают не только обывателя.

\subsubsection*{Устаревший и вредный}

На сегодня ЗоАП не учитывает свойство Интернет, его динамичность, возможности быстрого обмена контентом и выстраивает различные преграды. Чего, например, стоит ограничение белорусского законодательства заключать лицензионные договоры в отличной от письменной формы виде. За бортом закона остаётся набор таких публичных лицензий как Creative Commons (для контента), GNU GPL, BSD, Apache, Mozilla, MIT, CPAL (для софта).

\subsubsection*{Последняя капля}

«Копирайт» "--- один из самых неэффективных законов. По оценкам Microsoft, 87\% белорусов являлись пиратами в 2013 году \cite{Volchak2}. 9 из 10 нарушают этот закон. Какие могут быть причины? Закон не соответствует интересам общества? Незнание закона? Финансовая невозможность приобретать дорогие лицензии? Возможность получать гораздо более дешёвые «контрафактные» программы? Неуважение к автору или правообладателю? Разобраться в этом призван филиал Творческих Общин.

\subsection*{Филиал Творческих Общин}

В конце 2013 года активистами молодёжной организации Фаланстер началась подготовка к официальному открытию филиала (мероприятие пройдёт 29 августа 2014). На сегодня составлена дорожная карта \cite{Volchak3}, а также запущен проект на площадке Талака \cite{Volchak4}. Можно выделить основные три направления на этот период.

\textbf{Во"=первых}, это комплексное исследование белорусского закона «Об авторском праве» на предмет возможности имплементации публичного лицензирования для обычных и сетевых произведений.

\textbf{Во"=вторых}, это опубликование и открытое обсуждение данных исследования для всех заинтересованных сообществ.

\textbf{И в"=третьих}, это создание обновленного закона об авторском праве методом краудсорсинга, а также сбор 1000 подписей в поддержку этих правок через инструменты электронного участия "--- петицию.

Помимо этого будут проходить лекции, дискуссии по проблемам копирайта, консультации и информационная поддержка проектов использующих свободные лицензии.

Эти шаги с одной стороны наполняют созидательную деятельность пиратов, а с другой "--- делают возможным через присутствие филиала Сreative Сommons дополнительно вовлечь белорусское общество в мировой цифровой контекст.

\begin{thebibliography}{9}
\bibitem{Volchak1} \url{http://bit.ly/1sk1bZt}
\bibitem{Volchak2} \url{http://bit.ly/1sk7uMP}
\bibitem{Volchak3} \url{http://wiki.creativecommons.org/Belarus}
\bibitem{Volchak4} \url{http://www.talaka.by/projects/521}
\end{thebibliography}

\end{document}
