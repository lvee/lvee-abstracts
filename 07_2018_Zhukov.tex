\documentclass[10pt, a5paper]{article}
\usepackage{pdfpages}
\usepackage{parallel}
\usepackage[T2A]{fontenc}
\usepackage{ucs}
\usepackage[utf8x]{inputenc}
\usepackage[polish,english,russian]{babel}
\usepackage{hyperref}
\usepackage{rotating}
\usepackage[inner=2cm,top=1.8cm,outer=2cm,bottom=2.3cm,nohead]{geometry}
\usepackage{listings}
\usepackage{graphicx}
\usepackage{wrapfig}
\usepackage{longtable}
\usepackage{indentfirst}
\usepackage{array}
\newcolumntype{P}[1]{>{\raggedright\arraybackslash}p{#1}}
\frenchspacing
\usepackage{fixltx2e} %text sub- and superscripts
\usepackage{icomma} % коскі ў матэматычным рэжыме
\PreloadUnicodePage{4}

\newcommand{\longpage}{\enlargethispage{\baselineskip}}
\newcommand{\shortpage}{\enlargethispage{-\baselineskip}}

\def\switchlang#1{\expandafter\csname switchlang#1\endcsname}
\def\switchlangbe{
\let\saverefname=\refname%
\def\refname{Літаратура}%
\def\figurename{Іл.}%
}
\def\switchlangen{
\let\saverefname=\refname%
\def\refname{References}%
\def\figurename{Fig.}%
}
\def\switchlangru{
\let\saverefname=\refname%
\let\savefigurename=\figurename%
\def\refname{Литература}%
\def\figurename{Рис.}%
}

\hyphenation{admi-ni-stra-tive}
\hyphenation{ex-pe-ri-ence}
\hyphenation{fle-xi-bi-li-ty}
\hyphenation{Py-thon}
\hyphenation{ma-the-ma-ti-cal}
\hyphenation{re-ported}
\hyphenation{imp-le-menta-tions}
\hyphenation{pro-vides}
\hyphenation{en-gi-neering}
\hyphenation{com-pa-ti-bi-li-ty}
\hyphenation{im-pos-sible}
\hyphenation{desk-top}
\hyphenation{elec-tro-nic}
\hyphenation{com-pa-ny}
\hyphenation{de-ve-lop-ment}
\hyphenation{de-ve-loping}
\hyphenation{de-ve-lop}
\hyphenation{da-ta-ba-se}
\hyphenation{plat-forms}
\hyphenation{or-ga-ni-za-tion}
\hyphenation{pro-gramming}
\hyphenation{in-stru-ments}
\hyphenation{Li-nux}
\hyphenation{sour-ce}
\hyphenation{en-vi-ron-ment}
\hyphenation{Te-le-pathy}
\hyphenation{Li-nux-ov-ka}
\hyphenation{Open-BSD}
\hyphenation{Free-BSD}
\hyphenation{men-ti-on-ed}
\hyphenation{app-li-ca-tion}

\def\progref!#1!{\texttt{#1}}
\renewcommand{\arraystretch}{2} %Іначай формулы ў матрыцы зліпаюцца з лініямі
\usepackage{array}

\def\interview #1 (#2), #3, #4, #5\par{

\section[#1, #3, #4]{#1 -- #3, #4}
\def\qname{LVEE}
\def\aname{#1}
\def\q ##1\par{{\noindent \bf \qname: ##1 }\par}
\def\a{{\noindent \bf \aname: } \def\qname{L}\def\aname{#2}}
}

\def\interview* #1 (#2), #3, #4, #5\par{

\section*{#1\\{\small\rm #3, #4. #5}}

\def\qname{LVEE}
\def\aname{#1}
\def\q ##1\par{{\noindent \bf \qname: ##1 }\par}
\def\a{{\noindent \bf \aname: } \def\qname{L}\def\aname{#2}}
}

\begin{document}
\title{Как поссорились Иван Интелович с Иваном Опёнковичем\footnote{\url{zhuk@openbsd.org}, \url{https://lvee.org/en/abstracts/281}}}
\author{Vadim Zhukov, Moscow, Russian Federation}
\maketitle
\begin{abstract}
This is the story about latest Intel tries to discriminate OpenBSD on getting pre-public disclosure infromation about vulnerabilities in its products.
\end{abstract}
Многие слышали историю о том, как OpenBSD «отлучили» от информации об уязвимостях Meltdown и Spectre.

В середине 2017 года Мэти Ванхоф (Mathy Vanhoef), молодой и при этом уже известный в определённых кругах специалист в области безопасности, уже несколько лет занимавшийся вопросами защиты беспроводных сетей IEEE 802.11 (Wi-Fi), последовательно обнаруживает ряд проблем в реализации протоколов WPA и WPA2, разной степени серьёзности. Самая крупная из проблем позднее получит персональное имя~--- KRACK.

Будучи добропорядочным исследователем, Ванхоф извещает разработчиков реализаций WPA о найденных проблемах с тем, чтобы каждый смог подготовить исправления. В том числе, 15 июля 2017 года такое извещение было отправлено приватно разработчикам OpenBSD. Выработанное решение было согласовано с Ванхофом и 30 августа, по согласованию с ним же, внесено в кодовую базу OpenBSD под видом минорного исправления. Подобная политика позволяет выступать open source-проектам на равных с разработчиками пропиетарных ОС, которые могут не боясь огласки выпускать обновления безопасности.

Тем не менее, несмотря на все предпринятые меры, данный коммит привлёк внимание специалистов по безопасности, исходная проблема была выявлена и раскрытие сведений об уязвимости пришлось провести безотлагательно. Поскольку переписка задействованных разработчиков OpenBSD с Ванхофом, по понятным причинам, не была публичной, то в широких кругах сложилось впечатление, что проект OpenBSD самовольно нарушил эмбарго на разглашение сведений об уязвимости.

В это же время ряд исследователей обнаружил уязвимость в микропроцессорах Intel, позволявшую медленно, но верно получать по косвенному каналу содержимое произвольных участков адресного пространства текущего процесса (включая содержимое ядра операционной системы). К делу подключились специалисты Intel а позднее, когда стала очевидна универсальность подхода, и других производителей микропроцессоров. Когда стало понятно, что данную уязвимость невозможно закрыть одним только изменением микрокода, к работе стали подключать разработчиков операционных систем. Однако в свете упомянутых выше событий, связанных с KRACK, компания Intel (и, возможно, другие заинтересованные лица) настояла на невключении проекта OpenBSD как неблагонадёжного партнёра.

Изначально эмбарго планировалось снять в середине января, однако допущенная в LKML утечка вынудила сдвинуть сроки. Фактически, повторилась описанная выше ситуация, но теперь уже с Linux: журналисты из The Register (и не только они) обратили внимание на поток серьёзных патчей, в результате чего анонс был срочно перенесён на 4-е января. Именно тогда разработчики OpenBSD (как и DragonFly BSD? Кто ещё? \% ) узнали о проблеме~--- наравне с широкой общественностью. Из-за крайне высокой сложности проблемы работа над исправлением ситуации заняла не один месяц. В ходе работы по конструированию обходных путей для Meltdown/Spectre были решены следующие задачи:

\begin{itemize}
  \item разделение адресных пространств ядра и приложения;
  \item модификация компиляторов для генерации защищающего от Meltdown/Specre кода;
  \item реализация инфраструктуры обновления микрокода ЦП Intel.
\end{itemize}

Однако на этом дело не закончилось: с каждым месяцем исследователи находили всё новые уязвимости в ЦП. И далеко не все из них разделяли мнение о том, что OpenBSD надо «наказать». Например, Бен Грас (Ben Gras) из VUSec по собственной воле поделился с разработчиками OpenBSD подробностями найденной им уязвимости TLBleed, позволяющей организовать утечку секретных ключей за счёт эксплуатации особенностей Intel Hyper Threading (на процессорах с одновременной многопоточностью других производителей организовать эффективную атаку оказалось затруднительно). Ещё более интересная ситуация сложилась с уязвимостью LazyFP: до разработчиков OpenBSD дошли слухи о наличии уязвимости, связанной с сохранением состояния сопроцессора. По этим скудным обрывкам был создан и закоммичен патч, о котором Тео де Раадт, лидер проекта, рассказал на конференции BSDCan. Через пять часов после этого Колин Персиваль, бывший FreeBSD security officer, смог создать рабочий эксплоит~--- в результате проект OpenBSD, не нарушив эмбарго, таки поспособствовал досрочному анонсу уязвимости (полностью подробности на момент написания этих строк по-прежнему не раскрыты).

Таким образом можно констатировать, что изоляция проекта OpenBSD не удалась, и следует ожидать изменения позиции Intel (и других производителей, разделявших отношение Intel к OpenBSD) по вопросам предварительного информирования об уязвимости.

\end{document}
