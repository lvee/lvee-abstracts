\documentclass[10pt, a5paper]{article}
\usepackage[T2A]{fontenc}
\usepackage{ucs}
\usepackage[utf8x]{inputenc}
\usepackage[polish,english,russian]{babel}
\usepackage{hyperref}
\usepackage[inner=2cm,top=1.8cm,outer=2cm,bottom=2.3cm,nohead]{geometry}
\usepackage{listings}
\usepackage{graphicx}
\usepackage{wrapfig}
\usepackage{longtable}
\usepackage{indentfirst}
\frenchspacing
\usepackage{fixltx2e} %text sub- and superscripts
\usepackage{icomma} % коскі ў матэматычным рэжыме
\PreloadUnicodePage{4}

\newcommand{\longpage}{\enlargethispage{\baselineskip}}
\newcommand{\shortpage}{\enlargethispage{-\baselineskip}}

\def\switchlang#1{\expandafter\csname switchlang#1\endcsname}
\def\switchlangbe{
\let\saverefname=\refname%
\def\refname{Літаратура}%
\def\figurename{Іл.}%
}
\def\switchlangen{
\let\saverefname=\refname%
\def\refname{References}%
\def\figurename{Fig.}%
}
\def\switchlangru{
\let\saverefname=\refname%
\let\savefigurename=\figurename%
\def\refname{Литература}%
\def\figurename{Рис.}%
}

\hyphenation{admi-ni-stra-tive}
\hyphenation{ex-pe-ri-ence}
\hyphenation{fle-xi-bi-li-ty}
\hyphenation{Py-thon}
\hyphenation{ma-the-ma-ti-cal}
\hyphenation{re-ported}
\hyphenation{imp-le-menta-tions}
\hyphenation{pro-vides}
\hyphenation{en-gi-neering}
\hyphenation{com-pa-ti-bi-li-ty}
\hyphenation{im-pos-sible}
\hyphenation{desk-top}
\hyphenation{elec-tro-nic}
\hyphenation{com-pa-ny}
\hyphenation{de-ve-lop-ment}
\hyphenation{de-ve-loping}
\hyphenation{de-ve-lop}
\hyphenation{da-ta-ba-se}
\hyphenation{plat-forms}
\hyphenation{or-ga-ni-za-tion}
\hyphenation{pro-gramming}
\hyphenation{in-stru-ments}
\hyphenation{Li-nux}
\hyphenation{en-vi-ron-ment}
\hyphenation{Te-le-pathy}
\hyphenation{Li-nux-ov-ka}

\def\progref!#1!{\texttt{#1}}
\renewcommand{\arraystretch}{2} %Іначай формулы ў матрыцы зліпаюцца з лініямі
\usepackage{array}

\def\interview #1 (#2), #3, #4, #5\par{

\section[#1, #3, #4]{#1, #5}
\def\qname{LVEE}
\def\aname{#1}
\def\q ##1\par{{\noindent \bf \qname: ##1 }\par}
\def\a{{\noindent \bf \aname: } \def\qname{L}\def\aname{#2}}
}

\begin{document}
\title{Интервью с участниками}
%\author{}
\date{}
\maketitle

По традиции в сборник материалов входят интервью, взятые представителями оргкомитета во время зимней сессии конференции. Датой LVEE Winter 2014 оказался день 14 февраля, и это подвигло организаторов на романтическую идею: выбрать из программы конференции трех представительниц прекрасной половины человечества и узнать их мнение о свободном ПО, роли и месте GNU/Linux, и собственно о LVEE.

\section[Ирина Шубина "--- senior software developer, EPAM Systems, Минск, Беларусь]{Ирина Шубина "--- senior software \linebreak developer, EPAM Systems, Минск, Беларусь}
%\begin{figure}[ht]
%\centering{\includegraphics[width=4cm]{49_spons_altoros.jpg}}
%\end{figure}

{\noindent \bf LVEE: Скажи, помнишь ли ты тот момент, когда ты первый раз услышала о свободном программном обеспечении?}

{\noindent \bf И:} Услышать \ldots даже не так: я его первый раз даже увидела. Я о нем не слышала до этого. Для меня это было естественно новостью. После третьего курса я искала работу "--- очень долго, целое лето, активно.

{\noindent \bf L: А курс это\ldots}

{\noindent \bf И:} БГУ. Гуманитарный факультет. Специальность <<Информатика, web"=дизайн и компьютерная графика>> на тот момент.  И вот я искала работу, после того как прошла курсы тестировщиков "--- меня почему"=то все лето не брали в тестировщики, но зато в конце лета взяли в программисты.

{\noindent \bf L: На чем писать?}

{\noindent \bf И:} Да в общем"=то я и не знала, на чем :) Как оказалось, на С++ и Python.

{\noindent \bf L: <<Потом я узнала, что это С++, С и Python\ldots>>}

{\noindent \bf И:} Да, потом я узнала. А еще, когда я в первый раз пришла на работу, меня посадили за компьютер, там был двойной bootloader, и меня сразу загрузили в Linux и сказали: <<Знакомься, это Archlinux>>. Да, как"=то так\ldots Я посмотрела и сказала: <<О\ldots >>.

{\noindent \bf L: Это была любовь с первого взгляда? Или не была?}

{\noindent \bf И:} Ну, на тот момент я еще два года работала и в Windows, и в Linux, а позже перешла полностью на Linux.

{\noindent \bf L: А почему у вас, собственно, Linux"=то использовался? Особенно Archlinux "--- не очень простой в настройке дистрибутив, и не самый стабильный, из"=за постоянного обновления пакетов.}

{\noindent \bf И:} Archlinux использовался собственно на моем только компьютере\ldots

{\noindent \bf L: Кто"=то очень тебя любил :)}

{\noindent \bf И:} На тот момент я, как вы сами понимаете, не могла его установить с нуля "--- это же Arch!

{\noindent \bf L: Да, тогда у него еще был инсталлятор, но все равно книжка по установке была большая :)}

{\noindent \bf И:} Вот. Книжку по установке я тоже никогда не видела "--- до сих пор\ldots

{\noindent \bf L: А она есть!}

{\noindent \bf И:} Верю :)

{\noindent \bf L: Хорошо. И значит, Archlinux запустился, открылся какой"=то Gnome или KDE\ldots И какие были первые впечатления?}

{\noindent \bf И:} Он был на KDE. Мне сказали: <<Вот здесь есть тоже большая кнопка. Она, правда, вверху\ldots И нужно вот нажать вот здесь, и будет тебе меню>>.

{\noindent \bf L: Поскольку девушка"=дизайнер собиралась работать программистом, наверное это было не самое большое потрясение на общем фоне.}

{\noindent \bf И:} Ну, я не собиралась работать дизайнером, это точно, поскольку я искала работу тестировщика. Но в общем"=то мечтая, где"=то там на бэкграунде, работать программистом "--- но думала, что надо начинать с чего"=то попроще, попроще\ldots

{\noindent \bf L: По поводу Linux. Потихонечку ты начала им пользоваться больше\ldots}

{\noindent \bf И:} Да, потихонечку. В итоге даже установила на домашнем компьютере. Пришел момент, когда я решила снести свою XP\ldots

{\noindent \bf L: А что повлияло?}

{\noindent \bf И:} Не знаю, в какой"=то момент захотелось, чтобы это было стилем жизни. Вот, наверное так.

{\noindent \bf L: Кто"=то из домашних пугался?}

{\noindent \bf И:} Да у меня домашние до сих пор пугаются. Это основной момент, когда меня что"=то спрашивают и сильно донимают этим, я всем угрожаю, что я приеду и установлю Ubuntu, страшную и ужасную. Да, это главная угроза "--- по отношению, по крайней мере, к младшей сестре.

{\noindent \bf L: Все"=таки не Archlinux?}

{\noindent \bf И:} Нет, домашний все"=таки Ubuntu, на данный момент основной Linux на работе это Fedora у меня\ldots

{\noindent \bf L: А дома Ubuntu "--- как у Дональда Кнута\ldots}

{\noindent \bf И:} Только не с конфигурационным файлом 30"=летней давности :)

{\noindent \bf L: Хорошо. Как в последнее время, на твой взгляд, меняется отношение к Linux и вообще к свободному программному обеспечению? Заметно что"=нибудь?}

{\noindent \bf И:} На самом деле очень заметно. Очень заметна положительная тенденция, потому что становится больше обычных простых пользователей.

{\noindent \bf L: Не разработчиков.}

{\noindent \bf И:} Да, не разработчиков, которые не знают о разработке вообще ничего. То есть у меня есть знакомая, которая работает в магазине продавщицей, и не знаю как у нее появился Linux\ldots

{\noindent \bf L: Где она его взяла\ldots}

{\noindent \bf И:} Где она его взяла, да, но откуда"=то он у нее появился, и мне, как специалисту, задавали некоторые вопросы по чистому банальному пользованию.

{\noindent \bf L: Это здорово.}

{\noindent \bf И:} Это здорово, это приятно, и это хорошо.

\section[Александра Кононова "--- доцент кафедры информатики и программного обеспечения вычислительных систем, МИЭТ, г. Зеленоград, Москва, РФ]{Александра Кононова "--- доцент кафедры информатики и программного обеспечения вычислительных систем, \linebreak{МИЭТ}, г. Зеленоград, Москва, РФ}

%\begin{figure}[ht]
%\centering{\includegraphics[width=4cm]{49_spons_altoros.jpg}}
%\end{figure}

{\noindent \bf Александра Кононова:} Я по отношению к свободному ПО скорее пользователь, для начала. Хотя мне приходится иногда подгонять что"=то под свои нужды\ldots

{\noindent \bf L: Ты говоришь, что ты пользователь. А что побуждает тебя приезжать на LVEE (уже второй раз), и выступать с докладами?}

{\noindent \bf А:} Любопытство. А кроме того, мне белая майка не к лицу, а у вас, вот, докладчикам выдают черные :)

{\noindent \bf L: Хороший стимул :) Так. Ты помнишь тот момент, когда ты впервые услышала про свободное программное обеспечение, или про Linux, или еще про что"=то в этом роде?}

{\noindent \bf А:} К сожалению, память у меня достаточно скверная :) Поэтому когда первый раз "--- не знаю\ldots

{\noindent \bf L: Но если примерно?}

{\noindent \bf А:} Может, когда в~своё время на малом мехмате МГУ я увидела, как верстался математический журнал\ldots{} В \TeX, кажется, и да, под Linux. И меня поразило, насколько это было красиво.

{\noindent \bf L: Что"=то изменилось в твоей жизни после этого? Linux, во всяком случае, теперь приходится пользоваться, да?}

{\noindent \bf А:} Ну не то, чтобы приходится. Оказалось, что это удобно и действительно красиво. Поэтому немножко не то слово выбрано. 

{\noindent \bf L: То есть это собственное желание, которое удается претворять в жизнь.}

{\noindent \bf А:} Ну поскольку эта система "--- логичная и интуитивно"=понятная, и к которой можно подобрать удобный интерфейс, а не унифицированный для среднестатистического пользователя, то да, безусловно. 

{\noindent \bf L: А в учебном процессе появляется какое"=то свободное программное обеспечение у вас?}

{\noindent \bf А:} Поскольку на ВЦ стоит во всех аудиториях Windows официально, поскольку однотипное ПО проще администрировать "--- то только прикладное: Octave, Scilab, среды разработки\ldots

{\noindent \bf L: То есть много математического ПО?}

{\noindent \bf А:} Да\ldots{} А еще OpenOffice, как же без него.

{\noindent \bf L: А студенты различают вообще как"=то: <<Вот это свободная программа, а это "--- несвободная, но мы ей пользуемся>>?}

{\noindent \bf А:} Студенты все разные. Но надо признать, что большинство на вопрос о лицензиях отвечают <<Плевали мы на эту лицензию, все равно есть все взломанное>>. 

{\noindent \bf L: То есть борьба за чистоту лицензий это все"=таки, до конечного пользователя когда доходит, носит довольно условный характер, да? В принципе, российские вузы "--- они пока не очень сильно озабочены этим вопросом, или в вузе все"=таки приходится\ldots}

{\noindent \bf А:} Это, честно говоря, зависит от ВЦ. Потому что где"=то программное обеспечение если не свободное, то купленное, а где"=то "--- обычно это дорогостоящие мощные комплексы "--- где"=то оно\ldots разное :)

{\noindent \bf L: По"=настоящему тактичный ответ. А тот же Linux удается использовать или на собственном компьютере, или на кафедральных как"=то?}

{\noindent \bf А:} На собственных и кафедральных безусловно стоит у нескольких человек: у кого"=то как учебное пособие, а у кого"=то "--- как основная система. Потом есть отдельный учебный курс, <<Основы Unix>>, и курс <<Операционные системы>>, но там используется виртуальная машина, что, наверное, не улучшает впечатление, и в виртуальной машине он: Redhat, Ubuntu.

{\noindent \bf L: Получается, что преподавательский состав его использует активнее, чем студенты?}

{\noindent \bf А:} Опять"=таки, и студенты и преподаватели разные, поэтому многие используют, а многие нет.

{\noindent \bf L: Если соотношение попытаться вывести "--- где можно чаще услышать об этом, от студента и ли от преподавателя?}

{\noindent \bf А:} Слово "--- ни от кого: кто пользуется, тот пользуется и молчит :)

{\noindent \bf L: Значит, персональное личное дело каждого. Как вегетарианство :)}

{\noindent \bf А:} Но есть ресурс \url{unix.miet.ru}, где находятся зеркала основных дистрибутивов Linux, BSD. Так что, судя по всему, эта тема хоть и не проговаривается, но поддерживается.

{\noindent \bf L: То есть, судя по всему, ваш вычислительный центр как раз достаточно активно его использует. Наверняка же ВЦ администрирует эти зеркала. А кстати, по сложности эксплуатации свободных программ какая"=то разница ощущается?}

{\noindent \bf А:} Проще.

{\noindent \bf L: Проще?}

{\noindent \bf А:} Во"=первых, установка проще, и не требуется искать\ldots{} лицензию, и отвечать на многие кнопочки. А во"=вторых, поскольку свободное ПО пишут в основном энтузиасты и для себя, то можно обычно найти программу, которая соответствует твоей логике.

{\noindent \bf L: И последний вопрос, традиционно: как ты вообще узнала про LVEE?}

{\noindent \bf А:} В основном из чувства противоречия. Ибо нам на кафедру сыплются конференции от Microsoft "--- есть рассылка, и в ней периодически спрашивают: не хотите поучаствовать? Ну и поскольку возник естественный вопрос, а нет ли конференций\ldots

{\noindent \bf L: \ldotsгде все наоборот\ldots}

{\noindent \bf А:} Да, где можно обсудить свободное ПО, то стали искать и нашли несколько завершившихся. Через стандартный поиск, google. И эта показалась наиболее вменяемой по времени "--- когда надо было ждать не год, а несколько месяцев. 

{\noindent \bf L: И как впечатление?}

{\noindent \bf А:} Очень приятное.

{\noindent \bf L: На что это вообще похоже? LVEE все"=таки во многом отличается от академических конференций\ldots}

{\noindent \bf А:} На капустник. Но весело :) А доклады "--- ну, я не знаю, с чем их сравнить\ldots На доклады. Можно узнать много новых слов. Много интересного. 

{\noindent \bf L: Спасибо :)}

{\noindent \bf А:} Они, наверное, более понятны, чем многие академические, здесь больше практики. И видно, что человек этим занимается, а не просто наскреб на доклад. 

\section{Ольга Карабутова "--- senior programmer, R\&D, EPAM Systems, Минск, Беларусь}

%\begin{figure}[ht]
%\centering{\includegraphics[width=4cm]{49_spons_altoros.jpg}}
%\end{figure}

{\noindent \bf L: Ты сегодня выступала с докладом. Расскажи об ощущениях "--- как это страшно, как это волнительно, какие чувства приносит такое публичное выступление?}

{\noindent \bf Ольга Карабутова:} Честно говоря, первое что я хочу сказать "--- что это было мое первое публичное выступление.

{\noindent \bf L: Кстати, у тебя очень хорошо получилось.}

{\noindent \bf О:} Ура. Но факт в том, например, что ту же магистерскую я не смогла нормально защитить, растерялась, потому что чувствовала, что я что"=то не доделала\ldots Диплом у меня был чем"=то, что я действительно сделала "--- аппаратный проект, да, я звезда была по сравнению со всеми остальными, мальчиками, девочками, дневниками, вечерниками. А магистерская была дана преподавателем, магистратура у меня была заочная, все это было недоработано\ldots

{\noindent \bf L: По специальности?..}

{\noindent \bf О:} Радиотехника. Включая радиолокацию, радионавигацию и телевидение. В общем, магистратура была заочной, и я чувствовала, что недоработана тема. Но поблажки мне за это не было\ldots C тех пор я успела сменить много работ. Сейчас я работаю в EPAM. Там замечательно :)

{\noindent \bf L: Что не устраивало в предыдущих работах?}

{\noindent \bf О:} Ну если взять эволюцию, то бывало что что"=то серьезно не устраивало, но с последнего места работы я ушла не потому, что что"=то не устраивало "--- было некоторое стечение обстоятельств, и здесь мне предложили и более интересный проект, и более денежный, но в первую очередь все"=таки более интересный. И я, как человек общительный,  уже знала достаточно много людей оттуда и так\ldots Возможно благодаря LVEE. 

{\noindent \bf L: А кстати, как ты на LVEE попала впервые?}

{\noindent \bf О:} Мне его рекламировали несколько человек.  То есть фактически, это была внутрикорпоративная среда, где какие"=то разработчики рассказывали, что есть такая замечательная конференция. И один человек мне ее прорекламировал\ldots Он ни разу, кстати, на LVEE не был, но очень рекламировал. Это было в 2009 или в 2010 году. Потом я встретила старого знакомого, и он тоже\ldots А потом мне прорекламировал мой бывший на тот момент начальник, с бывшего места работы. И тут уж было без вариантов. Три человека с разных точек взаимодействия сказали\ldots

{\noindent \bf L: Это была просто судьба. И какое было первое впечатление?}

{\noindent \bf О:} Во"=первых, очень интересные доклады. Опять же, формат летнего LVEE располагает к отдыху, общению и новым профессиональным знакомствам "--- и не только, к дружеским в том числе. То есть я увидела большую"=большую сборную из разных интересных людей, и теперь я постоянный посетитель.

{\noindent \bf L: Может быть это как"=то повлияло на карьеру?}

{\noindent \bf О:} Конечно же повлияло. Дело в том, что каждый раз, когда вы идете к какому"=то изменению, есть много факторов. И нельзя сказать, что первое: мой интерес к Linux, и поэтому LVEE,  или LVEE и поэтому мой усилившийся интерес к Linux.

{\noindent \bf L: А вообще, первое знакомство с Linux "--- оно как произошло? Это наш вечный вопрос.}

{\noindent \bf О:} Одно дело личное знакомство, а другое "--- почти знакомство. Услышала я в университете, от одногруппников, и даже когда"=то, когда я попросила помощи, человек мне сразу поставил еще и Linux, и Windows\ldots Поставил и сказал: резервная система, если какие"=то будут неполадки, то через Linux можно решить. И был еще Linux на работе, после, я тогда работала в публичных библиотеках города Минска.

{\noindent \bf L: А в публичных библиотеках города Минска есть Linux?}

{\noindent \bf О:} Да он давным"=давно был. И я думаю, что не только в публичных библиотеках города Минска. На самом деле еще, например, в налоговой, УВД\ldots А в библиотеке я им не занималась "--- там была отдельная девушка выделена, системный администратор, которая занималась Linux. В общем, эта тема просто была на слуху. Но ближе познакомилась я чуть позже. Не помню точный момент\ldots

{\noindent \bf L: То есть это просто система для себя, за какие"=то ее достоинства?}


{\noindent \bf О:} Да, для себя, для тех экспериментов, которые вы проводите дома "--- с этим же интересно поиграться. Ну конечно же, я работала с Linux на работе, но я не помню, чтобы это было требованием. Где"=то, возможно, были встраиваемые системы, где"=то еще что"=то, т.е. с Linux у меня есть опыт работы, но прямо сказать, что я работала разработчиком под Linux "--- такого не было. 

{\noindent \bf L: А какое"=то такое ощущение, что свободное программное обеспечение "--- это есть какой"=то отдельной разновидность или отдельная группа программ, чем"=то отличается\ldots}

{\noindent \bf О:} Есть же своя философия Linux, это все знают, и это красиво и стройно. Я вообще склонна к философии, поэтому это мне особенно понравилось. В этом есть какая"=то гармония. Мне, как инженеру по образованию, вполне понятно, что система может быть красивой, устройство может быть красивым, и обычно это говорит о том, что\ldots О его лаконичности, эффективности, без всего лишнего. И в этом плане я очень быстро разочаровалась в Windows, как только попробовала Linux. Ну, как есть так есть, что тут сделаешь :) Вот так :)


\end{document}


