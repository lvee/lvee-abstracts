\documentclass[10pt, a5paper]{article}
\usepackage{pdfpages}
\usepackage{parallel}
\usepackage[T2A]{fontenc}
\usepackage{ucs}
\usepackage[utf8x]{inputenc}
\usepackage[polish,english,russian]{babel}
\usepackage{hyperref}
\usepackage{rotating}
\usepackage[inner=2cm,top=1.8cm,outer=2cm,bottom=2.3cm,nohead]{geometry}
\usepackage{listings}
\usepackage{graphicx}
\usepackage{wrapfig}
\usepackage{longtable}
\usepackage{indentfirst}
\usepackage{array}
\newcolumntype{P}[1]{>{\raggedright\arraybackslash}p{#1}}
\frenchspacing
\usepackage{fixltx2e} %text sub- and superscripts
\usepackage{icomma} % коскі ў матэматычным рэжыме
\PreloadUnicodePage{4}

\newcommand{\longpage}{\enlargethispage{\baselineskip}}
\newcommand{\shortpage}{\enlargethispage{-\baselineskip}}

\def\switchlang#1{\expandafter\csname switchlang#1\endcsname}
\def\switchlangbe{
\let\saverefname=\refname%
\def\refname{Літаратура}%
\def\figurename{Іл.}%
}
\def\switchlangen{
\let\saverefname=\refname%
\def\refname{References}%
\def\figurename{Fig.}%
}
\def\switchlangru{
\let\saverefname=\refname%
\let\savefigurename=\figurename%
\def\refname{Литература}%
\def\figurename{Рис.}%
}

\hyphenation{admi-ni-stra-tive}
\hyphenation{ex-pe-ri-ence}
\hyphenation{fle-xi-bi-li-ty}
\hyphenation{Py-thon}
\hyphenation{ma-the-ma-ti-cal}
\hyphenation{re-ported}
\hyphenation{imp-le-menta-tions}
\hyphenation{pro-vides}
\hyphenation{en-gi-neering}
\hyphenation{com-pa-ti-bi-li-ty}
\hyphenation{im-pos-sible}
\hyphenation{desk-top}
\hyphenation{elec-tro-nic}
\hyphenation{com-pa-ny}
\hyphenation{de-ve-lop-ment}
\hyphenation{de-ve-loping}
\hyphenation{de-ve-lop}
\hyphenation{da-ta-ba-se}
\hyphenation{plat-forms}
\hyphenation{or-ga-ni-za-tion}
\hyphenation{pro-gramming}
\hyphenation{in-stru-ments}
\hyphenation{Li-nux}
\hyphenation{sour-ce}
\hyphenation{en-vi-ron-ment}
\hyphenation{Te-le-pathy}
\hyphenation{Li-nux-ov-ka}
\hyphenation{Open-BSD}
\hyphenation{Free-BSD}
\hyphenation{men-ti-on-ed}
\hyphenation{app-li-ca-tion}

\def\progref!#1!{\texttt{#1}}
\renewcommand{\arraystretch}{2} %Іначай формулы ў матрыцы зліпаюцца з лініямі
\usepackage{array}

\def\interview #1 (#2), #3, #4, #5\par{

\section[#1, #3, #4]{#1 -- #3, #4}
\def\qname{LVEE}
\def\aname{#1}
\def\q ##1\par{{\noindent \bf \qname: ##1 }\par}
\def\a{{\noindent \bf \aname: } \def\qname{L}\def\aname{#2}}
}

\def\interview* #1 (#2), #3, #4, #5\par{

\section*{#1\\{\small\rm #3, #4. #5}}

\def\qname{LVEE}
\def\aname{#1}
\def\q ##1\par{{\noindent \bf \qname: ##1 }\par}
\def\a{{\noindent \bf \aname: } \def\qname{L}\def\aname{#2}}
}


\begin{document}

\title{Kexecboot --- Linux как bootloader}%\footnote{Текст данных и последующих тезисов, кроме специально оговоренных случаев, доступен под лицензией Creative Commons Attribution-ShareAlike 3.0}

\author{Юрий Бушмелев\footnote{Ульяновск, РФ}}
\maketitle

\begin{abstract}
The article describes Kexecboot, an implementation of Linux-as-a-Bootloader. It's a C program able to scan the partitions on available devices, offering a graphical framebuffer menu and allowing user to select from which one to boot.
Typically kexecboot resides together with kexec in an initramfs, embedded in a custom-tailored kernel compiled with support for initramfs and kexec system call. Both binaries are built static, linked against klibc to optimize size. Kexecboot may be linked against other *libc (glibc, eglibc, uclibc) and may be used as standalone binary as well.
\end{abstract}

Современные загрузчики операционных систем (grub, u-boot), по сути, сами по себе являются своеобразной операционной системой. В них есть драйвера для дисковых контроллеров, для файловых систем, для устройств ввода/вывода. И все это нужно только для того, чтобы загрузить ядро пользовательской операционной системы и передать ему управление. Причем, если на архитектуре PC есть более-менее стандартные интерфейсы BIOS и EFI, то в сфере embedded hardware ситуация куда более плачевная, что приводит к бесчисленным форкам загрузчиков со всеми вытекающими неприятностями.

Kexecboot --- это реализация идеи «Linux как bootloader». Сама идея не нова, она появилась в виде ответа разработчиков на вышеописанную ситуацию. Предполагается иметь простой минимальный загрузчик, который умеет только загружать ядро Linux из определенного фиксированного места и передавать ему управление. Дальнейшие действия по инициализации оборудования и работе с периферией выполняет уже само ядро. Для, собственно, загрузки с других носителей применяется системный вызов kexec, который загружает следующее ядро в память и передает ему управление.

Изначально kexecboot был разработан для КПК Sharp Zaurus. Проблема с данными КПК была в том, что под ядро ОС в штатной схеме разметки NAND был выделен раздел размером всего 1.2Mb. Прошло несколько лет и ядро Linux со всеми желаемыми функциями перестало входить в этот лимит. Нужно было либо менять загрузчик, либо придумывать пути обхода. Но с другой стороны, в старых моделях Sharp Zaurus объем NAND был всего 16Mb, что накладывало ограничения и на содержимое «прошивки». Требовался загрузчик, который способен запустить систему с SD или CF-карты. Причем, желательно, с возможностью выбора, откуда и что загружать. Так появился kexecboot.

Сам по себе, kexecboot --- это программа, которая сканирует устройства и отображает в виде меню список ядер, которые могут быть загружены. Чтобы элемент появился в списке, должны быть выполнены следующие условия:
\begin{itemize}
\item на устройстве находится файловая система, которую kexecboot может идентифицировать, а ядро может смонтировать;
\item на файловой системе есть файл конфигурации /boot/boot.cfg, который удалось распарсить, либо на файловой системе есть один из файлов /zImage, /boot/zImage (или uImage).
\end{itemize}

Также в общем меню присутствует системное меню, где можно запустить повторное сканирование устройств, посмотреть некоторую отладочную информацию, а также перезагрузить или выключить устройство. Меню может быть графическим (framebuffer) или текстовым. На данный момент, выбор может быть произведен только  аппаратными клавишами устройства.

После выбора элемента в меню, kexecboot формирует строку запуска для утилиты kexec, чтобы сначала загрузить ядро (kexec -l), а потом передать ему управление (kexec -e).

На параметры загружаемого ядра можно влиять при помощи файла конфигурации. В нем можно задать название и иконку, которые будут отображены в меню, а также путь к ядру и параметры его командной строки.

Кexecboot можно использовать как отдельную программу, так и вместо init в составе initramfs. Во втором случае ядро компонуется вместе с образом initramfs, где находятся необходимые каталоги и два исполнимых файла — kexecboot и kexec. На данный момент нам удается держать размер образа ядра с initramfs в пределах 1Mb. Частично это удается за счет более тонкой настройки конфигурации ядра, частично--за счет использования lzma для сжатия образа.

К сожалению, узким местом при использовании kexecboot остается неработоспособность kexec на многих аппаратных платформах. Тем не менее, на данный момент известно о фактах успешного использования kexecboot на всей линейке КПК Sharp Zaurus, на некоторых моделях КПК iPAQ, на нетбуках Toshiba AC100 и некоторых планшетах Archos. Пакеты для произвольных архитектур можно подготовить с помощью систем сборки OpenEmbedded или OpenWRT.

В будущем планируется добавление альтернативных методов загрузки — switch\_root, losetup+switch\_root и nfs+switch\_root. Это позволит использовать kexecboot даже на системах, где kexec (пока) не работает.

Подробности о проекте можно узнать на сайте \url{http://kexecboot.org}



\end{document}




