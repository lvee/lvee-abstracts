\documentclass[10pt, a5paper]{article}
\usepackage{pdfpages}
\usepackage{parallel}
\usepackage[T2A]{fontenc}
\usepackage{ucs}
\usepackage[utf8x]{inputenc}
\usepackage[polish,english,russian]{babel}
\usepackage{hyperref}
\usepackage{rotating}
\usepackage[inner=2cm,top=1.8cm,outer=2cm,bottom=2.3cm,nohead]{geometry}
\usepackage{listings}
\usepackage{graphicx}
\usepackage{wrapfig}
\usepackage{longtable}
\usepackage{indentfirst}
\usepackage{array}
\newcolumntype{P}[1]{>{\raggedright\arraybackslash}p{#1}}
\frenchspacing
\usepackage{fixltx2e} %text sub- and superscripts
\usepackage{icomma} % коскі ў матэматычным рэжыме
\PreloadUnicodePage{4}

\newcommand{\longpage}{\enlargethispage{\baselineskip}}
\newcommand{\shortpage}{\enlargethispage{-\baselineskip}}

\def\switchlang#1{\expandafter\csname switchlang#1\endcsname}
\def\switchlangbe{
\let\saverefname=\refname%
\def\refname{Літаратура}%
\def\figurename{Іл.}%
}
\def\switchlangen{
\let\saverefname=\refname%
\def\refname{References}%
\def\figurename{Fig.}%
}
\def\switchlangru{
\let\saverefname=\refname%
\let\savefigurename=\figurename%
\def\refname{Литература}%
\def\figurename{Рис.}%
}

\hyphenation{admi-ni-stra-tive}
\hyphenation{ex-pe-ri-ence}
\hyphenation{fle-xi-bi-li-ty}
\hyphenation{Py-thon}
\hyphenation{ma-the-ma-ti-cal}
\hyphenation{re-ported}
\hyphenation{imp-le-menta-tions}
\hyphenation{pro-vides}
\hyphenation{en-gi-neering}
\hyphenation{com-pa-ti-bi-li-ty}
\hyphenation{im-pos-sible}
\hyphenation{desk-top}
\hyphenation{elec-tro-nic}
\hyphenation{com-pa-ny}
\hyphenation{de-ve-lop-ment}
\hyphenation{de-ve-loping}
\hyphenation{de-ve-lop}
\hyphenation{da-ta-ba-se}
\hyphenation{plat-forms}
\hyphenation{or-ga-ni-za-tion}
\hyphenation{pro-gramming}
\hyphenation{in-stru-ments}
\hyphenation{Li-nux}
\hyphenation{sour-ce}
\hyphenation{en-vi-ron-ment}
\hyphenation{Te-le-pathy}
\hyphenation{Li-nux-ov-ka}
\hyphenation{Open-BSD}
\hyphenation{Free-BSD}
\hyphenation{men-ti-on-ed}
\hyphenation{app-li-ca-tion}

\def\progref!#1!{\texttt{#1}}
\renewcommand{\arraystretch}{2} %Іначай формулы ў матрыцы зліпаюцца з лініямі
\usepackage{array}

\def\interview #1 (#2), #3, #4, #5\par{

\section[#1, #3, #4]{#1 -- #3, #4}
\def\qname{LVEE}
\def\aname{#1}
\def\q ##1\par{{\noindent \bf \qname: ##1 }\par}
\def\a{{\noindent \bf \aname: } \def\qname{L}\def\aname{#2}}
}

\def\interview* #1 (#2), #3, #4, #5\par{

\section*{#1\\{\small\rm #3, #4. #5}}

\def\qname{LVEE}
\def\aname{#1}
\def\q ##1\par{{\noindent \bf \qname: ##1 }\par}
\def\a{{\noindent \bf \aname: } \def\qname{L}\def\aname{#2}}
}

\begin{document}
\title{Merge-a-fork или скрестим вилки}
\author{Михаил Шигорин\footnote{Киев, Украина, \url{mike@altlinux.org}}}
\date{}
\maketitle

\begin{abstract}
Source code availability might tempt to <<just make a copy of it>>. This might lead to fragmentation which in its turn is capable of splintering the efforts, introducing incompatibilities, and even development stall --- but doing it properly may also heavily help with the feature diversity.
Questions to discuss are what's a fork and a merge, the price of divergence and convergence, and joint competition.
\end{abstract}

Когда мы что-то делаем, то нередко основываемся на уже существующем и дополняем или дорабатываем его. При этом вне зависимости от того, код это, документация или конфигурация "--- происходит фактическое <<раздвоение>> объекта. И если уже существующее продолжает развиваться, то это раздвоение называется <<форк>>. Если такие производные варианты сливаются полностью или частично, постоянно или периодически "--- такое объединение называется <<мерж>>.

Форк чреват тем, что либо усилия на развитие в основе схожих вещей будут дублироваться (без малого худший случай), либо потребуются дополнительные решимость, время и силы на периодическое <<втягивание>> наработок коллег, либо же произойдёт загнивание менее продуктивной ветки вместе с уникальными для неё наработками.

С другой стороны, ожидаемый форк (когда событие ответвления с целью дальнейшей разработки является ожидаемым и приветствуемым) может оказаться весьма мощным средством проверки самых различных идей реализацией: тогда наиболее удачные затем прямо или же после переработки (как правило, с учётом возникших замечаний коллег и параллельно происшедших изменений в затрагиваемой части) мержатся в более майнстримную ветку.

Мерж чреват в первую очередь неизбежной затратой времени "--- как на собственно техническую часть вопроса, так и на предварительное согласование с последующей утруской и усушкой вновь образовавшихся проблем.

Хорош же мерж тем, что уменьшение объёма разницы между развивающимися параллельно ветками приводит к уменьшению латентных затрат времени на отслеживание и втягивание интересных наработок коллег.

В докладе рассматриваются типичные виды, причины и последствия форков (причём всегда есть что добавить по опыту аудитории), а также более-менее соответствующие варианты мержей.


\end{document}


