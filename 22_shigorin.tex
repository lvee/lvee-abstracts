\documentclass[10pt, a5paper]{article}
\usepackage{ucs}
\usepackage[utf8]{inputenc}
\usepackage[T2A]{fontenc}
\usepackage[english, russian]{babel}
\usepackage{hyperref}
\usepackage{geometry}
\frenchspacing
\begin{document}
\title{Merge-a-fork или скрестим вилки}
\author{Михаил Шигорин\footnote{Киев, Украина, \url{mike@altlinux.org}}}
\date{}
\maketitle

\begin{abstract}
Source code availability might tempt to <<just make a copy of it>>. This might lead to fragmentation which in its turn is capable of splintering the efforts, introducing incompatibilities, and even development stall --- but doing it properly may also heavily help with the feature diversity.
Questions to discuss are what's a fork and a merge, the price of divergence and convergence, and joint competition.
\end{abstract}

Когда мы что-то делаем, то нередко основываемся на уже существующем и дополняем или дорабатываем его. При этом вне зависимости от того, код это, документация или конфигурация --- происходит фактическое <<раздвоение>> объекта. И если уже существующее продолжает развиваться, то это раздвоение называется <<форк>>. Если такие производные варианты сливаются полностью или частично, постоянно или периодически --- такое объединение называется <<мерж>>.

Форк чреват тем, что либо усилия на развитие в основе схожих вещей будут дублироваться (без малого худший случай), либо потребуются дополнительные решимость, время и силы на периодическое <<втягивание>> наработок коллег, либо же произойдёт загнивание менее продуктивной ветки вместе с уникальными для неё наработками.

С другой стороны, ожидаемый форк (когда событие ответвления с целью дальнейшей разработки является ожидаемым и приветствуемым) может оказаться весьма мощным средством проверки самых различных идей реализацией: тогда наиболее удачные затем прямо или же после переработки (как правило, с учётом возникших замечаний коллег и параллельно происшедших изменений в затрагиваемой части) мержатся в более майнстримную ветку.

Мерж чреват в первую очередь неизбежной затратой времени --- как на собственно техническую часть вопроса, так и на предварительное согласование с последующей утруской и усушкой вновь образовавшихся проблем.

Хорош же мерж тем, что уменьшение объёма разницы между развивающимися параллельно ветками приводит к уменьшению латентных затрат времени на отслеживание и втягивание интересных наработок коллег.

В докладе рассматриваются типичные виды, причины и последствия форков (причём всегда есть что добавить по опыту аудитории), а также более-менее соответствующие варианты мержей.


\end{document}


