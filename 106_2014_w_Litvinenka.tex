\documentclass [10pt, a5paper]{article}
\usepackage{pdfpages}
\usepackage{parallel}
\usepackage[T2A]{fontenc}
\usepackage{ucs}
\usepackage[utf8x]{inputenc}
\usepackage[polish,english,russian]{babel}
\usepackage{hyperref}
\usepackage{rotating}
\usepackage[inner=2cm,top=1.8cm,outer=2cm,bottom=2.3cm,nohead]{geometry}
\usepackage{listings}
\usepackage{graphicx}
\usepackage{wrapfig}
\usepackage{longtable}
\usepackage{indentfirst}
\usepackage{array}
\newcolumntype{P}[1]{>{\raggedright\arraybackslash}p{#1}}
\frenchspacing
\usepackage{fixltx2e} %text sub- and superscripts
\usepackage{icomma} % коскі ў матэматычным рэжыме
\PreloadUnicodePage{4}

\newcommand{\longpage}{\enlargethispage{\baselineskip}}
\newcommand{\shortpage}{\enlargethispage{-\baselineskip}}

\def\switchlang#1{\expandafter\csname switchlang#1\endcsname}
\def\switchlangbe{
\let\saverefname=\refname%
\def\refname{Літаратура}%
\def\figurename{Іл.}%
}
\def\switchlangen{
\let\saverefname=\refname%
\def\refname{References}%
\def\figurename{Fig.}%
}
\def\switchlangru{
\let\saverefname=\refname%
\let\savefigurename=\figurename%
\def\refname{Литература}%
\def\figurename{Рис.}%
}

\hyphenation{admi-ni-stra-tive}
\hyphenation{ex-pe-ri-ence}
\hyphenation{fle-xi-bi-li-ty}
\hyphenation{Py-thon}
\hyphenation{ma-the-ma-ti-cal}
\hyphenation{re-ported}
\hyphenation{imp-le-menta-tions}
\hyphenation{pro-vides}
\hyphenation{en-gi-neering}
\hyphenation{com-pa-ti-bi-li-ty}
\hyphenation{im-pos-sible}
\hyphenation{desk-top}
\hyphenation{elec-tro-nic}
\hyphenation{com-pa-ny}
\hyphenation{de-ve-lop-ment}
\hyphenation{de-ve-loping}
\hyphenation{de-ve-lop}
\hyphenation{da-ta-ba-se}
\hyphenation{plat-forms}
\hyphenation{or-ga-ni-za-tion}
\hyphenation{pro-gramming}
\hyphenation{in-stru-ments}
\hyphenation{Li-nux}
\hyphenation{sour-ce}
\hyphenation{en-vi-ron-ment}
\hyphenation{Te-le-pathy}
\hyphenation{Li-nux-ov-ka}
\hyphenation{Open-BSD}
\hyphenation{Free-BSD}
\hyphenation{men-ti-on-ed}
\hyphenation{app-li-ca-tion}

\def\progref!#1!{\texttt{#1}}
\renewcommand{\arraystretch}{2} %Іначай формулы ў матрыцы зліпаюцца з лініямі
\usepackage{array}

\def\interview #1 (#2), #3, #4, #5\par{

\section[#1, #3, #4]{#1 -- #3, #4}
\def\qname{LVEE}
\def\aname{#1}
\def\q ##1\par{{\noindent \bf \qname: ##1 }\par}
\def\a{{\noindent \bf \aname: } \def\qname{L}\def\aname{#2}}
}

\def\interview* #1 (#2), #3, #4, #5\par{

\section*{#1\\{\small\rm #3, #4. #5}}

\def\qname{LVEE}
\def\aname{#1}
\def\q ##1\par{{\noindent \bf \qname: ##1 }\par}
\def\a{{\noindent \bf \aname: } \def\qname{L}\def\aname{#2}}
}

\begin{document}
\title{Кандалы прогресса: авторское право и научные публикации}
\author{Антон Литвиненко, Киев, Украина\footnote{\url{tenebrosus.scriptor@gmail.com}, \url{http://lvee.org/en/abstracts/109}}}
\maketitle
\begin{abstract}
Copyright treats scientific publication equal to regular work of art, ignoring its specific nature, that gives some signs of natural monopolies to scientific publishing houses. This severely compli\-cates exchange of scientific information and, so, slows down the worldwide research. Present issues of access to publications to\-gether with some present and theoretical methods of their solu\-tion are discussed.
\end{abstract}
\subsection*{Открытость как основа мировой исследовательской деятельности}

Во времена Средневековья и более ранних цивилизаций, когда наука в современном виде еще не сформировалась, а объем знаний о мире был невелик, исследователи тщательно скрывали свои результаты, распространяя их максимум в узком круге учеников, зачастую придумывая специальные обозначения и шифры. При экспоненциальном нарастании количества информации в мире вообще и научного знания в частности наука перестала быть делом одиночек-энциклопедистов и вынуждена основываться на тесном сотрудничестве и интенсивном обмене информацией между исследователями и их группами. Каждая научная работа добавляет незначительный фрагмент информации в общую структуру научного знания, общий объем которого давно невозможно изучить, а тем более исследовать одному человеку. При этом, фактически, работа, не обнародованная в общедоступных источниках, не существует для научной общественности.

Доступность для исследователя научных работ его коллег исторически составляла некоторую проблему по ряду технических причин: необходимость физической доставки экземпляров журналов и книг, языковой барьер, сложность поиска, политические аспекты (например, «железный занавес»). Для решения этих проблем с переменным успехом применялся ряд технических подходов (например, реферативные журналы для облегчения поиска), пока они не были фактически решены за счет компьютеризации, четкого доминирования английского языка и краха биполярной политической системы.

Однако, на пути прогресса встали вопросы авторского права.

\subsection*{Особенности авторского права на научные публикации}

Фундаментальная информация о природе, получаемая в процессе научного познания, является общественным достоянием. Тем не менее, при работе исследования создается и продукт, который может становиться объектом интеллектуальной собственности. Это касается:
\begin{itemize}
  \item Разработки новых изобретений; устройств, методов, материалов и прочих результатов, которые могут быть запатентованы;
  \item Баз данных научной информации, которые, не обладая правами на содержащуюся информацию, обладают правами на ее компиляции и результаты поиска;
  \item Авторское право на научные публикации.
\end{itemize}

С точки зрения авторского права, научные публикации являются обычными литературными произведениями.

\subsection*{Принципиальные отличия между научной публикацией и литературным произведением в контексте авторского права}
\begin{enumerate}
  \item Литературное произведение можно опубликовать с помощью любой организации, способной подготовить его к печати (при необходимости) и физически создать нужное количество экземпляров. Именитость издания может иметь некоторое значение для дальнейшей судьбы произведения, но основным является физическое наличие тиража. В то же время, для научного произведения место опубликования имеет критическое значение:\begin{itemize}
  \item Научное произведение должно пройти рецензирование в той или оной форме. Именитость места публикации (издательства, периодического издания, представительства конференции и т.\,д.) играет значительную роль и выступает показателем качества произведения (в том числе через систему наукометрических параметров, характеризующих место публикации "--- например, импакт"=фактор журнала).
  \item Авторитет места публикации влияет на принятие работы как научной, а также ее добавление в общую структуру научного знания "--- будут ли другие исследователи читать работу, воспримут ли ее всерьез, будет ли она доступна в научных поисковых системах.
  \item В некоторых случаях в качестве приемлемых научных работ воспринимаются только опубликованные в местах из определенного авторизованного списка (например, \linebreak списки ВАКов).
  \item Выход на рынок научных публикаций достаточно затруднен, так как малоизвестному издателю сложно претендовать на поступление для публикации интересных высококачественных работ.
\end{itemize}


  \item Первичные научные работы (по материалам оригинальных исследований), как правило, содержат достаточно уникальную информацию, которая редко где-либо дублируется полностью. Таким образом, необходимость доступа к конкретной публикации (а в процессе текущих работ теоретически может понадобиться доступ к любой ранее опубликованной работе) может быть критической без возможности замены какой-либо другой публикацией.
  \item Как правило, исследователь не получает значительного дохода от продажи своих научных публикаций (часто не получает его вообще), мотивы публикации и распространения своих трудов более нематериальны.
\end{enumerate}

Таким образом, рынок научного издательства имеет черты природной монополии (олигополии), в котором интерес издателя и автора существенно различен. Однако, это совершенно игнорируется при юридическом регулировании.

\subsection*{Типичная политика научных издательств}

\begin{enumerate}
  \item Продается подписка на печатный журнал и/или на онлайн-доступ к электронным версиям статей за все время или определенный период; а также продается онлайн-доступ к отдельным публикациям при отсутствии общей подписки;
  \item Подписка стоит дорого. Например, годовая подписка на журнал Angewandte Chemie стоит \$11529 (печатный + онлайн) [1]. А таких журналов только по химии десятки самых важных, а по широкому набору дисциплин (например, для научной библиотеки) сотни и тысячи. Доступ же к единичной статье стоит десятки долларов. Так, библиотека Гарварда в 2013 году сделала заявление о непомерных расходах на журнальные подписки, призвав ученых публиковаться в бесплатно распространяемых журналах [2].
  \item Подписки могут продаваться сразу на группу журналов. Фактически, часть журналов может продаваться в довесок.
  \item Гонорары авторам невелики или вовсе отсутствуют.
  \item Рецензенты не получают денег за работу.
  \item Условия лицензионных соглашений предусматривают передачу практически всех прав издательству.
  \item Высокая стоимость не гарантирует однозначно высокого качества научных публикаций.
  \item В последнее время большинство научных журналов было скуплено тремя большими издательскими домами: Wiley, Elsevier, Springer.
\end{enumerate}

Издатель научных публикаций торгует воздухом, пользуясь монопольным положением. Такая ситуация ведет к затруднению научных исследований, повышению порога вхождения в научную деятельность, недоступности результатов оригинальных исследований для широких масс интересующихся людей.

В последние годы это (а также ряд других причин вроде поддержки SOPA и PIPA) вызвало даже попытки «бунта» среди ученых, в том числе и весьма известных, "--- в частности, бойкот издательства Elsevier [3,4].

\subsection*{Методы борьбы}

\begin{enumerate}
  \item Заграница нам поможет. Ссылки, которые нужно скачать, направляются знакомым «утекшим мозгам» или временно стажирующимся/работающим за бугром коллегам или друзьям. Они скачивают статью, пользуясь местной подпиской.
  \item Расширенные версии предыдущих. Сообщества для реквестов вроде жжшного pdf.livejournal.com, межбиблиотечные подписки и т.д.
  \item Связка проектов sci-hub.org и libgen.org. Первый представляет собой систему прокси с парсерами и прочими техническими дополнениями для автоматического доступа к журналам через компьютеры в западных университетах. Второй "--- огромный фонд научных публикаций, ставящий цель собрать большинство научных публикаций в истории мировой науки вообще. Скачанные через sci-hub статьи автоматически попадают в libgen, при повторной скачке предлагается уже готовая копия. При неудачной попытке поиска через libgen предлагается открыть статью через sci-hub.
  \item Опция издательства "--- авторы платят за публикацию, после чего она становится открытой всем.
  \item Open-access журналы "--- аналог предыдущего, но применяется для всего журнала. Как правило, не имеет печатной версии. В то же время, могут финансироваться как авторами, так и из других источников. Некоторые исследователи критикуют низкое качество рецензирования, но проводимые ими исследования не вполне строгие (традиционные журналы страдают теми же проблемами) [3,5], и эти проблемы могут быть объяснены новизной явления. В качестве примера open-access журнала можно привести PLOS Biology, который взнимает с авторов зависящую от страны плату и публикует работу под лицензией Creative Commons CC-BY [6].
  \item Тома журналов выкачиваются постатейно ботами и выкладываются на торрентах.
  \item Максимально используются оставшиеся у автора права "--- рассылка небольшого числа авторских копий, публикация препринтов (не все журналы). Публикация препринтов эффективна в интеграции с Google Scholar (рядом со ссылкой на статью позволяет скачать найденный препринт с другого адреса).
  \item Использование временного открытия некоторых статей во время рекламных кампаний, доступа по паролям для рецензентов через проприетарные поисковики и прочих лазеек.
\end{enumerate}

Таким образом, большинство имеющихся методов борьбы являются откровенно пиратскими, а существующие легально не решают целостной проблемы (open access распространяется только на работы, явно опубликованные по этой модели).

Для существенного прогресса в вопросе авторских прав на научные работы следует разрабатывать и вносить изменения в законодательство об авторском праве, выделяя научные публикации в особый вид произведений со специальным регулированием. Например, существенное уменьшение (до нескольких лет) времени перехода в общественное достояние.

Таким образом, несмотря на принципиальную приверженность открытости информации, научному сообществу еще только предстоит пройти путь по либерализации недопустимой ситуации с авторскими правами, который уже успешно проходит общество программистское.

P.\,S. Во время обсуждения доклада слушателями были названы еще некоторые цифровые хранилища и библиотеки, из которых можно бесплатно или за разумную цену поличить желаемую литературу по определенным областям знаний. В частности, сайт CiteSeerX [7] (спасибо Алексею Чеусову).

\begin{thebibliography}{9}
 \bibitem{L1} \url{http://ordering.onlinelibrary.wiley.com/subs.asp?ref=1521-3757&doi=10.1002/(ISSN)1521-3757}
 \bibitem{L2} \url{http://www.vestifinance.ru/articles/22492/print}
 \bibitem{L3} \url{http://theoryandpractice.ru/posts/8440-znaniya-dlya-vsekh}
 \bibitem{L4} \url{http://www.svoboda.org/content/article/24892099.html}
 \bibitem{L5} \url{http://habrahabr.ru/company/cyberleninka/blog/197946/}
 \bibitem{L6} \url{http://www.plosbiology.org/static/information}
 \bibitem{L7} \url{http://citeseerx.ist.psu.edu}
\end{thebibliography}
\end{document}
