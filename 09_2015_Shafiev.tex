\documentclass[10pt, a5paper]{article}
\usepackage{pdfpages}
\usepackage{parallel}
\usepackage[T2A]{fontenc}
\usepackage{ucs}
\usepackage[utf8x]{inputenc}
\usepackage[polish,english,russian]{babel}
\usepackage{hyperref}
\usepackage{rotating}
\usepackage[inner=2cm,top=1.8cm,outer=2cm,bottom=2.3cm,nohead]{geometry}
\usepackage{listings}
\usepackage{graphicx}
\usepackage{wrapfig}
\usepackage{longtable}
\usepackage{indentfirst}
\usepackage{array}
\newcolumntype{P}[1]{>{\raggedright\arraybackslash}p{#1}}
\frenchspacing
\usepackage{fixltx2e} %text sub- and superscripts
\usepackage{icomma} % коскі ў матэматычным рэжыме
\PreloadUnicodePage{4}

\newcommand{\longpage}{\enlargethispage{\baselineskip}}
\newcommand{\shortpage}{\enlargethispage{-\baselineskip}}

\def\switchlang#1{\expandafter\csname switchlang#1\endcsname}
\def\switchlangbe{
\let\saverefname=\refname%
\def\refname{Літаратура}%
\def\figurename{Іл.}%
}
\def\switchlangen{
\let\saverefname=\refname%
\def\refname{References}%
\def\figurename{Fig.}%
}
\def\switchlangru{
\let\saverefname=\refname%
\let\savefigurename=\figurename%
\def\refname{Литература}%
\def\figurename{Рис.}%
}

\hyphenation{admi-ni-stra-tive}
\hyphenation{ex-pe-ri-ence}
\hyphenation{fle-xi-bi-li-ty}
\hyphenation{Py-thon}
\hyphenation{ma-the-ma-ti-cal}
\hyphenation{re-ported}
\hyphenation{imp-le-menta-tions}
\hyphenation{pro-vides}
\hyphenation{en-gi-neering}
\hyphenation{com-pa-ti-bi-li-ty}
\hyphenation{im-pos-sible}
\hyphenation{desk-top}
\hyphenation{elec-tro-nic}
\hyphenation{com-pa-ny}
\hyphenation{de-ve-lop-ment}
\hyphenation{de-ve-loping}
\hyphenation{de-ve-lop}
\hyphenation{da-ta-ba-se}
\hyphenation{plat-forms}
\hyphenation{or-ga-ni-za-tion}
\hyphenation{pro-gramming}
\hyphenation{in-stru-ments}
\hyphenation{Li-nux}
\hyphenation{sour-ce}
\hyphenation{en-vi-ron-ment}
\hyphenation{Te-le-pathy}
\hyphenation{Li-nux-ov-ka}
\hyphenation{Open-BSD}
\hyphenation{Free-BSD}
\hyphenation{men-ti-on-ed}
\hyphenation{app-li-ca-tion}

\def\progref!#1!{\texttt{#1}}
\renewcommand{\arraystretch}{2} %Іначай формулы ў матрыцы зліпаюцца з лініямі
\usepackage{array}

\def\interview #1 (#2), #3, #4, #5\par{

\section[#1, #3, #4]{#1 -- #3, #4}
\def\qname{LVEE}
\def\aname{#1}
\def\q ##1\par{{\noindent \bf \qname: ##1 }\par}
\def\a{{\noindent \bf \aname: } \def\qname{L}\def\aname{#2}}
}

\def\interview* #1 (#2), #3, #4, #5\par{

\section*{#1\\{\small\rm #3, #4. #5}}

\def\qname{LVEE}
\def\aname{#1}
\def\q ##1\par{{\noindent \bf \qname: ##1 }\par}
\def\a{{\noindent \bf \aname: } \def\qname{L}\def\aname{#2}}
}

\begin{document}
\title{Опыт выбора и применения современных систем мониторинга}
\author{Наим Шафиев, Баку, Азербайджан\footnote{\url{shafiev@gmail.com}, \url{http://lvee.org/ru/abstracts/159}}}
\maketitle
\begin{abstract}
Modern heterogeneous infrastructure, which contains different types of network devices and servers, needs modern monitoring systems. Also the main problem of finding proper solution is that it should  fulfil requirements of flexibility, openness, good support from community. 
The article presents an overview and experience  of modern free monitoring systems, which fulfil spoken above requirements in case of a middle"=size ISP.
\end{abstract}
\subsection*{Основные требования}

Проблема роста и взаимозаменяемости сотрудников для решения задач мониторинга и DevOps проявляются в любой компании при росте от нескольких человек до так называемого среднего размера.

Рассмотрим, какие основные требования предъявляются к системам мониторинга в современных условиях:

\begin{itemize}
  \item опрос агентами ресурсов (CPU, Disk I/O, Ram, Network) серверов;
  \item опрос сетевых устройств различных устройств (очень желательно,чтобы были готовые MIB);
  \item опрос VMware"=серверов;
  \item гибкая система оповещений с многоуровневой системой зависимости;
  \item opensource.
\end{itemize}

Также очень важны понятная документация и легкий старт "--- как минимум, для взаимозаменяемости сотрудников.

\subsection*{Практика использования}

Ниже представлены особенности, родившиеся из опыта поиска и использования различных решений (таких, как Nagios, Zabbix, LibreNMS, NetXMS) на  размере более 500 активных узлов в ISP среднего размера.
Рассмотрев все эти решения мы пришли к следующим выводам:

\begin{itemize}
  \item Nagios "--- к сожалению, началось разделение на бесплатную Core"=версию и XI"=версию, все это привело к появлению более технологичного форка Icinga. Также из минусов стоит отметить не доведенную до промышленного использования многоуровневую систему взаимосвязи объектов и сервисов.
  \item Zabbix "--- лучший opensource"=продукт, возможен высокий уровень детализации данных, развитое community. Из минусов "--- сложность в настройке, сложно дорабатывать.
  \item LibreNMS "--- наиболее перспективный свободный продукт. \linebreak Очень легкий старт (сравнимый с PRTG), использование стандартных компонентов. Включает огромное количество готовых MIB. Из минусов "--- невозможность без осложнений задать детализированные графики (менее 5 минут), нету развитой системы зависимостей (в данный момент автор решает эту задачу).
  \item NetXMS "--- один из наиболее технологичных продуктов индустрии. Содержит высокопроизводительное ядро, возможен высокий уровень детализации и другие стандартные для проприетарных систем функции. Из минусов отмечается сложный старт, ручная настройка, малое community.
\end{itemize}

Автор в данный момент дорабатывает LibreNMS для замены связки Nagios/Zabbix+MRTG.

\end{document}
