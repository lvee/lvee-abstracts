\documentclass[10pt, a5paper]{article}
\usepackage{pdfpages}
\usepackage{parallel}
\usepackage[T2A]{fontenc}
\usepackage{ucs}
\usepackage[utf8x]{inputenc}
\usepackage[polish,english,russian]{babel}
\usepackage{hyperref}
\usepackage{rotating}
\usepackage[inner=2cm,top=1.8cm,outer=2cm,bottom=2.3cm,nohead]{geometry}
\usepackage{listings}
\usepackage{graphicx}
\usepackage{wrapfig}
\usepackage{longtable}
\usepackage{indentfirst}
\usepackage{array}
\newcolumntype{P}[1]{>{\raggedright\arraybackslash}p{#1}}
\frenchspacing
\usepackage{fixltx2e} %text sub- and superscripts
\usepackage{icomma} % коскі ў матэматычным рэжыме
\PreloadUnicodePage{4}

\newcommand{\longpage}{\enlargethispage{\baselineskip}}
\newcommand{\shortpage}{\enlargethispage{-\baselineskip}}

\def\switchlang#1{\expandafter\csname switchlang#1\endcsname}
\def\switchlangbe{
\let\saverefname=\refname%
\def\refname{Літаратура}%
\def\figurename{Іл.}%
}
\def\switchlangen{
\let\saverefname=\refname%
\def\refname{References}%
\def\figurename{Fig.}%
}
\def\switchlangru{
\let\saverefname=\refname%
\let\savefigurename=\figurename%
\def\refname{Литература}%
\def\figurename{Рис.}%
}

\hyphenation{admi-ni-stra-tive}
\hyphenation{ex-pe-ri-ence}
\hyphenation{fle-xi-bi-li-ty}
\hyphenation{Py-thon}
\hyphenation{ma-the-ma-ti-cal}
\hyphenation{re-ported}
\hyphenation{imp-le-menta-tions}
\hyphenation{pro-vides}
\hyphenation{en-gi-neering}
\hyphenation{com-pa-ti-bi-li-ty}
\hyphenation{im-pos-sible}
\hyphenation{desk-top}
\hyphenation{elec-tro-nic}
\hyphenation{com-pa-ny}
\hyphenation{de-ve-lop-ment}
\hyphenation{de-ve-loping}
\hyphenation{de-ve-lop}
\hyphenation{da-ta-ba-se}
\hyphenation{plat-forms}
\hyphenation{or-ga-ni-za-tion}
\hyphenation{pro-gramming}
\hyphenation{in-stru-ments}
\hyphenation{Li-nux}
\hyphenation{sour-ce}
\hyphenation{en-vi-ron-ment}
\hyphenation{Te-le-pathy}
\hyphenation{Li-nux-ov-ka}
\hyphenation{Open-BSD}
\hyphenation{Free-BSD}
\hyphenation{men-ti-on-ed}
\hyphenation{app-li-ca-tion}

\def\progref!#1!{\texttt{#1}}
\renewcommand{\arraystretch}{2} %Іначай формулы ў матрыцы зліпаюцца з лініямі
\usepackage{array}

\def\interview #1 (#2), #3, #4, #5\par{

\section[#1, #3, #4]{#1 -- #3, #4}
\def\qname{LVEE}
\def\aname{#1}
\def\q ##1\par{{\noindent \bf \qname: ##1 }\par}
\def\a{{\noindent \bf \aname: } \def\qname{L}\def\aname{#2}}
}

\def\interview* #1 (#2), #3, #4, #5\par{

\section*{#1\\{\small\rm #3, #4. #5}}

\def\qname{LVEE}
\def\aname{#1}
\def\q ##1\par{{\noindent \bf \qname: ##1 }\par}
\def\a{{\noindent \bf \aname: } \def\qname{L}\def\aname{#2}}
}

\begin{document}
\title{Qucs "--- свободный симулятор электронных схем: новые возможности релиза 0.0.19}
\author{Вадим Кузнецов, Калуга, РФ\footnote{\url{ra3xdh@gmail.com}, \url{http://lvee.org/en/abstracts/162}}}
\maketitle
\begin{abstract}
Qucs (Quite Universal Circuit Simulator) is open source cir\-cuit simulation CAD tool. The main features of Qucs are conside\-red. New features of upcoming 0.0.19 release and spice4qucs subsystem are reviewed. Spice4qucs allows to simulate circuits from Qucs using external simulator kernels such as Ngspice and Xyce.
\end{abstract}
В настоящее время существует не так уж и много open"=source САПР. Тем не менее, среди САПР для электроники (EDA) есть весьма достойные продукты. Доклад посвящён симулятору электронных схем с открытым исходным кодом Qucs \url{http://qucs.sourceforge.net}. Qucs написан на С++ с использованием фреймворка Qt4. Qucs является кроссплатформенным и выпущен для ОС Linux, Windows и MacOS. Текущей версией проекта является 0.0.18. В настоящее время ведётся подготовка к релизу версии 0.0.19.

Разработку данной САПР начали в 2004 году немцы Michael Margraf и Stefan Jahn (в настоящее время не активны). Сейчас Qucs разрабатывается интернациональной командой, в которую входит автор статьи. Руководителями проекта являются Frans Schreuder и Guilherme Torri.

Qucs позволяет проводить следующие виды моделирования:

\begin{enumerate}
\item Моделирование на постоянном токе (DC analysis)
\item Моделирование в частотной области (AC analysis)
\item Моделирование во временной области (Transient analysis)
\item Параметрический анализ (Parameter sweep)
\item Моделирование S"=параметров в частотной области (S"=parame\-ter)
\item Синтез пассивных фильтров, согласованных схем, расчёт коаксиальных 
и микрополосковых линий.
\end{enumerate}

Результаты моделирования можно визуализировать в виде графиков в декартовых (2D и 3D) и полярных координатах, а также в виде таблиц и диаграмм Смита.

В настоящее время существуют следующие open"=source средства моделирования 
электронных схем:
\begin{enumerate}
\item Ngspice \url{http://ngspice.org} "--- консольный симулятор элект\-ронных схем. 
Совместим с индустриальным стандартом моделей электронных компонентов SPICE.
\item Xyce \url{http://xyce.sandia.gov} "--- новейший spice"=совместимый консольный 
симулятор, поддерживает параллельные вычисления через openMPI. Вышел в 2014 
году. Совместим с индустриальным стандартом моделей электронных компонентов 
SPICE.
\end{enumerate}

Недостатком вышеперечисленных симуляторов является отсутствие графического 
интерфейса, что сильно затрудняет ввод схемы. Для преодоления данного 
недостатка был разработан набор патчей spice4qucs, разработанный автором 
совместно с Mike Brinson (London Metropolitan University). Данный набор патчей позволяет использовать Qucs в качестве фронтенда для Ngspice или Xyce. Включение данного набора патчей в основную ветку ожидается в версии 0.0.19. В настоящее время поддерживаются все основные виды моделирования и компоненты. Текущий статус разработки можно отследить в репозитории проекта: \url{http://github.com/Qucs/qucs/issues/77}

Подсистема spice4qucs позволяет:
\begin{enumerate}
\item Моделировать схему Qucs при помощи внешнего симулятора Ngspice или Xyce
\item Использовать систему параметрического моделирования, совместимую со SPICE
\item Использовать постпроцессор Ngnutmeg
\item Использовать SPICE"=модели из документации электронных компонентов без 
ограничений
\item Использовать специфические виды моделирования, совместимые с Ngspice и 
Xyce: (Fourier analysis, Distortion analysis, Noise analysis)
\item Проводить моделирование при помощи скрипта Ngnutmeg, задаваемого 
пользователем.
\end{enumerate}

Функция поддержки внешних симуляторов, реализуемая подсистемой Spice4qucs, не 
имеет аналогов в проприетарном ПО.

Будущими задачами является разработка в следующих перспективных направлениях:
\begin{enumerate}
\item Расширение библиотеки компонентов
\item Разработка редактора библиотек
\item Разработка системы связи с KiCAD
\item Поддержка электромагнитного симулятора openEMS
\item Разработка системы экспорта моделей в формат Verilog"=A
\end{enumerate}

Таким образом, можно сделать вывод о том, что Qucs представляет собой 
быстроразвивающуюся САПР, по отдельным параметрам не уступающую проприетарным аналогам.  
можно рекомендовать Qucs для моделирования электронных схем в академических 
целях, на малых предприятиях и индивидуальным разработчикам электроники, а в 
некоторых случаях Qucs можно использовать и на крупных предприятиях для замены проприетарного ПО, закупаемого за рубежом.

\end{document}
