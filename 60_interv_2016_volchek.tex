\documentclass[10pt, a5paper]{article}
\usepackage{pdfpages}
\usepackage{parallel}
\usepackage[T2A]{fontenc}
\usepackage{ucs}
\usepackage[utf8x]{inputenc}
\usepackage[polish,english,russian]{babel}
\usepackage{hyperref}
\usepackage{rotating}
\usepackage[inner=2cm,top=1.8cm,outer=2cm,bottom=2.3cm,nohead]{geometry}
\usepackage{listings}
\usepackage{graphicx}
\usepackage{wrapfig}
\usepackage{longtable}
\usepackage{indentfirst}
\usepackage{array}
\newcolumntype{P}[1]{>{\raggedright\arraybackslash}p{#1}}
\frenchspacing
\usepackage{fixltx2e} %text sub- and superscripts
\usepackage{icomma} % коскі ў матэматычным рэжыме
\PreloadUnicodePage{4}

\newcommand{\longpage}{\enlargethispage{\baselineskip}}
\newcommand{\shortpage}{\enlargethispage{-\baselineskip}}

\def\switchlang#1{\expandafter\csname switchlang#1\endcsname}
\def\switchlangbe{
\let\saverefname=\refname%
\def\refname{Літаратура}%
\def\figurename{Іл.}%
}
\def\switchlangen{
\let\saverefname=\refname%
\def\refname{References}%
\def\figurename{Fig.}%
}
\def\switchlangru{
\let\saverefname=\refname%
\let\savefigurename=\figurename%
\def\refname{Литература}%
\def\figurename{Рис.}%
}

\hyphenation{admi-ni-stra-tive}
\hyphenation{ex-pe-ri-ence}
\hyphenation{fle-xi-bi-li-ty}
\hyphenation{Py-thon}
\hyphenation{ma-the-ma-ti-cal}
\hyphenation{re-ported}
\hyphenation{imp-le-menta-tions}
\hyphenation{pro-vides}
\hyphenation{en-gi-neering}
\hyphenation{com-pa-ti-bi-li-ty}
\hyphenation{im-pos-sible}
\hyphenation{desk-top}
\hyphenation{elec-tro-nic}
\hyphenation{com-pa-ny}
\hyphenation{de-ve-lop-ment}
\hyphenation{de-ve-loping}
\hyphenation{de-ve-lop}
\hyphenation{da-ta-ba-se}
\hyphenation{plat-forms}
\hyphenation{or-ga-ni-za-tion}
\hyphenation{pro-gramming}
\hyphenation{in-stru-ments}
\hyphenation{Li-nux}
\hyphenation{sour-ce}
\hyphenation{en-vi-ron-ment}
\hyphenation{Te-le-pathy}
\hyphenation{Li-nux-ov-ka}
\hyphenation{Open-BSD}
\hyphenation{Free-BSD}
\hyphenation{men-ti-on-ed}
\hyphenation{app-li-ca-tion}

\def\progref!#1!{\texttt{#1}}
\renewcommand{\arraystretch}{2} %Іначай формулы ў матрыцы зліпаюцца з лініямі
\usepackage{array}

\def\interview #1 (#2), #3, #4, #5\par{

\section[#1, #3, #4]{#1 -- #3, #4}
\def\qname{LVEE}
\def\aname{#1}
\def\q ##1\par{{\noindent \bf \qname: ##1 }\par}
\def\a{{\noindent \bf \aname: } \def\qname{L}\def\aname{#2}}
}

\def\interview* #1 (#2), #3, #4, #5\par{

\section*{#1\\{\small\rm #3, #4. #5}}

\def\qname{LVEE}
\def\aname{#1}
\def\q ##1\par{{\noindent \bf \qname: ##1 }\par}
\def\a{{\noindent \bf \aname: } \def\qname{L}\def\aname{#2}}
}

%\switchlang{be}
%\usepackage{color}
\begin{document}
\title{Интервью с участниками}
%\author{}
\date{}
\maketitle

По традиции в сборник материалов входят интервью, в которых активные участники
сообщества open source делятся своим мнением о свободном ПО, открытых
технологиях, роли и месте свободных лицензий, рассказывают, как видят проблематику
свободных проектов. В этот раз мы решили выбрать в качестве тематического русла
использование свободных лицензий в сфере, отличной от программного обеспечения:
в таких областях, как свободный контент, свободное аппаратное обеспечение и криптовалюты.

\section{Евгений Хоружий "---  Минск, Беларусь}
%\begin{figure}[ht]
%\centering{\includegraphics[width=4cm]{49_spons_altoros.jpg}}
%\end{figure}

{\noindent \bf LVEE: Для начала, расскажи пару слов о себе в контексте разработки встраиваемых систем.}

{\noindent \bf Евгений Хоружий:} Ну и вопрос\ldots С 2005 года в этой сфере как разработчик, в разной степени приближенности к железу, и не к железу тоже, вообще к системному софту в основном. Да и паяльник с 3 класса в руках держал, то есть когда пошел в embedded, хорошо представлял, что там внутри.

{\noindent \bf L: Удачное сочетание одного с другим.}

{\noindent \bf Е:} Да, к сожалению почему-то в наше время это редко встречается, и очень больно смотреть на нынешних программистов, занимающихся embedded.
{\noindent \bf L: Как получился интерес к open hardware как таковому. От программного обеспечения, или как?} 

{\noindent \bf Е:}  Честно говоря, я как-то не занимался разработкой именно железа, не участвовал в аппаратных открытых проектах, участвовал в программных\ldots И вообще больше в области встраиваемоего ПО работал. Но в принципе, не вижу здесь большой разницы. Другое дело, что меня мучила ностальгия радиокружка, когда в одном месте собиралось много людей, которым было интересно в чем-то ковыряться, которые друг другу помогали, подтрунивали, мешали и так далее. И вот тогда, собственно говоря, идея хакерспейса начала кристаллизоваться. 

{\noindent \bf L: То есть просто периодически  то, с чем вы работали, было родственно open-hardware или может быть опенсорсным проектам, связанным с embedded.}

{\noindent \bf Е:}  Скорее, проектам, связанным с embedded. Например, был проект OpenInkpot,  прошивки для электронных книг. Мы ведь сами аппаратные электронные книги не делали, это именно прошивка, достаточно низкоуровневая, в общем-то, и да, это был открытый проект.

Ну и свои проекты если делаю, когда не стыдно что-то показать, обычно в гитхаб выкладываю. И понятно, что если делаю что-то в хакерспейсе, то всегда рассказываю что и как.

{\noindent \bf L: А кстати, как вообще возник Минский Хакерспейс? С чего это началось?}

{\noindent \bf Е:} С моего выступления на LVEE кажется\ldots Так, не скромничая :)

На самом деле,  да, где-то в году 12 я закинул идею, я уже не помню, на LVEE или на линуксовке. Идея, понятное дело, не моя, хакерспейсы существуют во многих странах и достаточно давно. И мне тоже давно такого хотелось, и хотелось верить, что не мне одному такая штука нужна.



{\noindent \bf L: А интерес начался с того, что ты заглянул на действующий  хакерспейс или тоже где-то что-то прочитал, посмотрел презентацию?}

{\noindent \bf Е:} Скорее я читал: периодически натыкался на материалы,  что в таком то хакерспейсе что-то сделали. Есть такой ресурс,  \url{http://hackaday.com}, который время от времени публикует всякие забавные штуки, которые как раз в этих хакерспесах и делаются.

Честно говоря, не помню, где я это сам это увидел, но нужно учитывать и радиокружок тот самый из школы. В общем, я прекрасно понимал, как эта штука работает, другое дело что в отличие от школьных времен, чуть-чтуь другая форма, но разница совсем незначительная.

{\noindent \bf L: От идеи до возникновения действующего хакерспейса прошло много времени?}

{\noindent \bf Е:} О много, я даже не помню сколько времени. Наверное год, не меньше. Может немного быстрее.  Сначала мы нашли, вариант в БГУ, помещение в здании юридического колледжа. Пока договаривались, узнали, что там еще по вечерам курсы английского будут проходить, и только потом можно будет его полностью использовать.

В общем это была первая попытка, она показала, что все-таки чувство собственности очень важно, что помещение должно полностью в распоряжении быть, и чтобы доступ был в любое время. Активность в том первом варианте была нулевой, в том числе и из-за режима доступа.

{\noindent \bf L: Да, я помню\ldots И невозможность попасть туда на выходных, кажется\ldots}

{\noindent \bf Е:} Да-да, куча проблем, некоторые но вроде были преодолимы, но все равно в сумме это давало фактически ноль. Поэтому пришлось оттуда уйти. И опять же, немного оказались разные представления у нас и у пригласившей нас стороны о том, что такое хакерспейс. Они надеялись, что мы будем площадкой, из которой вырастают публичные проекты. А по нашему определению хакерспейс "--- площадка для каких-то проектов типа хобби, для, людей которые в первую очередь просто хотят этим заниматься. Если это вырастает в какой-то проект "--- отлично, но это необязательно. 

Потом мы пытались в МЕ100 что-то делать, когда это был заводской цех: большая площадка для мероприятий, коворкинги "--- что там только ни пытались организовать. Но мы опять же столкнулись с тем, что находиться внутри чужой площадки очень сложно.  Особенно учитывая, что она не была отремонтирована, и мы сами участвовали в ремонте. А к тому времени, когда ремонт стал приближаться к какой-то точке, когда помещением стало можно пользоваться, МЕ100 вдруг стало ЦЭХом, художественной галереей. И тут уже встали опять вопросы, находиться внутри галереи совсем неудобно. В общем, оттуда тоже пришлось уйти.

Но к счастью подвернулось текущее помещение. В плане годности для хакерспейса почти идеально кроме местоположения, пожалуй. Во всяком случае, на фоне предыдущих оно лучше соответствует. Отдельный вход, никаких коворкингов, готовое подвальное помещение, в котором не требуется никакого ремонта. И оно работает. 

{\noindent \bf L: В общем, одиссея пока закончилась.}

{\noindent \bf Е:} Да, если по срокам сказать, два года назад приблизительно мы заселились.

{\noindent \bf L: Вопрос, который часто возникает в интервью по поводу хакерспейсов, он наверняка всем имеющим отношение к хакерспейсам надоел, но все равно он и тут прозвучит тоже. Что по поводу регламентации техники безопасности? Как-нибудь специально прорабатывали эту тему? }

{\noindent \bf Е:} Пару раз конечно поднимался вопрос, но пока людей все-таки так уж много у нас, и поэтому удается все это в неформальном русле проговаривать.  Но тем не мене, в неформальном русле с новыми участником оговариваем\ldots Стараемся следить конечно, когда видим что это надо, но такого критичного оборудования у нас там в принципе нет, так, электробезопасность. Сейчас вот зарегистрировали организацию, юрлицо, на которое переоформляться будем. Действительно, есть вопросы, кто будет ответственным, если что, но тем  не менее, эти вопросы проблем не создают.

{\noindent \bf L: Ну а вот мы, кстати, плавно перешли к официальному статусу. Не так давно все-таки состоялась официальная регистрация.}

{\noindent \bf Е:} Да, эти перипетии  чуть ли не дольше чем поиск места для хакерспейса тянулись. Когда начали его создавать, было понятно, что нужен легальный статус, чтобы иметь возможность собирать деньги, официальные членские взносы, возможность арендовать помещение, именно на организацию. Конечно сразу встали вопросы, какая должна быть организация. Проще всего зарегистрировать коммерческую, или некоммерческкую, учрежденную не коллективно, а самим собственником. Но лучше всего фактический структуре хакерспейса соответствует конечно общественное объединение. Есть у нас в законодательстве такое понятие: разные клубы, например, сюда относятся.

{\noindent \bf L: Опыт Минского велосипедного общества, наверное, здорово пригодился при оформлении всех этих вещей.}

{\noindent \bf Е:} Естественно что у меня уже был опыт, я в этом обществе был долгое время председателем правления, хорошо знал, как все это делается,  внутреннюю кухню и требования.

И начали пытаться регистрировать организацию в разных вариантах, в том числе и как республиканское общественное объединение. По-моему, 5 или 6 попыток было. Последний вариант не совсем удобный, но по крайней мере работающий: это первичная организация Белорусского общества изобретателей и рационализаторорв. Оно существует с 90-х годов, а может и раньше, с советских времен. У него есть даже первичные ячейки на заводах. 

{\noindent \bf L: Нельзя не отметить, что вообще говоря, тематика родственная.}

{\noindent \bf Е:} Да, устав практически полностью имел все что нам надо, плюс теоретически будучи их  подразделением можно получить льготы на аренду гос. собственности. Но главное, получилось зарегистрироваться. И первичная организация заинтересована, чтобы такая ячейка была. У них сейчас не очень много активности\ldots
 
{\noindent \bf L: Последний вопрос наверное, хоть и большой. Насколько активность хакерспейса, его участников, близка к понятиям свободного программного и аппаратного обеспечения как-такого. Можно ли сказать, что примерно столько-то участников Минского Хакерспейса имеет представление о свободных лицензиях?}

{\noindent \bf Е:} На самом деле как раз те, кто приходит в к нам, уже имеют представление. Да, конечно есть разные люди, но чисто <<железячников>> на самом деле почти нету. Приходят именно те, кто либо имел до этого опыт с программированием, либо уже немного владеют темой и про open source знают достаточно. Сказывается, что инициировалось все в среде линуксойдов, и среди участников достаточно много тех, кто достаточно давно с Linux связан. Ну и есть пара человек чисто по аппаратной части.


{\noindent \bf L: Ну и кстати, как я так понимаю, тему открытого аппаратного обеспечения очень здорово распиарили несколько проектов типа Arduino, на которых достаточно легко  создаются разные аппаратные устройства?}

{\noindent \bf Е:} Вообще в мире да, и Arduino такой интересный пример потому что вообще-то ничего такого особенно революционного в нем нет. Обычный микроконтроллер почти ничего на него не навесили, просто сделали возможность любому человеку очень легко стартовать свой проект "--- что-то написать под него, загрузить, запустить\ldots Увидеть, что лампочка мигает :)

{\noindent \bf L: И на удивление каким то образом быстро собралась база проектов, почти готовых, совсем готовых, совсем не готовых "--- но тем не менее выложенных в открытый доступ. Как-то получилось, что их оказалось достаточно много, чтобы еще более упростить начальный этап.}

{\noindent \bf Е:} Да, вначале авторы Arduino сами библиотеки какие-то выкладывали, примеры проектов, а потом уже сформировалось достаточно большое сообщество. Профессиональные <<железячники>> вначале с этого всего смеялись. Там такой подход, что из пушки по воробьям иногда стреляют\ldots Но тем не менее, сейчас многие производители каких то электронных компонентов уже сразу выкладывают готовые проекты, как это все подключить к Arduino, протестировать\ldots


{\noindent \bf L: Ну доходит до того, что и у них в проектах внутри Arduino, и это уже не считается чем-то несолидным.}

{\noindent \bf Е:} Ну да, в нынешнем мире скорость разработки очень важна, а Arduino как раз позволяет ее сильно увеличить.

Arduino "--- такая платформа, которая сделала программирование микроконтроллеров массовым. Раньше-то это все требовало достаточно  глубокого погружения.

{\noindent \bf L: Вспоминаешь свой опыт.}

{\noindent \bf Е:} Да, конечно. Ну, а потом следующим проектом таким же успешным стал Raspberry Pi. 


{\noindent \bf L: Который тоже, надо сказать, в аппаратном смысле ничего выдающегося не представлял в сравнении с конкурентами.}

{\noindent \bf Е:} В общем-то тоже, да. Система на чипе, минимум обвязки, но зато стоит это все в пределах 50 долларов. Разные версии, и опять-таки ты подключаешь SD-карту и у тебя живая система с возможностью подключения всякой разной аппаратуры.  Что очень важно, потому что есть очень много именно аппаратных проектов с его использованием. Кстати, если посмотреть на наши трамвайные табло, которые в Минске висят по городу "--- когда их включают, там логотип Raspberry Pi виден.

{\noindent \bf L: Замечательный пример.}

{\noindent \bf Е:} Наши разработчики матерые тоже давно прочувствовали и поняли, что нечего изобретать велосипед и самому эти модули делать или покупать какие-то промышленные  и дорогие, когда вот есть дешевая массовая платформа.


{\noindent \bf L: В общем, подводя итог: получается, что секрет успеха открытого аппаратного обеспечения "--- низкий порог вхождения и скорость разработки.}

{\noindent \bf Е:} Да. И доступ к готовым наработкам тоже играет достаточно значительную роль.

\switchlang{be}

\section{Мiхаiл Волчак "---  Мiнск, Беларусь}
%\begin{figure}[ht]
%\centering{\includegraphics[width=4cm]{49_spons_altoros.jpg}}
%\end{figure}

{\noindent \bf LVEE: Ты прымаў удзел у некалькіх папярэдніх LVEE ў досыць розных амплуа: ад прэзентацыі беларускай партыі піратаў да воркшопа па меш-сетках. Як бы ты ідэнтыфікаваў сябе ў першую чаргу?
}

{\noindent \bf Мiхаiл Волчак:} Перш за ўсё для мне цікавыя такія праекты, у якіх удзельнічае супольнасць, таму што, супольнасць дае разнастайнасць, а я лічу што разнастайнасць "--- гэта такая базавая ўмова для развіцця. Гэта датычыцца розных праектаў, у тым ліку і развіцця размеркаваных меш-сетак. Але вядома я "--- пірат.

{\noindent \bf L: Які менавіта тэрмін ты ўкладаеш у слова «пірат»?}

{\noindent \bf М:} Піраты для мяне "--- гэта пэўная ідэалогія, калі Інтэрнэт на сваю карысць выкарыстоўваюць людзі, а не «крывавы карпарэйшн», напрыклад\ldots Гэта пірацкая палітычная ідэалогія, якая зараз узнікае, яе тэзіс, калі лаканічна, дэмакратыя з выкарыстаннем лічбавых сродкаў, праграмных і апаратных.




{\noindent \bf L: І гэтая твая цікавасць неяк перасякаецца з вольнымі ліцэнзіямі і кантэнтам, да якіх ты маеш непасрэднае дачыненне?} 

{\noindent \bf М:}  Свабодны кантэнт у нашу эпоху ведаў, пост-інфармацыйную эпоху\ldots

{\noindent \bf L: А пост-інфармацыйная, гэта?..}

{\noindent \bf М:}  Усе кажуць што мы ў інфармацыйнай, а насамрэч зараз важна пост-інфармацыя, калі мы не толькі ўспрымаем дадзеныя, мы іх ўмеем пераўтвараць у веды. Для спажыўцоў "--- інфармацыйная. Але iнфармацыя гэта не мэта, а сродак або інструмент.

{\noindent \bf L: І як гэта і пірацтва звязана са свабодным кантэнтам?}

{\noindent \bf М:} А як пірацтва звязана с кантэнтам? Кантэнт можна не толькі спажываць, а яго можна яшчэ змяняць, i гэта асноўны падыход, які мне блізкі. Яго можа зменяць кожны, калі мы заходзім у сеціва і маем там вікі ў сваіх свабодных праектах, і такім часам мы не проста спажыўцы, а сустваральнікі. Сустварэнне "--- гэта такая важная штука для любога апенсорса і ўвогуле любой супольнасці, якая хоча развівацца. Ўключаць новае ў свой дыскурс.

Першы кампанент "--- гэта сумесны ўдзел і магчымасць змяняць кантэнт, і другі кампанент "--- абараніць гэты кантэнт у эпоху, калі яго могуць прыватызаваць. Першапачатковую маёмасць супольнасці часам могуць прыватызаваць праз такія інструменты, як інтэлектуальная ўласнасць, гэта цэлы набор законаў, які дазваляе эканамічным суб'ектам, фірмам там, вялікім, малым "--- прысвойваць. Другі бок "--- гэта юрыдычнае спараджэнне свабоднага кантэнта, так званыя свабодныя ліцэнзіі, якія мы маем. GNU, Creative Commons (Творчыя Суполкі), яшчэ шэраг. Пад свабодай кантэнту мы маем тое ж самае, што з сафтом, тыя знакамітыя рычардавскія свабоды: свабода вывучаць, змяняць і распаўсюджваць. Калі ў кантэнту адна свабода знікае "--- знікае і цэласнае разуменне гэтай свабоды. Палітычныя піраты на сёння гэта разумеюць. Напрыклад, з забаронай распаўсюджвання знікае магчымасць арганічнага развіцця супольнасці. Карпаратыўнае развіццё можа мець месца, але тады арганічнасць знікае\ldots


{\noindent \bf L: Аднак, твая цікавасць да свабодных ліцэнзіях з'явілася яшчэ раней? Як гэта было?}

{\noindent \bf М:} Упершыню я пачуў пра гэта дзесьці ў годзе 2009-2010. У той перыяд, калі я пачаў больш глыбока разумець вікі-тэнхалогію і пачаў працаваць. Гэта прыйшло ад сафта. Я доўга мучыўся ад дыхатаміі Wіndows-Lіnux, першы мой Lіnux быў у 1999 годзе, я яго роўна на паўгадзіны устанавіў\ldots

{\noindent \bf L: Які дыстрыбутыў?}

{\noindent \bf М:} Asp Lіnux, там было некалькі дыскаў, якія я засоўваў і высоўваў "--- проста набыў дыскі і пачаў з імі іграцца\ldots Але потым была Kubuntu дзесь у 2010 годзе, у мяне стабільна сталі дзве сістэмы. Другі крок гэта былi web-applіcatіon, Drupal. Я актыўна удзельнічаў у Drupal-супольнасці, актыўна вывучаў Drupal і пасля пачаў чытаць лiцэнзiйныя дамовы GNU GPL, разоў пяць пачынаў чытаць і не заканчваў, але мне спадабаўся канцэпт. Канцэпт не з узору легальнасцi насамрэч "--- тое, што ён выяўляўся не для вырашэння задач камерцыі, а для вырашэння задач супольнасці. Я ў Drupal-супольнасць пагрузіўся не як спажывец праграмы, ці там кода, а як удзельнік, ну і тады я пачаў вывучаць пакеты, з якімі яны працуюць, і тады пакацілася. 

Ліцэнзійныя прынцыпы рэфлексуюць прынцыпы супольнасці, і калі ты адымаеш ад супольнасці штосьці з гэтых прынцыпаў, атрымліваецца нейкая карпаратыўная культура, а калі ты працуеш з трыма свабодамі, атрымліваецца супольнасць. А потым з Creatіve Commons платнічок пайшоў ў 2012 годзе. Я чытаў шмат кніг пра капірайт, і так арганічна гэта пайшло туды\ldots
 

{\noindent \bf L: Лоўрэнс Лессіг распіярыў?}

{\noindent \bf М:} Ведаеш, магчыма ад яго пайшлі некаторыя словы. Яго «Свабодная культура» дала мне нейкі стартавы слоўнікавы запас, на які можна было абапірацца далей.

{\noindent \bf L: А раскажы трохі аб Вікіпедыі. Пра сваё знаёмства з самой тэхналогіяй, падыходам, калі кам'юніці рэгулюе артыкулы.}

{\noindent \bf М:} Знаёмства адбылося практычна. Напісаў артыкул, яго праз не\-калькі хвілін выдалілі, а я на яго патраціў тры дні. Тады я зрабіў паўзу, таму што не разумеў, як працуе гэта супольнасць\ldots Вядома, я выкарыстоўваў той жа падыход, які выкарыстоўваў у Drupal-супольнасці, але яно не спрацавала. І хутчэй па культурна-палі\-тычным плане\ldots

{\noindent \bf L: Як гэта?}

{\noindent \bf М:} Я напісаў артыкул, які не адпавядаў крытэрыям значнасці для рускай вікіпедыі\ldots Потым я зноў пачаў пісаць, напісаў артыкул у беларускай вікіпедыі, яго заапрувілі і жыццё пайшло\ldots І вось у гэты момант, калі заапрувілі (быў статус «на вычытку» і раптам змяніўся) "--- мне стала цікава, як адбываецца самарэгуляцыя, як кантэнт застаецца рэлевантным.

{\noindent \bf L: У нейкім сэнсе самарэгуляцыю ты адчуў на сабе, пры першай спробе\ldots}

{\noindent \bf М:} Не зусім так. Спачатку мне здавалася, што гэта самарэгуляцыя, але потым прыйшло разуменне. Моўныя раздзелы, якія маюць 100-200 тысяч артыкулаў, выкарыстоўваюць стратэгію «набраць колькасць», калі прымаюцца любыя артыкулы, якія маюць 2-3 крыніцы. Калі раздзел перасякае рысу 500-600 тысяч, яны трохі змяняюць стратэгію,але вось гэтая свабода дадавання колькасці існуе. А калі яны перасякаюць мільён, або 800-900 тысяч, яны кардынальна мяняюць стратэгію і павялічваюць крытэрый значнасці. Такую стратэгію спачатку выкарыстоўвала і руская вікіпедыя, але пасля гэтага парога яны пачалі касіць усё, што нязначна для Расіі і выкарыстоўваюць тую ж стратэгію для беларускіх артыкулаў. Іншымі словамі, для “свадобных” ведаў у рувікі выкарыстоўваюцца геаграфічныя і культурна-палітычныя межы.


{\noindent \bf L: Патлумач гэты момант.}

{\noindent \bf М:} Ну, мы ж з большасці рускамоўныя, таму і пішам там.

У вікі-супольнасці ёсць 5 прынцыпаў асноўных, адзін з іх "--- крытэрый значнасці, notable sources, і ў многіх вялікіх раздзелах яна дэфармавалася ў крытэрый мега-якасці, які стварае такі парог уваходу, што ў невялікіх супольнасцяў не знойдзецца столькі спецыялістаў, якія будуць пісаць на адмысловую тэму. Атрымліваецца што невялікая супольнасць, напрыклад беларуская, якая прымае такія ж стандарты, якія існуюць у польскай або ў рускай wіkі (у польскай мякчэй) "--- узнікае такі парог уваходу, што пачаткоўца, які захоча нешта напісаць, выкідваюць з удзелу.


{\noindent \bf L: А чаму такі падыход не назіраецца ў ангельскай?}

{\noindent \bf М:} Да, з ангельскай вікі такія праблемы не ўзнікалі, я публікаваў нават такі эксперыментальны матэрыял, які быў бы з рускай выключаны з тымі крытэрыямі. Кожная супольнасць, нават і анлайнавая, рэфлексуе, як бы ты не хацеў называць яе анлайнавай, пэўную мадэль лакацыі\ldots Ангельская "--- выключэнне, таму што гэта вельмі глабальная супольнасць, яна больш інтэрнацыянальная, там і Еўропа, і Амерыка, і Азія. Ўсюды англійскае камьюніці, і таму там культурнага рэлятывізму або нейкай палітычнай скіраванасці не існуе.
 
{\noindent \bf L: І напрыканцы, як ты ўяўляеш сабе перспектывы адкрытых ліцэнзій ў вобласці, не звязанай з софтам: у open media, open hardware?}

{\noindent \bf М:} Асноўная праблема кантэнтных і сафтовых ліцэнзій "--- тое, што яны не арыгінальныя для трох чвэрцей гэтай планеты. У іх вельмі класная задума, як хакнуць залішнія абмежаванні, напрыклад, капірайта, але яны вельмі арганічна глядзяцца ў краіне, дзе капірайт найбольш развіты і стаў часткай штодзеннага жыцця для многіх. Для Беларусі гэта нехарактэрна, Беларусь ніколі не была краінай, дзе капірайт вельмі прывіваўся\ldots Ад Францыска Скарыны, які быў піратам і правозіў кантрабанднае абсталяванне праз польскую мяжу для распаўсюду свабодных ведаў у Вялікім Княстве Літоўскім.


{\noindent \bf L: ?!}

{\noindent \bf М:} Так, яго двойчы затрымлівалі, і толькі на трэці раз яму ўдалося правезці гэта hardware, і толькі пасля гэтага ён змог заняцца тут кнігадрукаваннем\ldots Умоўна кажучы, гэтая ліцэнзійная фішка павінна перасэнсавацца нашым грамадствам, і ўсімі не Western Europe і амерыканскімі супольнасцямі. Заходняя Еўропа "--- гэта індывідуалістычны падыход да ўсяго, а open source "--- гэта communіty ownershіp, tradіtіonal knowledge, традыцыйныя веды, калі усё “племя” нешта ведае, і ніхто не робіць гэтыя веды прапрыятарнымі, каб нажыцца. Сеткавыя супольнасці зараз працуюць па гэтым жа прынцыпе як плямёны там\ldots з Лацінскай Амерыкі ці Аўстраліі, і таму тут пытанне з гэтымі ліцэнзіямі, якія індывідуальна накіраваныя.

{\noindent \bf L: Але яны досыць паспяхова прымяняюцца?}

{\noindent \bf М:} Не зусім так. Спачатку мне здавалася, што гэта самарэгуляцыя, але потым прыйшло разуменне. Моўныя раздзелы, якія маюць 100-200 тысяч артыкулаў, выкарыстоўваюць стратэгію «набраць колькасць», калі прымаюцца любыя артыкулы, якія маюць 2-3 крыніцы. Калі раздзел перасякае рысу 500-600 тысяч, яны трохі змяняюць стратэгію,але вось гэтая свабода дадавання колькасці існуе. А калі яны перасякаюць мільён, або 800-900 тысяч, яны кардынальна мяняюць стратэгію і павялічваюць крытэрый значнасці. Такую стратэгію спачатку выкарыстоўвала і руская вікіпедыя, але пасля гэтага парога яны пачалі касіць усё, што нязначна для Расіі і выкарыстоўваюць тую ж стратэгію для беларускіх артыкулаў. Іншымі словамі, для “свадобных” ведаў у рувікі выкарыстоўваюцца геаграфічныя і культурна-палітычныя межы.


{\noindent \bf L: Тут выразна чутны голас пірацкай партыі :-)}

{\noindent \bf М:} Я пакуль спрабую перасэнсаваць гэтае пытанне\ldots Я за тое, каб гэтыя ліцэнзіі прысутнічалі, я сам прасоўваю ліцензіі Creative Com\-mons ў Беларусі. Але калі займаешся іх прамоўшнам\ldots Так, пад вікіпедыяй ёсць Creatіve Commons, але 90\% вікіпедыстаў разумеюць гэтую ліцэнзію хутчэй праз паводзіны ў супольнасці, а не ў юрыдычным плане. Паовдзіны супольнасці фарміруюць правілы "--- змест гэтых ліцэнзій, а не наадварот. Гэта для мяне важный момант\ldots

А ліцэнзіі на жалеза "--- гэта больш накшталт патэнтаў. У патэнтаў ёсць адзін плюс\ldots У іх ёсць шмат мінусаў, але адзін плюс, якому пакуль не прыдумана альтэрнатыва. Яны ўсё ўключаюць у адну базу дадзеных, якую чалавек можа праглядаць. Вось калі open source community, якое працуе з hardware, выпрацуе нейкі падобны механізм з класіфікатарам, катэгарызацыяй "--- тады гэта можа быць годнай альтэрнатывай, але пакуль такога механізму няма. Гэта выклік "--- ці могуць актывісты open hardware такое зрабіць, але такое трэба. Першая ўмова росту папулярнасці вольных ліцэнзій для hardware "--- гэта стварыць сістэму, якая будзе больш эфектыўная, больш адкрытая, чым патэнтнае бюро, патэнтныя офісы па розных краінах.

\switchlang{ru}  

\section{Даниэль Надь "---  Будапешт, Венгрия}
%\begin{figure}[ht]
%\centering{\includegraphics[width=4cm]{49_spons_altoros.jpg}}
%\end{figure}

{\noindent \bf LVEE: Поскольку нынешние интервью связаны с комьюни\-ти-ориентированными и свободными технологиями за пределами непосредственно свободного программного обеспечения "--- поговорим о криптовалютах. Как возник твой интерес к этой сфере?}

{\noindent \bf Даниэль Надь:} Это уходит корнями в глубокое детство. В 1984 году была выпущена такая компьютерная игра, Elite...\ldots 

{\noindent \bf L: Ух ты! Всё началось с неё?}

{\noindent \bf Д:} Вот, ты её тоже помнишь. Я в неё вложил "--- т. е. закопал "--- многие тысячи часов, наверное. И когда уже появился Интернет, у меня была такая мечта "--- сделать что-то похожее, но для большого количества игроков. Сейчас уже такие игры существуют, а когда я занимался этим  своим юношеским проектом "--- тогда возникали две проблемы. Во-первых, достаточно быстро стало ясно, что на централизованном сервере хранить все это очень затратно. Хотелось использовать вычислительные мощности клиентов, сделать какую-то распределенную систему. А во-вторых, я заметил, что в Elite наступал такой момент, когда у игрока становилось очень много денег, и игра из-за этого делалась несколько скучной. Я задумался, почему так не случается в жизни, начал читать про денежную систему "--- как это работает, что такое деньги вообще, почему, например, в игре никогда не случается так, что у меня есть какой-то товар, а у покупателей на космической станции нет денег\ldots

{\noindent \bf L: Действительно :)} 

{\noindent \bf Д:} ...а в реальном мире это достаточно частая ситуация. В общем, я начал вчитываться в суть денег, потом наступил крах доткомов 2001 года, когда у многих появился интерес к этой теме, и появилось с кем об этом поговорить. Сначала я начал вчитываться в то, как можно вести распределенный бухгалтерский учет для большой онлайн-игры , а через некоторое время задумался: зачем это делать в игрушечном варианте, когда чего-то подобного миру не хватает по-настоящему. 

{\noindent \bf L: Как обстояли на тот момент дела с электронными платёжными системами?}

{\noindent \bf Д:}  В 1998 году в русскоязычном пространстве появился Webmoney, а за несколько лет до этого был DigiCash, так что ростки криптовалют "--- они появлялись в конце 90-х, а теоретические основы (академические статьи на эту тему) начали появляться ещё в 80-х.

{\noindent \bf L: А в распределенном, децентрализованном виде?}

{\noindent \bf Д:} Всем хотелось децентрализации. Это был такой Святой Грааль, который всем хотелось сделать. И естественно, была проблема византийского консенсуса. Если описать простыми словами "--- это задача, как обеспечить, чтобы состояние бухгалтерского учета было для всех одинаково, с какой точки ни смотри на систему, т. е. чтобы всегда сохранялась целостность. Это была достаточно сложная проблема, и хорошего практического решения она не имела, пока вдруг из ниоткуда не появился Bitcoin.

{\noindent \bf L:А блокчейн?}

{\noindent \bf Д:} Он не существовал до Bitcoin. 

{\noindent \bf L: Насколько понимаю, распределенная децентрализованная система по идеологии и принципам достаточно близка открытому ПО. И что, Bitcoin оказался первой реально успешной системой электронных денег, по своим принципам родственной свободному ПО?}

{\noindent \bf Д:} Не сказал бы, что первой. Были свободные проекты до Bitcoin. И ePoint ведь тоже был основан на открытом ПО, и еще было много разных проектов. Это в начале двухтысячных была достаточно горячая тема, многие ею занимались, были разные эксперименты, но они были не очень успешны именно из-за той нерешенной проблемы, о которой мы только что говорили.

{\noindent \bf L: Тот кто первый сумел ее решить, тот завоевал мир?}

{\noindent \bf Д:} Да. Хотя я этот параметр и не очень высоко ценю, но тем не менее рыночная капитализация Bitcoin в разы превосходит рыночную капитализацию остальных криптовалют вместе взятых.

{\noindent \bf L: С технической точки зрения, со времен появления Bitcoin криптовалюты как-то развивались?}

{\noindent \bf Д:} Конечно. Есть разные направления развития. Кстати не все они положительные. Да и направление развития Bitcoin не совсем соответствует тем ожиданиям и предсказаниям, которые ему сопутствовали в 2009 году.  На сегодняшний день децентрализация Bitcoin во многом условна. На самом деле обработка транзакций происходит на достаточно маленьком количестве узлов, которое к тому же еще и сокращается.

{\noindent \bf L: И территориально, насколько я помню, есть проблема, что почти все они расположены там, где быстро делают микросхемы, и\ldots}

{\noindent \bf Д:} ...и где легко воровать электричество, называя вещи своими именами.

Когда эти проблемы Bitcoin стали более-менее очевидными, многие взялись их решать. Первая более-менее успешная альтернатива "--- Litecoin, базируется на коде Bitcoin, и тут как раз видно преимущество открытого ПО, потому что можно было не реализовывать всё с нуля, а просто взять Bitcoin и поправить там некоторые вещи, чтобы достичь тех целей, которые разработчики себе поставили. В случае Litecoin основной задачей была децентрализация майнинга. В отличие от ожиданий Сатоши Накамото, оказалось что идеальный майнер в Bitcoin "--- это не универсальный компьютер, а некая микросхема, созданная именно под эту задачу. И создатели Litecoin задались целью создать такой алгоритм майнинга, который лучше всего выполняется на компьютере общего назначения. Это им в какой-то степени удалось. 

А еще в тот момент, когда запустили Litecoin, люди уже были знакомы с Bitcoin, поэтому у этой системы не было ещё одной проблемы. Около трети общего количества биткоинов находятся в руках максимум 18 юридических или физических лиц. Эта очень большая централизация денежной массы случилась из-за того, что в самом начале мало кто знал о Bitcoin, и ранние майнеры очень много себе намайнили. С Litecoin такой проблемы не было, т. к. когда его объявили, очень многие бросились заниматься майнингом. И сам майнинг там децентрализован благодаря тому, что очень долго не было (а по-моему, и до сих пор нет) специализированных микросхем. 

Это было такое первое развитие темы Bitcoin "--- достаточно успешный проект ,у которого тоже существенная рыночная капитализация, и кроме того есть ещё некоторые преимущества перед Bitcoin. Например, существенно быстрее время блоков. 


{\noindent \bf L: Думаю, это нелишним будет пояснить.}

{\noindent \bf Д:} Для тех кто не знаком с основными принципами блокчейна, можно сказать, что Bitcoin "--- это такая большая публичная бухгалтерия, куда записываются все транзакции, и страница в этой бухгалтерской книге "--- это блок.

{\noindent \bf L: Копии всех страниц должны быть на каждом узле, и благодаря системе указателей с цифровыми подписями ни у кого нет возможности незаметно модифицировать часть блокчейна.}

{\noindent \bf Д:} Да. И пока транзакция не попала в эту общую книгу, она теоретически может быть ещё изменена или удалена. Транзакция становится точной и необратимой, когда попадает в этот децентрализованный распределенный блокчейн. В случае Bitcoin это занимает в среднем около 10 минут, а в случае Litecoin "--- 2.5 минуты. 

{\noindent \bf L: После этого были еще похожие системы?}

{\noindent \bf Д:} Очень много. Многие пытались что-то улучшить. С технической точки зрения им это может быть и удалось, но в случае с криптовалютами есть очень силён эффект сети. Даже если какая-то криптовалюта по параметрам  лучше Bitcoin, для того, чтобы у нее появилась существенная база пользователей, она должна найти для себя нишу, потому что сам тот факт, что огромное количество людей и бизнесов пользуется Bitcoin, делает остальные альтернативы непривлекательными.
 
{\noindent \bf L: Кстати, по поводу блокчейна, который присутствует на всех узлах. Вроде бы криптовалюты считаются такими анонимными, но при этом все твои транзакции сохраняются навсегда в публичном доступе. Они вроде бы пока не привязаны к твоей персоне, но никогда нет гарантии что\ldots}

{\noindent \bf Д:} ...Что кто-нибудь потом не привяжет.


{\noindent \bf L: Да. Так насколько корректно считать это способом анонимнизации платежей?}

{\noindent \bf Д:} По-моему, совершенно некорректно. Bitcoin я бы анонимным не назвал. Есть отдельные анонимизаторы, такие сервисы, принимающие грязные биткоины, а возвращающие отмытые. Потом есть, скажем так, теоретически разработанные технологии, при помощи которых можно создать действительно анонимную криптовалюту. Этим сейчас тоже занимаются люди, и думаю, это тоже одно из направлений, куда в будущем будут развиваться технологии.

{\noindent \bf L: По поводу анонимности. Получается, если какие-то правительства косо смотрят на Bitcoin по поводу возможности сокрытия доходов "--- то это скорее ошибочная точка зрения?}

{\noindent \bf Д:} Нет, с этим уже не соглашусь. Несмотря на то, что биткоины не очень сложно привязать к какому-то конкретному имени, их очень сложно отнять и очень сложно воспрепятствовать их движению. Само по себе использование Bitcoin еще не спасает в случае чего, но если у человека значительная часть сбережений в биткоинах "--- он, грубо говоря, бежит значительно быстрее в случае каких-то потрясений. Когда случаются какие-то финансовые катаклизмы, массовые неплатежи или финансовый кризис в банковской системе, тогда бывает паника, люди пытаются перемещать свои капиталы, но многим это не удаётся или удаётся только частично. И вот в этом случае Bitcoin очень сильно помогает, это было продемонстрировано в случае кипрского кризиса, потом греческого кризиса, последнее время в Восточной Европе тоже, в Украине, и в Беларуси тоже, по-моему, те кто сделал ставку на Bitcoin, не ошиблись. 

Но есть ещё и такое преимущество Bitcoin и вообще систем распределённого бухгалтерского учета: если надо, они достаточно хорошо сочетаются с системой налогообложения, контроля. Например, я сейчас получаю значительную часть моей прибыли в криптовалютах, получаю от фонда, фонд должен держать свою бухгалтерию открытой, тот факт что это получаю я, а не кто-то другой "--- это не секрет, я с этих денег конечно же плачу налоги. В налоговой было достаточно просто показать, что вот этот счет, вот транзакция "---  в общем, из-за этого не возникает подозрений, что где-то тут мошенничество или что я что-то скрываю. Т. е. если я хочу играть с открытыми картами "--- я имею такую возможность.

Подводя итог,  такой подход в значительной степени отдает контроль над средствами индивиду или организации, даёт возможность решать, какую часть своих операций держать в какой юрисдикции. Это ставит юрисдикции в положение конкуренции друг с другом.


{\noindent \bf L: Еще вопрос про вопрос про применение блокчейна за пределами платёжных систем. Насколько понимаю, с тех пор появились и такие проекты?}

{\noindent \bf Д:} Да. Я работаю как раз над таким проектом. Сейчас наиболее известен и успешен среди таких проектов Etherium "--- проект, который действительно превратил распределенный бухгалтерский учет Bitcoin в распределенную базу данных общего назначения. Там можно хранить в блокчейне любые данные, и записывать условия изменения этих данных на тьюринг-полном языке.
 То есть на этих данных можно проводить практически любые вычисления так, что все могут убедиться: действительно результатом этих вычислений стало то-то и то-то. Здесь открываются такие возможности, о которых даже трудно говорить. Это как пытаться в начале 80-х годов рассказать, что такое персональный компьютер. Подавляющее большинство будущих применений еще не видны из нашей перспективы, потому что их еще не изобрели.

{\noindent \bf L: Можешь привести хотя бы парочку примеров, случайно выбранных?}

{\noindent \bf Д:} Попытаюсь. Например, можно сделать страховку без страховой компании. Если какие-то люди хотят распределить между собой какие-то риски, они могут записать условие в умный контракт, что, тот кто регулярно делал какие-то взносы и с ним что-то случилось, получает определённые деньги. Это записывается в контракт, и таким образом страхование становится распределенным и децентрализованным: нет страховой компании, есть только люди, которые между собой поделили риски. 

Или, что меня больше всего интересует "--- это общее пользование капиталом. Самое простое, то над чем я сейчас конкретно работаю "--- чтобы любой человек мог сдавать в аренду свои вычислительные ресурсы. В данном случае это хранение и передача данных,  такое распределенное хранилище, к которому любой человек может подключить свои накопители и зарабатывать на том, что его накопителями пользуются другие люди.
{\noindent \bf L: То есть мы говорим не строго о вычислительной мощности процессора, мы понимаем более широко.}

{\noindent \bf Д:} Да, более широко, но и процессор тоже. В будущем, я думаю, можно будет давать в аренду и процессор. Есть какая-то программа, которая записывается в блокчейн, и потом данные можно рассылать разным участникам, они эти данные будут обрабатывать. Конечно, должна быть некоторая избыточность. Если те же  самые данные два узла обработали иначе, можно автоматически, без участия человека выяснить, кто из них обработал данные неправильно, и исключить этот узел. Можно, например, распределенно рендерить фильмы. Но это, мне кажется, только первые ласточки, это достаточно просто. А вот, например, есть такая проблема, что огромное количество капитала человечества существует в виде припаркованных автомобилей, которые ничего не делают, а просто загораживают дорогу.

{\noindent \bf L: Неожиданно.}

{\noindent \bf Д:} Из-за того, что мы друг другу не доверяем, я не могу подойти к какому-то припаркованному автомобилю, быстро доказать владельцу, что я надёжный и хороший человек и у меня есть достаточно сбережений чтобы компенсировать ему ущерб, если я его машину сломаю. Несмотря на то что это так, я не могу легко и просто это доказать. Но при помощи блокчейн-технологий это все становится доказуемо: как репутация, так и финансовое положение. Можно сделать такую автоматическую систему, когда к любому автомобилю можно подойти, быстро доказать, что мне его можно дать в аренду, заплатить, поехать куда-то и там его оставить для следующего желающего. Таким образом, количество машин, которое должно существовать, оказывается значительно меньшим. Многим не нужно будет иметь собственный автомобиль. И мне кажется что в итоге от этого человечество станет значительно богаче:  90\% автомобилей не нужно будет производить, и эти мощности можно будет использовать для чего-то другого.

{\noindent \bf L: По существу, блокчейн "--- такой достаточно мощный механизм, который «перетащит» комьюнити-технолгоии в осязаемую сферу?}

{\noindent \bf Д:} Да. Научная фантастика на эту тему существует, мы ее читали, и действительно сказку сейчас делаем былью, в том смысле, что, как мы  считаем, он перевернет мир так же, как в свое время это сделал доступный компьютер. Это очень амбициозный проект, и над ним очень приятно работать :)

 
\end{document}


