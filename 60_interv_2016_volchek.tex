\documentclass[10pt, a5paper]{article}
\usepackage{pdfpages}
\usepackage{parallel}
\usepackage[T2A]{fontenc}
\usepackage{ucs}
\usepackage[utf8x]{inputenc}
\usepackage[polish,english,russian]{babel}
\usepackage{hyperref}
\usepackage{rotating}
\usepackage[inner=2cm,top=1.8cm,outer=2cm,bottom=2.3cm,nohead]{geometry}
\usepackage{listings}
\usepackage{graphicx}
\usepackage{wrapfig}
\usepackage{longtable}
\usepackage{indentfirst}
\usepackage{array}
\newcolumntype{P}[1]{>{\raggedright\arraybackslash}p{#1}}
\frenchspacing
\usepackage{fixltx2e} %text sub- and superscripts
\usepackage{icomma} % коскі ў матэматычным рэжыме
\PreloadUnicodePage{4}

\newcommand{\longpage}{\enlargethispage{\baselineskip}}
\newcommand{\shortpage}{\enlargethispage{-\baselineskip}}

\def\switchlang#1{\expandafter\csname switchlang#1\endcsname}
\def\switchlangbe{
\let\saverefname=\refname%
\def\refname{Літаратура}%
\def\figurename{Іл.}%
}
\def\switchlangen{
\let\saverefname=\refname%
\def\refname{References}%
\def\figurename{Fig.}%
}
\def\switchlangru{
\let\saverefname=\refname%
\let\savefigurename=\figurename%
\def\refname{Литература}%
\def\figurename{Рис.}%
}

\hyphenation{admi-ni-stra-tive}
\hyphenation{ex-pe-ri-ence}
\hyphenation{fle-xi-bi-li-ty}
\hyphenation{Py-thon}
\hyphenation{ma-the-ma-ti-cal}
\hyphenation{re-ported}
\hyphenation{imp-le-menta-tions}
\hyphenation{pro-vides}
\hyphenation{en-gi-neering}
\hyphenation{com-pa-ti-bi-li-ty}
\hyphenation{im-pos-sible}
\hyphenation{desk-top}
\hyphenation{elec-tro-nic}
\hyphenation{com-pa-ny}
\hyphenation{de-ve-lop-ment}
\hyphenation{de-ve-loping}
\hyphenation{de-ve-lop}
\hyphenation{da-ta-ba-se}
\hyphenation{plat-forms}
\hyphenation{or-ga-ni-za-tion}
\hyphenation{pro-gramming}
\hyphenation{in-stru-ments}
\hyphenation{Li-nux}
\hyphenation{sour-ce}
\hyphenation{en-vi-ron-ment}
\hyphenation{Te-le-pathy}
\hyphenation{Li-nux-ov-ka}
\hyphenation{Open-BSD}
\hyphenation{Free-BSD}
\hyphenation{men-ti-on-ed}
\hyphenation{app-li-ca-tion}

\def\progref!#1!{\texttt{#1}}
\renewcommand{\arraystretch}{2} %Іначай формулы ў матрыцы зліпаюцца з лініямі
\usepackage{array}

\def\interview #1 (#2), #3, #4, #5\par{

\section[#1, #3, #4]{#1 -- #3, #4}
\def\qname{LVEE}
\def\aname{#1}
\def\q ##1\par{{\noindent \bf \qname: ##1 }\par}
\def\a{{\noindent \bf \aname: } \def\qname{L}\def\aname{#2}}
}

\def\interview* #1 (#2), #3, #4, #5\par{

\section*{#1\\{\small\rm #3, #4. #5}}

\def\qname{LVEE}
\def\aname{#1}
\def\q ##1\par{{\noindent \bf \qname: ##1 }\par}
\def\a{{\noindent \bf \aname: } \def\qname{L}\def\aname{#2}}
}

\switchlang{be}
%\usepackage{color}
\begin{document}
\title{Интервью с участниками}
%\author{}
\date{}
\maketitle

По традиции в сборник материалов входят интервью, в которых активные участники
сообщества open source делятся своим мнением о свободном ПО, открытых
технологиях, роли и месте свободных лицензий, рассказывают, как видят проблематику
свободных проектов. В этот раз мы решили выбрать в качестве тематического русла
использование свободных лицензий в сфере, отличной от программного обеспечения:
в таких областях, как свободный контент, свободное аппаратное обеспечение и др.

\section{Мiхаiл Волчак "---  Мiнск, Беларусь}
%\begin{figure}[ht]
%\centering{\includegraphics[width=4cm]{49_spons_altoros.jpg}}
%\end{figure}

{\noindent \bf LVEE: Ты прымаў удзел у некалькіх папярэдніх LVEE ў досыць розных амплуа: ад прэзентацыі беларускай партыі піратаў да воркшопа па меш-сетках. Як бы ты ідэнтыфікаваў сябе ў першую чаргу?
}

{\noindent \bf Мiхаiл Волчак:} Перш за ўсё для мне цікавыя такія праекты, у якіх удзельнічае супольнасць, таму што, супольнасць дае разнастайнасць, а я лічу што разнастайнасць "--- гэта такая базавая ўмова для развіцця. Гэта датычыцца розных праектаў, у тым ліку і развіцця размеркаваных меш-сетак. Але вядома я "--- пірат.

{\noindent \bf L: Які менавіта тэрмін ты ўкладаеш у слова «пірат»?}

{\noindent \bf М:} Піраты для мяне "--- гэта пэўная ідэалогія, калі Інтэрнэт на сваю карысць выкарыстоўваюць людзі, а не «крывавы карпарэйшн», напрыклад\ldots Гэта пірацкая палітычная ідэалогія, якая зараз узнікае, яе тэзіс, калі лаканічна, дэмакратыя з выкарыстаннем лічбавых сродкаў, праграмных і апаратных.




{\noindent \bf L: І гэтая твая цікавасць неяк перасякаецца з вольнымі ліцэнзіямі і кантэнтам, да якіх ты маеш непасрэднае дачыненне?} 

{\noindent \bf М:}  Свабодны кантэнт у нашу эпоху ведаў, пост-інфармацыйную эпоху\ldots

{\noindent \bf L: А пост-інфармацыйная, гэта?..}

{\noindent \bf М:}  Усе кажуць што мы ў інфармацыйнай, а насамрэч зараз важна пост-інфармацыя, калі мы не толькі ўспрымаем дадзеныя, мы іх ўмеем пераўтвараць у веды. Для спажыўцоў "--- інфармацыйная. Але iнфармацыя гэта не мэта, а сродак або інструмент.

{\noindent \bf L: І як гэта і пірацтва звязана са свабодным кантэнтам?}

{\noindent \bf A:} А як пірацтва звязана с кантэнтам? Кантэнт можна не толькі спажываць, а яго можна яшчэ змяняць, i гэта асноўны падыход, які мне блізкі. Яго можа зменяць кожны, калі мы заходзім у сеціва і маем там вікі ў сваіх свабодных праектах, і такім часам мы не проста спажыўцы, а сустваральнікі. Сустварэнне — гэта такая важная штука для любога апенсорса і ўвогуле любой супольнасці, якая хоча развівацца. Ўключаць новае ў свой дыскурс.

Першы кампанент "--- гэта сумесны ўдзел і магчымасць змяняць кантэнт, і другі кампанент — абараніць гэты кантэнт у эпоху, калі яго могуць прыватызаваць. Першапачатковую маёмасць супольнасці часам могуць прыватызаваць праз такія інструменты, як інтэлектуальная ўласнасць, гэта цэлы набор законаў, які дазваляе эканамічным суб'ектам, фірмам там, вялікім, малым — прысвойваць. Другі бок "--- гэта юрыдычнае спараджэнне свабоднага кантэнта, так званыя свабодныя ліцэнзіі, якія мы маем. GNU, Creative Commons (Творчыя Суполкі), яшчэ шэраг. Пад свабодай кантэнту мы маем тое ж самае, што з сафтом, тыя знакамітыя рычардавскія свабоды: свабода вывучаць, змяняць і распаўсюджваць. Калі ў кантэнту адна свабода знікае — знікае і цэласнае разуменне гэтай свабоды. Палітычныя піраты на сёння гэта разумеюць. Напрыклад, з забаронай распаўсюджвання знікае магчымасць арганічнага развіцця супольнасці. Карпаратыўнае развіццё можа мець месца, але тады арганічнасць знікае\ldots


{\noindent \bf L: Аднак, твая цікавасць да свабодных ліцэнзіях з'явілася яшчэ раней? Як гэта было?}

{\noindent \bf A:} Упершыню я пачуў пра гэта дзесьці ў годзе 2009-2010. У той перыяд, калі я пачаў больш глыбока разумець вікі-тэнхалогію і пачаў працаваць. Гэта прыйшло ад сафта. Я доўга мучыўся ад дыхатаміі Wіndows-Lіnux, першы мой Lіnux быў у 1999 годзе, я яго роўна на паўгадзіны устанавіў\ldots

{\noindent \bf L: Які дыстрыбутыў?}

{\noindent \bf A:} Asp Lіnux, там было некалькі дыскаў, якія я засоўваў і высоўваў "--- проста набыў дыскі і пачаў з імі іграцца\ldots Але потым была Kubuntu дзесь у 2010 годзе, у мяне стабільна сталі дзве сістэмы. Другі крок гэта былi web-applіcatіon, Drupal. Я актыўна удзельнічаў у Drupal-супольнасці, актыўна вывучаў Drupal і пасля пачаў чытаць лiцэнзiйныя дамовы GNU GPL, разоў пяць пачынаў чытаць і не заканчваў, але мне спадабаўся канцэпт. Канцэпт не з узору легальнасцi насамрэч "--- тое, што ён выяўляўся не для вырашэння задач камерцыі, а для вырашэння задач супольнасці. Я ў Drupal-супольнасць пагрузіўся не як спажывец праграмы, ці там кода, а як удзельнік, ну і тады я пачаў вывучаць пакеты, з якімі яны працуюць, і тады пакацілася. 

Ліцэнзійныя прынцыпы рэфлексуюць прынцыпы супольнасці, і калі ты адымаеш ад супольнасці штосьці з гэтых прынцыпаў, атрымліваецца нейкая карпаратыўная культура, а калі ты працуеш з трыма свабодамі, атрымліваецца супольнасць. А потым з Creatіve Commons платнічок пайшоў ў 2012 годзе. Я чытаў шмат кніг пра капірайт, і так арганічна гэта пайшло туды\ldots
 

{\noindent \bf L: Лоўрэнс Лессіг распіярыў?}

{\noindent \bf A:} Ведаеш, магчыма ад яго пайшлі некаторыя словы. Яго «Свабодная культура» дала мне нейкі стартавы слоўнікавы запас, на які можна было абапірацца далей.

{\noindent \bf L: А раскажы трохі аб Вікіпедыі. Пра сваё знаёмства з самой тэхналогіяй, падыходам, калі кам'юніці рэгулюе артыкулы.}

{\noindent \bf A:} Знаёмства адбылося практычна. Напісаў артыкул, яго праз не\-калькі хвілін выдалілі, а я на яго патраціў тры дні. Тады я зрабіў паўзу, таму што не разумеў, як працуе гэта супольнасць\ldots Вядома, я выкарыстоўваў той жа падыход, які выкарыстоўваў у Drupal-супольнасці, але яно не спрацавала. І хутчэй па культурна-палі\-тычным плане\ldots

{\noindent \bf L: Як гэта?}

{\noindent \bf М:} Я напісаў артыкул, які не адпавядаў крытэрыям значнасці для рускай вікіпедыі\ldots Потым я зноў пачаў пісаць, напісаў артыкул у беларускай вікіпедыі, яго заапрувілі і жыццё пайшло\ldots І вось у гэты момант, калі заапрувілі (быў статус «на вычытку» і раптам змяніўся) "--- мне стала цікава, як адбываецца самарэгуляцыя, як кантэнт застаецца рэлевантным.

{\noindent \bf L: У нейкім сэнсе самарэгуляцыю ты адчуў на сабе, пры першай спробе\ldots}

{\noindent \bf М:} Не зусім так. Спачатку мне здавалася, што гэта самарэгуляцыя, але потым прыйшло разуменне. Моўныя раздзелы, якія маюць 100-200 тысяч артыкулаў, выкарыстоўваюць стратэгію «набраць колькасць», калі прымаюцца любыя артыкулы, якія маюць 2-3 крыніцы. Калі раздзел перасякае рысу 500-600 тысяч, яны трохі змяняюць стратэгію,але вось гэтая свабода дадавання колькасці існуе. А калі яны перасякаюць мільён, або 800-900 тысяч, яны кардынальна мяняюць стратэгію і павялічваюць крытэрый значнасці. Такую стратэгію спачатку выкарыстоўвала і руская вікіпедыя, але пасля гэтага парога яны пачалі касіць усё, што нязначна для Расіі і выкарыстоўваюць тую ж стратэгію для беларускіх артыкулаў. Іншымі словамі, для “свадобных” ведаў у рувікі выкарыстоўваюцца геаграфічныя і культурна-палітычныя межы.


{\noindent \bf L: Патлумач гэты момант.}

{\noindent \bf A:} Ну, мы ж з большасці рускамоўныя, таму і пішам там.

У вікі-супольнасці ёсць 5 прынцыпаў асноўных, адзін з іх "--- крытэрый значнасці, notable sources, і ў многіх вялікіх раздзелах яна дэфармавалася ў крытэрый мега-якасці, які стварае такі парог уваходу, што ў невялікіх супольнасцяў не знойдзецца столькі спецыялістаў, якія будуць пісаць на адмысловую тэму. Атрымліваецца што невялікая супольнасць, напрыклад беларуская, якая прымае такія ж стандарты, якія існуюць у польскай або ў рускай wіkі (у польскай мякчэй) "--- узнікае такі парог уваходу, што пачаткоўца, які захоча нешта напісаць, выкідваюць з удзелу.


{\noindent \bf L: А чаму такі падыход не назіраецца ў ангельскай?}

{\noindent \bf М:} Да, з ангельскай вікі такія праблемы не ўзнікалі, я публікаваў нават такі эксперыментальны матэрыял, які быў бы з рускай выключаны з тымі крытэрыямі. Кожная супольнасць, нават і анлайнавая, рэфлексуе, як бы ты не хацеў называць яе анлайнавай, пэўную мадэль лакацыі\ldots Ангельская "--- выключэнне, таму што гэта вельмі глабальная супольнасць, яна больш інтэрнацыянальная, там і Еўропа, і Амерыка, і Азія. Ўсюды англійскае камьюніці, і таму там культурнага рэлятывізму або нейкай палітычнай скіраванасці не існуе.
 
{\noindent \bf L: І напрыканцы, як ты ўяўляеш сабе перспектывы адкрытых ліцэнзій ў вобласці, не звязанай з софтам: у open media, open hardware?}

{\noindent \bf A:} Асноўная праблема кантэнтных і сафтовых ліцэнзій "--- тое, што яны не арыгінальныя для трох чвэрцей гэтай планеты. У іх вельмі класная задума, як хакнуць залішнія абмежаванні, напрыклад, капірайта, але яны вельмі арганічна глядзяцца ў краіне, дзе капірайт найбольш развіты і стаў часткай штодзеннага жыцця для многіх. Для Беларусі гэта нехарактэрна, Беларусь ніколі не была краінай, дзе капірайт вельмі прывіваўся\ldots Ад Францыска Скарыны, які быў піратам і правозіў кантрабанднае абсталяванне праз польскую мяжу для распаўсюду свабодных ведаў у Вялікім Княстве Літоўскім.


{\noindent \bf L: ?!}

{\noindent \bf М:} Так, яго двойчы затрымлівалі, і толькі на трэці раз яму ўдалося правезці гэта hardware, і толькі пасля гэтага ён змог заняцца тут кнігадрукаваннем\ldots Умоўна кажучы, гэтая ліцэнзійная фішка павінна перасэнсавацца нашым грамадствам, і ўсімі не Western Europe і амерыканскімі супольнасцямі. Заходняя Еўропа "--- гэта індывідуалістычны падыход да ўсяго, а open source "--- гэта communіty ownershіp, tradіtіonal knowledge, традыцыйныя веды, калі усё “племя” нешта ведае, і ніхто не робіць гэтыя веды прапрыятарнымі, каб нажыцца. Сеткавыя супольнасці зараз працуюць па гэтым жа прынцыпе як плямёны там\ldots з Лацінскай Амерыкі ці Аўстраліі, і таму тут пытанне з гэтымі ліцэнзіямі, якія індывідуальна накіраваныя.

{\noindent \bf L: Але яны досыць паспяхова прымяняюцца?}

{\noindent \bf М:} Не зусім так. Спачатку мне здавалася, што гэта самарэгуляцыя, але потым прыйшло разуменне. Моўныя раздзелы, якія маюць 100-200 тысяч артыкулаў, выкарыстоўваюць стратэгію «набраць колькасць», калі прымаюцца любыя артыкулы, якія маюць 2-3 крыніцы. Калі раздзел перасякае рысу 500-600 тысяч, яны трохі змяняюць стратэгію,але вось гэтая свабода дадавання колькасці існуе. А калі яны перасякаюць мільён, або 800-900 тысяч, яны кардынальна мяняюць стратэгію і павялічваюць крытэрый значнасці. Такую стратэгію спачатку выкарыстоўвала і руская вікіпедыя, але пасля гэтага парога яны пачалі касіць усё, што нязначна для Расіі і выкарыстоўваюць тую ж стратэгію для беларускіх артыкулаў. Іншымі словамі, для “свадобных” ведаў у рувікі выкарыстоўваюцца геаграфічныя і культурна-палітычныя межы.


{\noindent \bf L: Тут выразна чутны голас пірацкай партыі :-)}

{\noindent \bf A:} Я пакуль спрабую перасэнсаваць гэтае пытанне\ldots Я за тое, каб гэтыя ліцэнзіі прысутнічалі, я сам прасоўваю ліцензіі Creative Com\-mons ў Беларусі. Але калі займаешся іх прамоўшнам\ldots Так, пад вікіпедыяй ёсць Creatіve Commons, але 90\% вікіпедыстаў разумеюць гэтую ліцэнзію хутчэй праз паводзіны ў супольнасці, а не ў юрыдычным плане. Паовдзіны супольнасці фарміруюць правілы "--- змест гэтых ліцэнзій, а не наадварот. Гэта для мяне важный момант\ldots

А ліцэнзіі на жалеза "--- гэта больш накшталт патэнтаў. У патэнтаў ёсць адзін плюс\ldots У іх ёсць шмат мінусаў, але адзін плюс, якому пакуль не прыдумана альтэрнатыва. Яны ўсё ўключаюць у адну базу дадзеных, якую чалавек можа праглядаць. Вось калі open source community, якое працуе з hardware, выпрацуе нейкі падобны механізм з класіфікатарам, катэгарызацыяй "--- тады гэта можа быць годнай альтэрнатывай, але пакуль такога механізму няма. Гэта выклік "--- ці могуць актывісты open hardware такое зрабіць, але такое трэба. Першая ўмова росту папулярнасці вольных ліцэнзій для hardware "--- гэта стварыць сістэму, якая будзе больш эфектыўная, больш адкрытая, чым патэнтнае бюро, патэнтныя офісы па розных краінах.

  
 
\end{document}


