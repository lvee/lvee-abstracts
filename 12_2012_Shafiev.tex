\documentclass[10pt, a5paper]{article}
\usepackage{pdfpages}
\usepackage{parallel}
\usepackage[T2A]{fontenc}
\usepackage{ucs}
\usepackage[utf8x]{inputenc}
\usepackage[polish,english,russian]{babel}
\usepackage{hyperref}
\usepackage{rotating}
\usepackage[inner=2cm,top=1.8cm,outer=2cm,bottom=2.3cm,nohead]{geometry}
\usepackage{listings}
\usepackage{graphicx}
\usepackage{wrapfig}
\usepackage{longtable}
\usepackage{indentfirst}
\usepackage{array}
\newcolumntype{P}[1]{>{\raggedright\arraybackslash}p{#1}}
\frenchspacing
\usepackage{fixltx2e} %text sub- and superscripts
\usepackage{icomma} % коскі ў матэматычным рэжыме
\PreloadUnicodePage{4}

\newcommand{\longpage}{\enlargethispage{\baselineskip}}
\newcommand{\shortpage}{\enlargethispage{-\baselineskip}}

\def\switchlang#1{\expandafter\csname switchlang#1\endcsname}
\def\switchlangbe{
\let\saverefname=\refname%
\def\refname{Літаратура}%
\def\figurename{Іл.}%
}
\def\switchlangen{
\let\saverefname=\refname%
\def\refname{References}%
\def\figurename{Fig.}%
}
\def\switchlangru{
\let\saverefname=\refname%
\let\savefigurename=\figurename%
\def\refname{Литература}%
\def\figurename{Рис.}%
}

\hyphenation{admi-ni-stra-tive}
\hyphenation{ex-pe-ri-ence}
\hyphenation{fle-xi-bi-li-ty}
\hyphenation{Py-thon}
\hyphenation{ma-the-ma-ti-cal}
\hyphenation{re-ported}
\hyphenation{imp-le-menta-tions}
\hyphenation{pro-vides}
\hyphenation{en-gi-neering}
\hyphenation{com-pa-ti-bi-li-ty}
\hyphenation{im-pos-sible}
\hyphenation{desk-top}
\hyphenation{elec-tro-nic}
\hyphenation{com-pa-ny}
\hyphenation{de-ve-lop-ment}
\hyphenation{de-ve-loping}
\hyphenation{de-ve-lop}
\hyphenation{da-ta-ba-se}
\hyphenation{plat-forms}
\hyphenation{or-ga-ni-za-tion}
\hyphenation{pro-gramming}
\hyphenation{in-stru-ments}
\hyphenation{Li-nux}
\hyphenation{sour-ce}
\hyphenation{en-vi-ron-ment}
\hyphenation{Te-le-pathy}
\hyphenation{Li-nux-ov-ka}
\hyphenation{Open-BSD}
\hyphenation{Free-BSD}
\hyphenation{men-ti-on-ed}
\hyphenation{app-li-ca-tion}

\def\progref!#1!{\texttt{#1}}
\renewcommand{\arraystretch}{2} %Іначай формулы ў матрыцы зліпаюцца з лініямі
\usepackage{array}

\def\interview #1 (#2), #3, #4, #5\par{

\section[#1, #3, #4]{#1 -- #3, #4}
\def\qname{LVEE}
\def\aname{#1}
\def\q ##1\par{{\noindent \bf \qname: ##1 }\par}
\def\a{{\noindent \bf \aname: } \def\qname{L}\def\aname{#2}}
}

\def\interview* #1 (#2), #3, #4, #5\par{

\section*{#1\\{\small\rm #3, #4. #5}}

\def\qname{LVEE}
\def\aname{#1}
\def\q ##1\par{{\noindent \bf \qname: ##1 }\par}
\def\a{{\noindent \bf \aname: } \def\qname{L}\def\aname{#2}}
}


\begin{document}

\title{Применение Erlang и Perl в ISP}%\footnote{Текст данных и последующих тезисов, кроме специально оговоренных случаев, доступен под лицензией Creative Commons Attribution-ShareAlike 3.0}

\author{Наим Шафиев\footnote{Баку, Азербайджан}}
\maketitle

\begin{abstract}
The article provides a comparison of Erlang and Perl being used by ISPs.
In telecommunication we have a couple of tasks which can be resolved by different ways, and also with different languages. The Erlang is a "king" in telecommunication, but the Perl is really applicable to many tasks. An attempt is presented to compare their most applicable features. Language comparing is done along with following criteria: amount and quality of modules (CEAN, CPAN), education aspect, tools for debug, tools for profiling/benchmarking, IDE, development of language by type, tools for ISPs (sniffer, netflow tools, network monitors).
\end{abstract}


В сфере телекоммуникаций, как и в других сферах ИТ при разработке действует следующее правило. Для решения определенной задачи лучше всего использовать инструмент, который:

\begin{enumerate}
  \item спроектирован под решение этого рода задач
  \item имеет набор готовых библиотек (<<изобретать велосипед>> как правило увлекательно, но  недальновидно)
\end{enumerate}

Но данные постулаты хорошо применимы только когда разработка начата с чистого листа, задачи хорошо конкретизированы и не меняются со временем.
Как известно, в сфере телекоммуникаций (ISP) сложно заранее закладывать риски по вполне понятным причинам (рост траффика, изменение его качественной структуры), а следовательно, приходиться комбинировать инструменты. Кроме того накладываются как ограничивающий фактор исторические элементы системы (так называемые унаследованные свойства или legacy).

На основе суммированного опыта нами раcсматривается вопрос применимости языков Perl и Erlang для различных задач при различных условиях.

Сравнение языков проведено по следующим прикладным моментам:

\begin{enumerate}
  \item Объем и качество библиотек
  \item Вопрос обучения (взаимозаменяемости программистов)
  \item Инструменты для отладки
  \item Инструменты для профайлинга/бенчмаркинга
  \item IDE
  \item Схема и основные коммитеры в развитии языка
  \item Инструменты специфические для сферы ISP (снифферы, сетевые мониторы)
\end{enumerate}

Суммированно к плюсам Erlang как платформы  по отношению к Perl можно отнести её стабильность, высокую адаптацию под задачи отрасли (встроенные механизмы интеркоммуникации по сети, механизмы балансировки и избыточности, библиотека OTP), поддержку со стороны крупных компаний, встроенные механизмы параллелизации.
Но есть и минусы по отношению к Perl: размер библиотек в разы меньше чем у CPAN, более высокая сложность обучения, более слабое сообщество, меньше сторонней документации, отчасти сильное влияние крупных компаний на развитие языка.




\end{document}




