\documentclass[10pt, a5paper]{article}
\usepackage{pdfpages}
\usepackage{parallel}
\usepackage[T2A]{fontenc}
\usepackage{ucs}
\usepackage[utf8x]{inputenc}
\usepackage[polish,english,russian]{babel}
\usepackage{hyperref}
\usepackage{rotating}
\usepackage[inner=2cm,top=1.8cm,outer=2cm,bottom=2.3cm,nohead]{geometry}
\usepackage{listings}
\usepackage{graphicx}
\usepackage{wrapfig}
\usepackage{longtable}
\usepackage{indentfirst}
\usepackage{array}
\newcolumntype{P}[1]{>{\raggedright\arraybackslash}p{#1}}
\frenchspacing
\usepackage{fixltx2e} %text sub- and superscripts
\usepackage{icomma} % коскі ў матэматычным рэжыме
\PreloadUnicodePage{4}

\newcommand{\longpage}{\enlargethispage{\baselineskip}}
\newcommand{\shortpage}{\enlargethispage{-\baselineskip}}

\def\switchlang#1{\expandafter\csname switchlang#1\endcsname}
\def\switchlangbe{
\let\saverefname=\refname%
\def\refname{Літаратура}%
\def\figurename{Іл.}%
}
\def\switchlangen{
\let\saverefname=\refname%
\def\refname{References}%
\def\figurename{Fig.}%
}
\def\switchlangru{
\let\saverefname=\refname%
\let\savefigurename=\figurename%
\def\refname{Литература}%
\def\figurename{Рис.}%
}

\hyphenation{admi-ni-stra-tive}
\hyphenation{ex-pe-ri-ence}
\hyphenation{fle-xi-bi-li-ty}
\hyphenation{Py-thon}
\hyphenation{ma-the-ma-ti-cal}
\hyphenation{re-ported}
\hyphenation{imp-le-menta-tions}
\hyphenation{pro-vides}
\hyphenation{en-gi-neering}
\hyphenation{com-pa-ti-bi-li-ty}
\hyphenation{im-pos-sible}
\hyphenation{desk-top}
\hyphenation{elec-tro-nic}
\hyphenation{com-pa-ny}
\hyphenation{de-ve-lop-ment}
\hyphenation{de-ve-loping}
\hyphenation{de-ve-lop}
\hyphenation{da-ta-ba-se}
\hyphenation{plat-forms}
\hyphenation{or-ga-ni-za-tion}
\hyphenation{pro-gramming}
\hyphenation{in-stru-ments}
\hyphenation{Li-nux}
\hyphenation{sour-ce}
\hyphenation{en-vi-ron-ment}
\hyphenation{Te-le-pathy}
\hyphenation{Li-nux-ov-ka}
\hyphenation{Open-BSD}
\hyphenation{Free-BSD}
\hyphenation{men-ti-on-ed}
\hyphenation{app-li-ca-tion}

\def\progref!#1!{\texttt{#1}}
\renewcommand{\arraystretch}{2} %Іначай формулы ў матрыцы зліпаюцца з лініямі
\usepackage{array}

\def\interview #1 (#2), #3, #4, #5\par{

\section[#1, #3, #4]{#1 -- #3, #4}
\def\qname{LVEE}
\def\aname{#1}
\def\q ##1\par{{\noindent \bf \qname: ##1 }\par}
\def\a{{\noindent \bf \aname: } \def\qname{L}\def\aname{#2}}
}

\def\interview* #1 (#2), #3, #4, #5\par{

\section*{#1\\{\small\rm #3, #4. #5}}

\def\qname{LVEE}
\def\aname{#1}
\def\q ##1\par{{\noindent \bf \qname: ##1 }\par}
\def\a{{\noindent \bf \aname: } \def\qname{L}\def\aname{#2}}
}


\begin{document}

\title{Построение частного облака на базе дистрибутива Proxmox Virtual Environment}%\footnote{Текст данных и последующих тезисов, кроме специально оговоренных случаев, доступен под лицензией Creative Commons Attribution-ShareAlike 3.0}

\author{Дмитрий Ванькевич\footnote{Львов, Украина}}
\maketitle

\begin{abstract}
Specifics and purposefullness of private cloud usage is discussed. Demands to the cloud are reviewed, as one one of the possible solutions, based on the Proxmox Virtual Environment distributive.
\end{abstract}


По модели развёртывания облака делятся на частные, публичные и смешанные.
Частное облако --- это вариант локальной реализации <<облачной концепции>>, когда компания создает ее для себя самой, в рамках одной организации. В отличие от него, публичное облако используется облачными провайдерами для предоставления сервисов внешним заказчикам, а смешанное или гибридное облако --- это вариант совместного использования двух вышеперечисленных моделей.

Национальным институтом стандартов и технологий США зафиксированы следующие обязательные характеристики облачных вычислений:

\begin{itemize}
  \item Самообслуживание по требованию (англ. self service on \linebreak demand), потребитель самостоятельно определяет и изменяет вычислительные потребности, такие как серверное время, скорости доступа и обработки данных, объём хранимых данных без взаимодействия с представителем поставщика услуг;
  \item Универсальный доступ по сети, услуги доступны потребителям по сети передачи данных вне зависимости от используемого терминального устройства;
  \item Объединение ресурсов (англ. resource pooling), поставщик \linebreak услуг объединяет ресурсы для обслуживания большого числа потребителей в единый пул для динамического перераспределения мощностей между потребителями в условиях постоянного изменения спроса на мощности; при этом потребители контролируют только основные параметры услуги (например, объём данных, скорость доступа), но фактическое распределение ресурсов, предоставляемых потребителю, осуществляет поставщик (в некоторых случаях потребители всё-таки могут управлять некоторыми физическими параметрами перераспределения, например, указывать желаемый центр обработки данных из соображений географической близости);
  \item Эластичность, услуги могут быть предоставлены, расширены, сужены в любой момент времени, без дополнительных издержек на взаимодействие с поставщиком, как правило, в автоматическом режиме;
  \item Учёт потребления, поставщик услуг автоматически исчисляет потреблённые ресурсы на определённом уровне абстракции (например, объём хранимых данных, пропускная способность, количество пользователей, количество транзакций), и на основе этих данных оценивает объём предоставленных потребителям услуг \cite{Vankevich1}.
\end{itemize}

Целесообразность внедрения облачных технологий определяется здравым смыслом, финансовыми возможностями заказчика, недостатками существующей <<безоблачной>> компьютерной инфраструктуры, осознанием потребности и готовностью внедрять новые технологии. При этом вариант частного облака часто оказывается единственным доступным вариантом --- не в последнюю очередь из-за отсутствия <<широких>> каналов связи. В качестве примера организаций, для которых автору приходилось решать задачу внедрения облачных вычислений по модели частного облака, можно привести:

\begin{enumerate}
  \item Компьютерные лаборатории общего использования Львовского национального университета имени Ивана Франко, где уже применялись технологии виртуализации в учебном процессе \cite{Vankevich3}.
  \item Небольшая фирма по разработке программного обеспечения, обнаружившая потребность в активном использовании технологий виртуализации, и имеющая около тридцати сотрудников, работающих с пятью независимыми серверами виртуальных машин.
\end{enumerate}

В первом случае внедрение облачных технологий представляет академический интерес, во втором, при планируемом увеличении количества серверов виртуальных машин, позволит эффективнее использовать аппаратное обеспечение и даст возможность удобного управления серверами виртуальных машин.

Среди платформ, подходящих для построения частного облака из доступных аппаратных компонентов, автором был выбран дистрибутив Proxmox Virtual Environment  \cite{Vankevich4}.

Proxmox Virtual Environment дает возможность использовать следующие технологии виртуализации:

\begin{itemize}
  \item OpenVZ --- виртуализация на уровне операционной системы. Позволяет на одном физическом сервере запускать множество изолированных копий операционной системы.
  \item KVM (Kernel-based Virtual Machine) --- позволяет запускать виртуализованную ОС при аппаратной поддержке процессора (Intel VT, AMD-V и др.) и эмуляции периферии при помощи QUEMU.
\end{itemize}

К числу достоинств данной платформы можно отнести:

\begin{itemize}
  \item Удобный инсталлятор (bare metal ISO-installer) позволяющий развернуть сервер виртуальных машин за 10--15 минут;
  \item Простое управление через веб-интерфейс (с сохранением возможности управления через интерфейс командной строки);
  \item Наличие библиотеки готовых шаблонов OpenVZ, готовых для промышленного использования (<<production ready>>);
  \item Встроенная система мониторинга;
  \item Поддержка различных типов аутентификации: MS ADS, LDAP, Linux PAM, Proxmox VE authentication;
  \item Гибкое управление правами доступа групп пользователей к консоли управления виртуальными машинами на основе ролей;
  \item Наличие Proxmox VE API;
  \item Возможность объединения серверов в кластер и живой миграции виртуальных машин (без остановки гостевой системы);
  \item Встроенная система автоматического резервного копирования
  \item Поддержка систем хранения Directory, LVM group, NFS share, iSCSI target.
  \item Возможность работы на распространённом аппаратном обеспечении (некоторые проприетарные системы виртуализации требуют сертифицированное аппаратное обеспечение). Это особенно \linebreak важно для учебных заведений, которые зачастую не обладают возможностью выбора при закупке оборудования.
\end{itemize}

По результатам работы с дистрибутивом  Proxmox Virtual \linebreak Environment можно сделать заключение, что он  соответствует перечисленным обязательным характеристикам для облачных вычислений.


\begin{thebibliography}{9}
\bibitem{Vankevich1} \url{http://ru.wikipedia.org/wiki/\%D0\%9E\%D0\%B1\%D0\%BB\%D0\%B0\%D1\%87\%D0\%BD\%D1\%8B\%D0\%B5_\%D0\%B2\%D1\%8B\%D1\%87\%D0\%B8\%D1\%81\%D0\%BB\%D0\%B5\%D0\%BD\%D0\%B8\%D1\%8F}
\bibitem{Vankevich2} И.П. Клементьев, В.А. Устинов. Введение в облачные вычисления. \url{http://www.intuit.ru/department/se/incloudc/3/3.html}
\bibitem{Vankevich3} Д. Ванькевич, Г. Злобин Быстрое развёртывание учебного полигона для проведения лабораторных работ в курсе «Системное администрирование ОС Linux» в компьютерных лабораториях общего пользования. \url{http://freeschool.altlinux.ru/wp-content/uploads/2012/01/zlobin.pdf}
\bibitem{Vankevich4} \url{http://pve.proxmox.com/wiki/Main\_Page}
\end{thebibliography}


\end{document}




