\documentclass[10pt, a5paper]{article}
\usepackage{pdfpages}
\usepackage{parallel}
\usepackage[T2A]{fontenc}
\usepackage{ucs}
\usepackage[utf8x]{inputenc}
\usepackage[polish,english,russian]{babel}
\usepackage{hyperref}
\usepackage{rotating}
\usepackage[inner=2cm,top=1.8cm,outer=2cm,bottom=2.3cm,nohead]{geometry}
\usepackage{listings}
\usepackage{graphicx}
\usepackage{wrapfig}
\usepackage{longtable}
\usepackage{indentfirst}
\usepackage{array}
\newcolumntype{P}[1]{>{\raggedright\arraybackslash}p{#1}}
\frenchspacing
\usepackage{fixltx2e} %text sub- and superscripts
\usepackage{icomma} % коскі ў матэматычным рэжыме
\PreloadUnicodePage{4}

\newcommand{\longpage}{\enlargethispage{\baselineskip}}
\newcommand{\shortpage}{\enlargethispage{-\baselineskip}}

\def\switchlang#1{\expandafter\csname switchlang#1\endcsname}
\def\switchlangbe{
\let\saverefname=\refname%
\def\refname{Літаратура}%
\def\figurename{Іл.}%
}
\def\switchlangen{
\let\saverefname=\refname%
\def\refname{References}%
\def\figurename{Fig.}%
}
\def\switchlangru{
\let\saverefname=\refname%
\let\savefigurename=\figurename%
\def\refname{Литература}%
\def\figurename{Рис.}%
}

\hyphenation{admi-ni-stra-tive}
\hyphenation{ex-pe-ri-ence}
\hyphenation{fle-xi-bi-li-ty}
\hyphenation{Py-thon}
\hyphenation{ma-the-ma-ti-cal}
\hyphenation{re-ported}
\hyphenation{imp-le-menta-tions}
\hyphenation{pro-vides}
\hyphenation{en-gi-neering}
\hyphenation{com-pa-ti-bi-li-ty}
\hyphenation{im-pos-sible}
\hyphenation{desk-top}
\hyphenation{elec-tro-nic}
\hyphenation{com-pa-ny}
\hyphenation{de-ve-lop-ment}
\hyphenation{de-ve-loping}
\hyphenation{de-ve-lop}
\hyphenation{da-ta-ba-se}
\hyphenation{plat-forms}
\hyphenation{or-ga-ni-za-tion}
\hyphenation{pro-gramming}
\hyphenation{in-stru-ments}
\hyphenation{Li-nux}
\hyphenation{sour-ce}
\hyphenation{en-vi-ron-ment}
\hyphenation{Te-le-pathy}
\hyphenation{Li-nux-ov-ka}
\hyphenation{Open-BSD}
\hyphenation{Free-BSD}
\hyphenation{men-ti-on-ed}
\hyphenation{app-li-ca-tion}

\def\progref!#1!{\texttt{#1}}
\renewcommand{\arraystretch}{2} %Іначай формулы ў матрыцы зліпаюцца з лініямі
\usepackage{array}

\def\interview #1 (#2), #3, #4, #5\par{

\section[#1, #3, #4]{#1 -- #3, #4}
\def\qname{LVEE}
\def\aname{#1}
\def\q ##1\par{{\noindent \bf \qname: ##1 }\par}
\def\a{{\noindent \bf \aname: } \def\qname{L}\def\aname{#2}}
}

\def\interview* #1 (#2), #3, #4, #5\par{

\section*{#1\\{\small\rm #3, #4. #5}}

\def\qname{LVEE}
\def\aname{#1}
\def\q ##1\par{{\noindent \bf \qname: ##1 }\par}
\def\a{{\noindent \bf \aname: } \def\qname{L}\def\aname{#2}}
}


\def\vv!#1!{\texttt{#1}}
\begin{document}

\switchlang{be}

\title{Лінуксоўка ў новым фармаце, занатоўкі выпадковага ўдзельніка аргкамітэта}%\footnote{Текст данных и последующих тезисов, кроме специально оговоренных случаев, доступен под лицензией Creative Commons Attribution-ShareAlike 3.0}

\author{Андрэй Захарэвіч\footnote{Мінск, Беларусь; \url{}}}
\maketitle

\begin{abstract}
Minsk LUG monthly meetings. Explaining of new meeting format, giving a to do list of organizational Board and discussing organizational aims which was (or was not) reached.
\end{abstract}

\subsection*{Крыху пра гісторыю лінуксовак MLUG}

Безумоўна, лінуксоўкі ў Мінску — з'ява не новая. Яны налічваюць мінімум тры «пакаленні» ўдзельнікаў і мінімум адну спробу афіцыйна аформіць юрыдычную асобу, ад імя якой маглі б праводзіцца афіцыйныя камунікацыі MLUG, калі гэта спатрэбіцца.

І гэта цалкам натуральна, таму што Linux і іншыя Unix-падобныя сістэмы з'явіліся ў Мінску некалькі дзесяцігоддзяў таму, яшчэ ў мінулым стагоддзі і, вядома ж, карыстальнікі і аматары гэтых сістэм мелі патрэбу ў камунікацыі з аднадумцамі не толькі з выкарыстаннем нейкіх сеткавых тэхналогій, але і пры асабістай сустрэчы.

Я не даследваў пытанне глыбока, але першыя тэмы ў раздзеле «Линуксовка» на forum.linux.by датуюцца пачаткам 2002 года. Апошняя версія кіраўніцтва (HOWTO) па правядзенню лінуксовак на Wiki-старонцы \url{http://wiki.linux.by/wiki/Linuxovka\_HOWTO} была створаная 10 ліпеня 2007 года карыстальнікам Mend0za.

\subsection*{Інфармацыйныя рэсусы, якія аб'ядноўвалі Linux-супольнасць на пачатку XXI стагоддзя}

Карыстаючыся толькі прыгаданым вышэй HOWTO, можна зазначыць, што існавалі як мінімум:

\begin{enumerate}
  \item Форум \url{forum.linux.by}. Месца, адносна жывое і зараз, калі значная частка актыўнасці, асабліва ў моладзі (да 25 год) перайшла ў сацыяльныя сеткі. Даволі даўно існуе больш для пачынаючых.
  \item Паштовыя рассылкі. Крыніца прыгадвае наступныя: \textit{«Весьма живая рассылка LVEE. На рассылки mlug-talks@ и mlug-announce@ подписано около 100 человек Подробности про рассылки. Наиболее серьёзная и многочисленная часть из подписчиков форумом пренебрегает, заслуженно считая его песочницей для пионеров.»}
  \item IRC: Каналы \vv!\#unix! і \vv!\#linux! на \vv!irc.bynets.org! і \vv!irc.by!.
  \item Fidonet: Эха \vv!bel.softw.unix!
\end{enumerate}

Такім чынам, магчымасці абвесціць лінуксоідам пра мерапрыемства на пачатку стагоддзя існавалі. І, як мне здаецца, казаць што ў спадзе лінуксовак вінаватая адсутнасць пляцовак для анонсаў і камунікацыі нельга.

\subsection*{Спад актыўнасці лінуксовак}

Наколькі я ведаю, памяньшэнне актыўнасці лінуксовак гэта з'ява не спецыфічная для Беларусі, такое адбываецца ва ўсім свеце.

На маю думку, гэтак адбываецца таму, што лінуксоўкі ў тым выглядзе, як яны існавалі на той час, маглі прапанаваць усё менш і менш перавагаў у параўнанні з анлайн-камунікацыямі:
\begin{itemize}
  \item З гадамі ўсё больш дасканалая праца пашуковых сістэм дала магчымасць шукаць рэсурсы з адказамі на пытанні мінімальна разумеючы што ты шукаеш без асаблівых навыкаў складання пашуковых запытаў. А пашырэнне папулярнасці Linux дае магчымасць адшукаць адказы на большасць пытанняў, бо нехта ўжо імі цікавіўся.
  \item Пашырэнне выкарыстання камунікацый з дапамогай сацыяльных сетак прыводзіць да таго, што для сустрэчы ў афлайне павінна быць досыць важкая нагода. Проста адшукаць патрэбнага чалавека і абмеркаваць адно нейкае асобнае пытанне збольшага можна ўжо не толькі не ўставаючы з-за кампутара, але ўжо нават не ўключаючы яго, з дапамогай смартфона ці планшэта.
\end{itemize}

Безумоўна, адна нейкая праблема не магла пахаваць мерапрыемства, таму я бы яшчэ прыгадаў:
\begin{itemize}
  \item Нерэгулярнасць і непрадказальнасць раскладу. Арганізацыя лінуксовак, не кажучы пра змястоўны ўдзел у іх — гэта праца, якая патрабуе выдаткаваць сілы і час. Не маючы нейкага сур'ёзнага выніка гэтай працы ніхто не захоча займацца гэтым рэгулярна колькі-небуць працягчы час
  \item Фармат лінуксовак не даваў магчымасці расці далей
\end{itemize}

\subsection*{Класічны фармат лінуксовак у Мінску}

Безумоўна, не ўсе лінуксоўкі праходзілі так. Часам бывалі нейкія мерапрыемствы з дакладамі ў прыстасаваным для гэтага памяшканні, але большасць праходзіла ў нейкіх грамадскіх месцах, кшталту сквера на плошчы Якуба Коласа, ці на Бульвары Мулявіна.

З пэўнага часу часцей за ўсё выкарыстоўвалася пляцоўка збору «каля фантана», удзельнікі лінуксоўкі сустракаліся каля фантана «Вянок» (1972 год, скульптары А. Анікейчык, Л. Гумілеўскі, А. Заспіцкі) ў парку імя Я. Купалы. Як самае вядомае месца правядзення, «каля фантана» дало назву фармату лінуксовак.

\subsection*{Чаму б не адкапаць яшчэ раз сцюардэсу «сустрэчы каля фантана»}

З'яўленне новага фармату лінуксовак было выкліканае тым, што класічны фармат «сустрэч з півам каля фантану» не задавальняў найбольш актыўных удзельнікаў з-за яго натуральных абмежаванняў:
\begin{enumerate}
  \item Неканструктыўнасць фармату не дазваляла праводзіць маштабных абмеркаванняў тэхнічных пытанняў і праблем. Цяжка расказваць пра нешта сур'ёзнае групе чалавек ў 10—15 пасярод парка ў цэнтры горада. Апроч таго, што нязручна паказваць нейкія схемы ці скрыншоты, дадаецца яшчэ той факт, што для такой сустрэчы паводле заканадаўства з мінулага года патрэбны дазвол гарадскіх ўлад з доўгім узгадненнем.
  \item Далёка не ўсім падабаўся сам фармат размовы «пад піва». Раней ці пазней лінуксоўка распадаецца на групы і драбіцца. Цытуючы прыгаданы вышэй HOWTO па лінуксоўках, \textit{«Да простят меня начинающие — но им обычно около-линуксовых тем хватает на полчаса-час (чтобы исчерпать круг интересов и личный опыт), потом начинается про бухло, тёлок и за жизнь.»}
  \item Сутракацца пад аткрытым небам зручна толькі пакуль пагода камфортная. То бок паводле самых аптымістычных ацэнак — 3—5 месяцаў на год.
\end{enumerate}

Апошні момант спрабавалі вырашаць, арганізуючы лінуксоўкі, напрыклад, у кафэ (\cite{zack1}). Аднак две першыя праблемы гэта не вырашала. Плюс, паколькі кафэ ў Мінску не хапае, то рэстаратары пачынаюць патрабаваць заказ не менш як на пэўную суму, папярэдні заказ. Магчымасць падсілкавацца пад час мерапрыемства можа быць не лішняй, але мы ж не для таго збіраліся?

Існуе меркаванне, што адной з прычын спаду (ці, нават, галоўнай прычынай ў некаторых інтэрпрэтацыях) было тое, што большасць удзельнікаў пасталелі і былі вельмі занятыя. Незгодны і лічу, што гэта звычайная адгаворка. Пры ўмове папярэдняга анонса даты мерапрыемства мінімум за 3—4 тыдні нават вельмі занятыя асобы знаходзяць час і сілы актыўна паўдзельнічаць. Некалькі прыкладаў (у выпадковым парадку):

\begin{itemize}
  \item {Максім Мельнікаў}, спецыяліст з кампаніі Wargaming, шчаслівы бацька (што не дадае вольнага часу і сіл) рэгулярна робіць цікавыя і змястоўныя даклады
  \item {Аляксей Чэвусаў}, таксама спецыяліст, бацька, FOSS актывіст і распрацоўшчык, што не зашкодзіла яму прыняць удзел у большасці лінуксовак, практычна кожны раз з дакладам, актыўна удзельнічаць у абмеркаваннях чужых дакладаў.
  \item {Уладзімір Шахаў}, зноўку ж, IT-спецыяліст, бацька, трэнер, галоўны сакратар аднаго з самых буйных спаборніцтваў па каратэ версіі WKF не толькі ў краіне, але і ў рэгіёне. Удзельнічае ў падрыхтоўцы лінуксовак ад з'яўлення іх у новым фармаце, рэгулярна робіць даклады, актыўна ўдзельнічае ў абмеркаваннях.
\end{itemize}

\subsection*{Што рабіць каб пераадолець спад актыўнасці}

Можа гэта не агульны рэцэпт, але падобна, што наш вопыт не ўнікальны і можа быць выкарыстаны іншымі.

Узяўшы за аснову фармат LVEE, арганізатары паспрабавалі стварыць нешта кшталту «міні-LVEE», які дазваляў бы і выказацца па нейкіх глабальных праблемах, і, пры неабходнасці, камерна абмеркаваць нешта ў кулуарах.

Стварэнне менавіта асобнага мерапрыемства было неабходнае таму, што LVEE праводзіцца раз на год, і далёка не усё будзе сэнс абмяркоўваць праз паўгады і больш. Павінна быць месца, куды не трэба выязджаць, каб задаць пытанні ці абмеркаваць нешта, куды могуць прыехаць і падзяліцца досведам людзі, якія не часта бываюць у Мінску. Снежаньская лінуксоўка традыцыйна з'яўляецца «эмігранцкай», у ёй удзельнічаюць спецыялісты з Мінску і Беларусі, якія зараз не жывуць на радзіме.

\subsection*{Што трэба, каб арганізаваць лінуксоўку}

\begin{enumerate}
  \item Патрэбнае памяшканне, якое зможа змясціць усіх ахвотных. У нашым выпадку павінна хапаць месца, каб рассадзіць 2—3 тузіны чалавек, плюс крыху вольнай прасторы, каб можна было рухацца і абмяркоўваць у купках.
  \item Патрэбныя актыўныя удзельнікі, якія змогуць і захочуць патраціць час на падрыхтоўку цікавых і змястоўных дакладаў, адпавядаючых тэматыцы мерапрыемства.
  \item Патрэбныя актыўныя удзельнікі, хто зможа і захоча ўдзельнічаць у абмеркаваннях і задаваць пытанні (ці даваць парады) дакладчыкам.
  \item Патрэбнае размяшчэнне інфармацыі пра лінуксоўку на тэматычных пляцоўках (гл. раздзел «Дзе трэба размясціць інфармацыю пра лінуксоўку»).
  \item Месца правядзення павінна быць спланавана такім чынам, каб анонс мог быць змешчаны прыблізна за месяц. Большасць актыўных удзельнікаў мінімум прафесіяналы, шмат сярод іх і бацькоў, якім трэба папярэдне планаваць вольны час.
\end{enumerate}

\subsection*{Дзе трэба размясціць інфармацыю пра лінуксоўку}

Усе больш-менш звязаныя з тэматыкай пляцоўкі павінны быць накрытыя. На дадзены момант у мяне склаўся такі спіс пляцовак, на якія варта змясціць навіну пра лінуксоўку MLUG:

\begin{enumerate}
  \item Сайт MLUG. Ствараецца навіна з анонсам. (\cite{mlug}).
  \item Група MLUG на Google+. Эфектыўней за ўсё стварыць падзею у межах групы. (\cite{mlug-gplus}).
  \item Форум на Linux.by. Стварыць новую тэму ў раздзеле пра лінуксоўкі (\cite{mlug-forum}).
  \item Ліст з анонсам у групу рассылкі \vv!mlug-talks@! (\cite{mlug-talks}).
  \item Анонс падзеі на \url{meetup.by}
  \item Падзея ў Facebook, у групе MLUG (\cite{mlug-fb}).
  \item Падзея з анонсам у групах у Вконтакте MLUG (\cite{mlug-vk-1}) і Linux.by (\cite{mlug-vk-2}).
  \item Скінуць спасылку на анонс у Twitter (хэштэгі \vv!\#mlug! і \vv!\#linux!).
  \item Анонс у ЖЖ, можна ў тэматычных групах.
  \item Ліст з анонсам у LVEE mailing list.
\end{enumerate}

\subsection*{Што не атрымалася}

Не атрымалася пляцоўка для абмеркаванняў і навучання пачынаючых. Паколькі арганізатары не фільтруюць тэмы паводле ўзроўню аўдыторыі, а толькі паводле тэматыкі, каб пляцоўка давала магчымасці для пачынаючых яны іх павінны браць. То бок, агульная тэма і накірунак лінуксоўкі вызначаецца самімі ўдзельнікамі. Калі нехта прыйшоў пасядзець моўчкі ў куточку — ён пасядзіць, але тады на развіццё падзей ён не ўплывае наогул аніяк.

Як прымусіць чалавека быць актыўным, я не ведаю. Але калі нехта параіць — буду вельмі ўдзячны.

\subsection*{Ці самотныя мы на гэтым свеце?}

Нашчасце, не! Друпал-клуб (falanster.by) праводзіць свае лінуксоўкі і іншыя мерапрыемствы.

Плюсы:

\begin{itemize}
  \item мерапрыемствы праводзяцца на ўзроўні зручным для тых, хто пачынае свой шлях у Лінукс;
  \item лінуксоўка вельмі камерная, менш за 10 чалавек, таму можа быць не так цяжка наважыцца задаць пытанні, абмеркаваць нешта;
  \item большасць удзельнікаў лінуксоўкі рыхтуецца з нейкімі тэмамі, па выніках дакладаў па тэме праходзіць абмеркаванне.
\end{itemize}

І гэта не кажучы пра LVEE, якога з мінулага года стала ў два разы больш штогод.

\subsection*{Што рабіць далей?}

Лінукс у масы! То бок, кожны можа прыняць удзел у бліжэйшай рэгіянальнай лінуксоўцы ці стварыць сваю. Не хапае групы людзей для камунікацыі і самаразвіцця? Ствары сваё кола знаёмых для камунікацыі і сваю групу!

\let\saverefname=\refname%
\def\refname{Спасылкі}%
\begin{thebibliography}{9}
\bibitem{zack1} \url{https://forum.linux.by/viewtopic.php?f=7\&t=9951}
\bibitem{mlug} \url{http://mlug.linux.by/}
\bibitem{mlug-gplus} \url{https://plus.google.com/communities/110681015343287679167}
\bibitem{mlug-forum} \url{https://forum.linux.by/viewforum.php?f=7}
\bibitem{mlug-talks} \url{http://groups.google.com/group/mlug-talks}
\bibitem{mlug-fb} \url{http://www.facebook.com/groups/46184759521/}
\bibitem{mlug-vk-1} \url{http://vk.com/club1271407}
\bibitem{mlug-vk-2} \url{http://vk.com/club8911620}
\end{thebibliography}
\let\refname=\saverefname%

\end{document}




