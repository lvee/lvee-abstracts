\documentclass[10pt, a5paper]{article}
\usepackage{pdfpages}
\usepackage{parallel}
\usepackage[T2A]{fontenc}
\usepackage{ucs}
\usepackage[utf8x]{inputenc}
\usepackage[polish,english,russian]{babel}
\usepackage{hyperref}
\usepackage{rotating}
\usepackage[inner=2cm,top=1.8cm,outer=2cm,bottom=2.3cm,nohead]{geometry}
\usepackage{listings}
\usepackage{graphicx}
\usepackage{wrapfig}
\usepackage{longtable}
\usepackage{indentfirst}
\usepackage{array}
\newcolumntype{P}[1]{>{\raggedright\arraybackslash}p{#1}}
\frenchspacing
\usepackage{fixltx2e} %text sub- and superscripts
\usepackage{icomma} % коскі ў матэматычным рэжыме
\PreloadUnicodePage{4}

\newcommand{\longpage}{\enlargethispage{\baselineskip}}
\newcommand{\shortpage}{\enlargethispage{-\baselineskip}}

\def\switchlang#1{\expandafter\csname switchlang#1\endcsname}
\def\switchlangbe{
\let\saverefname=\refname%
\def\refname{Літаратура}%
\def\figurename{Іл.}%
}
\def\switchlangen{
\let\saverefname=\refname%
\def\refname{References}%
\def\figurename{Fig.}%
}
\def\switchlangru{
\let\saverefname=\refname%
\let\savefigurename=\figurename%
\def\refname{Литература}%
\def\figurename{Рис.}%
}

\hyphenation{admi-ni-stra-tive}
\hyphenation{ex-pe-ri-ence}
\hyphenation{fle-xi-bi-li-ty}
\hyphenation{Py-thon}
\hyphenation{ma-the-ma-ti-cal}
\hyphenation{re-ported}
\hyphenation{imp-le-menta-tions}
\hyphenation{pro-vides}
\hyphenation{en-gi-neering}
\hyphenation{com-pa-ti-bi-li-ty}
\hyphenation{im-pos-sible}
\hyphenation{desk-top}
\hyphenation{elec-tro-nic}
\hyphenation{com-pa-ny}
\hyphenation{de-ve-lop-ment}
\hyphenation{de-ve-loping}
\hyphenation{de-ve-lop}
\hyphenation{da-ta-ba-se}
\hyphenation{plat-forms}
\hyphenation{or-ga-ni-za-tion}
\hyphenation{pro-gramming}
\hyphenation{in-stru-ments}
\hyphenation{Li-nux}
\hyphenation{sour-ce}
\hyphenation{en-vi-ron-ment}
\hyphenation{Te-le-pathy}
\hyphenation{Li-nux-ov-ka}
\hyphenation{Open-BSD}
\hyphenation{Free-BSD}
\hyphenation{men-ti-on-ed}
\hyphenation{app-li-ca-tion}

\def\progref!#1!{\texttt{#1}}
\renewcommand{\arraystretch}{2} %Іначай формулы ў матрыцы зліпаюцца з лініямі
\usepackage{array}

\def\interview #1 (#2), #3, #4, #5\par{

\section[#1, #3, #4]{#1 -- #3, #4}
\def\qname{LVEE}
\def\aname{#1}
\def\q ##1\par{{\noindent \bf \qname: ##1 }\par}
\def\a{{\noindent \bf \aname: } \def\qname{L}\def\aname{#2}}
}

\def\interview* #1 (#2), #3, #4, #5\par{

\section*{#1\\{\small\rm #3, #4. #5}}

\def\qname{LVEE}
\def\aname{#1}
\def\q ##1\par{{\noindent \bf \qname: ##1 }\par}
\def\a{{\noindent \bf \aname: } \def\qname{L}\def\aname{#2}}
}


\begin{document}

\title{Разработка сетевых устройств на базе дистрибутива OpenWRT}%\footnote{Текст данных и последующих тезисов, кроме специально оговоренных случаев, доступен под лицензией Creative Commons Attribution-ShareAlike 3.0}

\author{Виктор Полстюк\footnote{Минск, Беларусь}}
\maketitle

\begin{abstract}
The report analyses OpenWRT distribution as a basis for developing network devices. It also highlights limitations that constrain its use in commercial purposes.
\end{abstract}


Сначала OpenWRT развивался как альтернативный дистрибутив для серийно выпускаемых маршрутизаторов, который позволял получить более гибкую по сравнению с предоставляемой производителем систему, и дополнить ее необходимыми сервисами. Позже в дистрибутиве появилась поддержка X11, XFCE и LXDE, что позволяет использовать его на устройствах с графическими дисплеями. Сейчас проект перешел в категорию дистрибутивов общего применения (подобно OpenEmbedded), обладая наиболее широкими возможностями в области телекоммуникаций.

OpenWRT поддерживает большое число аппаратных платформ, построенных на процессорах ARM, MIPS, x86, и содержит свежее ядро с набором специфических для маршрутизаторов патчей. 

Сборочная система, используемая дистрибутивом, построена на пакетной системе, что позволяет выбрать перечень программ для добавления в прошивку и включает 2000+ пакетов в официальном репозитории. Система конфигурирования и инициализации позволяет получить согласованно функционирующую систему из набора выбранных пакетов.

Веб-интерфейс является обязательным компонентом сетевого оборудования.  Для OpenWRT существует несколько фреймворков для пользовательского интерфейса: LuCI, X-wrt. Фреймворк LuCI построен по принципу Model-View-Controller, что предполагает отделение графического представления (шаблонов) от логики работы интерфейса управления.

Имеются примеры коммерческого использования дистрибутива производителями процессоров, OEM-модулей и отладочных плат в качестве сопутствующего SDK. Однако ряд недостатков препятствует использованию mainline OpenWRT в коммерческих целях. К их числу можно отнести отсутствие поддержки аппаратных ускорителей сетевых контроллеров процессоров, отсутствие поддержки централизованных протоколов управления (семейство протоколов TR-069).



\end{document}




