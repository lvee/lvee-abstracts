\documentclass[10pt, a5paper]{article}
\usepackage{pdfpages}
\usepackage{parallel}
\usepackage[T2A]{fontenc}
\usepackage{ucs}
\usepackage[utf8x]{inputenc}
\usepackage[polish,english,russian]{babel}
\usepackage{hyperref}
\usepackage{rotating}
\usepackage[inner=2cm,top=1.8cm,outer=2cm,bottom=2.3cm,nohead]{geometry}
\usepackage{listings}
\usepackage{graphicx}
\usepackage{wrapfig}
\usepackage{longtable}
\usepackage{indentfirst}
\usepackage{array}
\newcolumntype{P}[1]{>{\raggedright\arraybackslash}p{#1}}
\frenchspacing
\usepackage{fixltx2e} %text sub- and superscripts
\usepackage{icomma} % коскі ў матэматычным рэжыме
\PreloadUnicodePage{4}

\newcommand{\longpage}{\enlargethispage{\baselineskip}}
\newcommand{\shortpage}{\enlargethispage{-\baselineskip}}

\def\switchlang#1{\expandafter\csname switchlang#1\endcsname}
\def\switchlangbe{
\let\saverefname=\refname%
\def\refname{Літаратура}%
\def\figurename{Іл.}%
}
\def\switchlangen{
\let\saverefname=\refname%
\def\refname{References}%
\def\figurename{Fig.}%
}
\def\switchlangru{
\let\saverefname=\refname%
\let\savefigurename=\figurename%
\def\refname{Литература}%
\def\figurename{Рис.}%
}

\hyphenation{admi-ni-stra-tive}
\hyphenation{ex-pe-ri-ence}
\hyphenation{fle-xi-bi-li-ty}
\hyphenation{Py-thon}
\hyphenation{ma-the-ma-ti-cal}
\hyphenation{re-ported}
\hyphenation{imp-le-menta-tions}
\hyphenation{pro-vides}
\hyphenation{en-gi-neering}
\hyphenation{com-pa-ti-bi-li-ty}
\hyphenation{im-pos-sible}
\hyphenation{desk-top}
\hyphenation{elec-tro-nic}
\hyphenation{com-pa-ny}
\hyphenation{de-ve-lop-ment}
\hyphenation{de-ve-loping}
\hyphenation{de-ve-lop}
\hyphenation{da-ta-ba-se}
\hyphenation{plat-forms}
\hyphenation{or-ga-ni-za-tion}
\hyphenation{pro-gramming}
\hyphenation{in-stru-ments}
\hyphenation{Li-nux}
\hyphenation{sour-ce}
\hyphenation{en-vi-ron-ment}
\hyphenation{Te-le-pathy}
\hyphenation{Li-nux-ov-ka}
\hyphenation{Open-BSD}
\hyphenation{Free-BSD}
\hyphenation{men-ti-on-ed}
\hyphenation{app-li-ca-tion}

\def\progref!#1!{\texttt{#1}}
\renewcommand{\arraystretch}{2} %Іначай формулы ў матрыцы зліпаюцца з лініямі
\usepackage{array}

\def\interview #1 (#2), #3, #4, #5\par{

\section[#1, #3, #4]{#1 -- #3, #4}
\def\qname{LVEE}
\def\aname{#1}
\def\q ##1\par{{\noindent \bf \qname: ##1 }\par}
\def\a{{\noindent \bf \aname: } \def\qname{L}\def\aname{#2}}
}

\def\interview* #1 (#2), #3, #4, #5\par{

\section*{#1\\{\small\rm #3, #4. #5}}

\def\qname{LVEE}
\def\aname{#1}
\def\q ##1\par{{\noindent \bf \qname: ##1 }\par}
\def\a{{\noindent \bf \aname: } \def\qname{L}\def\aname{#2}}
}

\begin{document}
\title{Свободные лицензии и российское законодательство: перспективы развития}
\author{Сергей Середа\footnote{Москва, Россия, Движение <<ПОтребитель>>, \url{serge_sereda@hotmail.com}}}
\date{}
\maketitle

\begin{abstract}
There is a number of legal issues with FOSS licensing in Russia which in some circumstances could be very serious. Moreover, prepared bill with changes to Civil Code contains initiatives which could only worsen this situation. FOSS enthusiasts in these circumstances should actively participate in debates on developement of the Russian Civil Code to prevent potential and fix actual compatibility problems of FOSS licences.
\end{abstract}

 До относительно недавнего времени в России <<свободное>> программное обеспечение существовало просто как явление, <<независимо>> от гражданского законодательства. Но по мере его распространения росло количество конечных пользователей, значительное число разработчиков стали участниками или даже основателями проектов разработки свободных программ, на внутреннем рынке появились сначала модемы и маршрутизаторы, работающие на GNU/Linux, затем стали продаваться телевизоры, видеопроигрыватели и приставки с этой же ОС <<на борту>>, приобрели популярность <<умные>> телефоны и планшетные ПК на базе ОС Android, развился бизнес по разработке программно-аппаратных решений на базе FOSS и оказанию услуг по их поддержке. Что же касается собственно операционной системы GNU/Linux, то на уровне Правительства она была названа целевой для сферы образования и госсектора в целом. 

На этом этапе правовая оценка свободного ПО стала необходимостью. Эта работа была проведена независимо разными исследователями, однако её результаты в целом схожи. В результате анализа текстов свободных лицензий и оценки используемого ими механизма лицензирования в контексте отечественного правового поля можно выделить следующие проблемные моменты:
\begin{enumerate}
	\item английский язык, на котором написано большинство свободных лицензий, начиная с GNU GPL;
	\item отсутствие явного указания территории, на которой действует лицензия;
	\item отсутствие явного указания срока действия лицензии;
	\item отсутствие явного указания суммы авторского вознаграждения (или безвозмездности лицензионного договора);
	\item несоблюдение письменной формы сделки при заключении лицензионного договора на свободное ПО;
	\item противоречие <<вирусного механизма>> лицензий на свободное ПО положениям п. 2 ст. 428 ГК РФ, п. 4 ст. 1233 ГК РФ и п. 4 ст. 1260 ГК РФ.
	\item эквивалентность безвозмездной лицензии на свободное ПО дарению, запрещённому для юридических лиц;
	\item отнесение экономии от бесплатного использования свободного ПО к внереализационным доходам, которые должны облагаться налогом, исчисляемым на основании рыночной стоимости использования аналогичного коммерческого ПО.
\end{enumerate}

Мнения специалистов о перечисленных коллизиях с отечественным законодательством разошлись: одна часть исследователей считает, что, несмотря на проблемные моменты, свободное ПО в России вполне законно (к ним, по большей части, относятся энтузиасты FOSS), другая же их часть считает, что противоречия с российским законодательством слишком серьёзны, чтобы считать свободные лицензии соответствующими Российским законам (этой точки зрения придерживаются <<ортодоксальные>> юристы). 

Следует также отметить, что это расхождение мнений касается лишь программ для ЭВМ и баз данных (для которых допустимы <<обёрточные>> лицензии). В отношении же свободных лицензий на другие объекты авторского права, в первую очередь --- литературные и музыкальные произведения, разногласий практически нет --- если подобные лицензионные договоры не заключаются в письменной форме, они (за небольшими исключениями) являются ничтожными на территории РФ.

Теоретически, указанная ситуация может быть разрешена совместными действиями общественных организаций, осуществляющих поддержку экосистемы свободных лицензий, и отечественного законодателя. Что касается общественных организаций, то, например, Creative Commons занимается адаптацией своих лицензий под законодательство РФ, в то время как Free Software Foundation лишь допускает такую возможность. Отечественный же законодатель, как выяснилось, придерживается в этом контексте довольно радикальной точки зрения. Авторы подготовленного пакета поправок в Гражданский Кодекс РФ считают, что несовместимость свободных лицензий и российского законодательств является системной, и проблему можно решить лишь созданием альтернативного механизма, дающего те же результаты, что и свободные лицензии.

Для этого предлагается ввести новый способ распоряжения исключительным правом --- временное предоставление произведения в пользование неограниченному кругу лиц на условиях, определённых правообладателем. Указанный механизм планируется зафиксировать в новой, шестой части статьи 1233 ГК РФ. Однако, чтобы им воспользоваться, правообладателю придётся разместить на официальном сайте федерального органа исполнительной власти по интеллектуальной собственности (\url{www.fips.ru}) заявление <<о предоставлении любым лицам возможности безвозмездно использовать принадлежащий ему результат интеллектуальной деятельности>> с указанием условий, на которых предоставляется эта возможность, а также срока действия заявления. При этом предоставлять произведение для использования публике можно только безвозмездно и только если на его использование не заключено ни одного возмездного лицензионного договора. Отказаться от подобного заявления или изменить зафиксированные в нём условия до истечения указанного в нём срока нельзя. 

Что касается собственно свободных лицензий, то авторы законопроекта предлагают трактовать их не как лицензионные договоры, а как описанные выше заявления. 

Однако, несмотря на озвученную выше позицию авторов поправок, этот же законопроект предполагает и расширение  положений об <<обёрточных>> лицензиях, к которым весьма близки лицензии на свободное ПО. Согласно предлагаемой формулировке новой части 5 статьи 1286 ГК РФ, заменяющей по смыслу часть 3, <<обёрточная>> лицензия \footnote{сам этот жаргонный термин в законе, естественно, не используется} --- это лицензионный договор, заключаемый <<в упрощенном порядке>> (в форме договора присоединения), но являющийся безвозмездным, <<если договором не предусмотрено иное>>. Следует отметить, что, намеренно или нет, но авторы поправок указанной формулировкой устраняют проблему отсутствия явного указания суммы авторского вознаграждения в текстах свободных лицензий. Правда, упрощённый порядок заключения лицензионного договора, к сожалению, так и остался привилегией программ для ЭВМ и баз данных, несмотря на активную торговлю <<электронными>> экземплярами  аудиовизуальных и литературных произведений в российской рознице.

Предлагаемые же поправки в текст статьи 1211 ГК РФ скажутся на свободных лицензиях однозначно негативно. Указанная статья определяет, право какой из сторон внешнеэкономической сделки применяется, если в соответствующем договоре об этом ничего не указано. Заключение лицензионного договора (в т.ч. свободной лицензии) с иностранным физическим или юридическим лицом, соответственно, регулируется указанной статьёй ГК. В соответствии с нынешней формулировкой этой статьи к лицензионным договорам, фиксирующим внешнеэкономическую сделку, по умолчанию применяется право страны лицензиара. В контексте существующих коллизий между свободными лицензиями и российским законодательством такая формулировка позволяет считать большую их часть несущественной в случаях, когда в России используется или дорабатывается свободное ПО, изначально созданное за рубежом.

В предлагаемой же новой редакции статьи 1211 ГК РФ, наоборот, говорится, что к лицензионным договорам должно применяться <<право страны, на территории которой лицензиату разрешается использование результата интеллектуальной деятельности>>. Это положение, правда, содержит оговорку о том, что <<если такое использование разрешается на территории одновременно нескольких стран>> то применяется <<право страны, где находится место жительства или основное место деятельности лицензиара>>, но она, к сожалению, проблемы не решает. Согласно части 3 статьи 1235 ГК РФ, в отсутствие явного указания на территорию, на которой согласно лицензионному договору предоставляется исключительное право, ею считается территория России, а поскольку в свободных лицензиях о территории ничего явно не говорится, то согласно будущей редакции статьи 1211 ГК РФ, к лицензиям на иностранное свободное ПО должно применяться право РФ.

Таким образом, общий эффект от планируемых изменений гражданского законодательства РФ на правовое положение свободных лицензий следует признать отрицательным. В отличие от Европейского союза, не просто признавшего свободные лицензии, но и разработавшего совместимую с ними собственную EUPL, в России лишь создаются дополнительные препятствия для этой <<инициативы снизу>> по саморегулированию рынка программного обеспечения. Такой подход, при его последовательном применении, приведёт к невозможности использования преимуществ свободного лицензирования российскими разработчиками и конечными пользователями, остановит обмен технологиями в области обработки данных и усилит зависимость России от поставщиков коммерческого программного обеспечения. Также, он прямо противоречит существующим и активно спонсируемым инициативам по внедрению свободного ПО в российском госсекторе, и создаёт существенные препятствия для развития российского инновационного бизнеса в сфере информационных технологий.

По мнению автора, свободное сообщество должно уделить особое внимание анализу планируемых изменений в российском гражданском законодательстве и их последствий для развития свободного ПО, а также принять активное участие в публичном обсуждении законопроекта и внесении в него корректив, которые бы позволили сократить число коллизий свободных лицензий с ГК РФ, вместо того, чтобы их усугублять.

\begin{thebibliography}{9}

	\bibitem{s1} Белопольский Э. Язык ничтожной сделки // Бизнес-адвокат, 1997, № 22. – Режим доступа к электрон. дан.: \url{http://www.juristlib.ru/book_305.html}.
	\bibitem{s2} Брауде–Золотарев М., Гребнев Г., Протасов П., Ралько А., Сербина Е. Лицензионные договоры и Российское законодательство. – Режим доступа к электрон. дан.: \url{http://vvv.srcc.msu.su/%7Eserbina/INFO-FOSS.RU/Digest3.pdf}
	\bibitem{s3} Проект ФЗ <<О внесении изменений в части первую, вторую, третью и четвертую Гражданского кодекса Российской Федерации, а также в отдельные законодательные акты Российской Федерации>>. – Режим доступа к электрон. дан.: \url{http://www.economy.gov.ru/minec/activity/sections/corpmanagment/civil_code/full_text_civil_code/140411_gk?presentationtemplate=docHTMLTemplate1&presentationtemplateid=2dd7bc8044687de796f0f7af753c8a7e&WCM_Page.ResetAll=TRUE&CACHE=NONE&CONTENTCACHE=NONE&CONNECTORCACHE=NONE}
	\bibitem{s4} Середа С.А. Открытое программное обеспечение: проблемы лицензирования и доказательства легальности. – Режим доступа к электрон. дан.: \url{http://consumer.nm.ru/osp_law.htm}
	\bibitem{s5} Середа С.А. Свободны ли в России <<свободные лицензии?>> // "Патенты и лицензии". – 2009. N4. – Режим доступа к электрон. дан.: \url{http://consumer.stormway.ru/foss&law.htm}
	\bibitem{s6} Тяпкина Е. Правовой статус GPL в России // <<Компьютерра>>. – 2002. №13 от 9 апреля. – Режим доступа к электрон. дан.: \url{http://offline.computerra.ru/print/offline/2002/438/17257/}.
\end{thebibliography}


\end{document}


