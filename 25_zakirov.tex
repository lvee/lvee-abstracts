\usepackage{pdfpages}
\usepackage{parallel}
\usepackage[T2A]{fontenc}
\usepackage{ucs}
\usepackage[utf8x]{inputenc}
\usepackage[polish,english,russian]{babel}
\usepackage{hyperref}
\usepackage{rotating}
\usepackage[inner=2cm,top=1.8cm,outer=2cm,bottom=2.3cm,nohead]{geometry}
\usepackage{listings}
\usepackage{graphicx}
\usepackage{wrapfig}
\usepackage{longtable}
\usepackage{indentfirst}
\usepackage{array}
\newcolumntype{P}[1]{>{\raggedright\arraybackslash}p{#1}}
\frenchspacing
\usepackage{fixltx2e} %text sub- and superscripts
\usepackage{icomma} % коскі ў матэматычным рэжыме
\PreloadUnicodePage{4}

\newcommand{\longpage}{\enlargethispage{\baselineskip}}
\newcommand{\shortpage}{\enlargethispage{-\baselineskip}}

\def\switchlang#1{\expandafter\csname switchlang#1\endcsname}
\def\switchlangbe{
\let\saverefname=\refname%
\def\refname{Літаратура}%
\def\figurename{Іл.}%
}
\def\switchlangen{
\let\saverefname=\refname%
\def\refname{References}%
\def\figurename{Fig.}%
}
\def\switchlangru{
\let\saverefname=\refname%
\let\savefigurename=\figurename%
\def\refname{Литература}%
\def\figurename{Рис.}%
}

\hyphenation{admi-ni-stra-tive}
\hyphenation{ex-pe-ri-ence}
\hyphenation{fle-xi-bi-li-ty}
\hyphenation{Py-thon}
\hyphenation{ma-the-ma-ti-cal}
\hyphenation{re-ported}
\hyphenation{imp-le-menta-tions}
\hyphenation{pro-vides}
\hyphenation{en-gi-neering}
\hyphenation{com-pa-ti-bi-li-ty}
\hyphenation{im-pos-sible}
\hyphenation{desk-top}
\hyphenation{elec-tro-nic}
\hyphenation{com-pa-ny}
\hyphenation{de-ve-lop-ment}
\hyphenation{de-ve-loping}
\hyphenation{de-ve-lop}
\hyphenation{da-ta-ba-se}
\hyphenation{plat-forms}
\hyphenation{or-ga-ni-za-tion}
\hyphenation{pro-gramming}
\hyphenation{in-stru-ments}
\hyphenation{Li-nux}
\hyphenation{sour-ce}
\hyphenation{en-vi-ron-ment}
\hyphenation{Te-le-pathy}
\hyphenation{Li-nux-ov-ka}
\hyphenation{Open-BSD}
\hyphenation{Free-BSD}
\hyphenation{men-ti-on-ed}
\hyphenation{app-li-ca-tion}

\def\progref!#1!{\texttt{#1}}
\renewcommand{\arraystretch}{2} %Іначай формулы ў матрыцы зліпаюцца з лініямі
\usepackage{array}

\def\interview #1 (#2), #3, #4, #5\par{

\section[#1, #3, #4]{#1 -- #3, #4}
\def\qname{LVEE}
\def\aname{#1}
\def\q ##1\par{{\noindent \bf \qname: ##1 }\par}
\def\a{{\noindent \bf \aname: } \def\qname{L}\def\aname{#2}}
}

\def\interview* #1 (#2), #3, #4, #5\par{

\section*{#1\\{\small\rm #3, #4. #5}}

\def\qname{LVEE}
\def\aname{#1}
\def\q ##1\par{{\noindent \bf \qname: ##1 }\par}
\def\a{{\noindent \bf \aname: } \def\qname{L}\def\aname{#2}}
}

\begin{document}
\title{<<Open Source Бизнес>> или о том как открыть 90\% и остаться со штанами}
\author{Руслан Закиров\footnote{Москва, Россия}}
\date{}
\maketitle
\begin{abstract}
The topic of the report is combining business with open source products. 
The personal experience of working in company pro\-ducing mostly free / open source 
products is represented in se\-ve\-ral field-tested principles.
\end{abstract}

Данный рассказ "--- о том, как можно совместить бизнес с открытыми продуктами. 
В его основе нет никакой книжной теории, а исключительно личный опыт:
более 5 лет автор работает в небольшой бостонской компании, Best Practical Solutions, 
90\% продуктов которой распространяются под открытыми лицензиями.
Ниже приведены основные принципы, примененные и доказавшие на практике, 
что построенный на них бизнес может быть успешным.

\subsection*{Флагманский продукт под открытой лицензией}

Этим продуктом является Request tracker (RT) "--- система обработки заявок пользователей 
(ticket tracking). Продукт приносит основной доход компании. В чем его успех? Мы не пытались
создать что-то, что заработает для всех и сразу. 
Вместо этого мы создаем программу, которую можно настроить,
расширить и интегрировать с другими решениями. Отсутствие границ для
применения увеличивает количество возможных пользователей. Каждый
новый пользователь "--- потенциальный клиент. Продукт ориентирован на
бизнес: так сложилось, что бизнес готов платить за инструменты,
когда они приносят прибыль или снижают расходы.

\subsection*{Плюсы и минусы Open Source}

Для клиентов основной плюс заключается в цене продукта. При этом наш
продукт не является полностью бесплатным. В какой-то момент клиенту не хватит функционала
<<из коробки>>, и тогда ему приходится потратить деньги. Но клиент
сам решает когда, на что и сколько потратить. 
В случае открытых программ выбор гораздо шире. Однако, у бизнеса есть свои опасения на счет
открытых продуктов. Одно из самых главных опасений "--- что завтра, когда понадобиться
помощь, никого не окажется рядом. Поэтому существуют такие компании, как Best Practical
Solutions.

\subsection*{Роль сообщества}

Сообщество пользователей вашего продукта "--- это ваши потенциальные клиенты,
т.~е. самое ценное, что у вас есть. Поэтому
сообщество нужно холить и лелеять, а когда его еще нет, его нужно
взращивать. Хорошее сообщество приносит не только клиентов, но и
маркетологов, которые будут рекламировать продукт, распространителей,
службу бесплатной поддержки, авторов технической документации,
специалистов для консультации, разработчиков.

\subsection*{Опыт велосипедов}

Если у вас классическая закрытая компания, то ничего плохого в этом нет, но
это не значит, что вы не можете заработать дополнительные деньги на открытии
ряда технологий. 
Многие компании разрабатывают вспомогательные инструменты внутри, и часто
это <<велосипеды>>, возникшие, когда доступные решения не заработали, и пришлось написать
свое. Если это заработало для вас, то возможно заработает и для других.
В Best Practical разрабатывали SVK "--- еще одну систему контроля версий,
которая решала ряд внутренних задач. В итоге к разработке присоединилось
28 разработчиков, и пришло несколько клиентов, которых мы и не ожидали.

\subsection*{Заключение}

Определенно, опыт показывает, что компания может существовать, разрабатывая
только программы с открытым исходным кодом. Открытые решения дают ряд
финансовых преимуществ пользователям, что делает их привлекательными для
бизнеса. Сообщество пользователей становится важным фактором финансового
успеха продукта. Регулярные инвестиции в построение сообщества "--- насущная
необходимость. Открытый код может стать дополнительным источником дохода
от вспомогательных технологий, разрабатываемых внутри закрытых компаний.


\end{document}


