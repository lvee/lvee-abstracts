\documentclass[10pt, a5paper]{article}
\usepackage{ucs}
\usepackage[utf8]{inputenc}
\usepackage[T2A]{fontenc}
\usepackage[english, russian]{babel}
\usepackage{hyperref}
\usepackage{geometry}
\frenchspacing
\begin{document}
\title{<<Open Source Бизнес>> или о том как открыть 90\% и остаться со штанами}
\author{Руслан Закиров\footnote{Москва, Россия}}
\date{}
\maketitle
\begin{abstract}
The topic of the report is combining business with open source products. 
The personal experience of working in company producing mostly free / open source 
products is represented in several field-tested principles.
\end{abstract}

Данный рассказ "--- о том, как можно совместить бизнес с открытыми продуктами. 
В его основе нет никакой книжной теории, а исключительно личный опыт:
более 5 лет автор работает в небольшой бостонской компании, Best Practical Solutions, 
90\% продуктов которой распространяются под открытыми лицензиями.
Ниже приведены основные принципы, примененные и доказавшие на практике, 
что построенный на них бизнес может быть успешным.

\section*{Флагманский продукт под открытой лицензией}

Этим продуктом является Request tracker (RT) "--- система обработки заявок пользователей 
(ticket tracking). Продукт приносит основной доход компании. В чем его успех? Мы не пытались
создать что-то, что заработает для всех и сразу. 
Вместо этого мы создаем программу, которую можно настроить,
расширить и интегрировать с другими решениями. Отсутствие границ для
применения увеличивает количество возможных пользователей. Каждый
новый пользователь "--- потенциальный клиент. Продукт ориентирован на
бизнес: так сложилось, что бизнес готов платить за инструменты,
когда они приносят прибыль или снижают расходы.

\section*{Плюсы и минусы Open Source}

Для клиентов основной плюс заключается в цене продукта. При этом наш
продукт не является полностью бесплатным. В какой-то момент клиенту не хватит функционала
<<из коробки>>, и тогда ему приходится потратить деньги. Главное в том, что клиент
сам решает когда, на что и куда их потратить. 

В случае открытых программ выбор гораздо шире. Однако, у бизнеса есть свои опасения на счет
открытых продуктов. Одно из самых главных опасений "--- что завтра, когда понадобиться
помощь, никого не окажется рядом. Поэтому существуют такие компании, как Best Practical
Solutions.

\section*{Роль сообщества}

Сообщество пользователей вашего продукта "--- это ваши потенциальные клиенты,
т.~е. самое ценное, что у вас есть. Поэтому
сообщество нужно холить и лелеять, а когда его еще нет, его нужно
взращивать. Хорошее сообщество приносит не только клиентов, но и
маркетологов, которые будут рекламировать продукт, распространителей,
службу бесплатной поддержки, авторов технической документации,
специалистов для консультации, а также разработчиков.

\section*{Опыт велосипедов}

Если у вас классическая закрытая компания, то ничего плохого в этом нет, но
это не значит, что вы не можете заработать дополнительные деньги на открытии
ряда технологий. 
Многие компании разрабатывают вспомогательные инструменты внутри, и часто
это <<велосипеды>>, возникшие, когда доступные решения не заработали, и пришлось написать
свое. Если это заработало для вас, то возможно заработает и для других.

В Best Practical разрабатывали SVK "--- еще одну систему контроля версий,
которая решала ряд внутренних задач. В итоге к разработке присоединилось
28 разработчиков, и пришло несколько клиентов, которых мы и не ожидали.

\section*{Заключение}

Определенно, опыт показывает, что компания может существовать, разрабатывая
только программы с открытым исходным кодом. Открытые решения дают ряд
финансовых преимуществ пользователям, что делает их привлекательными для
бизнеса. Сообщество пользователей становится важным фактором финансового
успеха продукта. Регулярные инвестиции в построение сообщества "--- насущная
необходимость. Открытый код может стать дополнительным источником дохода
от вспомогательных технологий, разрабатываемых внутри закрытых компаний.


\end{document}


