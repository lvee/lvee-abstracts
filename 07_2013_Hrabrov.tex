\documentclass[10pt, a5paper]{article}
\usepackage{pdfpages}
\usepackage{parallel}
\usepackage[T2A]{fontenc}
\usepackage{ucs}
\usepackage[utf8x]{inputenc}
\usepackage[polish,english,russian]{babel}
\usepackage{hyperref}
\usepackage{rotating}
\usepackage[inner=2cm,top=1.8cm,outer=2cm,bottom=2.3cm,nohead]{geometry}
\usepackage{listings}
\usepackage{graphicx}
\usepackage{wrapfig}
\usepackage{longtable}
\usepackage{indentfirst}
\usepackage{array}
\newcolumntype{P}[1]{>{\raggedright\arraybackslash}p{#1}}
\frenchspacing
\usepackage{fixltx2e} %text sub- and superscripts
\usepackage{icomma} % коскі ў матэматычным рэжыме
\PreloadUnicodePage{4}

\newcommand{\longpage}{\enlargethispage{\baselineskip}}
\newcommand{\shortpage}{\enlargethispage{-\baselineskip}}

\def\switchlang#1{\expandafter\csname switchlang#1\endcsname}
\def\switchlangbe{
\let\saverefname=\refname%
\def\refname{Літаратура}%
\def\figurename{Іл.}%
}
\def\switchlangen{
\let\saverefname=\refname%
\def\refname{References}%
\def\figurename{Fig.}%
}
\def\switchlangru{
\let\saverefname=\refname%
\let\savefigurename=\figurename%
\def\refname{Литература}%
\def\figurename{Рис.}%
}

\hyphenation{admi-ni-stra-tive}
\hyphenation{ex-pe-ri-ence}
\hyphenation{fle-xi-bi-li-ty}
\hyphenation{Py-thon}
\hyphenation{ma-the-ma-ti-cal}
\hyphenation{re-ported}
\hyphenation{imp-le-menta-tions}
\hyphenation{pro-vides}
\hyphenation{en-gi-neering}
\hyphenation{com-pa-ti-bi-li-ty}
\hyphenation{im-pos-sible}
\hyphenation{desk-top}
\hyphenation{elec-tro-nic}
\hyphenation{com-pa-ny}
\hyphenation{de-ve-lop-ment}
\hyphenation{de-ve-loping}
\hyphenation{de-ve-lop}
\hyphenation{da-ta-ba-se}
\hyphenation{plat-forms}
\hyphenation{or-ga-ni-za-tion}
\hyphenation{pro-gramming}
\hyphenation{in-stru-ments}
\hyphenation{Li-nux}
\hyphenation{sour-ce}
\hyphenation{en-vi-ron-ment}
\hyphenation{Te-le-pathy}
\hyphenation{Li-nux-ov-ka}
\hyphenation{Open-BSD}
\hyphenation{Free-BSD}
\hyphenation{men-ti-on-ed}
\hyphenation{app-li-ca-tion}

\def\progref!#1!{\texttt{#1}}
\renewcommand{\arraystretch}{2} %Іначай формулы ў матрыцы зліпаюцца з лініямі
\usepackage{array}

\def\interview #1 (#2), #3, #4, #5\par{

\section[#1, #3, #4]{#1 -- #3, #4}
\def\qname{LVEE}
\def\aname{#1}
\def\q ##1\par{{\noindent \bf \qname: ##1 }\par}
\def\a{{\noindent \bf \aname: } \def\qname{L}\def\aname{#2}}
}

\def\interview* #1 (#2), #3, #4, #5\par{

\section*{#1\\{\small\rm #3, #4. #5}}

\def\qname{LVEE}
\def\aname{#1}
\def\q ##1\par{{\noindent \bf \qname: ##1 }\par}
\def\a{{\noindent \bf \aname: } \def\qname{L}\def\aname{#2}}
}


\begin{document}

\title{Использование Ejudge для проведения олимпиад по программированию}%\footnote{Текст данных и последующих тезисов, кроме специально оговоренных случаев, доступен под лицензией Creative Commons Attribution-ShareAlike 3.0}

\author{Дмитрий Храбров\footnote{Гомель, Беларусь; ГГТУ им. П.О. Сухого; \url{root@dexp.in}}}
\maketitle

\begin{abstract}
The paper considers usage of open source online programming competitions server ejudge. In addition to technical aspects \linebreak author's personal experience is described, concerning both \linebreak technical and social issues.
\end{abstract}

 Ejudge "--- это система для проведения различных мероприятий, в которых необходима автоматическая проверка программ. Система может применяться для проведения олимпиад и поддержки учебных курсов. Ejudge распространяется под лицензией GPL, имеет многоязычный веб"=интерфейс и поддерживает защищённое исполнение программ (если установлен патч к ядру Linux).

Для проведения олимпиады необходимо зарегистрировать участников: лично или командно. Далее нужно дать участникам возможность читать условия задач и отправлять решения на тестирование. Перед отсылкой на тестирование участник выбирает компилятор и файл с исходным кодом решения. Далее система на сервере пытается скомпилировать решение с помощью выбранного компилятора. Если произошли ошибки, то участнику выдаётся сообщение. Если компиляция прошла успешно, то происходит непосредственно тестирование. Исполняемому файлу на вход (STDIN или файл) подаются входные данные, заранее сформированные автором задачи. На выполнение обычно ставятся ограничения по времени и по памяти. Если решение участника уложилось в лимиты и выдало ответ, система сверяет этот ответ с авторским. Кроме того система должна вести статистику, показывать положение участников. 

Установка и настройка Ejudge подробно описана на сайте проекта. На случай, если нет желания или возможности выполнять установку и настройку, на официальном сайте лежит готовый и настроенный VirtualBox"=образ. На официальном сайте Ejudge написано, что она имеет настраиваемый внешний вид. При беглом просмотре такая возможность без перекомпиляции найдена не была. Все страницы свёрстаны абсолютным позиционированием элементов, через CSS трудно поддаются изменениям.

На текущий момент нами успешно проведена на Ejudge внутривузовская олимпиада по программированию. Наиболее востребованы языки C\#, Java, C, Pascal. Были опасения, что реализация  C\# будет отличаться, и студенты будут жаловаться, что C\# не работает. Однако Mono отработало без нареканий. Java поддерживается нативно, однако требует минимум 512 мегабайт памяти. Компиляторов С/С++ нами предоставлялось студентам два: gcc и clang. Студенты периодически путали их и отправляли Си"=программу на тестирование компилятором g++ (для языка С++), однако негативных последствий это не имело. Наибольшее недопонимание вызывало отсутствие заголовочных файлов windows.h и conio.h, которые студенты не задумываясь вставляли в свои программы. Приходилось ходить по аудиториям и повторять, что эти файлы подключать нельзя. 

Двое человек спросили о поддержке PHP и Brainfuck. Последний системой Ejudge по умолчанию не поддерживается; студенту пообещали, что язык будет добавлен, если студент гарантирует, что будет на нём писать олимпиаду, после чего вопрос иссяк. Интерпретатор РНР на сервере установлен и доступен для тестирования приложений. Студенту был предоставлен пример программы, но писать на РНР он не рискнул, так как в аудиториях РНР не установлен. 

Из неожиданных моментов следует отметить то, что Ejudge по умолчанию считает C\# <<небезопасным>> языком, и он отключен для использования в турнирах. Ещё один тонкий момент был обнаружен после проведения олимпиады "--- система отказалась показать таблицу результатов студентам (пришлось воспользоваться для показа администраторским интерфейсом). Как выяснилось, Ejudge по умолчанию показывает таблицу через 2 часа после завершения олимпиады, однако это настраивается в веб"=интерфейсе.

В целом система тестирования Ejudge показала себя очень неплохо: свою задачу выполняет, поддерживает все современные языки программирования, имеет большое количество настраиваемых возможностей, почти всё доступно через веб"=интерфейс. Из отрицательных черт можно отметить не совсем логичный и удобный интерфейс, причём как турнирный, так и администраторский.


\end{document}




