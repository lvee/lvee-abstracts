\documentclass[10pt, a5paper]{article}
\usepackage[T2A]{fontenc}
\usepackage{ucs}
\usepackage[utf8x]{inputenc}
\usepackage[polish,english,russian]{babel}
\usepackage{hyperref}
\usepackage[inner=2cm,top=1.8cm,outer=2cm,bottom=2.3cm,nohead]{geometry}
\usepackage{listings}
\usepackage{graphicx}
\usepackage{wrapfig}
\usepackage{longtable}
\usepackage{indentfirst}
\frenchspacing
\usepackage{fixltx2e} %text sub- and superscripts
\usepackage{icomma} % коскі ў матэматычным рэжыме
\PreloadUnicodePage{4}

\newcommand{\longpage}{\enlargethispage{\baselineskip}}
\newcommand{\shortpage}{\enlargethispage{-\baselineskip}}

\def\switchlang#1{\expandafter\csname switchlang#1\endcsname}
\def\switchlangbe{
\let\saverefname=\refname%
\def\refname{Літаратура}%
\def\figurename{Іл.}%
}
\def\switchlangen{
\let\saverefname=\refname%
\def\refname{References}%
\def\figurename{Fig.}%
}
\def\switchlangru{
\let\saverefname=\refname%
\let\savefigurename=\figurename%
\def\refname{Литература}%
\def\figurename{Рис.}%
}

\hyphenation{admi-ni-stra-tive}
\hyphenation{ex-pe-ri-ence}
\hyphenation{fle-xi-bi-li-ty}
\hyphenation{Py-thon}
\hyphenation{ma-the-ma-ti-cal}
\hyphenation{re-ported}
\hyphenation{imp-le-menta-tions}
\hyphenation{pro-vides}
\hyphenation{en-gi-neering}
\hyphenation{com-pa-ti-bi-li-ty}
\hyphenation{im-pos-sible}
\hyphenation{desk-top}
\hyphenation{elec-tro-nic}
\hyphenation{com-pa-ny}
\hyphenation{de-ve-lop-ment}
\hyphenation{de-ve-loping}
\hyphenation{de-ve-lop}
\hyphenation{da-ta-ba-se}
\hyphenation{plat-forms}
\hyphenation{or-ga-ni-za-tion}
\hyphenation{pro-gramming}
\hyphenation{in-stru-ments}
\hyphenation{Li-nux}
\hyphenation{en-vi-ron-ment}
\hyphenation{Te-le-pathy}
\hyphenation{Li-nux-ov-ka}

\def\progref!#1!{\texttt{#1}}
\renewcommand{\arraystretch}{2} %Іначай формулы ў матрыцы зліпаюцца з лініямі
\usepackage{array}

\def\interview #1 (#2), #3, #4, #5\par{

\section[#1, #3, #4]{#1, #5}
\def\qname{LVEE}
\def\aname{#1}
\def\q ##1\par{{\noindent \bf \qname: ##1 }\par}
\def\a{{\noindent \bf \aname: } \def\qname{L}\def\aname{#2}}
}

\begin{document}
\title{Создание 3D"=мультфильма средствами СПО}
\author{Виктория Бабахина "--- Рязань, РФ\footnote{\url{vitorry@gmail.com}, \url{http://lvee.org/en/abstracts/134}}}
\maketitle
\begin{abstract}
The article gives an overview about using free software in film making industry. It covers all the basic steps of creating an animated film with sample use of Blender, Krita, Gimp, Mypaint, Alchemy, etс.
\end{abstract}
В последнее время СПО все активнее применяется в киноиндустрии. Не последнюю роль в этом играет студия Blender Foundation, некоммерческая организация, занимающаяся разработкой програм\-много пакета трёхмерного моделирования с открытым исходным кодом под названием Blender. Основателем фонда и главным разработчиком является Тон Розендаль.

Blender Foundation выпустили уже не один мультфильм, среди которых «Big Buck Bunny» и «Sintel». Мультфильмы эти были полностью, от концепта до готового продукта, созданы при помощи СПО.

Кроме  Blender Foundation в последнее время все больше и больше мультипликационных студий обращаются к СПО во всех странах мира. В большинстве случаев это небольшие студии, однако есть примеры использования, к примеру Blender, такими гигантами, как Columbia Pictures при съемках фильма «Человек"=Паук 2».

\subsection*{Разработка концепции}

На этапе разработки концептов неоценимо важна возможность создать много быстрых и выразительных эскизов"=идей, некоторые из которых в дальнейшем лягут в основу готового продукта. В этом отношении  программа  Alchemy может стать хорошим подспорьем. Еще одним полезным графическим редактором является Krita "--- созданная непосредственно для рисования, для этого в ней есть все необходимое и даже более того: обширный набор кистей, удобные инструменты для построения перспективы и прочие необходимые вещи. Однако, она практически не пригодна для обработки изображения. Для дальнейшей обработки можно воспользоваться GIMP.

\subsection*{Раскадровка}

После появления первых эскизов и написания сценария, необходима предварительная визуализация будущего мультфильма. Для этого создается раскадровка. Как правило, раскадровка выглядит как небольшие пронумерованные картинки, зачастую довольно схематичные. Однако, благодаря этим картинкам становится понятно чередование планов, тоновой разбор, монтажные стыки кадров, да и в принципе визуальный ряд сюжета. Целесообразно использование Krita или MyPaint. Помимо использования обычных контурных кистей возможно применение широкого спектра кистей, имитирующих натуральные материалы, что позволяет сделать раскадровку более «живой» и подходящей настрою мультфильма.

\subsection*{Аниматик}

Аналогичную роль выполняет аниматик, но в отличии от ракадровки, он дает уже более конкретное представление о том, что будет происходить в мультфильме, не просто последовательность действий, но и более подробные движения персонажей. Фактически аниматик "--- это черновой мультфильм. Как правило, аниматиков рисуется довольно много. В случае трехмерного мультфильма первый аниматик все равно создается рисованный, как черновая последовательность картинок. Для чернового аниматика вполне возможно использовать программу для 2D"=анимации sunfig. Или просто покадрово нарисовать основные фазы в Krita и собрать их в том же Blender или в любом другом видео"=редакоре. Большим плюсом специальных программ для 2D"=анимации является то, что есть возможность прорисовать некоторые фазы более подробно, и просто в  одном редакторе, и сразу там же их и собрать.

Следующие варианты аниматика создаются уже в 3D. Выглядят они поначалу достаточно неказисто "--- для упрощения процесса рендеринга в них не используются текстуры, убрана большая часть мелочей. В первых вариантах аниматика отсутствует модель как таковая, вместо нее упрощенная шарнирная болванка. Анимированных движений еще тоже нет. 
Затем аниматик начинает прорабатываться подробнее, добавляются персонажи, текстуры, мебель. И так до финального композитинга.

Эта работа проходит уже целиком и полностью в Blender. Для оптимизации процесса все предметы и персонажи, находящиеся в сцене, расположены в отдельных файлах. В главном документе на них присутствуют только ссылки, что значительно облегчает работу и ускоряет рендер.

\subsection*{Моделирование персонажей}

Когда определены первые эскизы персонажей, начинается подготовка референсов для моделирования. На референсах представлены несколько видов будущей модели, а так же, при необходимости, детали костюма. Затем начинается процесс создания модели в Blender. Поскольку процесс создания высокополигональной модели человека очень труден и долог, а время часто поджимает, то имеет смыл воспользоваться программой  MakeHuman. Это программа для создания модели человека методом задания параметров, таких как рост, пол, раса, форма носа и так далее. Эта программа удобна для создания простой «Болванки» будущего персонажа, на основе которой уже можно будет создать более сложную модель.

\subsection*{Анимация}

Для анимации модели ее необходимо оснастить скелетом, так называемым «ригом». Как правило, процесс заключается в следующем. В модели в нужных местах располагаются «кости», затем они привязываются к мэшу. Для удобства, кости позже заменяют специальными рычажками "--- контроллерами. Получается этакая марионетка, которой очень просто и удобно управлять.

В Blender есть встроенный аддон для автоматического создания рига "--- Rigify. Это довольно"=таки удобный аддон, которым пользуется сейчас большинство аниматоров. Сначала он автоматически задает арматуру. Ее можно отредактировать в зависимости от вида модели: увеличить, уменьшить, добавить запасную пару рук и так далее. Затем, на основе этой арматуры, автоматически генерируются контроллеры. Опять же, все можно дополнять при необходимости.

\subsection*{Освещение и постобработка}

Одним из заключительных этапов работы над мультфильмом является чистовая анимация движений персонажей. На этом же этапе производятся настройки освещения (лайтинг). Лайтинг "--- это большой отдельный этап в создании мультфильма, так как освещение "--- один из основных способов создания атмосферы в фильме.
 
На этом же этапе происходит запекание симуляций, если таковые имеются. Blender располагает внушительным набором симуляций: одежды, дыма, огня, ветра и т.д.

Мультфильм рендерится, как правило, небольшими фрагментами, которые в последствии монтируются друг с другом согласно аниматику. В финале, после рендеринга, каждая сцена проходит этап композитинга "--- постобработки. К готовой картинке добавляются необходимые эффекты: виньетирование, размытие, цветокоррекция. В Blender этот процесс производится при помощи нодов "--- так называемых узлов. Каждый узел в этой системе "--- какое"=то действие, они последовательно соединяются друг с другом, в начале и в конце входные и выходные узлы.


\begin{thebibliography}{9}
\bibitem{b1} Хитрук Ф.С. Проффесия "--- аниматор М.: Гаятри, 2008.
\bibitem{b2} Burne Hogarth  Dynamic Wrinkles And Drapery New York: Watson"=Guptill Publications, 1995.
\bibitem{b3} Ken A. Priebe The Advanced Art of Stop"=Motion Animation New York: Course Technology, a part of Cengage Learning, 2011.
\bibitem{b4} http://disneyfrozen.tumblr.com/tagged/concept-art  Дата просмотра: 20.04.2014.
\bibitem{b5} http://www.blendernation.com/ Дата просмотра: 14.04.2014.
\bibitem{b6} http://www.disneyanimation.com/ Дата просмотра: 20.04.2014.
\end{thebibliography}

\end{document}
