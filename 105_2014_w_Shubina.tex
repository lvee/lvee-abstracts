\documentclass[10pt, a5paper]{article}
\usepackage{pdfpages}
\usepackage{parallel}
\usepackage[T2A]{fontenc}
\usepackage{ucs}
\usepackage[utf8x]{inputenc}
\usepackage[polish,english,russian]{babel}
\usepackage{hyperref}
\usepackage{rotating}
\usepackage[inner=2cm,top=1.8cm,outer=2cm,bottom=2.3cm,nohead]{geometry}
\usepackage{listings}
\usepackage{graphicx}
\usepackage{wrapfig}
\usepackage{longtable}
\usepackage{indentfirst}
\usepackage{array}
\newcolumntype{P}[1]{>{\raggedright\arraybackslash}p{#1}}
\frenchspacing
\usepackage{fixltx2e} %text sub- and superscripts
\usepackage{icomma} % коскі ў матэматычным рэжыме
\PreloadUnicodePage{4}

\newcommand{\longpage}{\enlargethispage{\baselineskip}}
\newcommand{\shortpage}{\enlargethispage{-\baselineskip}}

\def\switchlang#1{\expandafter\csname switchlang#1\endcsname}
\def\switchlangbe{
\let\saverefname=\refname%
\def\refname{Літаратура}%
\def\figurename{Іл.}%
}
\def\switchlangen{
\let\saverefname=\refname%
\def\refname{References}%
\def\figurename{Fig.}%
}
\def\switchlangru{
\let\saverefname=\refname%
\let\savefigurename=\figurename%
\def\refname{Литература}%
\def\figurename{Рис.}%
}

\hyphenation{admi-ni-stra-tive}
\hyphenation{ex-pe-ri-ence}
\hyphenation{fle-xi-bi-li-ty}
\hyphenation{Py-thon}
\hyphenation{ma-the-ma-ti-cal}
\hyphenation{re-ported}
\hyphenation{imp-le-menta-tions}
\hyphenation{pro-vides}
\hyphenation{en-gi-neering}
\hyphenation{com-pa-ti-bi-li-ty}
\hyphenation{im-pos-sible}
\hyphenation{desk-top}
\hyphenation{elec-tro-nic}
\hyphenation{com-pa-ny}
\hyphenation{de-ve-lop-ment}
\hyphenation{de-ve-loping}
\hyphenation{de-ve-lop}
\hyphenation{da-ta-ba-se}
\hyphenation{plat-forms}
\hyphenation{or-ga-ni-za-tion}
\hyphenation{pro-gramming}
\hyphenation{in-stru-ments}
\hyphenation{Li-nux}
\hyphenation{sour-ce}
\hyphenation{en-vi-ron-ment}
\hyphenation{Te-le-pathy}
\hyphenation{Li-nux-ov-ka}
\hyphenation{Open-BSD}
\hyphenation{Free-BSD}
\hyphenation{men-ti-on-ed}
\hyphenation{app-li-ca-tion}

\def\progref!#1!{\texttt{#1}}
\renewcommand{\arraystretch}{2} %Іначай формулы ў матрыцы зліпаюцца з лініямі
\usepackage{array}

\def\interview #1 (#2), #3, #4, #5\par{

\section[#1, #3, #4]{#1 -- #3, #4}
\def\qname{LVEE}
\def\aname{#1}
\def\q ##1\par{{\noindent \bf \qname: ##1 }\par}
\def\a{{\noindent \bf \aname: } \def\qname{L}\def\aname{#2}}
}

\def\interview* #1 (#2), #3, #4, #5\par{

\section*{#1\\{\small\rm #3, #4. #5}}

\def\qname{LVEE}
\def\aname{#1}
\def\q ##1\par{{\noindent \bf \qname: ##1 }\par}
\def\a{{\noindent \bf \aname: } \def\qname{L}\def\aname{#2}}
}

\switchlang{ru}
\begin{document}
\title{Краткий обзор базовых лицензий СПО}
\author{Ирина Шубина, Минск, Беларусь\footnote{\url{klasy@tut.by}, \url{http://lvee.org/en/abstracts/104}}}
\maketitle
\begin{abstract}
With this report we are going to make a quick and brief overview of the open-source licenses. Starting with the base ideas of the licensing we'll come to the most wirespread and most popular licenses used in the open-source software. Also one of the most important points of the report I would consider the differentiation of the permissive and copyleft licenses.
\end{abstract}
\section*{Введение}

\subsection*{Понятие лицензирования}

Лицензирование "--- соглашение сторон, по которому одна сторона предоставляет какие-либо права другой стороне. Лицензирование используется для защиты авторских прав.

Именно правила лицензирования диктуют различные права доступа и использования исходного кода в приложении к программному обеспечению.

\subsection*{Типы лицензирования}

Различают пермиссивные и копилефт лицензии.

\textbf{Пермиссивные лицензии} "--- это лицензии на программное \linebreak обеспечение, которые практически не ограничивают свободу действий пользователей ПО и разработчиков, работающих с исходным кодом. По своему духу, распространение работы под пермиссивной лицензией схоже с помещением работы в общественное достояние, не требующее отказа от авторского права.

Идея \textbf{копилефт} состоит в том, что каждый, кто распространяет программу как с изменениями, так и без них, не вправе ограничивать свободу её дальнейшего распространения либо модификации.

\begin{itemize}
  \item \textbf{«Сильная»} copyleft лицензия  разрешает использовать код только программам, созданным под такой же лицензией.
  \item \textbf{«Слабая»} copyleft лицензия разрешает вносить любые изменения в код данной программы. Но она ставит условие, что другая программа, использующая данный код, будет строиться с указанием изначальной в качестве библиотеки. Тогда новая программа может выходить под любой другой лицензией.
\end{itemize}

В рамках данного типа лицензирования выделяют также полный и частичный копилефт.

\textbf{Полный копилефт} "--- все части программы (за исключением самой лицензии) могут модифицироваться и распространяться только под лицензией копилефта.

\textbf{Частичный копилефт} "--- программа может исключать нес- \linebreak колько условий копилефт лицензии и при этом включать модификации в рамках какой-то не-копилефт лицензии. Или в некоторых случаях программа, распространяемая под такой лицензией, может следовать не всем принципам копилефта. Например, исключение, сделанное для некоторых программ для GPL связывания.

\section*{Краткий обзор лицензий}

\subsection*{GPL (General Public License)}

Стандартная Общественная Лицензия GNU (GNU General Public License, GNU GPL) "--- это свободная copyleft лицензия для программного обеспечения (ПО) и других видов произведений\footnote{Взято из Неофициального Перевода GNU GPLv3 (\url{http://code.google.com/p/gpl3rus/wiki/LatestRelease})}.

GNU GPL требует распространения с двоичными файлами (в том числе неизменными) исходного кода или письменного обязательства его предоставить (своего или чужого; способы зависят от версии лицензии).

Лицензии, созданные на базе GPL:
\begin{itemize}
  \item AGPL (Affero General Public License)
  \item LGPL (Lesser General Public License)
\end{itemize}

\subsection*{BSD (Berkley Software Distribution)}

Существуют две основные версии лицензии BSD, которые необходимо различать: «оригинальная» и так называемая «модифицированная» (вторую в англоязычной литературе часто называют New BSD License). Данная лицензия является пермиссивной.

Лицензия BSD допускает проприетарное коммерческое использование ПО. Для ПО, выпущенного под этой лицензией, допускается встраивание в проприетарные коммерческие продукты. Работы, основанные на таком ПО, даже могут распространяться под проприетарными лицензиями (но всё же обязаны соответствовать требованиям лицензии). Наиболее заметные примеры таких программ "--- использование сетевого кода BSD в продуктах корпорации Microsoft, а также использование многих компонентов FreeBSD в операционной системе Mac OS X.

\subsection*{Apache Software License}

Лицензия Apache даёт пользователю право использовать программное обеспечение для любых целей, свободно распространять, изменять, и распространять изменённые копии, за исключением названия. 
Данная лицензия не ставит условием неизменность лицензии распространения программного обеспечения, и не настаивает даже на сохранении его открытого статуса. Единственным условием, накладываемым лицензией Apache, является информирование получателя о факте использования исходного кода. В противоположность copyleft-лицензиям, получатель модифицированной версии не обязательно получает все права, изначально предоставляемые лицензией Apache.

В каждом лицензируемом файле должна быть сохранена вся исходная информация о копирайтах или патентах, в каждый изменённый файл должна добавляться информация о проведённых изменениях.

\section*{Совместимость лицензий}

\textbf{Совместимость}, понятие, возникающее при попытке комбинирования двух и более лицензий, определяет непосредственно возможность комбинации одной лицензии с другими. Совместимость может варьироваться в зависимости от типа и версии лицензии. Различают GPL-совсемстимые и GPL-несовместимые лицензии. \linebreak Также нужно отметить возможность сочетать закрытые (или проприетарные) лицензии с открытыми. Ссылка на ресурс, где можно посмотреть совместимость лицензий: \url{http://www.tldrlegal.com/compare}.


\end{document}
