\documentclass[10pt, a5paper]{article}
\usepackage{pdfpages}
\usepackage{parallel}
\usepackage[T2A]{fontenc}
\usepackage{ucs}
\usepackage[utf8x]{inputenc}
\usepackage[polish,english,russian]{babel}
\usepackage{hyperref}
\usepackage{rotating}
\usepackage[inner=2cm,top=1.8cm,outer=2cm,bottom=2.3cm,nohead]{geometry}
\usepackage{listings}
\usepackage{graphicx}
\usepackage{wrapfig}
\usepackage{longtable}
\usepackage{indentfirst}
\usepackage{array}
\newcolumntype{P}[1]{>{\raggedright\arraybackslash}p{#1}}
\frenchspacing
\usepackage{fixltx2e} %text sub- and superscripts
\usepackage{icomma} % коскі ў матэматычным рэжыме
\PreloadUnicodePage{4}

\newcommand{\longpage}{\enlargethispage{\baselineskip}}
\newcommand{\shortpage}{\enlargethispage{-\baselineskip}}

\def\switchlang#1{\expandafter\csname switchlang#1\endcsname}
\def\switchlangbe{
\let\saverefname=\refname%
\def\refname{Літаратура}%
\def\figurename{Іл.}%
}
\def\switchlangen{
\let\saverefname=\refname%
\def\refname{References}%
\def\figurename{Fig.}%
}
\def\switchlangru{
\let\saverefname=\refname%
\let\savefigurename=\figurename%
\def\refname{Литература}%
\def\figurename{Рис.}%
}

\hyphenation{admi-ni-stra-tive}
\hyphenation{ex-pe-ri-ence}
\hyphenation{fle-xi-bi-li-ty}
\hyphenation{Py-thon}
\hyphenation{ma-the-ma-ti-cal}
\hyphenation{re-ported}
\hyphenation{imp-le-menta-tions}
\hyphenation{pro-vides}
\hyphenation{en-gi-neering}
\hyphenation{com-pa-ti-bi-li-ty}
\hyphenation{im-pos-sible}
\hyphenation{desk-top}
\hyphenation{elec-tro-nic}
\hyphenation{com-pa-ny}
\hyphenation{de-ve-lop-ment}
\hyphenation{de-ve-loping}
\hyphenation{de-ve-lop}
\hyphenation{da-ta-ba-se}
\hyphenation{plat-forms}
\hyphenation{or-ga-ni-za-tion}
\hyphenation{pro-gramming}
\hyphenation{in-stru-ments}
\hyphenation{Li-nux}
\hyphenation{sour-ce}
\hyphenation{en-vi-ron-ment}
\hyphenation{Te-le-pathy}
\hyphenation{Li-nux-ov-ka}
\hyphenation{Open-BSD}
\hyphenation{Free-BSD}
\hyphenation{men-ti-on-ed}
\hyphenation{app-li-ca-tion}

\def\progref!#1!{\texttt{#1}}
\renewcommand{\arraystretch}{2} %Іначай формулы ў матрыцы зліпаюцца з лініямі
\usepackage{array}

\def\interview #1 (#2), #3, #4, #5\par{

\section[#1, #3, #4]{#1 -- #3, #4}
\def\qname{LVEE}
\def\aname{#1}
\def\q ##1\par{{\noindent \bf \qname: ##1 }\par}
\def\a{{\noindent \bf \aname: } \def\qname{L}\def\aname{#2}}
}

\def\interview* #1 (#2), #3, #4, #5\par{

\section*{#1\\{\small\rm #3, #4. #5}}

\def\qname{LVEE}
\def\aname{#1}
\def\q ##1\par{{\noindent \bf \qname: ##1 }\par}
\def\a{{\noindent \bf \aname: } \def\qname{L}\def\aname{#2}}
}

\switchlang{ru}
\begin{document}
\title{Taurus: тесты на любителя}
\author{Taras Svirinovski, Minsk, Belarus \footnote{\url{grey.fenrir@gmail.com} \url {https://lvee.org/ru/abstracts/246}}}
\maketitle
\begin{abstract}
It makes sense to automate anything you repeat 10+ times.
Taurus improves experience of JMeter, Selenium and others.
\end{abstract}
Статья ставит целью продемонстрировать некоторые примеры \linebreak практического применения Taurus tool. Материал предназначен в первую очередь разработчикам и другим техническим специалистам, непосредственно не занимающимся тестированием.

\subsection*{Введение}

Taurus ~--- это кроссплатформенный проект с открытым исходным кодом, предназначенный, в первую очередь, для упрощения автоматизации нагрузочных и функциональных тестов.
Установка весьма проста ~--- наш инструмент, в частности, распространяется через pip, поэтому для установки вам хватит Python'a (установщику под win не нужно и этого). Есть deb- и rpm-пакеты (что не особо актуально), есть образ для докера. 
Много дополнительного софта мы можем поставить сами, но часть ~--- нет (например, JVM/JDK, TestNG, \ldots{}), но мы можем проверять на наличие в системе и в случае остутствия понятно ругаться.

\subsection*{Минимальный скрипт}

Мало кто начинает работать с новым инструментом с чистого листа, совершенно не имея каких-либо наработок. Пусть у нас имеется каталог со скриптами для selenium, написанными на Python (а ещё Java, JavaScript,\ldots{}). Самым популярным способом озадачить Tarus является передача ему скрипта на языке YAML. Рассмотрим как выглядит выполнение готовых скриптов:

\begin{verbatim}
execution:
- executor: selenium
  scenario:
    script: python_tests/\end{verbatim}
Командная строка:

\begin{verbatim}
\$ bzt j1.yml \end{verbatim}
После запуска вы увидите псевдографическую консоль. Во многих случая она не обязательна, но гики любят, поэтому мы решили её добавить. Разумеется, возможности визуализации в текстовом режиме ограничены, но мы можем показать немало ~--- текущую нагрузку в хитах/сек, изменение времени отклика, возникающие ошибки и предупреждения, затраченное/оставшееся время теста и многое другое. После окончания работы Taurus'a можно видеть краткий отчет, кроме этого для каждого запуска создается каталог с файлами логов и отчетов (artifacts dir).

\subsection*{Тест с параметрами}

Усложним скрипт и разберем некоторые базовые опции и возможности.

\begin{verbatim}
execution:
- executor: jmeter
  concurrency: 10
  ramp-up: 30s
  hold-for: 1m
  scenario:
    requests:
    - http://blazedemo.com
modules:
  jmeter:
    version: 3.1\end{verbatim}
Здесь мы указали параметры теста (инструмент ~--- jmteter, десять виртуальных пользователей, полминуты линейного роста нагрузки, минута её поддержания), сценарий теста (получить страницу по URL'у) и параметр для инструмента (запрос конкретной версии).

\subsection*{Multi execution}

Рассмотрим несколько более развитый сценарий:

\begin{verbatim}
execution:
- executor: jmeter
  concurrency: 10
  hold-for: 1m
  ramp-up: 30s
  scenario: shared_scenario
- executor: pbench
  concurrency: 3
  hold-for: 20s
  delay: 10s
  scenario: shared_scenario
scenarios:
  shared_scenario:
    requests:
    - http://blazedemo.com\end{verbatim}
Обратите внимание на описание структуры в языке YAML ~--- иерархия определяется отступами, элементы списка начинаются с дефиса, все остальные элементы, имеющие подчиненную структуру, считают ключами словаря. 
Здесь мы видим запрос выполнить два теста с разными инструментами, временными и количественными показателями, но c одним выделенным сценарием. Для разнообразия запустим тест с ключом -report и полюбуемся на результаты.
Особой пользы пока нету, но что-то явно работает.

\subsection*{Дружелюбный selenium}

Следующий тест позволяет зайти и выйти из веб-почты. Оцените краткость, читабельность, легкость составления и визуальный контроль работы.

\begin{verbatim}
execution:
- executor: selenium
  scenario: 
    browser: Chrome
    requests:
    - url: \url{https://mail.ru
      actions:
      - keysByName(Login): strange_user
      - keysByName(Password): secret_password
      - clickByID(mailbox__auth__button)
      - waitByLinkText(выход)
      - clickByLinkText(выход)\end{verbatim}
\subsection*{Автоматизация}

Рассмотрим типичную автоматизацию посредством csv-файлов. Итак, пускай имеются следующие данные:

\begin{verbatim}
\$ cat urls_and_keywords.csv
URL,KEYWORD
\url{https://java.com,Java
\url{https://www.python.org,Python
\url{https://lvee.org,LVEE 3017\end{verbatim}
Здесь мы видим через запятую адреса и ключевые слова на странице. Последняя строчка должна вызывать ошибку, так как конференция с таким названием в этом году не проводится.

\begin{verbatim}
execution:
- concurrency: 2
  iterations: 2
  scenario:
    data-sources:
    - urls_and_keywords.csv
    requests:
    - url: \${URL}
      assert:
      - \${KEYWORD}\end{verbatim}
После выполнения этого скрипта мы заглянем в kpi.jtl, размещенный в artifacts dir. Этот файл является csv-отчетом, который вернул jmeter. Обратим внимание на следующую часть таблицы:

\begin{table}[h!]
  \centering
  \begin{tabular}{ l l l }
    \textbf{label} & \textbf{responseCode} & \textbf{success} \\
    \url{https://java.com} & 200 & true \\
    \url{https://www.python.org} & 200 & true \\
    \url{https://lvee.org} & 200 & false \\
  \end{tabular}
\end{table}
Здесь видно, как виртуальный пользователь по очереди запрашивает адреса из списка, определенного в csv-файле. Ответы получены (RC 200), но последний адрес не прошел проверку ~--- и мы уже знаем, почему.
На этом я желаю вам удачи с использованием нашего инструмента. Мы всегда рады помочь вам, выслушать предложения и рассмотреть пулл-реквесты. За контактами и документацией прошу на gettaurus.org.

\end{document}
