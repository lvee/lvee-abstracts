\documentclass[10pt, a5paper]{article}
\usepackage{pdfpages}
\usepackage{parallel}
\usepackage[T2A]{fontenc}
\usepackage{ucs}
\usepackage[utf8x]{inputenc}
\usepackage[polish,english,russian]{babel}
\usepackage{hyperref}
\usepackage{rotating}
\usepackage[inner=2cm,top=1.8cm,outer=2cm,bottom=2.3cm,nohead]{geometry}
\usepackage{listings}
\usepackage{graphicx}
\usepackage{wrapfig}
\usepackage{longtable}
\usepackage{indentfirst}
\usepackage{array}
\newcolumntype{P}[1]{>{\raggedright\arraybackslash}p{#1}}
\frenchspacing
\usepackage{fixltx2e} %text sub- and superscripts
\usepackage{icomma} % коскі ў матэматычным рэжыме
\PreloadUnicodePage{4}

\newcommand{\longpage}{\enlargethispage{\baselineskip}}
\newcommand{\shortpage}{\enlargethispage{-\baselineskip}}

\def\switchlang#1{\expandafter\csname switchlang#1\endcsname}
\def\switchlangbe{
\let\saverefname=\refname%
\def\refname{Літаратура}%
\def\figurename{Іл.}%
}
\def\switchlangen{
\let\saverefname=\refname%
\def\refname{References}%
\def\figurename{Fig.}%
}
\def\switchlangru{
\let\saverefname=\refname%
\let\savefigurename=\figurename%
\def\refname{Литература}%
\def\figurename{Рис.}%
}

\hyphenation{admi-ni-stra-tive}
\hyphenation{ex-pe-ri-ence}
\hyphenation{fle-xi-bi-li-ty}
\hyphenation{Py-thon}
\hyphenation{ma-the-ma-ti-cal}
\hyphenation{re-ported}
\hyphenation{imp-le-menta-tions}
\hyphenation{pro-vides}
\hyphenation{en-gi-neering}
\hyphenation{com-pa-ti-bi-li-ty}
\hyphenation{im-pos-sible}
\hyphenation{desk-top}
\hyphenation{elec-tro-nic}
\hyphenation{com-pa-ny}
\hyphenation{de-ve-lop-ment}
\hyphenation{de-ve-loping}
\hyphenation{de-ve-lop}
\hyphenation{da-ta-ba-se}
\hyphenation{plat-forms}
\hyphenation{or-ga-ni-za-tion}
\hyphenation{pro-gramming}
\hyphenation{in-stru-ments}
\hyphenation{Li-nux}
\hyphenation{sour-ce}
\hyphenation{en-vi-ron-ment}
\hyphenation{Te-le-pathy}
\hyphenation{Li-nux-ov-ka}
\hyphenation{Open-BSD}
\hyphenation{Free-BSD}
\hyphenation{men-ti-on-ed}
\hyphenation{app-li-ca-tion}

\def\progref!#1!{\texttt{#1}}
\renewcommand{\arraystretch}{2} %Іначай формулы ў матрыцы зліпаюцца з лініямі
\usepackage{array}

\def\interview #1 (#2), #3, #4, #5\par{

\section[#1, #3, #4]{#1 -- #3, #4}
\def\qname{LVEE}
\def\aname{#1}
\def\q ##1\par{{\noindent \bf \qname: ##1 }\par}
\def\a{{\noindent \bf \aname: } \def\qname{L}\def\aname{#2}}
}

\def\interview* #1 (#2), #3, #4, #5\par{

\section*{#1\\{\small\rm #3, #4. #5}}

\def\qname{LVEE}
\def\aname{#1}
\def\q ##1\par{{\noindent \bf \qname: ##1 }\par}
\def\a{{\noindent \bf \aname: } \def\qname{L}\def\aname{#2}}
}


\begin{document}

\title{О <<дырах>> в безопасности}%\footnote{Текст данных и последующих тезисов, кроме специально оговоренных случаев, доступен под лицензией Creative Commons Attribution-ShareAlike 3.0}

\author{Олег Бойцев\footnote{Минск, Беларусь; \url{boytsev_om@mail.ru}}}
\maketitle

\begin{abstract}
During the past years cybersecurity issues become extremely relevant and have already arisen to the government level. Recent high-profile cases of Chinese hackers, the emergence of computer war bible ``The Tallinn Manual on the International Law Applicable to Cyber Warfare'' and the creation of a Russian cyber-army shows how serious this issue is.

The report provides answers to the following highly relevant issues: the way a server may got hacked, how servers are hacked, what to do if the server was owned by a stranger, and what could be the end of this terrible game. Practical exploitation examples of popular vulnerabilities are considered, as far as small research of hack tools.
\end{abstract}

Последние годы вопросы кибербезопасности стоят особенно остро и уже поднимаются на уровне государств. Последние громкие дела о китайских хакерах, появление библии компьютерной войны <<Таллинский учебник>> и создание в России кибервойск показывает, насколько серьезным является этот вопрос.

Рассмотрим краткие ответы на следующие, порой очень актуальные вопросы:

\subsection*{Как <<ломают>> сервера?}

В зависимости от степени автоматизации, несанкционированные проникновения на сервера можно разделить на две большие категории:

\begin{itemize}
  \item Целенаправленный взлом;
  \item Массовый взлом (сервер был взломан роботом).
\end{itemize}

\subsection*{Каким способом?}

Можно выделить следующие способы осуществления взлома:

\begin{itemize}
  \item Эксплуатация уязвимостей;
  \item Подбор паролей;
  \item Социальная инженерия.
\end{itemize}

\subsection*{Что делать?}

Порядок действий непосредственно после обнаружения факта несанкционированного проникновения на сервер, должен включать следующие пункты:

\begin{itemize}
  \item Побыстрее сделать резервные копии;
  \item <<Убить>> процессы и удалить скрипты, принадлежащие хакеру;
  \item Закрыть уязвимость (то есть устранить первопричину взлома).
\end{itemize}

\subsection*{Последствия?}

Среди последствий взлома сервера, наступающих в случае непринятых своевременно мер, необходимо упомянуть:

\begin{itemize}
  \item Поднятие прав с веб-сервера до рута;
  \item Шантаж владельца сервера;
  \item \verb!rm -Rf /!
\end{itemize}

Практическими примерами эксплуатации популярных уязвимостей являются:

\begin{itemize}
  \item SQL Injection (внедрение в запрос к базе данных постороннего SQL-кода, выполняющего, в зависимости от используемой СУБД, различные вредоносные действия, от хищения приватной информации до удаления таблиц или даже локальных файлов);
  \item XSS (межсайтовый скриптинг, то есть внедрение выполняемых на клиентском компьютере вредоносных скриптов в веб"=страницу, выдаваемую серевером, на котором авторизован данный клиент);
  \item PHP Include (выполнение на сервере постороннего PHP-кода, поступившего через HTTP"=запрос).
\end{itemize}

Исследование средств, применяемых хакерами для закрепления на сервере, выявляет следующие инструменты:

\begin{itemize}
  \item Orb web shell;
  \item Bind/Back Connect Backdoor;
  \item Suid scripts.
\end{itemize}



\end{document}




