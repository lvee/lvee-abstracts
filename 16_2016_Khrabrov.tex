\documentclass[10pt, a5paper]{article}
\usepackage{pdfpages}
\usepackage{parallel}
\usepackage[T2A]{fontenc}
\usepackage{ucs}
\usepackage[utf8x]{inputenc}
\usepackage[polish,english,russian]{babel}
\usepackage{hyperref}
\usepackage{rotating}
\usepackage[inner=2cm,top=1.8cm,outer=2cm,bottom=2.3cm,nohead]{geometry}
\usepackage{listings}
\usepackage{graphicx}
\usepackage{wrapfig}
\usepackage{longtable}
\usepackage{indentfirst}
\usepackage{array}
\newcolumntype{P}[1]{>{\raggedright\arraybackslash}p{#1}}
\frenchspacing
\usepackage{fixltx2e} %text sub- and superscripts
\usepackage{icomma} % коскі ў матэматычным рэжыме
\PreloadUnicodePage{4}

\newcommand{\longpage}{\enlargethispage{\baselineskip}}
\newcommand{\shortpage}{\enlargethispage{-\baselineskip}}

\def\switchlang#1{\expandafter\csname switchlang#1\endcsname}
\def\switchlangbe{
\let\saverefname=\refname%
\def\refname{Літаратура}%
\def\figurename{Іл.}%
}
\def\switchlangen{
\let\saverefname=\refname%
\def\refname{References}%
\def\figurename{Fig.}%
}
\def\switchlangru{
\let\saverefname=\refname%
\let\savefigurename=\figurename%
\def\refname{Литература}%
\def\figurename{Рис.}%
}

\hyphenation{admi-ni-stra-tive}
\hyphenation{ex-pe-ri-ence}
\hyphenation{fle-xi-bi-li-ty}
\hyphenation{Py-thon}
\hyphenation{ma-the-ma-ti-cal}
\hyphenation{re-ported}
\hyphenation{imp-le-menta-tions}
\hyphenation{pro-vides}
\hyphenation{en-gi-neering}
\hyphenation{com-pa-ti-bi-li-ty}
\hyphenation{im-pos-sible}
\hyphenation{desk-top}
\hyphenation{elec-tro-nic}
\hyphenation{com-pa-ny}
\hyphenation{de-ve-lop-ment}
\hyphenation{de-ve-loping}
\hyphenation{de-ve-lop}
\hyphenation{da-ta-ba-se}
\hyphenation{plat-forms}
\hyphenation{or-ga-ni-za-tion}
\hyphenation{pro-gramming}
\hyphenation{in-stru-ments}
\hyphenation{Li-nux}
\hyphenation{sour-ce}
\hyphenation{en-vi-ron-ment}
\hyphenation{Te-le-pathy}
\hyphenation{Li-nux-ov-ka}
\hyphenation{Open-BSD}
\hyphenation{Free-BSD}
\hyphenation{men-ti-on-ed}
\hyphenation{app-li-ca-tion}

\def\progref!#1!{\texttt{#1}}
\renewcommand{\arraystretch}{2} %Іначай формулы ў матрыцы зліпаюцца з лініямі
\usepackage{array}

\def\interview #1 (#2), #3, #4, #5\par{

\section[#1, #3, #4]{#1 -- #3, #4}
\def\qname{LVEE}
\def\aname{#1}
\def\q ##1\par{{\noindent \bf \qname: ##1 }\par}
\def\a{{\noindent \bf \aname: } \def\qname{L}\def\aname{#2}}
}

\def\interview* #1 (#2), #3, #4, #5\par{

\section*{#1\\{\small\rm #3, #4. #5}}

\def\qname{LVEE}
\def\aname{#1}
\def\q ##1\par{{\noindent \bf \qname: ##1 }\par}
\def\a{{\noindent \bf \aname: } \def\qname{L}\def\aname{#2}}
}

\begin{document}
\title{Написание своей StdLibC}
\author{Дмитрий Храбров, Homel, Belarus}
\maketitle
\begin{abstract}
This article shows how you can implement basic stdlib's functions. You can see here: string functions, formatted output, usage of *Nix kernel functions etc.
\end{abstract}
В последнее время на страницах информационных ресурсов можно увидеть статьи <<Привет из свободного от libc мира>> ~\cite{Khrabrov1}, <<Минимальный Elf>> ~\cite{Khrabrov2} и пр. Все они показывают настройку получаемого исполняемого файла, стараются его уменьшить. Например, получен работающий Elf-файл, занимающий всего 45 байт дискового пространства. Однако за данной минимизацией забывается такое важное свойство, как юзабилити. Elf-файл в 45 байт умеет только запускаться и корректно завершаться. При этом если добавить хотя бы вывод классического сообщения <<Hello World>>, то размер станет далеко не таким маленьким и от многих техник придётся отказаться.

Существует множество свободных реализаций стандартной библиотеки Си ~\cite{Khrabrov3}. Минимальный размер имеет dietlibc, которая позволяет скомпилировать приложение размером всего 200 байт. Или 6 килобайт если используется функция printf. Однако значительная часть функционала там не реализована, или реализована не в соответствии со стандартом. Кроме того, dietlibc лицензирована под GPL2, что ставит ряд значительных ограничений (свои приложения придётся так же лицензировать под GPL2). Ещё одной хорошей альтернативой в плане размера является musl: 1800 байт минимум, 13 килобайт с printf. Библиотека musl лицензирована под MIT и качественно реализует стандарт. Однако именно из-за полного охвата стандарта Си и получаются 13 килобайт, которые хотелось бы уменьшить.

Целью данной работы является создание библиотеки, с помощью которой можно было бы более-менее привычно программировать, но в то же время минимизировать размер получаемого исполняемого файла, вообще не использовать стандартную библиотеку. Данная работа представляет больше исследовательский интерес, чем практическую ценность. И в конечном счёте приближает к пониманию, как именно работают те или иные функции из стандартной библиотеки языка Си. Предложенные реализации далеко не оптимальны или безопасны, однако просты для понимания.

В итоге необходимо получить корректное функционирование \linebreak данного кода:
\lstset{ %
language=C,                 % выбор языка для подсветки (здесь это С)
basicstyle=\small\sffamily, % размер и начертание шрифта для подсветки кода
breaklines=true,           % автоматически переносить строки (да\нет)
breakatwhitespace=false, % переносить строки только если есть пробел
}
\begin{lstlisting}
printf("Hello %s LVEE %d!\n", "Summer", 2016);
\end{lstlisting}

В этой простейшей, казалось бы, строке кода, поднимается сразу несколько вопросов: 1) непосредственно сама функция printf и её форматный вывод; 2) функции работы со строками; 3) конвертирование чисел в строку; 4) непосредственно вывод символов на экран. Всё это обычно реализуется средствами стандартной библиотеки, и всё придётся реализовать самостоятельно.

Syscall "--- вызов функций ядра Linux, для x86 реализуется через $0\times80$ ассемблерное прерывание. Полный код для функции с 3 аргументами (взят из ~\cite{Khrabrov4}):

\begin{lstlisting}
static long _syscall3(int number, unsigned long arg1, unsigned long arg2, unsigned long arg3){
  long result;
    __asm__ volatile ("int $0x80"
    : "=a" (result)
    : "0" (number), "b" (arg1), "c" (arg2), "d" (arg3)
    : "memory", "cc");
  return result;
}
\end{lstlisting}

Если вызываемой функции необходимо менее 3 аргументов, то завершающие аргументы можно передать нулевыми. Например, \linebreak функция close принимает всего 1 аргумент "--- файловый дескриптор. Реализация может выглядеть следующим образом:

\begin{lstlisting}
return (int)syscall3(__NR_close, fd, 0, 0);
\end{lstlisting}

Константы названий функций (NR\_close и пр.) задаются для каждой архитектуры отдельно в файле unistd.h. Это заголовочный файл для доступа к API POSIX-совместимой операционной системы. В unistd.h можно посмотреть полный список функций ядра Linux. И не увидеть там ни строковых функций, ни потоков (thread), ни malloc/calloc/free, ни printf. Весь этот функционал реализуется на функциях более низкого уровня в стандартной библиотеке. Вывод символов на экран консоли осуществляется с помощью функции write:

\begin{lstlisting}
return (ssize_t)syscall3(__NR_write, fd, buf, count);
\end{lstlisting}

Как видно, функция write использует 3 аргумента: файловый дескриптор, буфер и размер буфера в байтах. Однако бывают ситуации, когда размер буфера заранее неизвестен. Обычно в таких случаях используется функция получения длины строки "--- strlen. Реализуется она средствами чистого Си, без вызова функций ядра. Это является преимуществом, так как такой код может быть скомпилирован любым компилятором, реализующим стандарт Си. Начиная от свободных GCC, Clang, Tiny C Compiler, и заканчивая проприетарной Visual Studio. Пример реализации поиска длины строки:

\begin{lstlisting}
len=0; while( str[len] != 0 ) len++;
\end{lstlisting}

Аналогично реализуются и многие другие функции работы со строками "--- происходит посимвольная обработка всей строки в цикле. В качестве примера обработки одного символа можно рассмотреть функцию перевода символа в верхний регистр. Если символ лежит в интервале от a до z, то от его кода отнимается <<a>> и прибавляется код <<A>>:

\begin{lstlisting}
return symb - 'a' + 'A';
\end{lstlisting}

Конвертирование целых чисел в строку "--- деление на 10 до тех пор, пока не останется 0. Остаток от деления складывать в буфер. Существует множество способов сделать перевод из числа в строку, но они всё-равно опираются на какой-либо один базовый, который и необходимо реализовать.

Функция printf может принимать разное количество аргументов. Эта <<перегрузка>> параметров функции осуществляется средствами Си через списки варьирующихся аргументов: va\_list, \linebreak va\_start, va\_arg, va\_end.

Пример реализации:

\begin{verbatim}
void Myprintf(char* format, ...){
  char* s;  // Буфер для вывода строк 
  int i; // Счётчик текущей позиции в форматной строке
  va_list arg; // Список варьирующихся аргументов
  va_start(arg, format); // Инициализируем список
  for(i=0; i<strlen(format); i++){
    if( format[i] == '%' ) switch( format[i+1] ){
      case 'd': // вызов своей функции для чисел
        MyPutInt( va_arg(arg, int) ); i++; break;
      case 's': // получение строки и её вывод
        s = va_arg(arg, char*);  
        MyWrite( s, strlen(s) ); i++; break;
      case 'c': // аргумент - в int
        MyPutchar( va_arg(arg, int) ); i++; break;
    } else MyPutchar( format[i] );
  } // вывод одного символа реализуется через write
  va_end(arg); // Необходимая очистка списка
}
\end{verbatim}

Суть так же проста "--- перебираем символы форматной строки, если встречаем символ `\%', то смотрим следующий за ним. По следующему символу узнаём тип аргумента (если символ <<d>>, то нужно конвертировать в int). Далее просто берём следующий аргумент из списка варьируемых, сразу просим привести его к нужному типу. Нужно помнить, что нельзя сразу конвертировать во float, char или short "--- изначально должно быть преобразование в double или int.

Теперь остаётся только скомпилировать приложение с флагами nodefaultlibs и nostartfiles (подробнее про это можно почитать в статье <<Привет из свободного от libc мира>> ~\cite{Khrabrov1}). Если всё собрано правильно, то при вызове ldd будет писать <<not a dynamic executable>>. Это означает, что полученное приложение вообще не зависит ни от каких системных библиотек. Но в то же время оно написано на Си и использует функцию printf, что и требовалось получить.

\begin{thebibliography}{99}

\bibitem{Khrabrov1} Hello from a libc-free world! \url{https://blogs.oracle.com/ksplice/entry/hello\_from\_a\_libc\_free}
\bibitem{Khrabrov2} A Whirlwind Tutorial on Creating Really Teensy ELF Executables for Linux. \url{http://www.muppetlabs.com/\~{}breadbox/software/tiny/teensy.html}
\bibitem{Khrabrov3} Comparison of C/POSIX standard library implementations for Linux. \url{http://www.etalabs.net/compare\_libcs.html}
\bibitem{Khrabrov4} Program in C without any C library. \url{https://github.com/fishilico/shared/tree/master/linux/nolibc}
\end{thebibliography}
\end{document}
