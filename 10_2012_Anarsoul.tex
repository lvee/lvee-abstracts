\documentclass[10pt, a5paper]{article}
\usepackage{pdfpages}
\usepackage{parallel}
\usepackage[T2A]{fontenc}
\usepackage{ucs}
\usepackage[utf8x]{inputenc}
\usepackage[polish,english,russian]{babel}
\usepackage{hyperref}
\usepackage{rotating}
\usepackage[inner=2cm,top=1.8cm,outer=2cm,bottom=2.3cm,nohead]{geometry}
\usepackage{listings}
\usepackage{graphicx}
\usepackage{wrapfig}
\usepackage{longtable}
\usepackage{indentfirst}
\usepackage{array}
\newcolumntype{P}[1]{>{\raggedright\arraybackslash}p{#1}}
\frenchspacing
\usepackage{fixltx2e} %text sub- and superscripts
\usepackage{icomma} % коскі ў матэматычным рэжыме
\PreloadUnicodePage{4}

\newcommand{\longpage}{\enlargethispage{\baselineskip}}
\newcommand{\shortpage}{\enlargethispage{-\baselineskip}}

\def\switchlang#1{\expandafter\csname switchlang#1\endcsname}
\def\switchlangbe{
\let\saverefname=\refname%
\def\refname{Літаратура}%
\def\figurename{Іл.}%
}
\def\switchlangen{
\let\saverefname=\refname%
\def\refname{References}%
\def\figurename{Fig.}%
}
\def\switchlangru{
\let\saverefname=\refname%
\let\savefigurename=\figurename%
\def\refname{Литература}%
\def\figurename{Рис.}%
}

\hyphenation{admi-ni-stra-tive}
\hyphenation{ex-pe-ri-ence}
\hyphenation{fle-xi-bi-li-ty}
\hyphenation{Py-thon}
\hyphenation{ma-the-ma-ti-cal}
\hyphenation{re-ported}
\hyphenation{imp-le-menta-tions}
\hyphenation{pro-vides}
\hyphenation{en-gi-neering}
\hyphenation{com-pa-ti-bi-li-ty}
\hyphenation{im-pos-sible}
\hyphenation{desk-top}
\hyphenation{elec-tro-nic}
\hyphenation{com-pa-ny}
\hyphenation{de-ve-lop-ment}
\hyphenation{de-ve-loping}
\hyphenation{de-ve-lop}
\hyphenation{da-ta-ba-se}
\hyphenation{plat-forms}
\hyphenation{or-ga-ni-za-tion}
\hyphenation{pro-gramming}
\hyphenation{in-stru-ments}
\hyphenation{Li-nux}
\hyphenation{sour-ce}
\hyphenation{en-vi-ron-ment}
\hyphenation{Te-le-pathy}
\hyphenation{Li-nux-ov-ka}
\hyphenation{Open-BSD}
\hyphenation{Free-BSD}
\hyphenation{men-ti-on-ed}
\hyphenation{app-li-ca-tion}

\def\progref!#1!{\texttt{#1}}
\renewcommand{\arraystretch}{2} %Іначай формулы ў матрыцы зліпаюцца з лініямі
\usepackage{array}

\def\interview #1 (#2), #3, #4, #5\par{

\section[#1, #3, #4]{#1 -- #3, #4}
\def\qname{LVEE}
\def\aname{#1}
\def\q ##1\par{{\noindent \bf \qname: ##1 }\par}
\def\a{{\noindent \bf \aname: } \def\qname{L}\def\aname{#2}}
}

\def\interview* #1 (#2), #3, #4, #5\par{

\section*{#1\\{\small\rm #3, #4. #5}}

\def\qname{LVEE}
\def\aname{#1}
\def\q ##1\par{{\noindent \bf \qname: ##1 }\par}
\def\a{{\noindent \bf \aname: } \def\qname{L}\def\aname{#2}}
}


\begin{document}

\title{Особенности возвращения наработок в апстрим}%\footnote{Текст данных и последующих тезисов, кроме специально оговоренных случаев, доступен под лицензией Creative Commons Attribution-ShareAlike 3.0}

\author{Василий Хоружик\footnote{Минск, Беларусь}}
\maketitle

\begin{abstract}
Patch submitting guide presented here briefly describes an iteration of getting ones code upstream, which includes following steps: code modification, creating patch, submitting patch, and getting review.
\end{abstract}

\subsection*{Введение}

Сообщество open source широко известно своим подходом <<Если тебе что-то не нравится, возьми и исправь>>. Но на самом деле испоправить недочеты в программном коде "--- это только половина задачи. Вторая и главная часть "--- добится включения патча в апстрим, т.\,е. в основное дерево исходных кодов проекта.

\subsection*{Зачем?}

Включение исправлений в апстрим избавляет их автора от необходимости ручной пересборки пакета с наложенным патчем после выхода каждой новой версии. Не менее утомительно и заниматься постоянной адаптацией патча под изменения, происходящие в коде проекта от версии к версии.

\subsection*{Мэйнтэйнеры}

На самом деле большой проект, даже такой сложный, как ядро Linux "--- всего лишь программа, а её мэйнтейнер(ы) "--- обычные люди. Всё что требуется от автора патча "--- это привести патч в приемлимый вид и отправить нужному человеку (людям).

\subsection*{Coding Style}

Важными нюансами являются особенности использования табуляций и пробелов. Аналогично обстоит дело с фигурными скобками, которые могут использоваться на той же самой или на следующей строке.

Необходимо выяснить, как принято писать код в данном конкретном проекте. Часто у проекта есть соответствующий документ, например linux/Documents/CodingStyle. Есть смысл заранее настроить соответствующим образом свой текстовый редактор, либо переформатировать код в соответствии со стандартом проекта. А в некоторых случаях такие параметры уже подобраны, и даже готов соответствующий скрипт, например Lindent (linux/scripts/Lindent).

В некоторых проектах (u-boot, Linux, qemu) есть скрипт для проверки патча на соответствие стиля кодирования принятым нормам (scripts/checkpatch.pl)

\subsection*{Система контроля версий}

Разработчики очень редко используют релизные версии (снапшоты) своих проектов. Пытаться добится включения патча в релизную версию кода "--- сомнительное предприятие. По всей видимости, для продвижения патча в апстрим понадобится освоить git (svn, hg, или другую систему контроля версий, которую используют в данном проекте).

В случае git всё выглядит просто:

\begin{itemize}
  \item  \verb!git clone git://project.url my_local_project! "--- создать \linebreak локальный клон удалённого репозитория
  \item  \verb!git diff [-cached]! "--- просмотреть текущие [добавленные] изменения
  \item  \verb!git add file_name.1 file_name.2! "--- добавить изменения для фиксации
  \item  \verb!git commit! "--- фиксация изменений, для большинства проектов обязательна опция \verb!-s! (т.н. Sign-Off)
  \item  \verb!git format-patch id_коммита! "--- сформировать патчи для коммитов, начиная с коммита, следующего за указанным.
\end{itemize}

\subsection*{Куда отсылать}

После очередной вычитки патча, прогона через checkpatch.pl наступает момент для отправки его на первый review. Кому именно отсылать "--- зависит от проекта, причем обычно это написано в README. Для Linux и u-boot есть скрипт get\_maintainer.pl (находится в linux/scripts), который анализирует патч и определяет, кому его отправить (выдает список адресов).

Отсылать патчи для проектов, которые используют git, лучше всего с помощью git send-email. Используя какой-либо почтовый клиент, следует убедиться, что он не переносит строки, не заменяет табуляцию пробелами, и вообще ничего не изменяет в патче. Слать патч прикреплённым к письму считается дурным тоном, т.\,к. его в таком виде неудобно просматривать и комментировать. Поэтому патч нужно <<включать>> в письмо (inline).

Как правило, приходится отправлять сообщение и в список рассылки (на который лучше всего подписаться), и конкретному человеку (добавляя его адрес в поле CC:).

Для ядра Linux, если вы хотите включения изменения в релизную ветку ядра, также стоит ставить в копию stable@vger.kernel.org.

\subsection*{Review}

Как правило, мэйнтейнеры отвечают на письмо с патчем, <<разбавляя>> его комментариями. Очень редко всё получается с первого раза; Как правило, автор получает в ответ пачку рекомендаций по доводке своего кода. Это нормальное явление, поскольку контрибуция в открытый проект является итеративным процессом, а вы в этот момент находитесь на первой итерации. Также не имеет смысла спорить с мэйнтейнером по поводу каждой строчки кода. Мэйнтейнер практически всегда прав. Несколько рекомендаций:

\begin{itemize}
    \item Процедура выглядит следующим образом: получить комментарий "--- поправить патч "--- отослать снова, с пометкой <<v2>> (или <<vN>>). Лучше не затягивать, т.к. при долгих итерациях патч может устареть или мэйнтейнер может забыть контекст.
    \item При долгом отсутствии ответа может потребоваться послать письмо-напоминание. Поскольку мэйнтейнер "--- обычный человек, он может быть занят или загружен работой, или просто забыть о Вас.
    \item Не имеет смысла ругайться с мэйнтейнером. В крайнем случае следует попросить рассудить спор какое-либо третье лицо.
    \item И, наконец, не следует отчаиваться.
\end{itemize}

Никогда не следует слать патчи, которые:

\begin{itemize}
    \item     Содержат большие блоки закомментированного кода
    \item Перемещают файлы без видимых на то причин
    \item Меняют систему сборки проекта с технологии N на технологию M, без видимых причин и плюсов
    \item Патчи, реализующие сразу несколько фич либо исправляющие несколько багов. Необходимо следовать правилу <<один баг "--- один патч>>, <<одна фича – один патч>>.
\end{itemize}

\subsection*{А что если...}

Может случится так, что предложенное изменение расходится с точкой зрения мэйнтейнера и/или общей архитектурой проекта. В таком случае существует два выхода:

\begin{enumerate}
\item    переделать патч
\item    инициировать и поддерживать свой собственный форк проекта
\end{enumerate}

\subsection*{Выводы}

Итак, работу можно разделить на 4 шага:

\begin{enumerate}
\item    Правка кода
\item Оформление изменений
\item Отсылка патча
\item Получение review, и при необходимости "--- возврат к пункту 2.
\end{enumerate}

\end{document}




