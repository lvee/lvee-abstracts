\documentclass[10pt, a5paper]{article}
\usepackage{pdfpages}
\usepackage{parallel}
\usepackage[T2A]{fontenc}
\usepackage{ucs}
\usepackage[utf8x]{inputenc}
\usepackage[polish,english,russian]{babel}
\usepackage{hyperref}
\usepackage{rotating}
\usepackage[inner=2cm,top=1.8cm,outer=2cm,bottom=2.3cm,nohead]{geometry}
\usepackage{listings}
\usepackage{graphicx}
\usepackage{wrapfig}
\usepackage{longtable}
\usepackage{indentfirst}
\usepackage{array}
\newcolumntype{P}[1]{>{\raggedright\arraybackslash}p{#1}}
\frenchspacing
\usepackage{fixltx2e} %text sub- and superscripts
\usepackage{icomma} % коскі ў матэматычным рэжыме
\PreloadUnicodePage{4}

\newcommand{\longpage}{\enlargethispage{\baselineskip}}
\newcommand{\shortpage}{\enlargethispage{-\baselineskip}}

\def\switchlang#1{\expandafter\csname switchlang#1\endcsname}
\def\switchlangbe{
\let\saverefname=\refname%
\def\refname{Літаратура}%
\def\figurename{Іл.}%
}
\def\switchlangen{
\let\saverefname=\refname%
\def\refname{References}%
\def\figurename{Fig.}%
}
\def\switchlangru{
\let\saverefname=\refname%
\let\savefigurename=\figurename%
\def\refname{Литература}%
\def\figurename{Рис.}%
}

\hyphenation{admi-ni-stra-tive}
\hyphenation{ex-pe-ri-ence}
\hyphenation{fle-xi-bi-li-ty}
\hyphenation{Py-thon}
\hyphenation{ma-the-ma-ti-cal}
\hyphenation{re-ported}
\hyphenation{imp-le-menta-tions}
\hyphenation{pro-vides}
\hyphenation{en-gi-neering}
\hyphenation{com-pa-ti-bi-li-ty}
\hyphenation{im-pos-sible}
\hyphenation{desk-top}
\hyphenation{elec-tro-nic}
\hyphenation{com-pa-ny}
\hyphenation{de-ve-lop-ment}
\hyphenation{de-ve-loping}
\hyphenation{de-ve-lop}
\hyphenation{da-ta-ba-se}
\hyphenation{plat-forms}
\hyphenation{or-ga-ni-za-tion}
\hyphenation{pro-gramming}
\hyphenation{in-stru-ments}
\hyphenation{Li-nux}
\hyphenation{sour-ce}
\hyphenation{en-vi-ron-ment}
\hyphenation{Te-le-pathy}
\hyphenation{Li-nux-ov-ka}
\hyphenation{Open-BSD}
\hyphenation{Free-BSD}
\hyphenation{men-ti-on-ed}
\hyphenation{app-li-ca-tion}

\def\progref!#1!{\texttt{#1}}
\renewcommand{\arraystretch}{2} %Іначай формулы ў матрыцы зліпаюцца з лініямі
\usepackage{array}

\def\interview #1 (#2), #3, #4, #5\par{

\section[#1, #3, #4]{#1 -- #3, #4}
\def\qname{LVEE}
\def\aname{#1}
\def\q ##1\par{{\noindent \bf \qname: ##1 }\par}
\def\a{{\noindent \bf \aname: } \def\qname{L}\def\aname{#2}}
}

\def\interview* #1 (#2), #3, #4, #5\par{

\section*{#1\\{\small\rm #3, #4. #5}}

\def\qname{LVEE}
\def\aname{#1}
\def\q ##1\par{{\noindent \bf \qname: ##1 }\par}
\def\a{{\noindent \bf \aname: } \def\qname{L}\def\aname{#2}}
}

\begin{document}
\title{<<Параноидальный>> офис "---  оборотная сторона безопасности\footnote{\url{admin@itg.by}, \url{http://lvee.org/ru/abstracts/206}}}
\author{Дмитрий Степанов, Минск, Беларусь}
\maketitle
\begin{abstract}
News about data breaches, hacking, discovered vulnerabilities are now typical. Common solutions deal with external threats along with international standards and best practices, not taking into account non-commercial threats of information leaks.
\end{abstract}
С каждым годом информационные ресурсы и системы развиваются, все глубже проникают в повседневную деятельность компаний и каждого человека в отдельности. Сейчас ни у кого не вызывает удивления новость об утечке данных пользователей того или иного сервиса, взломе системы, обнаружении уязвимости продукта. Многие руководители предприятий полагают, что это их не коснется, так как <<это где-то там, а мы здесь>>. Решения, внедряемые на предприятиях, защищают в большинстве от внешних угроз, взломов, атак, они согласованы с международными стандартами и рекомендациями (cobit, itil и т. д.), и в большинстве своем оценивают риски потери информации с точки зрения  убытков для предприятия, но не учитывают, что информация на предприятии может быть не только коммерческой но и чрезмерно конфиденциальной в ином плане.

Весь этот информационный поток порождает у руководителей частных предприятий вполне обоснованную паранойю о сохранности информации по деятельности предприятия.

Исходя из выше описанного мы и рассмотрим распространившуюся паранойю одного руководителя частной компании <<X>>  и реализацию его желаний и потребностей.

На момент созревания <<wish list>> система компании <<X>> представляла собой филиальную сеть между двумя странами, граничащими друг с другом. В стране <<А>> находился отдел продажи и маркетинга (собственно, мы находились в стране <<А>>), а в стране <<Б>> "--- производство, логистика и небольшие продажи.

\textbf{Инфраструктура} была реализована на основе платного ПО (все лицензировано) и представляла следующую картину:

\begin{itemize}
  \item 80 ПК с установленными операционными системами Windows;
  \item 3 Сервера: 1. Шлюз на платном ПО; 2. Бухгалтерия на платном ПО; 3. Файловый сервер на Платном ПО.
\end{itemize}

Структура описана кратко и не учитывает наличие АТС и структуры сети (наличия коммутационного оборудования и Wi-fi точек).

\subsection*{Задача.}

Была поставлена задача реализовать полное шифрование каналов связи с внешним миром, шифрование почтовой переписки, отслеживание сотрудников, максимально ограничить возможность утечки информации через сотрудников компании и в случае <<случайной пропажи какого-либо сервера или компьютера с жесткими дисками из компании>>. Важнейшим условием стало уменьшение стоимости лицензирования ПО и то, что внедряемый проект в последующем будет распространен на остальные подразделения компании.

\subsection*{План создания.}

Выбор ОС пал на дистрибутивы Linux по следующим причинам:

\begin{itemize}
  \item низкий уровень владения и грамотности конечных пользователей в среде Linux;
  \item стоимость дистрибутива Linux;
  \item возможность реализации шифрования из коробки;
  \item наличие альтернативного ПО на замену проприетарным пакетам в среде Windows;
  \item компания занимается продажами "--- количество узкоспециализированного по незначительного;
  \item производительность ОС на разношерстном парке ПК.
\end{itemize}

В результате сервера были реализованы на Ubuntu Linux, а клиентские места на Linux Mint Xfce Edition.

\subsection*{Защита каналов и выбор шлюза.}

Реализация защиты каналов была проведена посредством объединения филиалов Ipsec тоннелем, в качестве межсетевого экрана тестирровались системы Zentyal и ZeroShell. В городе <<А>> был установлен ZeroShell благодаря его возможности работать с LiveCD. На второй стороне в городе <<Б>> был установлен Zentyal "--- использовался как почтовый сервер и LDAP-сервер. Оба сервера работают под приложением OpenVPN, и таким образом реализация тоннеля не вызвала каких-либо существенных трудностей.

Маршруты были настроены так, что весь офис <<А>> выходит в интернет через офис <<Б>>. Также был реализован отдельный канал IPSec с SIP-провайдером посредством установки маршрутизатора на стороне провайдера.

В последствии система несколько видоизменялась в связи с улучшением интеграции.

Почтовая переписка была переведена на почтовый клиент \linebreak Thunderbird c реализованным модулем шифрования на основе PGP. На всех ПК был настроен Iptables для работы только в конкретной сети и с конкретными портами.

\subsection*{Шифрование серверов и офисного ПК.}

Полное шифрование диска (Full Disk Encryption, FDE) стало доступнее для простых пользователей Linux. Так начиналась статья (откровенно говоря, не помню какого журнала и тем более автора), ставшая в свое время основой для неоднократной реализации <<параноидальных офисов>>.

По очевидным причинам, мы не хотели стать жертвами \linebreak термально-ректального криптоанализа со стороны злоумышленников, в результате чего выбрали Luks-шифрование с использованием ключей на основе сертификатов, причем сертификат находился за пределами границ государства офиса <<А>>. Также были предприняты меры по уничтожению сертификата в случае необходимости сохранения информации в безопасности.

Контроль целостности файловой структуры был реализован посредством программного модуля Tripwire.

Таким образом мы получили полностью реализованную на \linebreak GNU/Linux структуру с шифрованием дисков, на которых оставалась исключительно загрузочная область для получения сертификата и разблокирования диска.

\subsection*{Реализация отслеживания персонала.}

Данный вопрос решился на удивление просто: у каждого сотрудника существовал корпоративный телефон на Android или iPhone, который проходил перед каждой командировкой подготовку и чистку у отдела IT. На данные аппараты были установлены скрытые приложения для программного СПО Traccar. Это позволило убить сразу двух зайцев: контроль транспорта и командированных сотрудников, так как телефоны были заведены через VPN-канал на АТС для осуществления звонков через SIP и бесплатной связи с офисом, отключения производились редко, и в результате трек и местоположение были корректны и отчетливы.

Для реализации удаленной работы и подачи приложений (с почтой и иными приложениями) на случай необходимости был рассмотрен продукт Ulteo, недостатком которого было использование Java, но переход на HTML5 вопросы снял.

\subsection*{Подводя итоги:}

Реализация данного проекта позволила в очередной раз пользователям с иной стороны рассмотреть СПО. Экономическая выгода, а так же явное преимущество <<коробочного решения>> по шифрованию, альтернативное офисное ПО и уровень защищенности платформы (благодаря низкой пользовательской популярности и своей структуре) от внешнего зловредного ПО делают гибридные интеграционные решения все более востребованными в современном мире.

\end{document}
