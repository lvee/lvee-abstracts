\documentclass[10pt, a5paper]{article}
\usepackage{pdfpages}
\usepackage{parallel}
\usepackage[T2A]{fontenc}
\usepackage{ucs}
\usepackage[utf8x]{inputenc}
\usepackage[polish,english,russian]{babel}
\usepackage{hyperref}
\usepackage{rotating}
\usepackage[inner=2cm,top=1.8cm,outer=2cm,bottom=2.3cm,nohead]{geometry}
\usepackage{listings}
\usepackage{graphicx}
\usepackage{wrapfig}
\usepackage{longtable}
\usepackage{indentfirst}
\usepackage{array}
\newcolumntype{P}[1]{>{\raggedright\arraybackslash}p{#1}}
\frenchspacing
\usepackage{fixltx2e} %text sub- and superscripts
\usepackage{icomma} % коскі ў матэматычным рэжыме
\PreloadUnicodePage{4}

\newcommand{\longpage}{\enlargethispage{\baselineskip}}
\newcommand{\shortpage}{\enlargethispage{-\baselineskip}}

\def\switchlang#1{\expandafter\csname switchlang#1\endcsname}
\def\switchlangbe{
\let\saverefname=\refname%
\def\refname{Літаратура}%
\def\figurename{Іл.}%
}
\def\switchlangen{
\let\saverefname=\refname%
\def\refname{References}%
\def\figurename{Fig.}%
}
\def\switchlangru{
\let\saverefname=\refname%
\let\savefigurename=\figurename%
\def\refname{Литература}%
\def\figurename{Рис.}%
}

\hyphenation{admi-ni-stra-tive}
\hyphenation{ex-pe-ri-ence}
\hyphenation{fle-xi-bi-li-ty}
\hyphenation{Py-thon}
\hyphenation{ma-the-ma-ti-cal}
\hyphenation{re-ported}
\hyphenation{imp-le-menta-tions}
\hyphenation{pro-vides}
\hyphenation{en-gi-neering}
\hyphenation{com-pa-ti-bi-li-ty}
\hyphenation{im-pos-sible}
\hyphenation{desk-top}
\hyphenation{elec-tro-nic}
\hyphenation{com-pa-ny}
\hyphenation{de-ve-lop-ment}
\hyphenation{de-ve-loping}
\hyphenation{de-ve-lop}
\hyphenation{da-ta-ba-se}
\hyphenation{plat-forms}
\hyphenation{or-ga-ni-za-tion}
\hyphenation{pro-gramming}
\hyphenation{in-stru-ments}
\hyphenation{Li-nux}
\hyphenation{sour-ce}
\hyphenation{en-vi-ron-ment}
\hyphenation{Te-le-pathy}
\hyphenation{Li-nux-ov-ka}
\hyphenation{Open-BSD}
\hyphenation{Free-BSD}
\hyphenation{men-ti-on-ed}
\hyphenation{app-li-ca-tion}

\def\progref!#1!{\texttt{#1}}
\renewcommand{\arraystretch}{2} %Іначай формулы ў матрыцы зліпаюцца з лініямі
\usepackage{array}

\def\interview #1 (#2), #3, #4, #5\par{

\section[#1, #3, #4]{#1 -- #3, #4}
\def\qname{LVEE}
\def\aname{#1}
\def\q ##1\par{{\noindent \bf \qname: ##1 }\par}
\def\a{{\noindent \bf \aname: } \def\qname{L}\def\aname{#2}}
}

\def\interview* #1 (#2), #3, #4, #5\par{

\section*{#1\\{\small\rm #3, #4. #5}}

\def\qname{LVEE}
\def\aname{#1}
\def\q ##1\par{{\noindent \bf \qname: ##1 }\par}
\def\a{{\noindent \bf \aname: } \def\qname{L}\def\aname{#2}}
}

\switchlang{en}
\begin{document}
\title{How open source changing and reshaping enterprises\footnote{\url{andreyrom@tut.by}, \url{https://lvee.org/en/abstracts/285}}}
\author{Андрей Романюк, Minsk, Belarus}
\maketitle
\begin{abstract}
Since the past 25 years relationship between open source and enterprise changed dramatically. While in the middle of 1990s open source was considered as a hobby for enthusiasts these days most enterprises build their success using open source products. There are three major trends that changed the shape of open source and enterprise relationship over the time 1. enterprise compatibility - we can see a lot of successful open source products used to leverage the business of enterprise products. Among them we can see NGINX, RedHat, etc\ldots 2. product innovation cycle for enterprise software is still stuck on 3 years while open source products have much more agile approach delivering\linebreak innovative releases much more frequently 3. platform compatibility is the major differentiator between open source and closed source platforms. Open source software helps users to use their products across multiple platforms while closed source application try to stick user on a single platform. The level of integration and trust between enterprise solutions and open source products grows every year. Future innovations and successful businesses start today with help of open source.
\end{abstract}

It is obvious that for enterprises the future is built on open source. In the last 10 years any innovation adopted in the enterprises has had its roots in open source. Cloud computing, containers, databases, new languages (node JS and GO)~--- they all start in open source. And enterprises starting to listen, they starting to embrace open source. Red Hat claims that 90\% of fortune 500 are using Red Hat products and services. That was unheard 25 years ago.

Let’s look at 3 major trends that shape the changes between\linebreak enterprise and open source in the last 2 decades.

\subsection*{Enterprise compatibility}

What was enterprise compatibility 25 years ago? 25 years is important because at this time enterprises started adopting open source.

OpenSource products started finding their users because they solved real problems. At the same time closed source companies tried to block open source solutions trying to communicate the main objection: open source is risky, and it is not enterprise ready.

There are many amazing quotes from this time about open source. Bill Gates from Microsoft said: <<Open Source is good for hobbies and enthusiasts\ldots>>. It was also famously said that the open source is like a free puppy, you’ve got this puppy home, you’ve got to feed it, you’ve got to train it and then you’ve got to clean up after it. But, guess what, we took that puppy home, and that puppy grew up to be a wolf, that’s got teeth and claws, and it’s got sights sit firmly on enterprise software. And it’s not going to stop.

25 years later open source is enterprise compatible. Everything has changed and enterprises now have choice that they never had before. Now the companies like NGINX propose the products available with open source, package, fully backed up with professional services, training, tested and supported. There are more things available 
\begin{itemize}
\item Army of developers and community that grew up with open source
\item open source is now proven, ready and nothing is going to stop it
\item open source now is less risky than closed source (some examples with NGINX and other software)
\end{itemize}

\subsection*{Product and innovation cycle}

Closed source is still stuck in 3 years product cycle (example~--- MS Office~--- known as a train that always arrives every three years). It is great for license renewals, its good for maximizing amount of money you can get from your customers, but lousy for delivering innovation.
 
Open source community believes that 3 years are too long to wait for innovation. Innovation is not happening in 3 year cycles, it is happening now. Open source approach is fundamentally different then the closed source. It encapsulates instant feedback from multiple sources~---\linebreak amateurs, professionals, students, engineers, architects, individual development companies, startups, midsize companies, enterprises, largest internet giants all over the world and delivers the changes very quickly. New ideas become materialized to the new projects, they are quickly released into the staging and stable branches at the speed enterprises are asking about it.

\subsubsection*{Platforms compatibility}

New technologies are getting more complex and the enterprises looking for the platforms to simplify this and to reduce risk. They see it like a set of the best practices and patterns and software that works better together. And the challenge with the closed source platforms is that you are always locked in. Closed source has to protect their intellectual property, their own developers, their own engineers being tainted by other peoples IP. Fundamentally they try to lock their customers to their platform (example iTunes that don’t support Android users). This kind of lock in creates hostages, not happy\linebreak customers.
 
Open source suggests fundamentally different approach over the platforms~--- it integrates with various platforms and this is part of DNA of OpenSource solutions. For example NGINX is supported on AWS, Azure, GCP, kubernetes; on premise, private cloud, containers, hybrid cloud, any kind of cloud that you want. Platform should free you, but not lock you in. 
Open source is a better way to deliver the software.

Now let’s see some examples how open source products help driving the business of the large corporations. 

I have experience working in well known producer of the network equipment from California Netgear Inc. This company is known more than 20 years for their famous network devices: routers, wi-fi access points, bridges and other equipment. When we take Netgear router and look to the basic firmware components of this device we can see that it uses linux kernel, major libraries: curl, openssl, gsoap. And there are UI modules Node JS, Angular JS, Apache and other components to organize effective user interface. Server side components include Spring framework, Tomcat, NGINX, MySQL, Redis and many more.

Another company that we can look at is ARLO Inc~--- the leading expert in cloud video systems. ARLO is located in USA, California and having more than 50\% of US market. Their solutions – cloud video systems embed wide range of open source products such as Busybox, uClibc, GCC tools, Open SSL \& TLS, libcurl, libjson and many others. Together these tools help defining the shape of the final product and create that amazing user experience that makes those enterprise\linebreak successful.

Both ARLO and Netgear are examples of how successful business can leverage its own enterprise products by using open source\linebreak technologies. The very possibility of using so many open source projects in an enterprise solution is a great indicator of maturity of open source.

If 20 years ago the company had to build the most part of its enterprise value from scratch, now the most efforts are put into the building the product from various blocks provided by open source\linebreak community.

Today we can say for sure that the future of software development is with open source.
 

\end{document}
