\documentclass[10pt, a5paper]{article}
\usepackage{pdfpages}
\usepackage{parallel}
\usepackage[T2A]{fontenc}
\usepackage{ucs}
\usepackage[utf8x]{inputenc}
\usepackage[polish,english,russian]{babel}
\usepackage{hyperref}
\usepackage{rotating}
\usepackage[inner=2cm,top=1.8cm,outer=2cm,bottom=2.3cm,nohead]{geometry}
\usepackage{listings}
\usepackage{graphicx}
\usepackage{wrapfig}
\usepackage{longtable}
\usepackage{indentfirst}
\usepackage{array}
\newcolumntype{P}[1]{>{\raggedright\arraybackslash}p{#1}}
\frenchspacing
\usepackage{fixltx2e} %text sub- and superscripts
\usepackage{icomma} % коскі ў матэматычным рэжыме
\PreloadUnicodePage{4}

\newcommand{\longpage}{\enlargethispage{\baselineskip}}
\newcommand{\shortpage}{\enlargethispage{-\baselineskip}}

\def\switchlang#1{\expandafter\csname switchlang#1\endcsname}
\def\switchlangbe{
\let\saverefname=\refname%
\def\refname{Літаратура}%
\def\figurename{Іл.}%
}
\def\switchlangen{
\let\saverefname=\refname%
\def\refname{References}%
\def\figurename{Fig.}%
}
\def\switchlangru{
\let\saverefname=\refname%
\let\savefigurename=\figurename%
\def\refname{Литература}%
\def\figurename{Рис.}%
}

\hyphenation{admi-ni-stra-tive}
\hyphenation{ex-pe-ri-ence}
\hyphenation{fle-xi-bi-li-ty}
\hyphenation{Py-thon}
\hyphenation{ma-the-ma-ti-cal}
\hyphenation{re-ported}
\hyphenation{imp-le-menta-tions}
\hyphenation{pro-vides}
\hyphenation{en-gi-neering}
\hyphenation{com-pa-ti-bi-li-ty}
\hyphenation{im-pos-sible}
\hyphenation{desk-top}
\hyphenation{elec-tro-nic}
\hyphenation{com-pa-ny}
\hyphenation{de-ve-lop-ment}
\hyphenation{de-ve-loping}
\hyphenation{de-ve-lop}
\hyphenation{da-ta-ba-se}
\hyphenation{plat-forms}
\hyphenation{or-ga-ni-za-tion}
\hyphenation{pro-gramming}
\hyphenation{in-stru-ments}
\hyphenation{Li-nux}
\hyphenation{sour-ce}
\hyphenation{en-vi-ron-ment}
\hyphenation{Te-le-pathy}
\hyphenation{Li-nux-ov-ka}
\hyphenation{Open-BSD}
\hyphenation{Free-BSD}
\hyphenation{men-ti-on-ed}
\hyphenation{app-li-ca-tion}

\def\progref!#1!{\texttt{#1}}
\renewcommand{\arraystretch}{2} %Іначай формулы ў матрыцы зліпаюцца з лініямі
\usepackage{array}

\def\interview #1 (#2), #3, #4, #5\par{

\section[#1, #3, #4]{#1 -- #3, #4}
\def\qname{LVEE}
\def\aname{#1}
\def\q ##1\par{{\noindent \bf \qname: ##1 }\par}
\def\a{{\noindent \bf \aname: } \def\qname{L}\def\aname{#2}}
}

\def\interview* #1 (#2), #3, #4, #5\par{

\section*{#1\\{\small\rm #3, #4. #5}}

\def\qname{LVEE}
\def\aname{#1}
\def\q ##1\par{{\noindent \bf \qname: ##1 }\par}
\def\a{{\noindent \bf \aname: } \def\qname{L}\def\aname{#2}}
}

\begin{document}
\title{Голос спонсора: World of Tanks team}
%\author{}
\date{}
\maketitle

\subsection*{О проекте}

World of Tanks (Мир танков) "--- первый ММО проект ААА класса, созданный
белоруской командой. Разработчиками игра позиционируется как MMO"=экшн с
элементами ролевой игры, шутера и стратегии. Концепция <<World of Tanks>>
базируется на массовых командных танковых сражениях в режиме PvP. Онлайн
релиз русской версии игры состоялся 12 августа 2010 года, в марте 2011 года
состоялся <<китайский>> релиз, а в апреле 2011 года проект успешно вышел на
территории США и Европы.

Успех проекта World of Tanks можно оценить по целому ряду показателей:
количество активных игроков более 2 миллионов, рекордная цифра одновременной
игры "--- более 150.000 игроков на российском игровом кластере! В книгу
рекордов Гиннеса мы вошли 23 января 2011 года с показателем 91 311 игроков
он"=лайн. 

Дважды подряд в 2010 и 2011 году на Конференции Разработчиков Компьютерных
Игр (Москва) наш проект был признан Лучшей клиентской он"=лайн игрой (КРИ
2010) и Лучшей игрой (КРИ 2011).

В 2010 году по результатам международной выставки E3 (Лос"=Анжелес)
крупнейший ММО"=портал Massivly назвал наш проект New Concept 2010. В
настоящий момент определяются победители E3 2011. Мы сможем рассказать о
наших успехах уже при личной встрече на конференции LVEE 2011.

\subsection*{Подробности о проекте}

Игровые кластеры проекта находятся в дата"=центрах России, Германии, США и
Китая. Общее количество серверов в настоящий момент порядка 500, к концу
года мы планируем удвоить их количество.

Как игровые, так и прочие инфраструктурные сервера функционируют на базе
операционной системы CentOS.

Осенью 2010 года пики онлайна на российском игровом кластере достигали 30
тыс. В настоящий момент за счет высокотехнологичных решений наших
специалистов (включающих в себя оптимизацию сетевой инфраструктуры,
оптимизацию работы с базами данных, оптимизацию нашего серверного ПО) пик
он-лайна вырос до 150 тыс.

Высокая производительность достигается в т.~ч. за счет использования
free and open source software  как технологической основы для работы
серверной части самой масштабной ММО"=игры, а именно: CentOS,
MySQL, nginx, zabbix, nagios, cacti, python, django и т.~д.

\subsection*{О нас}

Над созданием проекта работает СООО <<Гейм Стрим>> "--- основной центр разработки
компании \url{Wargaming.net}. История компании "--- это 12"=летний опыт создания игр,
более 15 выпущенных проектов, среди которых <<Операция Багратион>>, а так же
<<Order of War>>, изданная Square Enix. В 2007 году мы объединились с минской
студией Arise. В 2010 из разработчика превратились в издателя "--- мы сами
осуществляем оперирование проекта World of Tanks. 

Наши награды на Конференции Разработчиков Компьютерных Игр (Москва): лучшая стратегическая игра КРИ"=2008 (проект <<Операция Багратион>>),
приз от прессы КРИ"=2009, лучшая компания"=разработчик КРИ"=2009 и КРИ"=2010, приз от индустрии КРИ 2011, приз зрительских симпатий КРИ 2011.

Сейчас в студии работает более 200 человек. Мы готовимся к запуску в
производство новых проектов и будем рады знакомству с талантливыми
специалистами. 

Нашим будущим сотрудникам мы предлагаем уникальную возможность работать над
онлайн"=проектами ААА класса, с применением передовых технологических
решений; реализовать себя, работая над сложными задачами; учиться у ведущих
специалистов отрасли. Со своей стороны мы создаём для этого все условия:
комфортный офис, современное техническое обеспечение рабочих мест, все
социальные гарантии, высокие белые зарплаты, обучение и т.~д. А ещё у нас
работают замечательные люди.

Наш контактный e-mail:  \url{rabota@wargaming.net}
Присоединяйтесь к команде World of Tanks!

\end{document}


