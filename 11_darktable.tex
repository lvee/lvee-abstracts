\documentclass[10pt, a5paper]{article}
\usepackage{ucs}
\usepackage[utf8]{inputenc}
\usepackage[T2A]{fontenc}
\usepackage[english, russian]{babel}
\usepackage{hyperref}
\usepackage{geometry}
\frenchspacing
\begin{document}
\title{Darktable, приложение для каталогизации и обработки RAW-файлов}
\author{Константин Шевцов\footnote{Новополоцк/Минск, Беларусь, \url{kanstantsin.sha@gmail.com}}, Александр Рабцевич\footnote{Минск, Беларусь, \url{alexander.v.rabtchevich@gmx.net}}}
\date{}

\def\progref!#1!{\texttt{#1}}

\maketitle

\begin{abstract}
Darktable is an open source application devoted to processing of RAW files. The program can manage collections of RAWs with rating, color labels and custom tags. A rich set of build-in filters (some of them are unique), used at processing, store their settings in a form of a history stack, which is saved alongside original RAW, providing original RAW untouched. Due to rawspeed RAW importing library, multi-threading and permanent optimizations, the program is fast and responsive.
\end{abstract}

\section*{История}
В докладе описывается история появления и развития проекта, возможности программы на примере работы с RAW файлами, а также планы разработчиков по развитию программы.

До появления darktable в ОС Linux отсутствовал открытый инструмент, сочетающий возможности каталогизации коллекции RAW файлов с их неразрушающей обработкой. UFRaw и  Rawtherapee не обладают достаточными возможностями для профессионального фотографа.

Восполнить пробел решил Йоханес Ханика (Johannes Hanika), который зарегистрировал проект darktable на SourceForge в феврале 2009 года. Вскоре к нему присоединились 3 разработчика: Хенрик Андерсон (Henrik Andersson), Паскаль де Брайн (Pascal de Bruijn), Александр Прокудин. В настоящее время у проекта около 20 контрибьюторов. 

\section*{Описание}
Спиосок основных возможностей программы при работе с коллекцией включает:
\begin{itemize}
\item импорт фотографий в коллекцию из папки, импорт отдельных фотографий. Фотографии физически не перемещаются.
\item импорт из фотоаппарата посредством gPhoto2
\item хранение данных о коллекции в собственной базе данных
присвоение фотографиям рейтинга (stars), система цветовых меток, пользовательские теги (метки).
\item поиск по произвольной комбинации: съемка, камера, метка, дата, наличие изменений после экспорта.
\item копирование истории обработки между фотографиями
\item экспорт в jpeg, tiff (8 и 16 бит) 
\item экспорт в picasa, flickr, email
Основные возможности программы по обработке RAW (версия из репозитария git):
\item внутреннее представление данных RGB float (32 бит/канал) или LCh (в зависимости от модуля)
\item быстрый предосмотр с масштабированием вплоть до 100\%, при предосмотре обрабатывается только часть изображения, показываемая в окне с кешированием операций
\item многопоточность
\item использование для ряда операций OpenCL при наличии драйвера и подходящего графического ускорителя
\item все изменения хранятся в виде стека истории с возможностью отката до произвольной точки, а также копирования истории изменений между фотографиями или сохранения ее в виде стиля для последующего применения
\item стек истории сохраняется в виде *.xmp файла вместе с оригинальным файлом RAW
\item модульная архитектура (плагины)
\item возможность создания и ручного/автоматического использования предустановок для всех модулей
\item наличие режимов смешивания для большинства модулей
\item алгоритмы дебайеризации ppg и AMaZE 
\item поканальный баланс белого через множители либо через  - сдвиг цветовой температуры и множителя зеленого канала
\item редактируемая тональная кривая камеры с возможностью выбора из набора готовых кривых, аналогичных используемым производителями цифровых фотоаппаратов
\item восстановление пересветов при отрицательной экспокоррекции
\item профили камеры: стандартные (Adobe), улучшенные, либо пользовательские
\item трансформации: кадрирование с выбираемым соотношением сторон, поворот, перспектива
\item редактируемая тональная кривая (L канал в LCh пространстве
\item возможность работать с тональной кривой в виде последовательности зон Адамса
\item мощнейший инструмент --- эквалайзер, позволяющий регулировать локальный контраст (L и С) в зависимости от пространственной частоты (аналога радиуса в USM) с помощью избегающих краев вейвлетов
\item подавление шума
\item исправление геометрических искажений оптики с помощью библиотеки lensfun
\item регулирование яркости/насыщенности или сдвига тонов в плагине цветовые зоны
\item микшер каналов
\item обесцвечивание с регулируемым цветовым фильтром
\item градиентный фильтр (GND) с возможностью поворота и радиального смещения
\item эффект виньетирования с регулируемыми на холсте радиусом, формой (окружность/эллипс)
\item вельвия
\item смягчающий фильтр
\item и другие фильтры
\end{itemize}

\section*{Будущее}
В настоящее время разработчики готовят программу к выпуску версии 0.9. 

В ближайших планах разработчиков --- добавление функционала масок (Henrik Andersson) и переработка интерфейса с централизованной обработкой горячих клавиш (GSoC студент Robert Bieber). В состоянии обсуждения — возможность не  однократного использования плагинов. В более отдаленных планах — использование GEGL внутри darktable с возможностью передачи данных в GIMP и обратно для более сложного редактирования; однако этот функционал может появиться не раньше выхода GIMP 3.0.

Проект активно развивается и радует постоянное растущее сообщество пользователей новыми выпусками с периодичностью примерно раз в три месяца. Можно ответственно заявить, что текущая версия пригодна для использования профессиональным фотографом.

\begin{thebibliography}{9}
\bibitem{pDynamo} Официальный сайт проекта --- \url{http://darktable.sf.net}
\end{thebibliography}
\end{document}
