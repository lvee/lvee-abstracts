\documentclass[10pt, a5paper]{article}
\usepackage{pdfpages}
\usepackage{parallel}
\usepackage[T2A]{fontenc}
\usepackage{ucs}
\usepackage[utf8x]{inputenc}
\usepackage[polish,english,russian]{babel}
\usepackage{hyperref}
\usepackage{rotating}
\usepackage[inner=2cm,top=1.8cm,outer=2cm,bottom=2.3cm,nohead]{geometry}
\usepackage{listings}
\usepackage{graphicx}
\usepackage{wrapfig}
\usepackage{longtable}
\usepackage{indentfirst}
\usepackage{array}
\newcolumntype{P}[1]{>{\raggedright\arraybackslash}p{#1}}
\frenchspacing
\usepackage{fixltx2e} %text sub- and superscripts
\usepackage{icomma} % коскі ў матэматычным рэжыме
\PreloadUnicodePage{4}

\newcommand{\longpage}{\enlargethispage{\baselineskip}}
\newcommand{\shortpage}{\enlargethispage{-\baselineskip}}

\def\switchlang#1{\expandafter\csname switchlang#1\endcsname}
\def\switchlangbe{
\let\saverefname=\refname%
\def\refname{Літаратура}%
\def\figurename{Іл.}%
}
\def\switchlangen{
\let\saverefname=\refname%
\def\refname{References}%
\def\figurename{Fig.}%
}
\def\switchlangru{
\let\saverefname=\refname%
\let\savefigurename=\figurename%
\def\refname{Литература}%
\def\figurename{Рис.}%
}

\hyphenation{admi-ni-stra-tive}
\hyphenation{ex-pe-ri-ence}
\hyphenation{fle-xi-bi-li-ty}
\hyphenation{Py-thon}
\hyphenation{ma-the-ma-ti-cal}
\hyphenation{re-ported}
\hyphenation{imp-le-menta-tions}
\hyphenation{pro-vides}
\hyphenation{en-gi-neering}
\hyphenation{com-pa-ti-bi-li-ty}
\hyphenation{im-pos-sible}
\hyphenation{desk-top}
\hyphenation{elec-tro-nic}
\hyphenation{com-pa-ny}
\hyphenation{de-ve-lop-ment}
\hyphenation{de-ve-loping}
\hyphenation{de-ve-lop}
\hyphenation{da-ta-ba-se}
\hyphenation{plat-forms}
\hyphenation{or-ga-ni-za-tion}
\hyphenation{pro-gramming}
\hyphenation{in-stru-ments}
\hyphenation{Li-nux}
\hyphenation{sour-ce}
\hyphenation{en-vi-ron-ment}
\hyphenation{Te-le-pathy}
\hyphenation{Li-nux-ov-ka}
\hyphenation{Open-BSD}
\hyphenation{Free-BSD}
\hyphenation{men-ti-on-ed}
\hyphenation{app-li-ca-tion}

\def\progref!#1!{\texttt{#1}}
\renewcommand{\arraystretch}{2} %Іначай формулы ў матрыцы зліпаюцца з лініямі
\usepackage{array}

\def\interview #1 (#2), #3, #4, #5\par{

\section[#1, #3, #4]{#1 -- #3, #4}
\def\qname{LVEE}
\def\aname{#1}
\def\q ##1\par{{\noindent \bf \qname: ##1 }\par}
\def\a{{\noindent \bf \aname: } \def\qname{L}\def\aname{#2}}
}

\def\interview* #1 (#2), #3, #4, #5\par{

\section*{#1\\{\small\rm #3, #4. #5}}

\def\qname{LVEE}
\def\aname{#1}
\def\q ##1\par{{\noindent \bf \qname: ##1 }\par}
\def\a{{\noindent \bf \aname: } \def\qname{L}\def\aname{#2}}
}


\begin{document}

\title{pkgsrc4unix}%\footnote{Текст данных и последующих тезисов, кроме специально оговоренных случаев, доступен под лицензией Creative Commons Attribution-ShareAlike 3.0}

\author{Алексей Чеусов\footnote{Минск, Беларусь;\url{vle@gmx.net}}}
\maketitle

\begin{abstract}
pkgsrc is a cross"=platform packaging system. Besides NetBSD where it was born in 1997, pkgsrc supports Linux, Solaris, all BSDs, Minix, QNX and a lot of others. In total 15 platforms are supported. Pkgsrc has a number of advantages over existing packaging systems: easy packaging, support for many compilers, efficient binary package management and source"=based upgrades but the killer"=feature is a support for diverse operating systems. Adapting pkgsrc for Linux as an additional yum/zypper/apt repository of binary packages is considered here.
\end{abstract}

pkgsrc --- кросс"=платформная пакетная система, созданная в 1997 году как ответвление FreeBSD ports. Несмотря на то, что разрабатывается она в основном разработчиками NetBSD, pkgsrc поддерживает практически все живые операционные системы, существующие на данный момент. Среди них Linux, Solaris, все варианты BSD, QNX, Haiku, AIX, HP"=UX и многие другие. pkgsrc является по сути единственной пакетной системой, поддерживающей на неплохом уровне такой широкий набор операционных систем, что дает пользователю, будь то крупная компания или один человек, возможность на любой платформе использовать один и тот же знакомый набор пакетов для работы. Это очень удобно особенно в тех случаях, когда пользователю необходимо достаточно редкое программное обеспечение, отсутствующее по той или иной причине в типичных, в последнее время Десктоп"=ориентированных, Линукс"=истрибутивах.

В подобных случаях есть два принципиально разных пути решения этой проблемы. Путь первый --- если пакета нет в системе, значит он нам не нужен (однако это неподходящий метод). Второй путь --- самостоятельное пакетирование. Опытный специалист в состоянии самостоятельно запакетировать необходимое ПО под необходимый дистрибутив или систему. В этом случае требуется изучить правила пакетирования для конкретной системы и строго следовать им. На мой взгляд, с этим связан один существенный недостаток. Эти правила часто весьма специфичны, а полученные от их изучения знания лишь отчасти могут быть применены при работе с другими системами. Так, например, диалекты rpm spec и доступный набор макроопределений довольно сильно отличаются в различных дистрибутивах Линукс и практически не используются в других ОС, а правила пакетирования для OpenBSD, FreeBSD, Gentoo или Arch linux вообще мало похожи на остальные системы пакетирования. Другой недостаток, еще более серьезный --- это ориентированность разработанного пакета на одну определенную систему, а чаще всего, дистрибутив. Если выбор ОС и дистрибутива уже сделан, и сделан на всю жизнь, это не представляет большой проблемы, но едва ли найдется достаточно много людей, осознанно использующих одну единственную систему более, скажем, пяти лет. Обычно система подбирается под конкретную решаемую задачу, а не наоборот, поэтому часто складывается ситуация, когда система уже определена, а потому усилия потраченные на самостоятельное пакетирование необходимого ПО под другую систему могут оказаться напрасными. Частично эту проблему решает система OpenBuildService (OBS), разрабатываемая сообществом OpenSuSE. Она позволяет, имея в распоряжении единый язык для описания сценариев пакетирования (rpm spec) создать пакет для различных дистрибутивов Линукс. Но\ldots{} только Линукса! Инфраструктура инфтраструктурой, но для успешного пакетирования тысяч и тысяч пакетов под различные операционные системы и среды необходимо сообщество, в котором уже сложилась традиция разработки переносимого ПО. Едва ли Линукс"=сообщество в целом и сообщество OpenSuSE/OBS в частности на это способно. Скорее наоборот, в последнее время все чаще в мире Linux раздаются призывы отказаться от POSIX и разработывать ПО исключительно под единственно"=верную систему.

К счастью, не все разработчики забыли о пользе переносимости. Одно из сообществ таких разработчиков --- разработчики pkgsrc, где, как уже было сказано, пакеты изначально разрабатываются с расчетом на то, чтобы обеспечить максимально возможное количество поддержиаемых систем. pkgsrc обладает всем необходимым для серьезной работы: удобным механизмом сборки пакетов, включая разработку заплаток для исправления ошибок, поддержкой работы с бинарными пакетами, системой массовой сборки пакетов (bulk builders) и прочим. По ряду параметров pkgsrc превосходит форматы rpm и deb и соответствующие программы (dpkg, yum, apt, zypper и т.д.), наиболее широко используемые в Линукс и считающиеся стандартом de facto. 

Из коробки pkgsrc предоставляет весь необходимый набор инструментов как для разработки и сборки пакетов, так и для управления установленными пакетами в системе. С точки зрения системного администратора pkgsrc в системе будет иметь отдельную базу данных установленных пакетов (pkgdb) параллельно с основной использующейся в системе, например, rpmdb и, соответственно, отдельный набор пакетов и утилит для работы с ними. Будучи небольшой проблемой для профессионала, это может оказаться неудобным для массового пользователя. Один из способов решения данной проблемы --- преобразование <<родных>> пакетов pkgsrc в <<родные>> пакеты системы, формирование на их основе репозитория дополнительного ПО и установка/удаление/поиск/\ldots{} по общим для системы правилам, будь то yum, zypper или aptitude.

Цель моего собственного проекта pkgsrc4unix --- создание регулярно обновляемого полноценного yum"=репозитория пакетов для RHEL"=6 (для начала), создаваемого на основе pkgsrc. Его особенности: 
\begin{itemize}
\item для сборки ПО используется инфраструктура pkgsrc, а не rpm spec; 
\item для массовой сборки пакетов также используются средства pkgsrc; 
\item пакеты в формате pkgsrc преобразуются в .rpm для RHEL"=6 и далее из rpm"=пакетов создается yum"=репозиторий; 
\item для исключения конфликтов на уровне файлов с пакетами RHEL, repoforge, epel и др. все ПО устанавливается в \linebreak /opt/pkgsrc4unix, для исключения конфликтов на уровне названий пакетов каждый пакет имеет префикс <<nb->>; 
\item для управления конфигурационными файлами используется подход pkgsrc; 
\item для старта демонов используется механизм NetBSD/pkgsrc, при этом все необходимое для этого ПО реализуется в виде отдельного rpm"=пакета.
\end{itemize}



\end{document}




