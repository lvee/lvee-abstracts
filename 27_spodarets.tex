\documentclass[10pt, a5paper]{article}
\usepackage[T2A]{fontenc}
\usepackage{ucs}
\usepackage[utf8x]{inputenc}
\usepackage[polish,english,russian]{babel}
\usepackage{hyperref}
\usepackage[inner=2cm,top=1.8cm,outer=2cm,bottom=2.3cm,nohead]{geometry}
\usepackage{listings}
\usepackage{graphicx}
\usepackage{wrapfig}
\usepackage{longtable}
\usepackage{indentfirst}
\frenchspacing
\usepackage{fixltx2e} %text sub- and superscripts
\usepackage{icomma} % коскі ў матэматычным рэжыме
\PreloadUnicodePage{4}

\newcommand{\longpage}{\enlargethispage{\baselineskip}}
\newcommand{\shortpage}{\enlargethispage{-\baselineskip}}

\def\switchlang#1{\expandafter\csname switchlang#1\endcsname}
\def\switchlangbe{
\let\saverefname=\refname%
\def\refname{Літаратура}%
\def\figurename{Іл.}%
}
\def\switchlangen{
\let\saverefname=\refname%
\def\refname{References}%
\def\figurename{Fig.}%
}
\def\switchlangru{
\let\saverefname=\refname%
\let\savefigurename=\figurename%
\def\refname{Литература}%
\def\figurename{Рис.}%
}

\hyphenation{admi-ni-stra-tive}
\hyphenation{ex-pe-ri-ence}
\hyphenation{fle-xi-bi-li-ty}
\hyphenation{Py-thon}
\hyphenation{ma-the-ma-ti-cal}
\hyphenation{re-ported}
\hyphenation{imp-le-menta-tions}
\hyphenation{pro-vides}
\hyphenation{en-gi-neering}
\hyphenation{com-pa-ti-bi-li-ty}
\hyphenation{im-pos-sible}
\hyphenation{desk-top}
\hyphenation{elec-tro-nic}
\hyphenation{com-pa-ny}
\hyphenation{de-ve-lop-ment}
\hyphenation{de-ve-loping}
\hyphenation{de-ve-lop}
\hyphenation{da-ta-ba-se}
\hyphenation{plat-forms}
\hyphenation{or-ga-ni-za-tion}
\hyphenation{pro-gramming}
\hyphenation{in-stru-ments}
\hyphenation{Li-nux}
\hyphenation{en-vi-ron-ment}
\hyphenation{Te-le-pathy}
\hyphenation{Li-nux-ov-ka}

\def\progref!#1!{\texttt{#1}}
\renewcommand{\arraystretch}{2} %Іначай формулы ў матрыцы зліпаюцца з лініямі
\usepackage{array}

\def\interview #1 (#2), #3, #4, #5\par{

\section[#1, #3, #4]{#1, #5}
\def\qname{LVEE}
\def\aname{#1}
\def\q ##1\par{{\noindent \bf \qname: ##1 }\par}
\def\a{{\noindent \bf \aname: } \def\qname{L}\def\aname{#2}}
}

\begin{document}
\title{GRID-технологии в исследованиях дымовой плазмы}
\author{Сподарец Д.В., Драган Г.С.\footnote{Одесса, Украина, Одесский национальный университет имени И.И.Мечникова, \url{m31@root-ua.com}}}
\date{}
\maketitle
\begin{abstract}
The paper describes the use of GRID systems and computer clusters for smoky plasma research. Offers insight into the experience of expanding the functionality of TERM software and hardware complex for GRID systems.
\end{abstract}
В экспериментальных исследованиях все чаще используют  вычислительные кластеры и GRID-системы. Связано это с тем, что количество информации, полученное в следствии проведения эксперимента с использованием современного измерительного оборудования, достаточно велико. Немаловажным фактом, который играет значительную роль при выборе супперкомпьютеров для автоматизации исследований и обработки результатов, является уменьшение ошибки измерений, а также возможность использовать новые технологии диагностики, например, позволяющие наблюдать процессы в реальном времени.

Основной объект наших исследований --- дымовая плазма. Она является разновидностью низкотемпературной термической плазмы с конденсированной дисперсной фазой, образуется в продуктах сгорания натуральных и синтетических видов топлива и состоит из газовой фазы с легкоионизируемой примесью атомов щелочного металла и конденсированной в виде частиц оксидов металлов и сажи \cite{spod1}.

Для проведения диагностики таких параметров дымовой плазмы, как температура газовой и конденсированной фаз, концентрации электронов и атомов, нами был разработан аппаратно"=программный комплекс TERM \cite{spod2}. В его основе лежат оптические методы диагностики плазмы, а программная часть построена на свободном программном обеспечении. 

Среди новых возможностей комплекса, которые уже реализованы или находятся на стадии реализации, можно отметить следующие: дополнение его собственным вычислительным кластером, адаптация для работы в GRID-системах, разработка механизмов проведения виртуальных экспериментов \cite{spod3}.

Аппаратно-программный комплекс TERM работает следующим образом:
\begin{enumerate}
	\item Экспериментально определяется температура газовой и конденсированной фаз, а также концентрация электронов и атомов в плазме.
	\item На основе теории дымовой плазмы и измеренных экспериментально её параметров, проводиться расчёт заряда частиц, а также определяется функция распределения частиц по заряду.
	\item Полученные данные зарядов используются для определения пространственного концентрационного неравновесия и сил \linebreak диффузионно-дрейфового давления свободных носителей заряда на частицы. Основой для данных расчётов служит теория обобщённого потенциала.
	\item Значения сил диффузионно-дрейфового давления учитываются при  расчёте парного взаимодействия частиц, а также в уравнениях молекулярной динамики для дымовой плазмы.
	\item Результатом работы комплекса является моделирование 3D структур заряженных конденсированных частичек в плазме. Этот этап проводится с использованием GRID Украины.
\end{enumerate}
Базовой ОС комплекса является ALT Linux. Для захвата и обработки видео-потоков применяется библиотека компьютерного зрения OpenCV. В качестве инструментария, при проведении расчётов, задействованы Octave и Mathematica. Разработан ряд собственных специфических алгоритмов, например, для проведения анализа спектров излучения плазмы. 

При совмещении реального физического эксперимента и виртуального моделирования на базе GRID Украины, стало возможным более глубокое изучение влияния объёмных и поверхностных процессов, происходящих в области пространственного заряда конденсированной частицы, на кинетику и динамику дымовой плазмы.
\begin{thebibliography}{9}
	\bibitem{spod1}Dragan G.S., Malgota A.A. et al. // Proc. Sc. and Tech. Meet., Alma-Ata, USSR, October 25-31, 1982. – Inst. High Temp., Acad. Sci. of the USSR, 1984. – P. 191 – 192. 
	\bibitem{spod2} http://term.m31.org.ua
	\bibitem{spod3} Spodarets D.V., Dragan G.S. Method of Experimental Research of Long-Range Interactions in Smoky Plasmas // Book of Abstracts of the ICPDP 2011, Garmasch-Partenkirchen. Germany.: 2011. P. 74. 
\end{thebibliography}
\end{document}


