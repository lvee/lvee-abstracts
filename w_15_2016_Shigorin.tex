\documentclass[10pt, a5paper]{article}
\usepackage{pdfpages}
\usepackage{parallel}
\usepackage[T2A]{fontenc}
\usepackage{ucs}
\usepackage[utf8x]{inputenc}
\usepackage[polish,english,russian]{babel}
\usepackage{hyperref}
\usepackage{rotating}
\usepackage[inner=2cm,top=1.8cm,outer=2cm,bottom=2.3cm,nohead]{geometry}
\usepackage{listings}
\usepackage{graphicx}
\usepackage{wrapfig}
\usepackage{longtable}
\usepackage{indentfirst}
\usepackage{array}
\newcolumntype{P}[1]{>{\raggedright\arraybackslash}p{#1}}
\frenchspacing
\usepackage{fixltx2e} %text sub- and superscripts
\usepackage{icomma} % коскі ў матэматычным рэжыме
\PreloadUnicodePage{4}

\newcommand{\longpage}{\enlargethispage{\baselineskip}}
\newcommand{\shortpage}{\enlargethispage{-\baselineskip}}

\def\switchlang#1{\expandafter\csname switchlang#1\endcsname}
\def\switchlangbe{
\let\saverefname=\refname%
\def\refname{Літаратура}%
\def\figurename{Іл.}%
}
\def\switchlangen{
\let\saverefname=\refname%
\def\refname{References}%
\def\figurename{Fig.}%
}
\def\switchlangru{
\let\saverefname=\refname%
\let\savefigurename=\figurename%
\def\refname{Литература}%
\def\figurename{Рис.}%
}

\hyphenation{admi-ni-stra-tive}
\hyphenation{ex-pe-ri-ence}
\hyphenation{fle-xi-bi-li-ty}
\hyphenation{Py-thon}
\hyphenation{ma-the-ma-ti-cal}
\hyphenation{re-ported}
\hyphenation{imp-le-menta-tions}
\hyphenation{pro-vides}
\hyphenation{en-gi-neering}
\hyphenation{com-pa-ti-bi-li-ty}
\hyphenation{im-pos-sible}
\hyphenation{desk-top}
\hyphenation{elec-tro-nic}
\hyphenation{com-pa-ny}
\hyphenation{de-ve-lop-ment}
\hyphenation{de-ve-loping}
\hyphenation{de-ve-lop}
\hyphenation{da-ta-ba-se}
\hyphenation{plat-forms}
\hyphenation{or-ga-ni-za-tion}
\hyphenation{pro-gramming}
\hyphenation{in-stru-ments}
\hyphenation{Li-nux}
\hyphenation{sour-ce}
\hyphenation{en-vi-ron-ment}
\hyphenation{Te-le-pathy}
\hyphenation{Li-nux-ov-ka}
\hyphenation{Open-BSD}
\hyphenation{Free-BSD}
\hyphenation{men-ti-on-ed}
\hyphenation{app-li-ca-tion}

\def\progref!#1!{\texttt{#1}}
\renewcommand{\arraystretch}{2} %Іначай формулы ў матрыцы зліпаюцца з лініямі
\usepackage{array}

\def\interview #1 (#2), #3, #4, #5\par{

\section[#1, #3, #4]{#1 -- #3, #4}
\def\qname{LVEE}
\def\aname{#1}
\def\q ##1\par{{\noindent \bf \qname: ##1 }\par}
\def\a{{\noindent \bf \aname: } \def\qname{L}\def\aname{#2}}
}

\def\interview* #1 (#2), #3, #4, #5\par{

\section*{#1\\{\small\rm #3, #4. #5}}

\def\qname{LVEE}
\def\aname{#1}
\def\q ##1\par{{\noindent \bf \qname: ##1 }\par}
\def\a{{\noindent \bf \aname: } \def\qname{L}\def\aname{#2}}
}

\begin{document}
\title{Альт на <<Эльбрусе>>\footnote{\url{mike@altlinux.org}, \url{http://lvee.org/ru/abstracts/180}}}
\author{Михаил Шигорин, Москва, Россия}
\maketitle
\begin{abstract}
As soon as we've got a shell on Elbrus processor we wanted to port our RPM there; upon that, it was only natural to want hasher working too. The availability of a physical system didn't hurt at all.
\end{abstract}
Эльбрус "--- два семейства процессоров разработки российской компании МЦСТ: SPARC-совместимая ветка и оригинальная VLIW-архитектура. Речь пойдёт о второй. Особенностями платформы в настоящее время являются малодоступность (вследствие в т.ч. применения, например, в системах ПРО) и закрытость системного компилятора (вероятно, по тем же причинам). Используем рабочую станцию <<Эльбрус-401>>, которая автором доклада найдена вполне симпатичной на ощупь (подробнее в кулуарах). Работающая на ней хост-система "--- Linux (точнее, ОС <<Эльбрус>>, во многом близкая к Debian 5.0/7.0 и местами новее).

Я работаю в компании <<Базальт СПО>>, которая участвует в разработке репозитория ALT Linux Sisyphus. Как только у нас появился доступ на машину с процессором <<Эльбрус-4С>>, возникло вполне естественное желание портировать туда нашу пакетную базу. Первым этапом стало портирование пакетного менеджера (RPM версии ALT Linux, он же ALT-RPM). Когда заработал rpm, следующим этапом стал запуск hasher "--- инструмента, с помощью которого собираются пакеты Sisyphus (hasher спроектирован так, чтобы не допускать влияния собираемого пакета на хост-систему, а также взаимного влияния собирающихся пакетов).

Текущая работа опирается на труды многих других людей "---  начальное портирование RPM было выполнено glebfm@, процедуру бутстрапа альта ранее описал kas@ по мотивам ARM-порта, а код поддержки архитектуры мы получили от сотрудников МЦСТ.

На время написания тезисов доступна базовая сборочная среда ALT для сборки в автоматически создаваемом силами hasher чруте, за исключением архитектурнозависимых пакетов (binutils, glibc, компилятор), которые пока alien'изированы из предоставленных разработчиком системы deb-пакетов "---  примерно 230 исходных пакетов.

Основные пройденные стадии сборки:

\begin{enumerate}
  \item сборка/установка rpm вручную в хост-окружении;
  \item упаковывание всего, что попадает в hasher chroot;
  \item пересборка собранных пакетов уже в hasher.
\end{enumerate}

Производится итеративная пересборка с откручиванием гаек вроде "--- disable static "--- without-ssl и корректировка полученной начальной пакетной базы для возможности включения её в основной разработческий репозиторий ALT Linux Sisyphus.

В целом, работа позволила оценить достоинства и недостатки:

\begin{itemize}
  \item e2k как целевой платформы;
  \item ALT Linux как портабельного репозитория и набора инструментария;
  \item <<бутстрапа напролом>> и <<раннепакетного>>.
\end{itemize}

\begin{thebibliography}{99}
  \bibitem{Ghigorin1} \url{http://altlinux.org/bootstrap}
  \bibitem{Ghigorin2} \url{http://altlinux.org/ports}
  \bibitem{Ghigorin3} \url{http://altlinux.org/hasher}
  \bibitem{Ghigorin4} \url{http://sdelanounas.ru/blogs/71419/}
\end{thebibliography}

\end{document}
