\documentclass[10pt, a5paper]{article}
\usepackage[T2A]{fontenc}
\usepackage{ucs}
\usepackage[utf8x]{inputenc}
\usepackage[polish,english,russian]{babel}
\usepackage{hyperref}
\usepackage[inner=2cm,top=1.8cm,outer=2cm,bottom=2.3cm,nohead]{geometry}
\usepackage{listings}
\usepackage{graphicx}
\usepackage{wrapfig}
\usepackage{longtable}
\usepackage{indentfirst}
\frenchspacing
\usepackage{fixltx2e} %text sub- and superscripts
\usepackage{icomma} % коскі ў матэматычным рэжыме
\PreloadUnicodePage{4}

\newcommand{\longpage}{\enlargethispage{\baselineskip}}
\newcommand{\shortpage}{\enlargethispage{-\baselineskip}}

\def\switchlang#1{\expandafter\csname switchlang#1\endcsname}
\def\switchlangbe{
\let\saverefname=\refname%
\def\refname{Літаратура}%
\def\figurename{Іл.}%
}
\def\switchlangen{
\let\saverefname=\refname%
\def\refname{References}%
\def\figurename{Fig.}%
}
\def\switchlangru{
\let\saverefname=\refname%
\let\savefigurename=\figurename%
\def\refname{Литература}%
\def\figurename{Рис.}%
}

\hyphenation{admi-ni-stra-tive}
\hyphenation{ex-pe-ri-ence}
\hyphenation{fle-xi-bi-li-ty}
\hyphenation{Py-thon}
\hyphenation{ma-the-ma-ti-cal}
\hyphenation{re-ported}
\hyphenation{imp-le-menta-tions}
\hyphenation{pro-vides}
\hyphenation{en-gi-neering}
\hyphenation{com-pa-ti-bi-li-ty}
\hyphenation{im-pos-sible}
\hyphenation{desk-top}
\hyphenation{elec-tro-nic}
\hyphenation{com-pa-ny}
\hyphenation{de-ve-lop-ment}
\hyphenation{de-ve-loping}
\hyphenation{de-ve-lop}
\hyphenation{da-ta-ba-se}
\hyphenation{plat-forms}
\hyphenation{or-ga-ni-za-tion}
\hyphenation{pro-gramming}
\hyphenation{in-stru-ments}
\hyphenation{Li-nux}
\hyphenation{en-vi-ron-ment}
\hyphenation{Te-le-pathy}
\hyphenation{Li-nux-ov-ka}

\def\progref!#1!{\texttt{#1}}
\renewcommand{\arraystretch}{2} %Іначай формулы ў матрыцы зліпаюцца з лініямі
\usepackage{array}

\def\interview #1 (#2), #3, #4, #5\par{

\section[#1, #3, #4]{#1, #5}
\def\qname{LVEE}
\def\aname{#1}
\def\q ##1\par{{\noindent \bf \qname: ##1 }\par}
\def\a{{\noindent \bf \aname: } \def\qname{L}\def\aname{#2}}
}


\begin{document}

\switchlang{be}
\title{Экспресс-тест десктопных возможностей BSD дистрибутивов}%\footnote{Текст данных и последующих тезисов, кроме специально оговоренных случаев, доступен под лицензией Creative Commons Attribution-ShareAlike 3.0}

\author{Дмитрий Ванькевич\footnote{Львов, Украина}}
\maketitle

\begin{abstract}
Different flaws of BSD operating systems family are reviewed on the subject of their readiness as a desktop OS.
Criteria of desktop readiness covering the ease of configuration and maintenance, as far as needed desktop software are presented, and
the easy sequence of steps is proposed to make quick comparison covering large subset of these criteria.
\end{abstract}

К числу BSD-дистрибутивов, более или менее ориентированных на десктоп-системы, можно отнести следующие:

\begin{itemize}
  \item DragonFlyBSD
  \item FuguIta
  \item PC-BSD
  \item GhostBSD
  \item РУС-BSD
  \item DesktopBSD
  \item VirtualBSD 9.0
  \item Frenzy
  \item Jibbed 5.1
  \item TrueBSD
\end{itemize}

В списке сознательно не рассматривается FreeBSD, потому что изначально требует значительных усилий в настройке для применения в качестве десктопа.

Вопрос пригодности дистрибутива для десктоп-систем можно рассматривать исходя из его соответствия достаточно большому списку требований:

\begin{itemize}
  \item работа на персональных компьютерах, типичных для офисных рабочих
         мест
  \item лёгкость в использовании
  \item лёгкость в установке
  \item наличие качественной локализации
  \item тщательность подбора программ (одна задача --- одна программа)
  \item лёгкость в обновлении
  \item возможность установки дополнительных программ пользователями
  \item иметь стандартный и приятный внешний вид
  \item наличие необходимого для офисной работы инструмента: текстовой процессор, электронные таблицы, средства просмотра для разных файловых форматов и утилиты печати
  \item интеграция с Linux и Windows серверами
  \item возможность удалённого администрирования
  \item наличие стандартных программ для работы в Internet
  \item возможность сетевой установки на бездисковые рабочие станции
\end{itemize}

Критерии взяты из статьи \url{nklug.org.ua/lg/lg81/arndt.html}.

На практике была использована следующая последовательность действий для определения,
насколько дистрибутивов соответствует критериям пригодности для десктопа:

\begin{itemize}
  \item установка дистрибутива
  \item если дистрибутив ставится без значительных усилий и корректно установились драйвера для оборудования --- выход в интернет с выполнением обычных задач характерных для \linebreak пользователя.
  \item определение недостающих программ и их установка
  \item если установка и настройка необходимых программ требует <<титанических усилий>> --- переход к следующему дистрибутиву
\end{itemize}

Бета-версия доклада была прочитана на заседании \url{linux.lviv.ua/}. Автору  были даны ценные замечания по поводу презентации.

Выводы:

PC-BSD несмотря на некоторые недостатки и <<шероховатости>> --- самый дружелюбный дистрибутив. Подойдёт как начинающим пользователям, так и пользователям со стажем. 
В PC-BSD имеется графическая программа установки и удаления пакетов PBI. В то же время в ней есть и система портов (ports) и пакетов (packages) FreeBSD.

РУС-BSD --- Содержит все необходимые для повседневной работы программы. Но базируется на устаревшей версии FreeBSD.

VirtualBSD --- в ногу со временем (в свете всеобщей виртуализации и облачных технологий)



\end{document}




