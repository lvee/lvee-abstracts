\documentclass[10pt, a5paper]{article}
\usepackage[T2A]{fontenc}
\usepackage{ucs}
\usepackage[utf8x]{inputenc}
\usepackage[polish,english,russian]{babel}
\usepackage{hyperref}
\usepackage[inner=2cm,top=1.8cm,outer=2cm,bottom=2.3cm,nohead]{geometry}
\usepackage{listings}
\usepackage{graphicx}
\usepackage{wrapfig}
\usepackage{longtable}
\usepackage{indentfirst}
\frenchspacing
\usepackage{fixltx2e} %text sub- and superscripts
\usepackage{icomma} % коскі ў матэматычным рэжыме
\PreloadUnicodePage{4}

\newcommand{\longpage}{\enlargethispage{\baselineskip}}
\newcommand{\shortpage}{\enlargethispage{-\baselineskip}}

\def\switchlang#1{\expandafter\csname switchlang#1\endcsname}
\def\switchlangbe{
\let\saverefname=\refname%
\def\refname{Літаратура}%
\def\figurename{Іл.}%
}
\def\switchlangen{
\let\saverefname=\refname%
\def\refname{References}%
\def\figurename{Fig.}%
}
\def\switchlangru{
\let\saverefname=\refname%
\let\savefigurename=\figurename%
\def\refname{Литература}%
\def\figurename{Рис.}%
}

\hyphenation{admi-ni-stra-tive}
\hyphenation{ex-pe-ri-ence}
\hyphenation{fle-xi-bi-li-ty}
\hyphenation{Py-thon}
\hyphenation{ma-the-ma-ti-cal}
\hyphenation{re-ported}
\hyphenation{imp-le-menta-tions}
\hyphenation{pro-vides}
\hyphenation{en-gi-neering}
\hyphenation{com-pa-ti-bi-li-ty}
\hyphenation{im-pos-sible}
\hyphenation{desk-top}
\hyphenation{elec-tro-nic}
\hyphenation{com-pa-ny}
\hyphenation{de-ve-lop-ment}
\hyphenation{de-ve-loping}
\hyphenation{de-ve-lop}
\hyphenation{da-ta-ba-se}
\hyphenation{plat-forms}
\hyphenation{or-ga-ni-za-tion}
\hyphenation{pro-gramming}
\hyphenation{in-stru-ments}
\hyphenation{Li-nux}
\hyphenation{en-vi-ron-ment}
\hyphenation{Te-le-pathy}
\hyphenation{Li-nux-ov-ka}

\def\progref!#1!{\texttt{#1}}
\renewcommand{\arraystretch}{2} %Іначай формулы ў матрыцы зліпаюцца з лініямі
\usepackage{array}

\def\interview #1 (#2), #3, #4, #5\par{

\section[#1, #3, #4]{#1, #5}
\def\qname{LVEE}
\def\aname{#1}
\def\q ##1\par{{\noindent \bf \qname: ##1 }\par}
\def\a{{\noindent \bf \aname: } \def\qname{L}\def\aname{#2}}
}

\begin{document}
\title{Интервью с участниками}
%\author{}
\date{}
\maketitle

По традиции в сборник материалов входят интервью, в которых активные участники сообщества open source делятся своим мнением о свободном ПО, открытых технологиях, роли и месте GNU/Linux, рассказывают, как видят проблематику свободных проектов. В этот раз мы решили расспросить трёх участников конференции, какое-то время назад перебравшихся из Беларуси на территорию Европейского Союза.

\section[Александр Боковой "--- principal software engineer, Red Hat, Эспоо, Финляндия]{Александр Боковой "--- principal software\linebreak engineer, Red Hat, Эспоо, Финляндия}
%\begin{figure}[ht]
%\centering{\includegraphics[width=4cm]{49_spons_altoros.jpg}}
%\end{figure}

{\noindent \bf LVEE: Традиционно первый вопрос "--- как ты познакомился с открытым ПО?}

{\noindent \bf Александр Боковой:} В 1995 году. Я учился на третьем курсе БГПУ им. Максима Танка, и одна из курсовых
работ была посвящена фрактальной геометрии. Необходимо было написать
приложение, которое бы отрисовывало и в интерактивном режиме позволяло бы
исследовать множества Жюлиа для соответствующих точек из множества
Мандельброта. 

{\noindent \bf L: Звучит очень наукоёмко\ldots}

{\noindent \bf А:} Программу я писал на Паскале, и в какой"=то момент стало не хватать
стандартной памяти в 16"=битном режиме.

{\noindent \bf L: Под MS DOS.} 

{\noindent \bf А:}  Да. Вариантов использования 32"=битного
режима было немного, поскольку требовалось еще и приличный интерфейс
пользователя обеспечить. Значит, нужна была не только графическая библиотека,
но и виджеты, обработка клавиатуры и так далее. И я нашел такую библиотеку "--- SWORD,
написанную французом Эриком Николя на C++ и поставлявшуюся вместе с DJGPP.

{\noindent \bf L: А DJGPP "--- это\ldots}

{\noindent \bf А:}  DJGPP "--- это первый порт программ проекта GNU на
платформу Intel x86, сделанный еще в 1989 году DJ Delorie. Ричард Столлман
выступал на встрече Northern England Unix Users Group в компании Data General,
где работал тогда DJ Delorie, и на вопрос о переносе GCC под MS DOS, ответил,
что это невозможно, поскольку gcc слишком большая программа, а MS-DOS работает
в 16"=битном режиме. DJ понял, что это вызов, и принял его. Так что в 1990 мы
уже имели компиляторы GNU, Emacs, binutils и много разных библиотек, все под
MS DOS в 32"=битном режиме "--- DJ пришлось написать свой DOS Extender для
того, чтобы компилировать gcc под MS DOS.

{\noindent \bf L: Как скоро ты осознал, что это вот "--- свободное ПО, сообщество, коллективная разработка?}

{\noindent \bf A:} Практически сразу. DJGPP поставлялся со всеми исходными текстами, в документации было
написано, где можно задавать вопросы. Я подписался на рассылки и первое время просто читал "--- и переписку,
где люди отвечали не только на вопросы об использовании тех или иных компонент системы, но и обсуждали бытовые
темы. Выглядело все это очень по"=домашнему, а если кто"=то предлагал патчи, то это предполагало прежде всего
устранение необходимости патчить то же самое место в следующей версии "--- rsync еще не был написан (он появился
только в 1996), а DJGPP распространялся по FTP. На наших узких линиях (64Кбит/с на весь университет) тогда
приходилось прежде всего думать, а потом делать.

{\noindent \bf L: Итак, ты использовал DJGP. А каким образом пользователь СПО стал его разработчиком?}

{\noindent \bf A:} Со SWORD и DJGPP я и начал. В 1996 вышла вторая версия DJGPP, независящая от
коммерческих компонент для своей пересборки. Главное, что случилось с DJGPP в
1994--96 годах "--- это взрывной рост популярности, привлекший огромное число
терпеливых и общительных людей в списки рассылки. Можно было задавать вопросы и
получать ответы на них, вне зависимости от того, насколько плох был твой
английский язык. В 1995 году сделали зеркало в рассылку в виде группы USENET
\url{comp.os.msdos.djgpp}, она стала доступна на локальном NNTP"=сервере университета.

{\noindent \bf L: Не помнишь, как вообще оформилась мысль: а сделаю"=ка я публичный патч? Или это получилось как"=то незаметно: пообсуждал, поисправлял "--- и вдруг люди уже пользуются?}

{\noindent \bf A:} Непубличные патчи поддерживать было неудобно, поскольку сам комплект
DJGPP распространялся в виде архивов. Так что старался отправлять исправления сразу.
К тому же, библиотеки были мне нужны для работы, но не являлись главным ее содержимым.
Лицензия SWORD "--- GNU General Public License, которую я уже читал и видел в применении
к остальным компонентам GNU. 

В 1998 я вместе с Эриком работал над третьей версией SWORD. Эта работа привела
к тому, что через несколько лет я ушел из аспирантуры, так и не закончив свою
работу над диссертацией, потому что вместо работы над методикой преподавания
фрактальной геометрии сосредоточился над SWORD "--- нужда в нормальной
интерфейсной библиотеке, работающей под MS DOS и GNU/Linux на тот момент еще не
отпала, поскольку Qt до 2000 года выходила под неудачной с точки зрения свободного ПО и написания
GPL"=программ лицензией и не поддерживала MS DOS.

Правда, после ухода из аспирантуры я сосредоточился на сетевых файловых системах,
а Эрик переписал SWORD с нуля с учетом прогресса в Qt и проект был перезапущен в 2005: 
\url{http://www.erik-n.net/software/sword/}. 

На GNU/Linux я перешел где"=то в 1996--1997, практически сразу, как появился
собственный компьютер.

{\noindent \bf L: На какой дистрибутив?}

{\noindent \bf A:} Начал со Slackware. А в декабре 1999 перевел на белорусский язык программу установки Mandrake Linux. Она вошла в Mandrake Linux Russian Edition, а потом и в основной Mandrake Linux.

Другой проект, который <<втянул>> меня в себя в приблизительно то же время, это
Midgard, система ведения веб"=сайтов. Изначально придуманная финнами Генри
Бергиусом и Юккой Зиттингом для сайта своего реконструкторского общества в
1998, система переросла викингов и стала довольно успешно использоваться как
конструктор различных сайтов, в том числе и для интранетов. Я выступал с докладом
о Midgard на первом FOSDEM в 2001 году, а в середине 2000"=х даже интегрировал
Midgard и Samba для того, чтобы обеспечить прозрачную авторизацию в
интранет"=приложениях на Midgard в среде Active Directory.

{\noindent \bf L: И тут мы наконец подобрались к твоему участию в проекте Samba. }

{\noindent \bf A:} Получается забавная ситуация: практически все проекты, над которыми я работал и
работаю, в той или иной мере связаны между собой. В 2001--2004 годах мы с Игорем
Вергейчиком работали над системой хранения, где требовалась поддержка различных
сетевых файловых систем, и я столкнулся с необходимостью внести какие"=то
изменения в Samba.  Мы написали ряд патчей, отправили их в рассылку, часть из
них приняли, часть "--- нет.  Потом Игорь доработал Samba до поддержки Unicode.
Потом я написал поддержку множественных модулей виртуальной файловой системы. И в
2003 меня пригласили в Samba Team. Принцип был простой: мой код практически
не требовал дополнительных доработок, поэтому мне дали прямой доступ к
изменению исходного текста.

Когда в декабре 2003 мы получили заказ на разработку поддержки Active Directory
в нашей системе хранения, Эндрю Триджелл, создатель Samba, просто сказал нам:
<<Зачем пытаться добавить патчи в версию 2.0, лучше помогите мне закончить 3.0,
где я уже много добился>>. То есть, взгляды апстрима и даунстрима совпали,
получилось сделать многое. Конечно, не в тот срок, который обещал Эндрю, но в
2005 у нас был вполне работающий продукт.

{\noindent \bf L: Какие различия бросаются в глаза, если сравнивать опенсорс"=комьюнити в СНГ с англоязычным? }

{\noindent \bf А:} Если тебе нужны какие-то изменения к существующему коду, ты их пишешь,
оформляешь патчи, отправляешь в рассылку и обсуждаешь с другими разработчиками.
Патчи могут принять сразу, могут не принять совсем, но чаще всего приходится
объяснять и находить компромисс. Работа над отдельными изменениями может
затянуться на годы. В СНГ есть разработчики свободного ПО (и их много), но
очень мало сообществ разработчиков свободного ПО как таковых. Те, кто
заинтересован, участвуют в международных проектах различного масштаба. Новые
проекты с преимущественно русскоязычным общением -- редкость, они мало кому из
разработчиков нужны. Они, безусловно, нужны пользователям, но сколько времени
разработчики могут посвятить локальным пользователям?

С другой стороны, уровень знания английского языка может препятствовать
активному участию в существующих проектах, даже если кто"=то готов написать
код, часто сталкиваешься с тем, что довести работу до конца они не могут "---
нужна документация на английском, участие в дискуссиях, причем в темпе
активности конкретного проекта, а не разработчика.  В этом смысле разница в
Европе особенно бросается в глаза, здесь проблем с английским языком среди
разработчиков свободного ПО нет, даже в традиционно неанглоязычных странах.
Английский "--- lingua franca свободного ПО.

Другой аспект взаимодействия в проектах свободного ПО, это значительно меньший
накал страстей в рассылках по сравнению с тем, что я вижу в русскоязычной
среде. 

{\noindent \bf L: О да! В этом сезоне организаторская рассылка LVEE переживала 
как раз такую драму :) }

{\noindent \bf А:} Наблюдается заметное ослабление эмоций при общении с пользователями при
продвижении с востока на запад в Европе "--- если, скажем, польские пользователи
еще пишут с активным выражением своей позиции в отношении разработчиков на IRC"=каналах, 
то там, где преобладают английские или американские пользователи,
атмосфера менее накалена. В программном обеспечении есть и будут ошибки, никто
не идеален, поэтому поиск источника ошибки "--- рабочая ситуация, не требующая
перехода на личности. Почему"=то русскоязычное пространство переполнено полярным
выражением собственных эмоций.

{\noindent \bf L: Кстати, возвращаясь к теме работы над продуктами. Как бы ты охарактеризовал свой личный опыт использования СПО в корпоративном секторе?}

{\noindent \bf A:} Мне повезло, я последние лет пятнадцать использую свободное ПО в рабочем окружении. В
последние пять"=семь лет с этим стало совсем хорошо из"=за активного продвижения
мобильных платформ и веб"=приложений, которые вынесли из многих компаний
специализированные плагины и прочие платформо"=зависимые клиентские компоненты.

Работа над Samba и FreeIPA предполагает, что приходится иметь дело с
проприетарной инфраструктурой и клиентским ПО, но для обеспечения собственной
жизни в корпоративной среде мне они практически не нужны. Гораздо сложнее
с проприетарным ПО на серверной стороне "--- даже если интерфейс к нему позволяет
использовать свободное ПО на клиентской стороне, доступность данных в
большинстве таких систем завязана на производителя. Это данных наших компаний,
но извлечь их в структурированном виде и перенести куда"=то еще мы часто просто
не можем.

{\noindent \bf L: Последний вопрос, о Redhat. Как это выглядит изнутри?}

{\noindent \bf А:} По"=домашнему. В прямом смысле "--- большую часть времени
я работаю из дома. У нас небольшой офис в Эспоо, рабочее место у меня есть, но
появляюсь я в офисе нечасто, поскольку моя команда разбросана по миру. Из инструментов
общения "--- электронная почта, IRC, интернет"=телефония и видео"=конференции.
Раз или два в год получается встретиться лично, это время используется для интенсивных
дискуссий, особенно в феврале, когда в Брно (Чехия) проходит традиционная конференция
\url{devconf.cz} "--- на нее съезжаются ребята из многих команд и есть шанс обсудить
предстоящие задачи на год вперед с теми, с кем не получается пересекаться <<в эфире>>
из"=за часовых поясов.

Самым удивительным для меня четыре года назад было то, как мало информации скрыто от
посторонних глаз. Если Red Hat участвует в разработке какого"=то проекта, то вся информация
доступна на сайте апстрима. Внутри только детали планов интеграции конкретных апстримных
версий в продукты компании, а весь дизайн новых функций и их разработка ведутся публично.

А моя история, кстати, замкнулась: DJ Delorie работает в Red Hat и обеспечивает нас
работающими компиляторами вот уже более шестнадцати лет.


\section[Андрей Шадура "--- software engineer, Collabora, Братислава, Словакия]{Андрей Шадура "--- software engineer, \linebreak Collabora, Братислава, Словакия}

%\begin{figure}[ht]
%\centering{\includegraphics[width=4cm]{49_spons_altoros.jpg}}
%\end{figure}

{\noindent \bf Андрей Шадура:} Моё знакомство со свободным ПО произошло, когда я в школьные времена ещё пользовался DOS и Windows и программировал для них на турбопаскале. Некоторые библиотеки, которыми я пользовался, поставлялись в бинарном виде и без исходных кодов, некоторые были с исходниками и README о том, что коммерческое использование запрещено, а к некоторым прилагался объёмный файл COPYING с текстом лицензии. Примерно в то же время я узнал об альтернативных операционных системах из тогдашних компьютерных журналов, и идея того, что ОС можно «похачить» изнутри меня очень заинтересовала. Позднее я скачал несколько однодискетных дистрибутивов (кроме как по dial"=up, мне Интернет был слабодоступен) и поиграл с ними, но «настоящий» Linux я не попробовал до учёбы в университете.

{\noindent \bf L: И какие были впечатления от этого самого первого опыта, от Unix"=подобных систем?}

{\noindent \bf А:} Как раз первый из этих однодисковых дистрибутивов и был сертифицированный Unix "--- демо"=версия QNX. Было конечно интересно увидеть что-то совсем другое, и там был такой GUI! Но возможности у этой версии были сильно ограниченными. Затем был очередной однодискетный Linux-дистрибутив, Trinux. Я о нём где"=то прочитал, не то в <<Компьютерной газете>>, не то в <<Хакере>>\ldots

{\noindent \bf L: Эксперимент прошёл с тем же примерно успехом?}

{\noindent \bf А:} Да. А затем меня увлекло местное движение <<даунгрейдеров>>, и я провел несколько лет в окружении FreeDOS, GEM, ViewMAX, других древних систем и их опенсорсных реинкарнаций. Но, кстати, в DOS меня буквально бесили все тамошние недокументированные функции.

{\noindent \bf L: Ты имеешь в виду тамошний зоопарк системных вызовов? Все эти int 21h?}

{\noindent \bf А:} Да. Ещё в школьные годы я часами в библиотеке просиживал, выуживая прерывания и номера функций из старой литературы и новых журналов. В сравнении с этим, а также недокументированными функциями Windows, жизнь в Linux "--- просто раздолье.

{\noindent \bf L: Итак, следующий этап "--- уже в университете?}

{\noindent \bf А:} Это был уже  2005 год, там я получил от Дениса Пынькина, который уже был ALT Linux developer, копию ALT Linux 2.2 Master. Это, кстати, было в преддверии выхода версии 2.4, и он меня уговаривал подождать, но мне хотелось здесь и сейчас, и я на следующий же день проинсталлировал 2.2. Получилось без звука, были проблемы с X"=сервером (конечно, без всякого опыта), и это конечно было круто "--- иметь действующую Linux-систему, но основной ОС она в тот раз для меня не стала.

{\noindent \bf L: А когда наконец стала?}

{\noindent \bf А:} Годом позже. Работал в лаборатории института ядерных проблем, я получил аккаунт на сервере под управлением Debian <<sarge>>, попросил "--- и мне сделали бутстрап этой инсталляции на мой жёсткий диск. Потом эта машина у меня дома занималась маршрутизацией, пока в 2009 году не заменил ее маленьким роутером на MIPS. Ну а через несколько месяцев общения с этой машиной Debian стал и моей десктоп"=системой. И тогда же захотелось как"=то контрибутить в проект. Сначала начал делать патчи и баг"=репорты к тому, чем пользовался. А в 2009 запакетировал первое приложение.

{\noindent \bf L: Что это было?}

{\noindent \bf А:} Случайное, практически на спор. Кто"=то  пожаловался, что это очень тяжело "--- делать пакеты для Debian, я ответил, что нет ничего проще, и услышал в ответ: <<Ну давай, запакетируй мой проект, посмотрим, сколько это у тебя займёт времени>>. И за пару часов подготовил пакет для gdigi.

А потом, уже на LVEE, Дмитрий Бородаенко (в то время "--- единственный Debian Developer из Беларуси) побудил меня на больший вклад. Первый пакет, который я по"=настоящему мэйнтэнил, был tclxml.

Позже, в 2010, занялся исправлением некоторых багов ifupdown, инструмента конфигурирования сети в Debian, ну и так далее.

{\noindent \bf L: Ты принципиальный Debian'щик?}

{\noindent \bf А:} Конечно, не только Debian. Я вообще вношу вклад время от времени в разные свободные проекты, да и свои собственные есть.
Кроме того, время от времени вношу правки в википедию, а еще, достаточно регулярно "--- в OpenStreetMap.

{\noindent \bf L: Теперь "--- к переезду. Скажи, отъезд из Беларуси как"=нибудь повлиял на твои взаимоотношения с миром свободного ПО?}

{\noindent \bf А:} В некотором смысле повлиял: проще, ближе и быстрее стало ездить на всевозможные конференции и прочие спринты и hackweeks. Из событий, на которых я побывал недавно: LinuxDays.cz в Праге (три раза), FOSDEM (два раза), Cambridge Debian Miniconf\ldots Ну и недавняя hack week в Копенгагене.

{\noindent \bf L: Неполиткорректные соотечественники должны в этот момент воскликнуть <<дорвался>> :)}

{\noindent \bf А:} Ага, так и есть. Но вообще, с кругом общения сложнее. Чтобы было понятнее, я за чуть более, чем три года в Словакии переезжал два раза. Первое время здесь я жил в деревне, с кругом было вообще никак.

Но я активно этот круг искал в других местах. Например, познакомился с местным сообществом OpenStreetMap (Freemap.sk) и через две недели после переезда поехал с ними на mapping party. Общаться было сложновато, потому как по"=английски я хоть и говорю, но мне хотелось научиться говорить по"=словацки, а знания были очень слабы. А по"=белорусски меня понимали слабо :)

{\noindent \bf L: Все эти переезды были связаны с работой?}

{\noindent \bf А:} Да, одна закончилась, новая находилась в другом месте. Через год, кстати, я приехал на университетскую конференцию OSSConf в Жилине, о которой узнал случайно, и познакомился там с кучей сторонников free software.

Да, еще забыл. Через пару месяцев после mapping party в результате активных поисков я узнал, что в Братиславе есть хакерспейс Progressbar, и направился туда на одно из мероприятий. мероприятий там вообще много проводилось, собирались какие-то питонисты, опенстритмапперы и прочие, но поездка туда занимала бы 5 часов в одну сторону, поэтому часто посещать их не получалось, пока я не переехал в конечном итоге в Братиславу.

{\noindent \bf L: Вопрос по поводу членов опенсорс"=комьюнити: какие-то отличия после переезда? Бросалась в глаза какая"=то разница?}

{\noindent \bf А:} С одной стороны, здесь я заметил, что линуксами, в основном Ubuntu, пользуются иногда люди, далекие от IT вообще. И это меня удивило.

{\noindent \bf L: Ну да, у нас это обычно члены семей линуксоидов.}

{\noindent \bf А:} Из примеров вспоминается одна знакомая, которая мне рассказывала о том, какая замечательная Ubuntu и какой ужасный Debian. При этом она ни разу в жизни, как мне кажется, не видела командную строку ни одного, ни другого. Её работа вообще связана с образовательными программами\ldots

{\noindent \bf L: Ты говоришь, что это с одной стороны. А с другой?}

{\noindent \bf А:} С другой стороны, пассивность «активистов». В Жилине, где я жил, есть некоторое количество людей, пользующихся Linux и знающих про свободное ПО (часть из них работает в местном университете). И за целый год проводится одно, максимум два события на тему: тот самый OSSConf, и иногда OSS Weekend. Это при том, что в городе вроде как третий по величине технический университет страны\ldots 

Ну да ладно, Жилина, маленький городок. Берем Братиславу. Здесь есть STU, Словацкий технический университет, здесь есть Progressbar.  Progressbar организует небольшие митапы иногда, но нерегулярно. Я предлагал создать регулярные встречи, подобие минских линуксовок. Никто энтузиазма не проявил, но один человек рассказал, что он когда"=то пробовал делать какие"=то встречи, но всё угасло.

Подобный вопрос я поднял на OSS Weekend, который в прошлом году (и в этом тоже) проводился в Братиславе. <<Надо бы\ldots>> был ответ :)

Но, с третьей стороны, как ни странно, местное Ruby"=сообщество достаточно активное. Встречи проводят несколько раз в месяц, называются Рубислава.


\section[Евгений Калюта "--- experienced developer, Ericsson, Хельсинки, Финляндия]{Евгений Калюта "--- experienced develo\-per, Ericsson, Хельсинки, Финляндия}

%\begin{figure}[ht]
%\centering{\includegraphics[width=4cm]{49_spons_altoros.jpg}}
%\end{figure}

{\noindent \bf L: Традиционный первый вопрос "--- твое первое знакомство с открытым ПО. Может быть первые впечатления, если они были?}

{\noindent \bf Евгений Калюта:} Я расскажу долгую историю :) 

{\noindent \bf L: Отлично :)}

{\noindent \bf Е:} Я из провинции. Доступность как информации, так и техники тогда была не на высоте. Учитель информатики у нас был молодой, активный, сразу после
института. Это был 7"=ой класс, когда нам поставили <<Корветы>>. 

{\noindent \bf L: Действительно, издалека :)}


{\noindent \bf Е:} Программы обучения толком не было, нас в класс пускали, но директор строго говорила
<<седьмому классу только игры>>. Однако учитель некоторым пытливым показал книжки по Basic и давал основы алгоритмизации (в моём классе нас таких пытливых было двое). Он же (учитель) как"=то рассказал, что для настоящего
программирования бывает ассемблер (что это я тогда представлял с трудом), и C.

{\noindent \bf L: Этого хотелось?}

{\noindent \bf Е:} Этого очень хотелось. Но книг в доступности не было (начало девяностых).

Однажды в книжном я таки увидел какую"=то брошюрку, то ли про C, то ли про что"=то ещё, но главное, что в предисловии было замечено, что вот такой вот он язык C, и на нём написали Unix, на котором работает Интернет.

Очень захотелось как C, так и Unix. При мысли о них в душе возникал некий трепет.

Заработать на первый PC мне удалось кажется на третьем курсе. Где"=то в это
время, кажется в <<Компьютерной газете>>, пробежала статья с заголовком <<Попробуйте Linux>>. Это был Unix, этого хотелось. Плюс мысль о том, что
можно посмотреть в исходный код настоящего ядра настоящей операционной
системы, вызывала ощущения на грани\ldots.

{\noindent \bf L: Напишем, что мысль вызывала катарсис.}

{\noindent \bf Е:} Хорошо :) Но этого негде было взять (из моего круга общения, ясное дело, который на
тот момент охватывал не очень много людей, приобщённых к IT). Первый диск, привезённый одной компьютерной фирмочкой, нёс на себе две безнадёжно испорченные версии дистрибутива <<Caldera>>, ни одна из них не могла поставиться
по объективным причинам.

{\noindent \bf L: Из"=за неумелой перепаковки?}

Ну как, если правильно подмонтировать распакованный tar.gz
одиного из них как umsdos, то может шансы и были бы, но я тогда я не имел
об этом ни малейшего представления. Я потрогал консоль инсталлятора,
смог даже перенести файл на ДОС"=раздел, испытал\ldots катарсис, понятное
дело, ну и как бы на этом всё.

{\noindent \bf L: А твой первый работающий Linux?}

{\noindent \bf Е:} Первым работающим оказался русский клон Redhat 4.2 "--- он назывался <<Красная шапочка
5.0>>, он умел ставиться, он умел грузиться, на нём собиралось ядро и, если
мне не изменяет память, KDE 1.0 (к тому моменту у меня уже были контакты,
у кого это можно было взять).

Подытоживая, пришёл к открытому ПО я случайно (я о нём ничего не знал) из
желания приобщиться к великому, к Unix, и (за неимением)  других вариантов не искал.

{\noindent \bf L: Несколько слов про твой путь из пользователей свободного ПО в разработчики?}

{\noindent \bf Е:} Ну, вообще контрибуций у меня не очень много. В детстве я был <<хорошим
советским мальчиком>> и очень боялся публичного порицания. Поэтому долгое
время в <<серьёзные>> проекты было лезть страшновато\ldots Что очень зря. С
большего, по отношению к открытым проектам, это прошло только пару лет
назад. А с Debian в период большого желания просто случился неприятный
казус, который затормозил мой путь в Debian Developer.



{\noindent \bf L: У нас в этом году снова тематическое интервью. Поэтому еще группа вопросов, инспирированная отъездом интервьюируемого из Беларуси. Какие различия бросаются в глаза, если сравнивать опенсорс-комьюнити в
СНГ с англоязычным? Европейских и белорусских (и вообще русскоязычных, наверное) разработчиков?}

{\noindent \bf Е:} Про англоязычные комьюнити особенно говорить бессмысленно, ибо белорусы "--- такие же полноправные участники этого коммунити. Сами и всё видят, и несут вклад в общую атмосферу.

{\noindent \bf L: Ну, речь скорее о локальных комьюнити. }

{\noindent \bf Е:} Локального, про Финляндию могу сказать чуть личного.

{\noindent \bf L: Очень хорошо. }

{\noindent \bf Е:} Я бы отметил, что различные открытые проекты тут занимают видимую часть 
общественной жизни "--- тут вообще модны всевозможные общественные обсуждения и
инициативы). Участие студентов в opensource очень естественно: помню, поразился количеством Debian Developers.  

{\noindent \bf L: Если подумать, не только Debian "--- в конце концов, происхождение Linux как такового\ldots}

{\noindent \bf Е:} Финляндия дала миру open source кроме ядра Linux и некоторые менее известные вещи, такие как протокол IRC, клиент  irssi, оконный менеджер ion (славящийся своим проблемным автором), протокол ssh, почтовый сервер dovecot "--- это навскидку.

Местные заинтересованные вполне чувствуют себя частью мирового движения, активно участвуют в проектах, конференциях, инициативах по всему миру, устраивают их у себя. Первый мой debconf был в Финляндии, пару раз был на
дебиановских bug\linebreak squashing party, они тоже как правило не совсем локальные. Можно, наверное, сказать, что открытости в сообществе порядочно поболе. 


А еще для пущего приобщения студентам в плюс к традиционным конференциям устраивают различного рода встречи и круглые столы с известными в мире open source людьми, и они как правило не ограничиваются студентами. Я был на лекциях Столлмана и Торвальдса (на той самой, про Nvidia).

И потом, обсуждать вопросы с местными ребятами мне лично очень приятно "--- это, как правило, спокойно, по делу, без излишнего давления.

В общем, все это очень повлияло и на личную систему <<свой "--- чужой>>. Она слабо коррелирует с границами и языками.

Кроме того, на момент переезда (2006 год), проникновения IT в общественную жизнь было порядочно больше, чем в Беларуси, поэтому и восприятие околоайтишных движений более серьёзно.

В остальном точно так же: лобби коммерческих компаний имеет больший вес. Слышал истории об оспаривании некоторых государственных тендеров, на что банально не хватило денег.

{\noindent \bf L: Еще интересный вопрос "--- твой личный опыт использования СПО в корпоративном секторе. Понятно, всегда есть работодатель\ldots}

{\noindent \bf Е:} Используется активно, там где не противоречит коммерческим интересам. Но, как мы знаем, построить коммерческую систему на базе свободного ПО и поддержки
комьюнити у Нокии не получилось. Вклад она при этом внесла очень порядочный, надо отметить.

В Эриксоне, безусловно, оно тоже используется в разных местах, и даже что"=то выползает наружу. 

{\noindent \bf L: В смысле наработок, которые отдаются сообществу?}

{\noindent \bf Е:} В целом, отдавать из корпорации назад обычно сопряжено с трудностями. Как правило, это связано с законодательством США и нежеланием рисковать, ну или и простая жадность иногда. Но есть и позитивные случаи: на память сразу приходят TIPC\footnote{\url{http://en.wikipedia.org/wiki/TIPC}} и \linebreak Eclipse.

А если про компании вообще, уже не из личного опыта "--- понятно, раньше Nokia и окружающие её компании задавали тон, но и сейчас тут присутствует некоторое количество компаний, серьёзно вкладывающих в разработку открытых проектов: это небольшой (по их меркам, в несколько сотен) офис OTC Intel, это наверное наиболее <<честный>> контрибьютор, но есть Huawei, Samsung, (члены Linaro), Nvidia опять же. В определённом смысле присутствуют Red Нat и TI.

\end{document}


