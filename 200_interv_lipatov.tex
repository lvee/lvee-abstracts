\documentclass[10pt, a5paper]{article}
\usepackage{pdfpages}
\usepackage{parallel}
\usepackage[T2A]{fontenc}
\usepackage{ucs}
\usepackage[utf8x]{inputenc}
\usepackage[polish,english,russian]{babel}
\usepackage{hyperref}
\usepackage{rotating}
\usepackage[inner=2cm,top=1.8cm,outer=2cm,bottom=2.3cm,nohead]{geometry}
\usepackage{listings}
\usepackage{graphicx}
\usepackage{wrapfig}
\usepackage{longtable}
\usepackage{indentfirst}
\usepackage{array}
\newcolumntype{P}[1]{>{\raggedright\arraybackslash}p{#1}}
\frenchspacing
\usepackage{fixltx2e} %text sub- and superscripts
\usepackage{icomma} % коскі ў матэматычным рэжыме
\PreloadUnicodePage{4}

\newcommand{\longpage}{\enlargethispage{\baselineskip}}
\newcommand{\shortpage}{\enlargethispage{-\baselineskip}}

\def\switchlang#1{\expandafter\csname switchlang#1\endcsname}
\def\switchlangbe{
\let\saverefname=\refname%
\def\refname{Літаратура}%
\def\figurename{Іл.}%
}
\def\switchlangen{
\let\saverefname=\refname%
\def\refname{References}%
\def\figurename{Fig.}%
}
\def\switchlangru{
\let\saverefname=\refname%
\let\savefigurename=\figurename%
\def\refname{Литература}%
\def\figurename{Рис.}%
}

\hyphenation{admi-ni-stra-tive}
\hyphenation{ex-pe-ri-ence}
\hyphenation{fle-xi-bi-li-ty}
\hyphenation{Py-thon}
\hyphenation{ma-the-ma-ti-cal}
\hyphenation{re-ported}
\hyphenation{imp-le-menta-tions}
\hyphenation{pro-vides}
\hyphenation{en-gi-neering}
\hyphenation{com-pa-ti-bi-li-ty}
\hyphenation{im-pos-sible}
\hyphenation{desk-top}
\hyphenation{elec-tro-nic}
\hyphenation{com-pa-ny}
\hyphenation{de-ve-lop-ment}
\hyphenation{de-ve-loping}
\hyphenation{de-ve-lop}
\hyphenation{da-ta-ba-se}
\hyphenation{plat-forms}
\hyphenation{or-ga-ni-za-tion}
\hyphenation{pro-gramming}
\hyphenation{in-stru-ments}
\hyphenation{Li-nux}
\hyphenation{sour-ce}
\hyphenation{en-vi-ron-ment}
\hyphenation{Te-le-pathy}
\hyphenation{Li-nux-ov-ka}
\hyphenation{Open-BSD}
\hyphenation{Free-BSD}
\hyphenation{men-ti-on-ed}
\hyphenation{app-li-ca-tion}

\def\progref!#1!{\texttt{#1}}
\renewcommand{\arraystretch}{2} %Іначай формулы ў матрыцы зліпаюцца з лініямі
\usepackage{array}

\def\interview #1 (#2), #3, #4, #5\par{

\section[#1, #3, #4]{#1 -- #3, #4}
\def\qname{LVEE}
\def\aname{#1}
\def\q ##1\par{{\noindent \bf \qname: ##1 }\par}
\def\a{{\noindent \bf \aname: } \def\qname{L}\def\aname{#2}}
}

\def\interview* #1 (#2), #3, #4, #5\par{

\section*{#1\\{\small\rm #3, #4. #5}}

\def\qname{LVEE}
\def\aname{#1}
\def\q ##1\par{{\noindent \bf \qname: ##1 }\par}
\def\a{{\noindent \bf \aname: } \def\qname{L}\def\aname{#2}}
}

\begin{document}
\title{Интервью с участниками}
%\author{}
\date{}
\maketitle

По сложившейся традиции в сборник материалов включены интервью, взятые представителями оргкомитета у участников конференции: как сразу после LVEE 2011, так и во время впервые проведенной в этом году выездной зимней сессии конференции LVEE Winter 2012. Интервью 2011 года было тематическим, посвященным использованию свободного программного обеспечения в сфере образования. Его участники принадлежат к разным странам и в той или иной степени имеют отношение к академическому процессу. Мы попросили их поделиться особенностями использования свободных программ в их вузах и услышали, какова ситуация в академических кругах Украины, Литвы и Венгрии.

\section[Григорий Злобин, Львов, Украина (LVEE 2011)]{Григорий Злобин, Львов, Украина \linebreak (LVEE 2011)}
%\begin{figure}[ht]
%\centering{\includegraphics[width=4cm]{49_spons_altoros.jpg}}
%\end{figure}

{\noindent \bf Григорий Злобин:} Я доцент Львовского университета, работаю на кафедре радиофизики факультета электроники. Уже шесть лет как у нас ведется подготовка студентов в направлении компьютерных наук. Возникло это благодаря тому, что раньше кафедра занималась разработкой программ машинного анализа электронных цепей, т.\,е. считалось, что мы имеем достаточный уровень профессиональной подготовки. А специальность, по которой мы готовим студентов "--- это <<Информационные системы>>.

{\noindent \bf L: Кем обычно работают выпускники этой специальности?}

{\noindent \bf Г:} Если честно, спектр очень широк. Работают и разработчиками, и системными администраторами, и разработчиками встроенного программного обеспечения, правда большая часть разрабатывает ПО для микроконтроллеров фирмы Cypress. Именно здесь наши выпускники имеют очень высокую подготовку, потому что для этих микроконтроллеров мало сделать ПО, нужно еще разработать аппаратную часть, как цифровую так и аналоговую. 

{\noindent \bf L: Учебный процесс по этой специальности как-нибудь задействует свободное ПО?}

{\noindent \bf Г:} Используется свободное ПО при чтении как общих курсов, так и при чтении спецкурсов. Общие курсы это <<Операционные системы>> и <<Системное программирование>>, а спецкурсы "--- здесь уже очень зависит от лектора. Поскольку мы по сути дела только в начале пути, то к сожалению наша кафедра не доросла до встраиваемых систем, где нужен Linux; пока мы занимаемся только микроконтроллерами. Правда, этому есть объяснение: очень большой объем заказов у фирм, занимающихся микроконтроллерами, и мы им не в состоянии дать столько студентов. Пока идет замыкание на микроконтроллеры Cypress, но я надеюсь, что мы будем двигаться дальше.

{\noindent \bf L: Программное обеспечение, используемое в вузе, это еще и средства обеспечения учебного процесса: те же текстовые редакторы и операционные системы...}

{\noindent \bf Г:} Да. Поскольку факультет наш принадлежит к естественным наукам, то курса информатики у нас просто нет. Но есть учебные практики после первого, второго, третьего курса, и это чудесная возможность вставить упреждающий по сути курс по ОС Linux, Openoffice, SciLab\ldots Могу похвастаться, что на практике мы используем оболочку <<Кузя>> "--- собственную разработку, наших студентов и для студентов.

{\noindent \bf L: А можно про нее пару слов?}

{\noindent \bf Г:}  Что было толчком? У нас на кафедре использовалась еще одна оболочка, сделанная под Windows для языка Паскаль. Я ее горячо поддерживал, а другие преподаватели относились с прохладой. Но когда у нас появилась компьютерная специальность, мы получили интересное явление: преподаватель, который читал программирование на Паскале, требовал, чтобы студенты этой оболочкой не пользовались. А студенты дома в ней работали, а потом на лабораторных работах сделанные программы сдавали в Turbo Pascal. И именно это натолкнуло их "--- студентов "--- на идею: если оболочка, сделанная именно под учебный процесс, дает возможность делать программы быстрее, давайте напишем ее сами. 

Поскольку <<Кузя>> использует компилятор gcc (и не только) "--- возможности его очень широкие. Правда, поскольку это работа просто по желанию "--- не все так хорошо движется, но мы с моим дипломником в следующем году собираемся сделать версию с веб-доступом. Так что ребята работают, это очень приятно\ldots У нас уже поддерживается шесть языков в интерфейсе, один из них белорусский, и есть арабский. Даже благодаря <<Кузе>> ребята, которые ее сделали, уже больше года работают разработчиками в фирме. 

Как ни парадоксально, изменить отношение к СПО помогла фирма Microsoft. Прошлой осенью я получил письмо от Сергея Байдачного с просьбой срочно, прямо завтра, предоставить резюме студентов, которые могли бы поехать на стажировку в Microsoft в США. Я быстро всех студентов проинформировал, ребята послали свои резюме, в т.~ч. и те, кто делали <<Кузю>>, причем один из них признался, что занимается СПО. И именно тот, который признался, прошел первый тур отбора. 

{\noindent \bf L: Это показатель :)}

{\noindent \bf Г:} Да. Противники СПО, когда об этом узнали, очень сильно озаботились. Это повлияло на преподавателей, у них начало меняться отношение. Хотя есть и другие примеры. Один из наших преподавателей написал методические указания к Matlab. Я предложил проверить, можно ли эти задания выполнить в Octave, и оказалось, что все выполняется. У нас в этом году открылась научно-исследовательская тема, в рамках которой мы собираемся сделать книгу <<Свободные системы компьютерной математики>>, и там будет раздел по Octave. Показательно, что я с этим же человеком дискутировал, когда он в других методических указаниях вставлял Matlab: <<Вот придет наш выпускник на производство и скажет: давайте купим Matlab, и я в нем такие вещи сделаю! А этого выпускника немедленно уволят, потому что Matlab стоит \$5000, а библиотека Simulink "--- \$45000. Они скажут, что им такие дорогие работники не нужны>>. А вот когда мы посмотрели, что то же самое можно сделать в Octave "--- это уже совсем другой поворот.

{\noindent \bf L: Какие требования у потенциальных работодателей ваших выпускников? Можно ли говорить о каком-то интересе работодателей к свободному ПО? Эта тема как-то просматривается?}

{\noindent \bf Г:} Ну, во-первых, фирмы, занимающиеся программированием во Львове, Softserve, Eleks, ведут разработки как под Windows, так и в СПО, и проявляют к этому большой интерес. Но на контакты с нами, тем не менее, идут не очень охотно, предпочитают напрямую работать со студентами. Потом, есть еще фирмы, нанимающие технический персонал для обслуживания облака в отдаленных от нас регионах, и работающие на открытом программном обеспечении. Но небольшая емкость рынка делает эту политику очень индивидуальной.

{\noindent \bf L: Как бы вы, коротко, охарактеризовали основные факторы в вузе, мешающие свободному программному обеспечению?}

{\noindent \bf Г:} Тотальная безответственность. Основная часть проприетарного программного обеспечения "--- пиратская, и несмотря на уголовный кодекс, который содержит очень жесткие уголовные нормы за пользование пиратским продуктом, вузы не трогают. И пока это так, вузы живут с позиции <<Мне все равно, а если даже будет проверка "--- проблема будет у ректора>>. Впрочем, я бы не хотел так уж обвинять своих коллег. Нужно еще учитывать, что уровень оплаты труда преподавателей не слишком соответствует решаемым задачам.

{\noindent \bf L: И последний вопрос: как вы заинтересовались свободным ПО сами?}

{\noindent \bf Г:} Это было лет 15 назад, еще до появления статьи в уголовном кодексе. Я как-то задумался: <<Неужели есть только Microsoft Office и больше ничего?>> Тогда мне под руки подвернулся Star Office, проприетарный продукт, бесплатный для учебных заведений. Я начал его пропагандировать и постепенно перешел в Linux. С Linux была проблема из-за той компьютерной техники, которую мы тогда имели. Некоторые легковесные дистрибутивы можно было поставить, но они не были локализированы. Это сразу означало, что мы не можем их использовать. А еще для нас было очень важным наличие украинского диалога. Долгое время фирма Microsoft игнорировала этот вопрос, считая, что достаточно русскоязычной версии. Только после появления локализированных дистрибутивов Linux они начали некоторую активность. Для некоторых регионов Украины наличие или отсутствие украиноязычного диалога не имеет никакого значения, но только не во Львове.

\section{Миколас Окулич-Казаринас, Вильнюс, Литва (LVEE 2011)}

%\begin{figure}[ht]
%\centering{\includegraphics[width=4cm]{49_spons_altoros.jpg}}
%\end{figure}

{\noindent \bf Миколас:} Я преподаватель университета им. Миколаса Ромериса в Вильнюсе. Уже три года как у нас создан новый факультет социальной информатики. Я работаю на кафедре информатики программных систем, и мы уже через год будем иметь первых выпускников "--- студентов по специальности <<Бизнес"=информатика>>.

{\noindent \bf L: Это достаточно новая специальность?}

{\noindent \bf М:} Новая, на стыке информатики и менеджмента. 

{\noindent \bf L: Как бы ты охарактеризовал потенциальный рынок труда для ваших выпускников?}

{\noindent \bf М:} Мое понимание этого вопроса чуть отличается от понимания некоторых других специалистов. Одна из предусмотренных специальностей "--- консультант по продажам информационных систем, но я надеюсь, что наши студенты также могут быть руководителями каких"=либо отделов или фирм, связанных с IT. Я хотел бы, чтобы высшее образование давало более высокое положение. А с другой стороны, т.\,к. мы имеем много предметов, связанных с бизнесом, мы не можем сделать из студентов хороших программистов, и я надеюсь сделать их настолько хорошими программистами, чтобы они могли сформулировать задачу для программистов"=профессионалов.

{\noindent \bf L: И как при такой специализации "--- свободное ПО находит какое-то применение в учебном процессе?}

{\noindent \bf М:} Конечно находит. Мы свободное программное обеспечение начали включать еще в 2004 году, т.\,е. до появления этой специальности и этого факультета. Что мы сделали "--- это инсталлировали OpenOffice на всех компьютерах кафедры, на которой я тогда работал, не удаляя ничего, что было прежде. Мне приходилось тогда преподавать социальную статистику, все задачи выполнялись на табличном процессоре. И когда я спросил: <<Какие табличные процессоры вы знаете?>> "--- около 5\% знали про OpenOffice. Из них около половины его раньше включали, и только один студент сказал, что имеет опыт работы. И я объяснил: <<Я вам преподаю статистические методы, и приходя на занятие, вы должны уже уметь пользоваться табличным процессором. В свободное время я могу показать вам программу, которую вы не знаете "--- OpenOffice. Вы можете выполнять задачи на чем угодно, но я вам предлагаю попробовать эту новую программу>>. После такого предложения один или два студента выполнили работу в Mircosoft Excel, а остальные выполнили их в OpenOffice или использовали обе программы. Причем я внимательно сравнивал качество этих работ "--- не было никакой корреляции между выбранной программой и качеством выполнения. Если бы студент начал работать с новой программой и столкнулся с проблемами "--- конечно он перешел бы обратно на уже известную программу. Не у всех хватило храбрости попробовать новую программу с первого занятия, но большинство из тех, кто раньше или позже попробовал, так до конца и делали задания в OpenOffice. Этот курс статистики, который я преподавал не для информатиков, а для совсем другой специальности, позволил мне убедиться, что в университете для непрограммистов можно без ущерба для учебного процесса использовать OpenOffice. Это был первый шаг.

Кроме того, еще до нового факультета у нас использовалась система Moodle. Мы сейчас развиваем систему сетевого обучения "--- чуть другой подход, чем дистанционное обучение, и в рамках этой системы у нас сервер работает на Linux. Однако общая  инфраструктура создана в университете еще до нас, она работает и нет смысла ее стирать. Может быть, если бы она была создана сейчас...

{\noindent \bf L: То есть свободное программное обеспечение все-таки находит применение и за рамками учебных дисциплин?}

{\noindent \bf М:} Когда появились информатики "--- появились задачи программирования. Программировали что-то локально. Я  создал сервер, к которому подключается каждый студент (но аутентификацию отдельную мы не делали, через LikwiseOpen наш сервер аутентифицируется в контроллере домена Active Directory), студенту создается автоматически акаунт в этом Linux-сервере, и веб-адрес. Каталог в домашнем каталоге студента отображается на веб-сервере в Apache. Первокурсник, получив задачу запрограммировать что-то на PHP, программирует и может сразу же показать что-то "--- в т.~ч. даже показать друзьям или позвонить домой. Возможно, в будущем~--- показать своему работодателю. С другой стороны, PHP "--- это язык, который дает слишком много возможностей, и я не уверен, что это хорошо с педагогической точки зрения.

{\noindent \bf L: А как у вас преподавательский состав относится к СПО?}

{\noindent \bf М:} Очень по-разному. Есть очень много хороших специалистов в области информатики, которые используют СПО, но без политического подтекста. Java, Netbeans, Eclipse\ldots Без разницы, Windows там или Linux. Есть такие, которые используют азартно, кто чувствует, что выучив что-то свободное, студент сможет потом предложить своему работодателю более дешевую и приспособленную систему. Задача университета "--- скорее подготовить таких специалистов, которые приносили бы деньги своему работодателю, а не указывали, что ему покупать. Есть люди, которые это понимают. Но есть конечно и такие, кто считает: <<Я столько версий этого монополиста изучал! Опять переучиваться? Нет уж, мне до пенсии и этого хватит>>. Самое неожиданное для меня было, когда один преподаватель сказал: <<Да, университет сэкономит очень много денег, но это же не я!>> Хотя я надеюсь, что это скорее исключения. 

Что больше всего радует "--- понимание на уровне администрации того, что сетевое обучение неотделимо от свободного ПО. Когда студент находится дома, он должен иметь все то же программное обеспечение, которое нужно для обучения, и мы не можем его обеспечить проприетарным ПО. И даже если какой-то поставщик сделает свой продукт для обучения бесплатным "--- все равно мы хотим, чтобы студент, чему-то научившись, сразу мог использовать этот продукт для своей работы, для частного бизнеса, для чего-то еще\ldots А для этого подходит только открытое программное обеспечение. 

Когда преподаватели почувствовали поддержку сверху, от администрации "--- возмущений стало все меньше и меньше. С другой стороны, свободное ПО совершенствуется быстро, и то что было несколько лет назад "--- хуже чем есть сейчас. И еще, когда появился compiz, люди не интересовавшиеся Linux, собрались в преподавательской смотреть на поворачивающийся куб\ldots :) 


{\noindent \bf L: Да, надо сказать, что хотя аппаратное ускорение графики придумали не в Linux, зато в Linux его внедрили быстрее всего. И заметнее. }

{\noindent \bf М:} Да. С другой стороны, есть люди, которые много лет используют Windows, и им нравится стабильность. 

{\noindent \bf L: А как вообще в Литве себя чувствует свободное программное обеспечение, в целом?}

{\noindent \bf М:} Рост есть. Я "--- один из учредителей ассоциации открытого кода (АКЛ), мы раньше несколько лет участвовали в региональной выставке информационных технологий. В первом году реакция была: <<Что, бесплатно?>> Люди не понимали. На следующий год "--- <<Да-да-да, я слышал>>. Потом "--- <<Я использовал>>. А на следующий год уже толпа народу: <<Я использую Ubuntu>>. Мы чувствовали, что с каждым годом все больше людей приходило, которые понимали, и что они все больше понимали. Это было очень ярко выраженно. Студенты тоже много используют Ubuntu, даже в тех университетах, которые ее не используют. Но на государственном уровне все пока не так хорошо. Есть очень много хороших специалистов в государственных структурах, которые хотели бы двигаться в этом направлении. У монополистов есть свои методы саботировать эти процессы.

{\noindent \bf L: А как ты сам заинтересовался свободным ПО?}

{\noindent \bf М:} В детстве я программировал. Был Turbo Pascal и  Quick Pascal. Более крупная фирма, победившая в университете, использовала не вполне честные методы продвижения своего продукта, заявленные возможности оказались не до конца реализованы, и мне это не понравилось. А потом я хотел иметь легальное программное обеспечение. Купил на какой-то выставке редактор Ami Pro, за \$100, и вдруг оказалось, что функциональность идентична редактору от Mircosoft, но я не могу им пользоваться, потому что невозможен ни с кем никакой обмен данными. Я понял, что вся система функционирует против потребителей. И потом, когда я работал в ассоциации потребителей, моя специализация была "--- защита потребителей от монополистов, главные направления "--- в IT  и в отоплении.

{\noindent \bf L: Т.~е. интерес к свободному ПО родился из соображений защиты потребителя?}

{\noindent \bf М:} Да. Причем в начале 2000-х, когда я первый раз инсталлировал себе Linux, я знал, что будут проблемы с драйверами, будет тяжело, но я хотел показать на своем примере, что это возможно. Наверное, полгода прошло, пока у меня начало работать все, что я хочу, а еще через полгода я начал чувствовать, что  обратно вернуться не хотелось бы. Сейчас\ldots У нас выпускают дистрибутив Baltix на базе Ubuntu, когда просто ставишь компакт"=диск, нажимаешь <<Инсталлировать>>, и имеешь полный набор программ для своего десктопа, в отличие от закрытого программного обеспечения, где инсталлируешь операционную систему, потом офисный пакет, потом антивирусы всякие\ldots Понимаешь, что в Linux  на порядок быстрее и проще все сделать. Сейчас уже совсем другая ситуация, все в пользу Linux. А почему люди не переходят?

{\noindent \bf L: А почему?}

{\noindent \bf М:} А потому, что люди консервативны, наверное. Я администрировал компьютеры в одной фирме, учредитель поручил проверить, все ли у нас легально, и оказалось, что Microsoft Office не оплачен. Тогда я предложил директору OpenOffice, бесплатно, и он конечно же был за. Но я опасался, что многие сотрудники будут против, и проинсталлировав его, не сказал им, что это новая программа, а сказал: <<Я обновляю наш офис>>. Все спокойно это восприняли, и даже сказали, что у них на порядок стало лучше управление таблицами. Люди боятся даже не программы, а нового названия.

\section[Даниэль Надь, Будапешт, Венгрия (LVEE 2011)]{Даниэль Надь, Будапешт, Венгрия \linebreak (LVEE 2011)}

%\begin{figure}[ht]
%\centering{\includegraphics[width=4cm]{49_spons_altoros.jpg}}
%\end{figure}

{\noindent \bf Даниэль Надь:} Каждый второй семестр я веду курсы по криптографии в Будапештском университете имени Лоранда Этвёша на факультете естественных наук, на кафедре операционных исследований.

{\noindent \bf L: Каких специалистов готовит кафедра ?}

{\noindent \bf Д:} В основном математиков, и кроме этого есть еще такая специальность, <<математик-программист>>. Ожидаемое место их последующей работы "--- это либо академическая сфера, либо индустрия информационных технологий.

{\noindent \bf L:  Как в учебном процессе используется свободное ПО?}

{\noindent \bf Д:} Достаточно широко. В общем-то, во всех сферах деятельности. Терминалы, которыми пользуются студенты, на Linux. Сейчас это тонкие клиенты. Очень много свободного софта типа Octave, но в любом случае есть конечно и коммерческие приложения, которые в академической сфере очень распространены. 

{\noindent \bf L: В том числе и потому, что приходится ориентироваться на рынок труда?}

{\noindent \bf Д:} Частично да, а частично потому, что некоторым приложениям, типа Mathematica, пока еще полноценной свободной замены нет. Но там где можно, совободное ПО широко используется. Большинство статей пишется в \TeX, графики рисуются в gnuplot "--- это две программы, которые знает каждый, кто прошел через кафедру математики.  

Unix-подобные системы, еще до того, как они стали свободными, получили достаточно широкое распространение в венгерских вузах, потому что еще в начале девяностых или даже в конце восьмидесятых по некоторым западным грантам было получено оборудование с Solaris от Sun, AIX от IBM, поэтому Unix в венгерской системе образования уже очень давно, и менять его по частям на свободное ПО было легко и безболезненно для преподавательского состава и для научных работников, и естественно, для студентов также. Резкого перехода в духе <<Раньше мы использовали что-то другое, а теперь будем использовать Linux>> "--- не было. Постепенно, терминал за терминалом это старое железо от Sun менялось на Linux-станции.

{\noindent \bf L: А как сам ты заинтересовался свободным ПО, помнишь?}

{\noindent \bf Д:}  Это было в другом университете. Я сам учился в техническом университете, и там я впервые увидел Интернет, как таковой, старые Unix'ы, которые тогда еще были новыми. Мне очень нравилась система Silicon Graphics. И когда в одной из лабораторий поставили на клиенты Linux, я понял, что на дешевом домашнем компьютере, который я могу иметь, тоже может быть что-то похожее. Тогда я начал интересоваться Linux, и\ldots пошло-поехало.

{\noindent \bf L: Как вообще свободное ПО чувствует себя в Венгрии?}

{\noindent \bf Д:} Достаточно комфортно. В Венгрии есть люди с большим энтузиазмом и большой энергией, которые его распространяют, так что венгерская локализация  достаточно качественная. Потом, по причине восточно-европейского безденежья очень многие принимают решение пользоваться свободным ПО. Единственное, где у свободного ПО серьезные трудности "--- это крупные компании и государственные учреждения. С бесплатного "--- и откатов нет :)

\section[Виталий Липатов, Санкт-Петербург, Россия (LVEE Winter 2012)]{Виталий Липатов, Санкт-Петербург, \linebreak Россия (LVEE Winter 2012)}

Незадолго до конца LVEE Winter 2012 мы взяли интервью у Виталия Липатова "--- члена команды разработчиков  дистрибутива ALT Linux и генерального директора компании Etersoft, известной в первую очередь своей собственной коммерческой версией проекта wine. Вопросы от имени LVEE задавали Дмитрий Костюк и Инна Рыкунина.

{\noindent \bf LVEE: Вопрос, который всегда легко задавать: а как получилось, что ты заинтересовался свободным ПО?}

{\noindent \bf Виталий Липатов:} Где-то году в 92-м или 93-м папа купил мне книжку <<UNIX "--- универсальная среда программирования>>. Тогда я её прочитал и ничего не понял, большие системы в начавшуюся эпоху персональных компьютеров казались далёкими, как динозавры. А лет через шесть, когда я заканчивал ВУЗ, меня пригласили работать в центральный НИИ судовой электротехники в лабораторию, где разрабатывалось программное обеспечение для пригородной электрички. Я попал в проект не с самого начала, и когда пришёл, выбор операционной системы уже был сделан, а выбирали между QNX и Linux. 

{\noindent \bf L: А что вообще операционная система делает в электричке?}

{\noindent \bf В:} Она стоит на контроллерах, которые управляют тормозами, дверями, скоростью движения,  пультом машиниста (там монитор, который выводит все показатели, лампочки зажигаются предупредительные)\ldots в поезде множество подсистем, они традиционно разрабатываются разными организациями, как, например, система безопасности электропоезда. Интересно, что тогда мы одни из первых начали применять Ethernet, у нас был проложен кабель сквозь весь поезд. Ранее обычно применялись разные стандартные и нестандартные протоколы на базе последовательного интерфейса.

{\noindent \bf L: Вагоны в локальной сети...}

{\noindent \bf В:} Да, это был 2000 год. И QNX не выбрали "--- потому, что если умножить стоимость лицензии QNX на количество вагонов, которые предполагалось выпускать серийно\ldots Выбрали Linux, тогда это был Slackware 3.5 с ядром 2.0.34, по-моему. Тогда уже, конечно, существовали и другие дистрибутивы...

{\noindent \bf L: Но люди любили Slackware, поэтому электричка работала на ней.}

{\noindent \bf В:} Это была первая Linux-система, которую я увидел, и она меня очень впечатлила. Так я познакомился с Linux. Немного поработав с ним, почувствовал, что освоить это невозможно. Бесчисленное количество команд с разными параметрами, и совершенно нереально запомнить, что у <<cp>>  есть параметр <<-a>>, а у <<tar>> нужно писать <<xvfz>>, чтобы что-нибудь распаковать\ldots Но потом мозг как-то со всем этим справился, и я понял, что это та самая система, о которой я всегда подсознательно мечтал. Недостатков даже не было заметно. И вообще, тогда в Linux не было недостатков, они появились намного позже :)

{\noindent \bf L: Тогда их ещё не успели написать :)}

{\noindent \bf В:} И вот, мы для 386 процессоров сделали систему, и всё очень быстро работало.

{\noindent \bf L: Проект был закончен?}

{\noindent \bf В:} Да, электричку мы сделали, но её не пустили в серию, к сожалению. Был изготовлен поезд в шесть вагонов, он прошёл все испытания.

{\noindent \bf L: В общем, интерес к Linux заметно пережил этот проект. А собственно интерес к свободному ПО?}

{\noindent \bf В:} Он появился немного позже. Эта мысль достаточно долго входила в сознание. Я перечитывал ту книжку про Unix, а ещё как раз недавно появился Интернет, и я узнал, что есть другие люди, которые под это что-то пишут, и втянулся. А потом попал в ALT Linux Team.

{\noindent \bf L: А кстати, как это произошло?}

{\noindent \bf В:} Для дальнейшей деятельности нам потребовался новый дистрибутив, а выбирали мы почему-то между только появившимся Mandriva RE Spring 2001 и ASP Linux. Я долго выбирал. ASP Linux выглядел красиво и профессионально. Но интуитивно меня тянуло к собранному в ALT Linux дистрибутиву. Сейчас я уже не помню подробностей. Возможно, уже тогда я заметил наличие сообщества разработчиков.

Параллельно с НИИ я работал в компании, где  писал конфигурации для 1С, причем с нуля. Работал один, написал их три штуки для разных направлений деятельности, заодно что-то администрировал "--- скучная такая деятельность для творческого человека. Там у меня был сервер на Linux, на котором лежали файлы баз данных для 1С:Предприятия. А и одновременно это был мой рабочий компьютер и мне нужно было на нём писать код, поэтому я в итоге озаботился, а как бы запустить 1С:Предприятие в Linux. Это был год, наверное, 2002 или 2003.  

{\noindent \bf L: Вот, оказывается, откуда растёт WINE@Etersoft!}

{\noindent \bf В:} Это уже позже, в 2004 году на выставке Softool в Москве мы от имени свежеорганизованного Etersoft показали, что на  WINE@Etersoft запускается 1С:Предприятие. Причём показывали, подключаясь к Linux-серверу с тонкого клиента, то есть в режиме терминального доступа. А в то время я только знал, что есть такой пакет wine, позволяющий запускать Windows-программы, попробовал его\ldots Долго подбирал настройки и библиотеки, и в итоге 1С:Предприятие запустилось. Работать оказалось не совсем возможным, слишком много багов\ldots Тогда я уже был пользователем ALT Linux. В то время wine там собирал генеральный директор ALT Linux Алексей Евгеньевич Новодворский. Пару раз он исправлял пакет по моим замечаниям, а потом прислал письмо в рассылку разработчиков ALT Linux: <<Уважаемые коллеги, представляю вам нового сопровождающего пакет wine Виталия Липатова, прошу любить и жаловать>>. То есть меня как бы поставили перед фактом. А я был очень благодарен за оказанное доверие\ldots 

Но Etersoft стал заниматься wine позже: когда мы наконец открылись и думали внедрять ALT Linux в офисах, оказалось, что никому это вообще-то не нужно, потому что для Linux нет необходимых прикладных программ. Тут мы поняли, что в офисах должно работать 1С:Предприятие, причем не просто запускаться, а реально работать, в сетевом режиме, когда есть много пользователей. Именно это востребовано, а особенно на терминальном сервере, т.\,е. в режиме, который позволяет получить от 1С:Предприятия версии 7.7 максимум производительности. Мы на это нацелились, сделали, и в конце 2005 года начали продавать первые версии. Собственно, первую версию продали уже чуть ли не под новый год.

{\noindent \bf L: Новогодний подарок. Оно работало?}

{\noindent \bf В:} Весьма относительно\ldots Но у нас были принципы, которых мы придерживались:  бухгалтерия в нашей компании использовала \linebreak 1С:Бухгалтерию, и мы запускали её под Linux.

{\noindent \bf L: А кстати, как по поводу изменений в wine? Этот проект ведь прославился тем, что периодически происходят изменения в апстриме, когда что-то внезапно может сломаться. Насколько часто и насколько болезненно такая ситуация бьёт по коммерческой версии?}

{\noindent \bf В:} Раньше, много лет назад, когда релизов wine почти не было, просто забирали из cvs какой-то срез кода и пытались им пользоваться. Стабильность действительно оставляла желать лучшего. Но потом разработчики wine перешли к концепции, что после любого изменения код не должен деградировать, стали требовать этого от разработчиков, появилась система тестирования. Требования ужесточили, в итоге сейчас wine перешёл на чёткие релизы. Каждый день новые коммиты, и каждую неделю выходит новый релиз. Причем невозможно прислать изменения, если ты перед этим не написал тест, который подтвердит, что  проблема есть, а после твоего исправления она исчезает. Эти тесты требуют не просто так "--- в проекте работает автомат, который прогоняет их на десяти разных конфигурациях Windows и Linux и проверяет, нет ли деградации.

Из-за достаточно серьёзного тестирования новых патчей сейчас так сильно всё не ломается. Но до последнего времени мы свою текущую ветку разработки синхронизировали очень редко. У нас мало ресурсов, поэтому мы выбирали некоторую стабильную для нас версию оригинального wine. Дальше мы уменьшали количество багов в ней, не синхронизируясь с основной веткой. Но за последние два года wine дошел наконец до версии 2.0. Мы синхронизировались и написали робота, который периодически  переносит изменения из апстрима в нашу ветку разработки. После сборки прогоняются тесты, проверяется, не появилось ли регрессий. Не так идеально, как в оригинальном wine, но всё-таки тесты мы запускаем.

{\noindent \bf L: Вы передаёте свои  исправления в апстрим?}

{\noindent \bf В:} Конечно. Раньше "--- больше, апстрим wine был менее строгим. И сейчас иногда принимают что-то. Теоретически мы должны от этого выигрывать, но на практике этого не замечаем. Количество хаков, которые мы не можем послать, просто потому они не будут приняты, больше, чем то, что у нас приняли. Мы специально для патчей, которые у нас должны бы принять, ведём отдельную ветку. Она свободная, и wine в ALT Linux собирается именно из неё. На нашем ftp-сервере доступны сборки и под другие системы.

{\noindent \bf L: Насколько широко расходится  WINE@Etersoft как коммерческий продукт?}

{\noindent \bf В:} Я так понимаю, что это зависит от роста популярности Linux. Процесс взаимосвязанный. Иногда я слышу, что без нашей деятельности по помощи в запуске Windows-программ в таком объёме движения не было бы. Раньше, когда компанию 1С просили сделать версию под Linux, они отвечали: <<У нас там нет пользователей>>. Мы этот порочный круг разомкнули, и появились пользователи под Linux. Теперь в компании 1С сделали Linux"=версию, пока только серверную часть, их самый дорогой продукт. Клиент может быть запущен либо под Windows, либо можно подключиться через браузер, но это доступно не для всех конфигураций, и не вся функциональность доступна, поэтому наше решение по запуску Windows"=клиента остаётся востребованным.

На этом проекте нам не удаётся много зарабатывать, и, соответственно, быстро его развивать. Специфика такова, что готовых разработчиков не существует, и новых сотрудников нам приходится обучать практически с нуля "--- как писать код, как отлаживать, какие приёмы использовать для поиска ошибок в бинарном коде. Деятельность не очень благодарная и мало кому нравится.

{\noindent \bf L: А вы не эксплуатируете практикантов?}

{\noindent \bf В:} Вообще-то студенты проходят к нам на практику, но в основном те, кто уже у нас работает. А так\ldots В городе множество компаний, предлагающих более высокие зарплаты, но у нас есть другие преимущества:  свободный график, особая атмосфера, интересная работа\ldots Для сотрудников разработана целая система,  интегрированная  Bugzilla, которая высчитывает рабочее время, учитывает решённые задачи.

{\noindent \bf L: Если в двух словах коснуться ваших продуктов, менее известных широкой публике, чем wine?}

{\noindent \bf В:} Наши другие продукты "--- во многом эксперимент, что же ещё будет востребовано по теме миграции на Linux. Есть SQL-транслятор SELTA@Etersoft, который позволяет отказаться от Microsoft SQL Server. Версию 2.0 хотим сделать в виде прокси-сервера, со стороны сети выглядящего точно как MS SQL. Там много сложностей: хранимые процедуры, встроенные функции\ldots Это пример полностью нашего коммерческого продукта, построенного, правда, на нашей собственной сборке PostgreSQL, которая конечно же свободно доступна. А пример нашего свободного продукта "--- UniOffice, который  подменяет Microsoft Office на OpenOffice для VBA-приложений. Есть у нас и решение для организации терминального доступа "--- RX@Etersoft, основанное на NX. Ещё у нас есть совершенно другая деятельность: мы делаем промышленные системы управления. 

{\noindent \bf L: Например?}

{\noindent \bf В:} Например, мы разрабатываем ПО для электростанций в России. К нам обратилась компания, по заказу которой производились преобразователи частоты в Китае. Там делали железо и программу управления к нему под Windows. Она закрывалась каждые несколько часов из-за различных ошибок… А тут требовалась бесперебойная работа в течение года. Там всё очень серьёзно, преобразователь управляет нагнетателем, подающим воздух для горения газа, и если пропорция смеси нарушается, всё чревато аварийной ситуацией. Нас нашли, уговорили, мы в очень сжатые сроки сделали этот заказ, последние штрихи наводили уже на электростанции. И вот система уже более 5 лет работает, без сбоев. Мы делаем и другие подобные системы управления для судов:. Не системы навигации, там свои поставщики, а мониторинг, контроль, системы сигнализации, пульты в рубку и др. И в том и том случае на базе ALT Linux. 

{\noindent \bf L: На нём плавают корабли и работают электростанции. А как твои сегодняшние впечатления? На что похоже LVEE?}

{\noindent \bf В:} Достаточно необычно. Я участвовал в двух видах мероприятий: тех, что проводятся ALT Linux "--- в основном с людьми, близкими к ALT Linux Team, в них ещё сильна академическая составляющая. И другого типа "--- те, что проводятся крупными спонсорами, такими как IBM или Sun, для своих целей, в фешенебельных отелях,  на тысячи людей.

У вас отличается формат и люди по глубине задействованности: совершенно другой пласт, с которым на тех мероприятиях не сталкиваешься. Мне очень понравилось мероприятие, рад был встретить такое количество небезразличных к свободному ПО людей.

\end{document}
