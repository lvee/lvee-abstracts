\documentclass[10pt, a5paper]{article}
\usepackage[T2A]{fontenc}
\usepackage{ucs}
\usepackage[utf8x]{inputenc}
\usepackage[polish,english,russian]{babel}
\usepackage{hyperref}
\usepackage[inner=2cm,top=1.8cm,outer=2cm,bottom=2.3cm,nohead]{geometry}
\usepackage{listings}
\usepackage{graphicx}
\usepackage{wrapfig}
\usepackage{longtable}
\usepackage{indentfirst}
\frenchspacing
\usepackage{fixltx2e} %text sub- and superscripts
\usepackage{icomma} % коскі ў матэматычным рэжыме
\PreloadUnicodePage{4}

\newcommand{\longpage}{\enlargethispage{\baselineskip}}
\newcommand{\shortpage}{\enlargethispage{-\baselineskip}}

\def\switchlang#1{\expandafter\csname switchlang#1\endcsname}
\def\switchlangbe{
\let\saverefname=\refname%
\def\refname{Літаратура}%
\def\figurename{Іл.}%
}
\def\switchlangen{
\let\saverefname=\refname%
\def\refname{References}%
\def\figurename{Fig.}%
}
\def\switchlangru{
\let\saverefname=\refname%
\let\savefigurename=\figurename%
\def\refname{Литература}%
\def\figurename{Рис.}%
}

\hyphenation{admi-ni-stra-tive}
\hyphenation{ex-pe-ri-ence}
\hyphenation{fle-xi-bi-li-ty}
\hyphenation{Py-thon}
\hyphenation{ma-the-ma-ti-cal}
\hyphenation{re-ported}
\hyphenation{imp-le-menta-tions}
\hyphenation{pro-vides}
\hyphenation{en-gi-neering}
\hyphenation{com-pa-ti-bi-li-ty}
\hyphenation{im-pos-sible}
\hyphenation{desk-top}
\hyphenation{elec-tro-nic}
\hyphenation{com-pa-ny}
\hyphenation{de-ve-lop-ment}
\hyphenation{de-ve-loping}
\hyphenation{de-ve-lop}
\hyphenation{da-ta-ba-se}
\hyphenation{plat-forms}
\hyphenation{or-ga-ni-za-tion}
\hyphenation{pro-gramming}
\hyphenation{in-stru-ments}
\hyphenation{Li-nux}
\hyphenation{en-vi-ron-ment}
\hyphenation{Te-le-pathy}
\hyphenation{Li-nux-ov-ka}

\def\progref!#1!{\texttt{#1}}
\renewcommand{\arraystretch}{2} %Іначай формулы ў матрыцы зліпаюцца з лініямі
\usepackage{array}

\def\interview #1 (#2), #3, #4, #5\par{

\section[#1, #3, #4]{#1, #5}
\def\qname{LVEE}
\def\aname{#1}
\def\q ##1\par{{\noindent \bf \qname: ##1 }\par}
\def\a{{\noindent \bf \aname: } \def\qname{L}\def\aname{#2}}
}

\begin{document}
\title{Интервью с участниками}
%\author{}
\date{}
\maketitle

По сложившейся традиции в сборник материалов включены интервью, взятые представителями оргкомитета у участников конференции: как сразу после LVEE 2011, так и во время впервые проведенной в этом году выездной зимней сессии конференции LVEE Winter 2012. Участники рассказывают о себе и своей работе, делятся планами, высказывают мнения по актуальным вопросам, волнующим сообщество. 

\section[Николай Маржан, Киев, Украина (LVEE 2010)]{Николай Маржан, Киев, Украина ~~~~\linebreak (LVEE 2010)}
%\begin{figure}[ht]
%\centering{\includegraphics[width=4cm]{49_spons_altoros.jpg}}
%\end{figure}

{\noindent \bf Николай Маржан:} Я работаю в компании PortaOne. Вот уже 3.5 года. Мои обязанности "--- это релиз-инжиниринг и, на данный момент, обеспечение работы системы мониторинга, которую я разработал с нуля.

{\noindent \bf LVEE: А чем занимается компания?}

{\noindent \bf Н:} Мы разрабатываем свои VOIP"=решения, основной продукт это VOIP"=биллинг и, соответственно, все сопутствующие сервисы.

{\noindent \bf L: В работе вам приходится использовать свободное ПО?}

{\noindent \bf Н:} Конкретно в моей работе "--- да. Я это очень приветствую, и все свои наработки мы отдаем под той же лицензией (GPL) нашим клиентам. Т.~е. если клиент купил наше решение, мы отдаем исходные коды, а если там GPL "--- под ней же свои доработки.

{\noindent \bf L: А как вообще возникла твоя симпатия к свободному ПО?}

{\noindent \bf Н:} Трудно сказать. Проще вспомнить, как появилась симпатия к Unix-way. В 16 лет я написал свой собственный форум на perl. И после этого влюбился в Unix. 
На бесплатном хостинге \url{agava.ru} тогда были только PHP и perl. Я посмотрел на PHP и понял, что это не мое. А perl\ldots Не считаю себя в нем гуру, но он мне очень нравится. 

{\noindent \bf L: И так мир Unix просочился\ldots}

{\noindent \bf Н:} Через эту лазейку.

{\noindent \bf L: Сейчас тебе приходится вплотную заниматься системным администрированием?}

{\noindent \bf Н:} Как раз системное администрирование от меня ушло почти полностью, кроме отладки системы мониторинга, и 95\% времени я являюсь разработчиком. Занимаюсь релиз-инжинирингом, т.е. подготавливаю наш продукт к деплойменту. Готовлю инсталляционные диски, апгрейд-процедуры, в общем делаю все, чтоб деплоймент прошел как можно легче. 

{\noindent \bf L: А мониторинг?}

{\noindent \bf Н:} На прошлой работе я 50\% времени занимался разработкой системы мониторинга, и вот, на новой "--- решили использовать накопленный опыт и предложили написать ее полностью с нуля.

В маленьких компаниях мы идем от готовых решений к реализации. В PortaOne я смог отталкиваться от того, что нужно потребителю "--- в данном случае отделу технической поддержки "--- к тому, как это реализовать. Невзирая на сроки, ориентируясь исключительно на качество. Это была главная идея, и благодаря ей я смог реализовать все что было задумано, без недостатков моих ранних систем мониторинга.

{\noindent \bf L: Какое свободное ПО было задействовано в процессе?}

{\noindent \bf Н:} Из свободного ПО был использован проект nagios, был использован транспорт nagios'а, а это отдельный проект. Были позаимствованы кое"=какие скрипты построения графиков. Но на данный момент nagios можно заменить буквально тремя тысячами строк кода на perl, он сейчас используется просто как планировщик заданий. Т.~е. разработка переросла стадию инфраструктуры, связывающей несколько проектов, и стала самостоятельным проектом.

{\noindent \bf L: За пределами работы ты, вероятно, тоже пользуешься свободным ПО?}

{\noindent \bf Н:} Да, на своем десктопе 3 года использовал FreeBSD, но потом неудачно купил ноутбук :). Так что пришлось поставить на него Ubuntu. И теперь она на всех моих машинах.

{\noindent \bf L: Переход с FreBSD дался легко?}

{\noindent \bf Н:} Настолько легко, что трудно даже представить (это ведь не наоборот). На самом деле я считаю себя скорее BSD-гуру, чем Linux. Я мэйнтейнер нескольких портов FreeBSD и посто ее люблю.

О портах\ldots На самом деле они исходят из моих потребностей по работе. Что"=то было нужно "--- сделал порт. А потом подготовил его для включения в мэйнстрим. Мы, как комьюнити, должны отдавть свою работу "--- иначе\ldots Иначе мы не достигнем вселенского счастья. Если я отдам свою крупицу сообществу, кто"=то ей воспользуется.

\section{Наим Шафиев, Москва, Россия (LVEE 2010)}

%\begin{figure}[ht]
%\centering{\includegraphics[width=4cm]{49_spons_altoros.jpg}}
%\end{figure}

{\noindent \bf Наим Шафиев:} Я из Москвы, работаю в компании ECUmoney Limited, параллельно учусь. Я могу себя позиционировать как поборник свободного ПО в его истинном проявлении, полностью без несвободных лицензий. Интересуюсь perl, пишу на нем, иногда приходится работать и с PHP, и с Pithon, и, боже мой, даже с Delphi. 

Но на самом деле я придерживаюсь главной теории экономики, что разделение труда "--- это великое благо, потому что при разделении труда выделяются наиболее способные люди для определенных вещей, соответственно повышается общая производительность труда. Соответственно, мой основной язык программирования это perl, я на нем программирую с 2001 года, но плотно занялся года три назад, можно сказать, к нему возвратился. Почему это так? Надо сказать, что perl 2001 года был не очень хорош, особенно для web. Использование CGI-обработки приложений является не очень хорошим способом обработки. Оно слишком низкоуровневое. Просто на тот момент ничего больше не было. Тот же PHP был посто ужасен на самом деле "--- он ничего не поддерживал. Кстати, я доклад прочитал миенно по поводу модернистического perl [на LVEE 2010], там я указал некоторые базовые направления, по которым стоит развиваться и на что стоит смотреть.

{\noindent \bf L: А свободное ПО "--- это больше профессиональный интерес или личные предпочтения?}

{\noindent \bf Н:} Я пытаюсь пользоваться только открытыми решениями, начиная от видеокарт и заканчивая\ldots буквально всем. Т.~е. если под устройство отсутствуют свободные драйвера "--- я его не использую. По крайней мере, стараюсь: сейчас все"=таки приходится использовать Radeon, там над открытыми драйверами, насколько я знаю, сейчас работают всего два челоаека\ldots Что в принципе ужасно. 

О профессиональной деятельности. В ECUmoney Limited мы используем практически полностью открытый стек. Используется CentOS. Хотя у нас был сервер на RHEL, но я считаю, что нет смысла платить за поддержку, которую я могу выполнять сам. ECUmoney Limited "--- мое основное место работы "--- это проект, который занимается веб-деньгами. Это новозеландский банк, и он осуществляет все операции вплоть до депонирования и работы на бирже. Менеджмент в Москве, соответственно там же я и работаю. 

{\noindent \bf L: Эта строгая позиция в отношении свободного ПО и непримиримость к двойным стандартам "--- как она возникла?}

{\noindent \bf Н:} С детства интересовался экономикой, особенно политэкономией. В принципе очевидно, что любые патенты, ограничения, проприетарщина "--- это ограничения свободы, и за счет этого происходит ограничение развития человечества. Могу привести одну цитату. Билл Гейтс, человек, который до недавнего времени руководил самой большой компанией по разработке проприетарного ПО, говорил: <<Если бы патенты существовали в том виде, в каком они существуют сейчас "--- никакая индустриальная революция, которая полностью перевернула жизнь людей в Европе в XVII веке, не прошла бы>>.

Что касается Unix-way "--- в 2001 году я начал изучать computer science, прочитал одну американскую книжку про альтернативные операционные системы. Тогда найти диск с FreeBSD даже в Москве было довольно сложно, но я нашел и попытался его установить. С первого раза конечно же ничего не получилось.  И жесткий диск я стер (но это ничего, я отношусь к тем админам, которые все"=таки делают бэкапы). Пришлось распечатать весь хэндбук, он был действительно огромный\ldots  И вот так я потихонечку пересел на свободное ПО. Сначала это была FreeBSD, потом Mandrake, потом Fedora, была еще пара тестовых дистрибутивов. Писал даже свой дистрибутив для встраиваемых решений. А в конце остановился на Debian: решил, что раз они наиболее сильные поборники свободного ПО, стоит на нем остановиться.

{\noindent \bf L: А участие в свободных проектах?}

{\noindent \bf Н:} В perl есть CPAN "--- репозиторий свободных проектов. Я в CPAN-е, можете набрать \url{search.cpan.org/~naim/} и увидите мой проект. На самом деле у меня разработок довольно много, но часть пока еще не готова для релиза. Моя разработка AnyEvent Benchmark "--- это довольно удобный фреймворк для осуществления собственных HTTP и HTTPS"=тестинга, для тестирования SOAP и других вещей. Сейчас она в разработке, я сам ее использую, как и еще пара членов \url{moscow.pm.org/}.

{\noindent \bf L: А как ты оказался на LVEE?}

{\noindent \bf Н:} Это довольно сложно. Информация о LVEE несколько раз появлялась на \url{linux.org.ru}, а я на этом ресурсе с 2003 года. 
Первый раз хотел приехать еще в 2009 году\ldots Для меня оказалось просто шоком, что в Беларуси проводится настолько прогрессивное мероприятие, формат которого на мой взгляд может в чем"=то соперничать с FOSSCON, плюс уровень разработчиков довольно хороший. 


%\section[Михаил Шалаев, Мюнхен, Германия (LVEE 2010)]{Михаил Шалаев, Мюнхен, Германия \linebreak (LVEE 2010)}

%\begin{figure}[ht]
%\centering{\includegraphics[width=4cm]{49_spons_altoros.jpg}}
%\end{figure}

%{\noindent \bf Михаил Шалаев:} Родом я из Харькова, с Украины. Работаю в фирме, которая находится в Германии, в Мюнхене, и называется GeNUA, сокращение от\ldots  как это по"=русски\ldots сервисы для юникс и нетворк администрэйшн.

%{\noindent \bf L: Как появился ваш интерес к свободному ПО?}

%{\noindent \bf М:} Ну, когда я был совсем маленьким еще, под стол проходил, меня мама привела на работу, где стояла СМ ЭВМ, на которой смурные дядьки гоняли BSD 2.9. Точнее, советскую версию его, Демос 2.0, который был сделан частично из BSD 2.9, частично из System 3, по"=моему. Понятное дело, с русификацией, с переписанными мануалами и всеми делами. И смурные бородатые дядьки рассказали мне, что это и есть на самом деле фри операционная система, распространяется не за деньги, а любой человек может приехать в Беркли, привезти с собой большую тэйпу, откатать на нее себе дистрибутив и поехать домой пользователем. Вот. И потом как-то получилось так, даже не помню как, что я пользовал разнообразные BSD, немножко баловался с 386 BSD, потом с NetBSD, а потом в 95-м году образовался проект OpenBSD, я в нем участвовал девелопером некоторое время, и вот так это все и происходило. И продолжает.

%{\noindent \bf L:  Удается совмещать интерес к свободному ПО и к BSD-системам в частности с профессиональной деятельностью?}

%{\noindent \bf М:} Какое-то время я работал сисадмином в разнообразных конторах, и мы использовали BSD"=системы для роутеров, серверов, DNS"=серверов, и в частности я участвовал в разработке пакет"=фильтров для OpenBSD в свое время, потому что работал в небольшом ISP, и мы занимались фильтрованием как раз. 

%{\noindent \bf L: Получается, что ваш код есть в одном из самых надежных\ldots}

%{\noindent \bf М:} И не только :)

%{\noindent \bf L: Чем вы сейчас занимаетесь по работе?}


%{\noindent \bf М:} Я работаю в конторе, в которой мы, опять же, делаем файрволы. В данный момент из OpenBSD, раньше это был BSDi, но BSDi загнулся. Собственно чем я занимаюсь, это починкой злобных бугов, которые вылазят под безумной нагрузкой, поскольку некоторые клиенты, которые используют наше железо\ldots Собственно железо это разнообразные PC, от маленьких совсем до многопроцессорных гробов. И при большой нагрузке как правило вылазят буги, которые никто никогда не видел. Тут неважно что использовать, Linux или FreeBSD, или OpenBSD, все равно эти буги будут вылазить, потому что то что мы делаем, скорее всего никто другой не делает. И собственно то, чем я занимаюсь "--- это починка ядер OpenBSD.  Кроме того что мы делаем файрволы, мы людям помогаем организовывать их нетворк на базе того, что у них уже есть, плюс того что они еще хотят "--- в частности, опять"=таки наши файрволы. Фирма по-моему около 15 лет существует, сертифицирована для использования в правительстве. Небольшая, порядка 100 чаловек, большинство девелоперов являются или являлись OpenBSD"=девелоперами. 

%{\noindent \bf L: За последнее время отношение к свободному ПО сильно меняется? В сфере, если не конечных пользователей, то решений?}

%{\noindent \bf М:} Насколько вижу я, конечным клиентам все равно, что оно и как оно. Их волнуют две вещи: делает ли оно, что должно, и сколько оно стоит. Ну а бюджетные организации даже не волнует, сколько оно стоит "--- их волнует, чтоб оно работало. А что лежит в основе? Есть вероятно в этом какая"=то маркетинговая цена, потому что все читают журналы, видят как все хорошо в мире свободного программного обеспечения, но в конечном итоге они не покупают продукт, основанный на Linux, или BSD, или еще на чем-нибудь, а то, что отвечает их нуждам. В частности, наши железяки покупают потому, что есть не так много альтернатив. Мы делаем не простой пакет"=фильтеринг, мы делаем и разборку протокола, и сканирование на вирусы, и все остальное.

%{\noindent \bf L: Иными словами, эти системы выигрывают из"=за преимуществ?}

%{\noindent \bf М:} Отчачти из"=за того что есть сорс"=код, который можно починить, если что. Потому что основная проблема с  коммерческими системами "--- чинить их невозможно. Или если хочешь сделать что"=то свое на их основе. С другой стороны, есть отрицательный вариант в плане фирм, использующих free software: не всегда они дают назад то, что наваяли, или с другой стороны вместо того чтобы самим ваять, они ждут пока кто"=нибудь сделает. В результате получается, что большие конторы наживаются на маленьких конторах, которые собственно занимаются разработкой. Но я думаю, это явление временное отчасти, потому что большие конторы в конечном итоге упрутся в то, что никто не делает то, что им нужно, и начнут делать это сами.

%А про LVEE мне товарищи сказали. Общались с товарищем из Минска, думали, как бы так организовать, чтобы повстречаться. И я просто спросил, не бывает ли каких смешных конференций в Беларуси. Cказали, есть такая конференция, даже смахивает на хакер"=кэмп, как их организуют в Европе, и я решил поинтересоваться: а как же оно там? Точнее, здесь, там-то я уже видел. 

%{\noindent \bf L: И как оно здесь?}

%{\noindent \bf М:} Хорошо. С поправкой на количество людей, типичное для Евросоюза, и на формат, как оно все проходит "--- вполне сопоставимо с тем, как оно проходит там. В частности небольшие организации такого рода тоже происходят, вот на рождество обычно. Один раз на рождество дядьки просто сняли трамвай длинный, <<колбасу>>, и гоняли его по Мюнхену, имели интернет по GSM"=у , загружались пивом на каждом углу, потому как места не было в трамвае, и так ездили целую ночь, вот был у них такой хакер"=эвент. Там же организовались у них выступления, там же ели, пили и всем остальным занимались. Небольшая для Европы организация, до ста человек их было, а народ где"=то похожий. Я так думаю, на подобных конференциях, или там симпозиумах, всегда есть люди, которые делают то же самое. Идея"=то простая: рассказать чего сам делаешь, послушать что другие делают, и кто"=то это все дело должен окучить, вне зависимости от страны и языка.

\section{Виталий Липатов, Санкт"=Петербург, Россия}

Незадолго до конца LVEE Winter 2012 мы взяли интервью у Виталия Липатова --- члена команды разработчиков  дистрибутива ALT Linux и генерального директора компании Etersoft, известной в первую очередь своей собственной коммерческой версией проекта wine. Вопросы от имени LVEE задавали Дмитрий Костюк и Инна Рыкунина.

{\noindent \bf LVEE: Вопрос, который всегда легко задавать: а как получилось, что ты заинтересовался свободным ПО?}

{\noindent \bf Виталий Липатов:} Где-то году в 92-м или 93-м папа купил мне книжку «UNIX --- универсальная среда программирования». Тогда я её прочитал и ничего не понял, большие системы в начавшуюся эпоху персональных компьютеров казались далёкими, как динозавры. А лет через шесть, когда я заканчивал ВУЗ, меня пригласили работать в центральный НИИ судовой электротехники --- в лабораторию, где разрабатывалось программное обеспечение для пригородной электрички. Я попал в проект не с самого начала, и когда пришёл, выбор операционной системы уже был сделан, а выбирали~--- между QNX и Linux. 

{\noindent \bf L: А что вообще операционная система делает в электричке?}

{\noindent \bf В:} Она стоит на контроллерах, которые управляют тормозами, дверями, скоростью движения,  пультом машиниста (там монитор, который выводит все показатели, лампочки зажигаются предупредительные)... в поезде множество подсистем, они традиционно разрабатываются разными организациями, как, например, система безопасности электропоезда. Интересно, что тогда мы одни из первых начали применять Ethernet, у нас был проложен кабель сквозь весь поезд. Ранее обычно применялись разные стандартные и нестандартные протоколы на базе последовательного интерфейса.

{\noindent \bf L: Вагоны в локальной сети...}

{\noindent \bf В:} Да, это был 2000 год. И QNX не выбрали --- потому, что если умножить стоимость лицензии QNX на количество вагонов, которые предполагалось выпускать серийно... Выбрали Linux, тогда это был Slackware 3.5 с ядром 2.0.34, по-моему. Тогда уже, конечно, существовали и другие дистрибутивы...

{\noindent \bf L: Но люди любили Slackware, поэтому электричка работала на ней.}

{\noindent \bf В:} Это была первая Linux-система, которую я увидел, и она меня очень впечатлила. Так я познакомился с Linux. Немного поработав с ним, почувствовал, что освоить это невозможно. Бесчисленное количество команд с разными параметрами, и совершенно нереально запомнить, что у «cp»  есть параметр «-a», а у «tar» нужно писать «xvfz», чтобы что-нибудь распаковать... Но потом мозг как-то со всем этим справился, и я понял, что это та самая система, о которой я всегда подсознательно мечтал. Недостатков даже не было заметно. И вообще, тогда в Linux не было недостатков, они появились намного позже :)

{\noindent \bf L: Тогда их ещё не успели написать :)}

{\noindent \bf В:} И вот, мы для 386-х процессоров сделали систему, и всё очень быстро работало.

{\noindent \bf L: Проект был закончен?}

{\noindent \bf В:} Да, электричку мы сделали, но её не пустили в серию, к сожалению. Был изготовлен поезд в шесть вагонов, он прошёл все испытания.

{\noindent \bf L: В общем, интерес к Linux заметно пережил этот проект. А собственно интерес к свободному ПО?}

{\noindent \bf В:} Он появился немного позже. Эта мысль достаточно долго входила в сознание. Я перечитывал ту книжку про Unix, а ещё как раз недавно появился Интернет, и я узнал, что есть другие люди, которые под это что-то пишут, и втянулся. А потом попал в ALT Linux Team.

{\noindent \bf L: А кстати, как это произошло?}

{\noindent \bf В:} Для дальнейшей деятельности нам потребовался новый дистрибутив, а выбирали мы почему-то между только появившимся Mandriva RE Spring 2001 и ASP Linux. Я долго выбирал. ASP Linux выглядел красиво и профессионально. Но интуитивно меня тянуло к собранному в ALT Linux дистрибутиву. Сейчас я уже не помню подробностей. Возможно, уже тогда я заметил наличие сообщества разработчиков.

Параллельно с НИИ я работал в компании, где  писал конфигурации для 1С, причем с нуля. Работал один, написал их три штуки для разных направлений деятельности, заодно что-то администрировал --- скучная такая деятельность для творческого человека. Там у меня был сервер на Linux, на котором лежали файлы баз данных для 1С:Предприятия. А и одновременно это был мой рабочий компьютер и мне нужно было на нём писать код, поэтому я в итоге озаботился, а как бы запустить 1С:Предприятие в Linux. Это был год, наверное, 2002 или 2003.  

{\noindent \bf L: Вот, оказывается, откуда растёт WINE@Etersoft!}

{\noindent \bf В:} Это уже позже, в 2004 году на выставке Softool в Москве мы от имени свежеорганизованного Etersoft показали, что на  WINE@Etersoft запускается 1С:Предприятие. Причём показывали, подключаясь к Linux-серверу с тонкого клиента, то есть в режиме терминального доступа. А в то время я только знал, что есть такой пакет wine, позволяющий запускать Windows-программы, попробовал его... Долго подбирал настройки и библиотеки, и в итоге 1С:Предприятие запустилось. Работать оказалось не совсем возможным, слишком много багов... Тогда я уже был пользователем ALT Linux. В то время wine там собирал генеральный директор ALT Linux Алексей Евгеньевич Новодворский. Пару раз он исправлял пакет по моим замечаниям, а потом прислал письмо в рассылку разработчиков ALT Linux: «Уважаемые коллеги, представляю вам нового сопровождающего пакет wine Виталия Липатова, прошу любить и жаловать». То есть меня как бы поставили перед фактом. А я был очень благодарен за оказанное доверие... 

Но Etersoft стал заниматься wine позже: когда мы наконец открылись и думали внедрять ALT Linux в офисах, оказалось, что никому это вообще-то не нужно, потому что для Linux нет необходимых прикладных программ. Тут мы поняли, что в офисах должно работать 1С:Предприятие, причем не просто запускаться, а реально работать, в сетевом режиме, когда есть много пользователей. Именно это востребовано, а особенно на терминальном сервере, т. е. в режиме, который позволяет получить от 1С:Предприятия версии 7.7 максимум производительности. Мы на это нацелились, сделали, и в конце 2005 года начали продавать первые версии. Собственно, первую версию продали уже чуть ли не под новый год.

{\noindent \bf L: Новогодний подарок. Оно работало?}

{\noindent \bf В:} Весьма относительно... Но у нас были принципы, которых мы придерживались:  бухгалтерия в нашей компании использовала 1С:Бухгалтерию, и мы запускали её под Linux.

{\noindent \bf L: А кстати, как по поводу изменений в wine? Этот проект ведь прославился тем, что периодически происходят изменения в апстриме, когда что-то внезапно может сломаться. Насколько часто и насколько болезненно такая ситуация бьёт по коммерческой версии? }

{\noindent \bf В:} Раньше, много лет назад, когда релизов wine почти не было, просто забирали из cvs какой-то срез кода и пытались им пользоваться. Стабильность действительно оставляла желать лучшего. Но потом разработчики wine перешли к концепции, что после любого изменения код не должен деградировать, стали требовать этого от разработчиков, появилась система тестирования. Требования ужесточили, в итоге сейчас wine перешёл на чёткие релизы. Каждый день новые коммиты, и каждую неделю выходит новый релиз. Причем невозможно прислать изменения, если ты перед этим не написал тест, который подтвердит, что  проблема есть, а после твоего исправления она исчезает. Эти тесты требуют не просто так --- в проекте работает автомат, который прогоняет их на десяти разных конфигурациях Windows и Linux и проверяет, нет ли деградации.

Из-за достаточно серьёзного тестирования новых патчей сейчас так сильно всё не ломается. Но до последнего времени мы свою текущую ветку разработки синхронизировали очень редко. У нас мало ресурсов, поэтому мы выбирали некоторую стабильную для нас версию оригинального wine. Дальше мы уменьшали количество багов в ней, не синхронизируясь с основной веткой. Но за последние два года wine дошел наконец до версии 2.0. Мы синхронизировались и написали робота, который периодически  переносит изменения из апстрима в нашу ветку разработки. После сборки прогоняются тесты, проверяется, не появилось ли регрессий. Не так идеально, как в оригинальном wine, но всё-таки тесты мы запускаем.

{\noindent \bf L: Вы передаёте свои  исправления в апстрим?}

{\noindent \bf В:} Конечно. Раньше --- больше, апстрим wine был менее строгим. И сейчас иногда принимают что-то. Теоретически мы должны от этого выигрывать, но на практике этого не замечаем. Количество хаков, которые мы не можем послать, просто потому они не будут приняты, больше, чем то, что у нас приняли. Мы специально для патчей, которые у нас должны бы принять, ведём отдельную ветку. Она свободная, и wine в ALT Linux собирается именно из неё. На нашем ftp-сервере доступны сборки и под другие системы.

{\noindent \bf L: Насколько широко расходится  WINE@Etersoft как коммерческий продукт?}

{\noindent \bf В:} Я так понимаю, что это зависит от роста популярности Linux. Процесс взаимосвязанный. Иногда я слышу, что без нашей деятельности по помощи в запуске Windows-программ в таком объёме движения не было бы. Раньше, когда компанию 1С просили сделать версию под Linux, они отвечали: «У нас там нет пользователей». Мы этот порочный круг разомкнули, и появились пользователи под Linux. Теперь в компании 1С сделали Linux-версию , пока только серверную часть, их самый дорогой продукт. Клиент может быть запущен либо под Windows, либо можно подключиться через браузер, но это доступно не для всех конфигураций, и не вся функциональность доступна, поэтому наше решение по запуску Windows-клиента остаётся востребованным.

На этом проекте нам не удаётся много зарабатывать, и, соответственно, быстро его развивать. Специфика такова, что готовых разработчиков не существует, и новых сотрудников нам приходится обучать практически с нуля — как писать код, как отлаживать, какие приёмы использовать для поиска ошибок в бинарном коде. Деятельность не очень благодарная и мало кому нравится.

{\noindent \bf L: А вы не эксплуатируете практикантов?}

{\noindent \bf В:} Вообще-то студенты проходят к нам на практику, но в основном те, кто уже у нас работает. А так... В городе множество компаний, предлагающих более высокие зарплаты, но у нас есть другие преимущества:  свободный график, особая атмосфера, интересная работа... Для сотрудников разработана целая система,  интегрированная  Bugzilla, которая высчитывает рабочее время, учитывает решённые задачи.

{\noindent \bf L: Если в двух словах коснуться ваших продуктов, менее известных широкой публике, чем wine?}

{\noindent \bf В:} Наши другие продукты --- во многом эксперимент, что же ещё будет востребовано по теме миграции на Linux. Есть SQL-транслятор SELTA@Etersoft, который позволяет отказаться от Microsoft SQL Server. Версию 2.0 хотим сделать в виде прокси-сервера, со стороны сети выглядящего точно как MS SQL. Там много сложностей: хранимые процедуры, встроенные функции... Это пример полностью нашего коммерческого продукта, построенного, правда, на нашей собственной сборке PostgreSQL, которая конечно же свободно доступна. А пример нашего свободного продукта --- UniOffice, который  подменяет Microsoft Office на OpenOffice для VBA-приложений. Есть у нас и решение для организации терминального доступа --- RX@Etersoft, основанное на NX. Ещё у нас есть совершенно другая деятельность: мы делаем промышленные системы управления. 

{\noindent \bf L: Например?}

{\noindent \bf В:} Например, мы разрабатываем ПО для электростанций в России. К нам обратилась компания, по заказу которой производились преобразователи частоты в Китае. Там делали железо и программу управления к нему под Windows. Она закрывалась каждые несколько часов из-за различных ошибок… А тут требовалась бесперебойная работа в течение года. Там всё очень серьёзно, преобразователь управляет нагнетателем, подающим воздух для горения газа, и если пропорция смеси нарушается, всё чревато аварийной ситуацией. Нас нашли, уговорили, мы в очень сжатые сроки сделали этот заказ, последние штрихи наводили уже на электростанции. И вот система уже более 5 лет работает, без сбоев. Мы делаем и другие подобные системы управления для судов:. Не системы навигации, там свои поставщики, а мониторинг, контроль, системы сигнализации, пульты в рубку и др. И в том и том случае на базе ALT Linux. 

{\noindent \bf L: На нём плавают корабли и работают электростанции. А как твои сегодняшние впечатления? На что похоже LVEE?}

{\noindent \bf В:} Достаточно необычно. Я участвовал в двух видах мероприятий: тех, что проводятся ALT Linux --- в основном с людьми, близкими к ALT Linux Team, в них ещё сильна академическая составляющая. И другого типа --- те, что проводятся крупными спонсорами, такими как IBM или Sun, для своих целей, в фешенебельных отелях,  на тысячи людей.

У вас отличается формат и люди по глубине задействованности: совершенно другой пласт, с которым на тех мероприятиях не сталкиваешься. Мне очень понравилось мероприятие, рад был встретить такое количество небезразличных к свободному ПО людей.

\end{document}