\documentclass[10pt, a5paper]{article}
\usepackage{pdfpages}
\usepackage{parallel}
\usepackage[T2A]{fontenc}
\usepackage{ucs}
\usepackage[utf8x]{inputenc}
\usepackage[polish,english,russian]{babel}
\usepackage{hyperref}
\usepackage{rotating}
\usepackage[inner=2cm,top=1.8cm,outer=2cm,bottom=2.3cm,nohead]{geometry}
\usepackage{listings}
\usepackage{graphicx}
\usepackage{wrapfig}
\usepackage{longtable}
\usepackage{indentfirst}
\usepackage{array}
\newcolumntype{P}[1]{>{\raggedright\arraybackslash}p{#1}}
\frenchspacing
\usepackage{fixltx2e} %text sub- and superscripts
\usepackage{icomma} % коскі ў матэматычным рэжыме
\PreloadUnicodePage{4}

\newcommand{\longpage}{\enlargethispage{\baselineskip}}
\newcommand{\shortpage}{\enlargethispage{-\baselineskip}}

\def\switchlang#1{\expandafter\csname switchlang#1\endcsname}
\def\switchlangbe{
\let\saverefname=\refname%
\def\refname{Літаратура}%
\def\figurename{Іл.}%
}
\def\switchlangen{
\let\saverefname=\refname%
\def\refname{References}%
\def\figurename{Fig.}%
}
\def\switchlangru{
\let\saverefname=\refname%
\let\savefigurename=\figurename%
\def\refname{Литература}%
\def\figurename{Рис.}%
}

\hyphenation{admi-ni-stra-tive}
\hyphenation{ex-pe-ri-ence}
\hyphenation{fle-xi-bi-li-ty}
\hyphenation{Py-thon}
\hyphenation{ma-the-ma-ti-cal}
\hyphenation{re-ported}
\hyphenation{imp-le-menta-tions}
\hyphenation{pro-vides}
\hyphenation{en-gi-neering}
\hyphenation{com-pa-ti-bi-li-ty}
\hyphenation{im-pos-sible}
\hyphenation{desk-top}
\hyphenation{elec-tro-nic}
\hyphenation{com-pa-ny}
\hyphenation{de-ve-lop-ment}
\hyphenation{de-ve-loping}
\hyphenation{de-ve-lop}
\hyphenation{da-ta-ba-se}
\hyphenation{plat-forms}
\hyphenation{or-ga-ni-za-tion}
\hyphenation{pro-gramming}
\hyphenation{in-stru-ments}
\hyphenation{Li-nux}
\hyphenation{sour-ce}
\hyphenation{en-vi-ron-ment}
\hyphenation{Te-le-pathy}
\hyphenation{Li-nux-ov-ka}
\hyphenation{Open-BSD}
\hyphenation{Free-BSD}
\hyphenation{men-ti-on-ed}
\hyphenation{app-li-ca-tion}

\def\progref!#1!{\texttt{#1}}
\renewcommand{\arraystretch}{2} %Іначай формулы ў матрыцы зліпаюцца з лініямі
\usepackage{array}

\def\interview #1 (#2), #3, #4, #5\par{

\section[#1, #3, #4]{#1 -- #3, #4}
\def\qname{LVEE}
\def\aname{#1}
\def\q ##1\par{{\noindent \bf \qname: ##1 }\par}
\def\a{{\noindent \bf \aname: } \def\qname{L}\def\aname{#2}}
}

\def\interview* #1 (#2), #3, #4, #5\par{

\section*{#1\\{\small\rm #3, #4. #5}}

\def\qname{LVEE}
\def\aname{#1}
\def\q ##1\par{{\noindent \bf \qname: ##1 }\par}
\def\a{{\noindent \bf \aname: } \def\qname{L}\def\aname{#2}}
}

\begin{document}
\title{Альт на Эльбрусе: обе вершины\footnote{\url{mike@altlinux.org}, \url{https://lvee.org/en/abstracts/286}}}
\author{Михаил Шигорин, Москва, Russian Federation}
\maketitle
\begin{abstract}
Elbrus 801-PC workstation supports three displays out-of-box but is further extendable to dual-seat configuration (with each seat still supporting up to three displays); that's what we've done in ALT installer.
\end{abstract}
Рабочая станция <<Эльбрус 801-РС(e801)>>~\cite{Shigorin-1} идёт в комплекте с трёхголовой видеокартой, а если добавить ещё одну такую же~--- возможно разместить на одной машине два полноценных рабочих места: ресурсов для них более чем достаточно (8 ядер, 32 Гб ОЗУ), а вот экономический эффект получается кратный.

Мы решили опробовать <<многоголовый>> вариант этой весной в рамках подготовки к конференции OS Day~\cite{Shigorin-2} 2018; сперва воспользовались свободными выходами уже имеющейся видеокарты, но такая конфигурация с промежуточной прослойкой Xephyr получилась небыстрой из-за отсутствия 3D-ускорения (у нас уже MATE 1.20, собранный с GTK+3), плюс достаточно громоздкой и неудобной в настройке и запуске (хотя по факту и она тоже возможна)~--- а вот с раздельными видеокартами всё стало простым и лаконичным, на данном этапе даже не потребовалось ничего патчить.

Описаны разные варианты~\cite{Shigorin-3}~\cite{Shigorin-4}~\cite{Shigorin-5} ~--- с одинаковыми или различными GPU, с задействованием logind или без него; с учётом особенностей платформы и своих предпочтений  остановились на двух Radeon R5 230 (один штатный и один дополнительный во втором слоте PCIe x8) и wdm~\cite{Shigorin-6}, который умеет поднимать несколько X-серверов штатным образом.

Реализация упакована в составе пакета xorg-conf-e801-dualseat~\cite{Shigorin-7} и вместе с доработкой инсталятора дистрибутива Альт Рабочая станция~\cite{Shigorin-8} для Эльбруса попала в августовский выпуск <<репозитория>> и образов(доступны клиентам МЦСТ)~\cite{Shigorin-9}~--- так что обладатели рабочих станций <<Эльбрус 801-РС>> могут уже сейчас обратиться в МЦСТ для получения дистрибутива и его развёртывания.

\begin{thebibliography}{20}

\bibitem{Shigorin-1} Эльбрус 801-РС \url{http://www.mcst.ru/elbrus_801-pc}
\bibitem{Shigorin-2} OS Day \url{http://osday.ru}
\bibitem{Shigorin-3} Ныне покойная multiseat wiki \url{http://web.archive.org/web/20101030044616/http://wiki.c3sl.ufpr.br/multiseat/index.php/Main_Page}
\bibitem{Shigorin-4} Multiseat in Gentoo Wiki \url{https://wiki.gentoo.org/wiki/Multiseat}
\bibitem{Shigorin-5} Multiseat in Arch Wiki \url{https://wiki.archlinux.org/index.php/xorg_multiseat}
\bibitem{Shigorin-6} wdm \url{http://altlinux.org/X11/DualSeat}
\bibitem{Shigorin-7} xorg-conf-e801-dualseat \url{http://git.altlinux.org/people/mike/packages/?p=xorg-conf-e801-dualseat.git}
\bibitem{Shigorin-8} Альт Рабочая станция \url{http://basealt.ru/products/alt-workstation/}
\bibitem{Shigorin-9} Репозиторий и образы \url{http://altlinux.org/ports/e2k}
\end{thebibliography}
\end{document}
