\documentclass[10pt, a5paper]{article}
\usepackage[T2A]{fontenc}
\usepackage{ucs}
\usepackage[utf8x]{inputenc}
\usepackage[polish,english,russian]{babel}
\usepackage{hyperref}
\usepackage[inner=2cm,top=1.8cm,outer=2cm,bottom=2.3cm,nohead]{geometry}
\usepackage{listings}
\usepackage{graphicx}
\usepackage{wrapfig}
\usepackage{longtable}
\usepackage{indentfirst}
\frenchspacing
\usepackage{fixltx2e} %text sub- and superscripts
\usepackage{icomma} % коскі ў матэматычным рэжыме
\PreloadUnicodePage{4}

\newcommand{\longpage}{\enlargethispage{\baselineskip}}
\newcommand{\shortpage}{\enlargethispage{-\baselineskip}}

\def\switchlang#1{\expandafter\csname switchlang#1\endcsname}
\def\switchlangbe{
\let\saverefname=\refname%
\def\refname{Літаратура}%
\def\figurename{Іл.}%
}
\def\switchlangen{
\let\saverefname=\refname%
\def\refname{References}%
\def\figurename{Fig.}%
}
\def\switchlangru{
\let\saverefname=\refname%
\let\savefigurename=\figurename%
\def\refname{Литература}%
\def\figurename{Рис.}%
}

\hyphenation{admi-ni-stra-tive}
\hyphenation{ex-pe-ri-ence}
\hyphenation{fle-xi-bi-li-ty}
\hyphenation{Py-thon}
\hyphenation{ma-the-ma-ti-cal}
\hyphenation{re-ported}
\hyphenation{imp-le-menta-tions}
\hyphenation{pro-vides}
\hyphenation{en-gi-neering}
\hyphenation{com-pa-ti-bi-li-ty}
\hyphenation{im-pos-sible}
\hyphenation{desk-top}
\hyphenation{elec-tro-nic}
\hyphenation{com-pa-ny}
\hyphenation{de-ve-lop-ment}
\hyphenation{de-ve-loping}
\hyphenation{de-ve-lop}
\hyphenation{da-ta-ba-se}
\hyphenation{plat-forms}
\hyphenation{or-ga-ni-za-tion}
\hyphenation{pro-gramming}
\hyphenation{in-stru-ments}
\hyphenation{Li-nux}
\hyphenation{en-vi-ron-ment}
\hyphenation{Te-le-pathy}
\hyphenation{Li-nux-ov-ka}

\def\progref!#1!{\texttt{#1}}
\renewcommand{\arraystretch}{2} %Іначай формулы ў матрыцы зліпаюцца з лініямі
\usepackage{array}

\def\interview #1 (#2), #3, #4, #5\par{

\section[#1, #3, #4]{#1, #5}
\def\qname{LVEE}
\def\aname{#1}
\def\q ##1\par{{\noindent \bf \qname: ##1 }\par}
\def\a{{\noindent \bf \aname: } \def\qname{L}\def\aname{#2}}
}

\begin{document}
\title{Протокол IF-MAP}
\author{Олег Орел\footnote{Минск, Беларусь, EPAM Systems, \url{Aleh_Arol@epam.com}}}
\date{}
\maketitle
\begin{abstract}
The Interface for Metadata Access Points (IF-MAP) is an open standard client/server protocol developed as one of the core protocols of the Trusted Network Connect (TNC) open architec\-ture. IF-MAP provides a common interface between the database server acting as a clearinghouse for information about security events and objects, and other elements of the TNC architecture.
\end{abstract}

TNC (Trusted Network Connect) "--- архитектура, описывающая возможную реализацию подхода к сетевой компьютерной безопасности, унифицирующего решения по обеспечению безопасности на конечных узлах (такие как антивирусное ПО, системы обнаружения вторжений и т.~д.), пользовательскую аутентификацию, элементы обеспечения безопасности. Компьютер, подключившийся в сеть, получает уровень доступа к ресурсам по результатам анализа таких его параметров, как уровень защищенности от вредоносных программ, конфигурации, роли пользователя, обновлений ОС. По любому параметру доступ может быть разграничен: например, пользователь, чей компьютер не обладает свежими антивирусными базами, не будет допущен в Интернет. IF"=MAP "--- открытый стандарт, описывающий клиент"=серверный протокол обмена данных между элементами (MAP) в TNC"=архитектуре.
IF-MAP сервер "--- это централизованное хранилище метаданных (например, о состоянии узлов сети, пользователях и т.~д.), предоставляющее механизмы для публикации данных, поиска и подписок всем заинтересованным MAP"=клиентам. Модель данных, предусмотренная для IF-MAP сервиса, представляет собой граф, вершины которого "--- идентификаторы (device, ip-address, mac-address, access-request, \ldots), а ребра "--- метаданные. Например, метаданные типа ip-mac, которые публикует MAP"=клиент, работающий совместно с DHCP сервером, соединяют идентификаторы типа ip"=address и mac"=address (DHCP lease) и содержат дополнительную информацию (время действия адреса). MAP"=клиентам доступен поиск на этом графе, а система подписок на изменения позволяет выполнять отложенный поиск.


\end{document}


