\documentclass[10pt, a5paper]{article}
\usepackage{pdfpages}
\usepackage{parallel}
\usepackage[T2A]{fontenc}
\usepackage{ucs}
\usepackage[utf8x]{inputenc}
\usepackage[polish,english,russian]{babel}
\usepackage{hyperref}
\usepackage{rotating}
\usepackage[inner=2cm,top=1.8cm,outer=2cm,bottom=2.3cm,nohead]{geometry}
\usepackage{listings}
\usepackage{graphicx}
\usepackage{wrapfig}
\usepackage{longtable}
\usepackage{indentfirst}
\usepackage{array}
\newcolumntype{P}[1]{>{\raggedright\arraybackslash}p{#1}}
\frenchspacing
\usepackage{fixltx2e} %text sub- and superscripts
\usepackage{icomma} % коскі ў матэматычным рэжыме
\PreloadUnicodePage{4}

\newcommand{\longpage}{\enlargethispage{\baselineskip}}
\newcommand{\shortpage}{\enlargethispage{-\baselineskip}}

\def\switchlang#1{\expandafter\csname switchlang#1\endcsname}
\def\switchlangbe{
\let\saverefname=\refname%
\def\refname{Літаратура}%
\def\figurename{Іл.}%
}
\def\switchlangen{
\let\saverefname=\refname%
\def\refname{References}%
\def\figurename{Fig.}%
}
\def\switchlangru{
\let\saverefname=\refname%
\let\savefigurename=\figurename%
\def\refname{Литература}%
\def\figurename{Рис.}%
}

\hyphenation{admi-ni-stra-tive}
\hyphenation{ex-pe-ri-ence}
\hyphenation{fle-xi-bi-li-ty}
\hyphenation{Py-thon}
\hyphenation{ma-the-ma-ti-cal}
\hyphenation{re-ported}
\hyphenation{imp-le-menta-tions}
\hyphenation{pro-vides}
\hyphenation{en-gi-neering}
\hyphenation{com-pa-ti-bi-li-ty}
\hyphenation{im-pos-sible}
\hyphenation{desk-top}
\hyphenation{elec-tro-nic}
\hyphenation{com-pa-ny}
\hyphenation{de-ve-lop-ment}
\hyphenation{de-ve-loping}
\hyphenation{de-ve-lop}
\hyphenation{da-ta-ba-se}
\hyphenation{plat-forms}
\hyphenation{or-ga-ni-za-tion}
\hyphenation{pro-gramming}
\hyphenation{in-stru-ments}
\hyphenation{Li-nux}
\hyphenation{sour-ce}
\hyphenation{en-vi-ron-ment}
\hyphenation{Te-le-pathy}
\hyphenation{Li-nux-ov-ka}
\hyphenation{Open-BSD}
\hyphenation{Free-BSD}
\hyphenation{men-ti-on-ed}
\hyphenation{app-li-ca-tion}

\def\progref!#1!{\texttt{#1}}
\renewcommand{\arraystretch}{2} %Іначай формулы ў матрыцы зліпаюцца з лініямі
\usepackage{array}

\def\interview #1 (#2), #3, #4, #5\par{

\section[#1, #3, #4]{#1 -- #3, #4}
\def\qname{LVEE}
\def\aname{#1}
\def\q ##1\par{{\noindent \bf \qname: ##1 }\par}
\def\a{{\noindent \bf \aname: } \def\qname{L}\def\aname{#2}}
}

\def\interview* #1 (#2), #3, #4, #5\par{

\section*{#1\\{\small\rm #3, #4. #5}}

\def\qname{LVEE}
\def\aname{#1}
\def\q ##1\par{{\noindent \bf \qname: ##1 }\par}
\def\a{{\noindent \bf \aname: } \def\qname{L}\def\aname{#2}}
}


\begin{document}

\title{Сеть хранения данных своими руками}%\footnote{Текст данных и последующих тезисов, кроме специально оговоренных случаев, доступен под лицензией Creative Commons Attribution-ShareAlike 3.0}

\author{Роман Шишнев\footnote{Минск, Беларусь; \url{rommer@active.by}}}
\maketitle

\begin{abstract}
There is a small gap in ready solutions between the systems with a local disks and a professional storage systems. The report highlights some technologies and approaches helpful at building personal storage area network of arbitrary size for using in various environments.
\end{abstract}

В настоящее время есть небольшой разрыв между готовыми системами, использующими локальные диски для хранения данных, и профессиональными системами хранения данных. Рассмотрим некоторые современные технологии, которые могут помочь самостоятельно построить сеть хранения данных (СХД) практически любого масштаба для применения в самых различных средах.

На первом шаге постороения личной СХД необходимо определить параметры её производительности и доступность. Ключеными моментами для СХД являются:

\begin{itemize}
  \item объем дискового простанства;
  \item производительность в IOPS (операций ввода-вывода в секунду);
  \item latency (среднее время доступа к данным);
  \item пропускная способность сети;
  \item допустимое время простоя.
\end{itemize}

На обьеме доступного простанства останавливаться не будем, так как тот параметр зависит исключительно от задачи.

Произодительность в количестве операций ввода/вывода в секунду следует рассматривать как ключевой параметр системы. Для примера, один диск SATA с 7200 оборотами в секунду позволяет выполнять около 100 операций при случайном доступе. Именно требуемое количество IOPS для СХД в основном и определяет тип и колическов дисков в нашей системе.

Latency в большей степени также зависит от дисков. Но при достаточно быстрых дисках, например SSD, стоит уже учитывать и задержки, которые возникают в сети при передаче данных.

Пропускная способность сети определяет максимальную скорость, на которой можно будет выполнять операции чтения и записи на системе хранения данных. В зависимости от задачи сеть можно создавать медленном медном гигабитном Ethernet, на 10-ти гигабитной оптике, либо на высокосторостном fiber channel.

Допустимое время простоя определяет количесво дублированных компонентов в сети и системе хранения данных.

Теперь рассмотрим самую простую задачу:

\begin{itemize}
  \item сервер \verb!A!, в котором установлен один SATA"=диск;
  \item сервер \verb!B!, которому нужно получить доступ к этому диску;
  \item гигабитные сетевые карты на каждом из серверов.
\end{itemize}

Для получения доступа к этому диску по сети Ethernet нужно выбрать протокол передачи данных. Самым простым и удобным протоколом в данном случае является iSCSI -- это по сути SCSI"=протокол поверх TCP/IP. На сервер \verb!A! нужно будет установить ПО, которое реализует так называемый iSCSI"=таргет "--- именно он будет осуществлять доступ к диску. На сервере \verb!B! должен быть установлен iSCSI"=инициатор "--- он будет подключаться к дискам. После установки и настройки сервер \verb!B! будет видеть диск сервера \verb!A! так, как будто он локальный.

Добавим в схему сервер \verb!C!, которому также нужен доступ к этому диску.

Для этого схему нужно просто дополнить гигабитным коммутатором, к которому теперь будут подключены все 3 сервера. На сервере \verb!C! достаточно установить / настроить iSCSI"=инициатор и подключить к iSCSI"=таргету на сервере \verb!A!.

После приведенных манипуляций серверы \verb!B! и \verb!C! <<видят>> один и тот же диск.

Имеет смысл предусмотреть также меры по устанению единых точек отказа в системе, и опционально "--- варианты маштабирования системы до десятков и сотен дисков и серверов.

\end{document}




