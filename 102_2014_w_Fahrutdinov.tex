\documentclass[10pt, a5paper]{article}
\usepackage{pdfpages}
\usepackage{parallel}
\usepackage[T2A]{fontenc}
\usepackage{ucs}
\usepackage[utf8x]{inputenc}
\usepackage[polish,english,russian]{babel}
\usepackage{hyperref}
\usepackage{rotating}
\usepackage[inner=2cm,top=1.8cm,outer=2cm,bottom=2.3cm,nohead]{geometry}
\usepackage{listings}
\usepackage{graphicx}
\usepackage{wrapfig}
\usepackage{longtable}
\usepackage{indentfirst}
\usepackage{array}
\newcolumntype{P}[1]{>{\raggedright\arraybackslash}p{#1}}
\frenchspacing
\usepackage{fixltx2e} %text sub- and superscripts
\usepackage{icomma} % коскі ў матэматычным рэжыме
\PreloadUnicodePage{4}

\newcommand{\longpage}{\enlargethispage{\baselineskip}}
\newcommand{\shortpage}{\enlargethispage{-\baselineskip}}

\def\switchlang#1{\expandafter\csname switchlang#1\endcsname}
\def\switchlangbe{
\let\saverefname=\refname%
\def\refname{Літаратура}%
\def\figurename{Іл.}%
}
\def\switchlangen{
\let\saverefname=\refname%
\def\refname{References}%
\def\figurename{Fig.}%
}
\def\switchlangru{
\let\saverefname=\refname%
\let\savefigurename=\figurename%
\def\refname{Литература}%
\def\figurename{Рис.}%
}

\hyphenation{admi-ni-stra-tive}
\hyphenation{ex-pe-ri-ence}
\hyphenation{fle-xi-bi-li-ty}
\hyphenation{Py-thon}
\hyphenation{ma-the-ma-ti-cal}
\hyphenation{re-ported}
\hyphenation{imp-le-menta-tions}
\hyphenation{pro-vides}
\hyphenation{en-gi-neering}
\hyphenation{com-pa-ti-bi-li-ty}
\hyphenation{im-pos-sible}
\hyphenation{desk-top}
\hyphenation{elec-tro-nic}
\hyphenation{com-pa-ny}
\hyphenation{de-ve-lop-ment}
\hyphenation{de-ve-loping}
\hyphenation{de-ve-lop}
\hyphenation{da-ta-ba-se}
\hyphenation{plat-forms}
\hyphenation{or-ga-ni-za-tion}
\hyphenation{pro-gramming}
\hyphenation{in-stru-ments}
\hyphenation{Li-nux}
\hyphenation{sour-ce}
\hyphenation{en-vi-ron-ment}
\hyphenation{Te-le-pathy}
\hyphenation{Li-nux-ov-ka}
\hyphenation{Open-BSD}
\hyphenation{Free-BSD}
\hyphenation{men-ti-on-ed}
\hyphenation{app-li-ca-tion}

\def\progref!#1!{\texttt{#1}}
\renewcommand{\arraystretch}{2} %Іначай формулы ў матрыцы зліпаюцца з лініямі
\usepackage{array}

\def\interview #1 (#2), #3, #4, #5\par{

\section[#1, #3, #4]{#1 -- #3, #4}
\def\qname{LVEE}
\def\aname{#1}
\def\q ##1\par{{\noindent \bf \qname: ##1 }\par}
\def\a{{\noindent \bf \aname: } \def\qname{L}\def\aname{#2}}
}

\def\interview* #1 (#2), #3, #4, #5\par{

\section*{#1\\{\small\rm #3, #4. #5}}

\def\qname{LVEE}
\def\aname{#1}
\def\q ##1\par{{\noindent \bf \qname: ##1 }\par}
\def\a{{\noindent \bf \aname: } \def\qname{L}\def\aname{#2}}
}

\begin{document}
\title{Операционная система Linux как основа для построения высокопроизводительных систем хранения данных}
\author{Александр Фахрутдинов, Сызрань, Russia}
\maketitle
\begin{abstract}
Report describes some Linux kernel subsystems which can be used to create a high-performance data storage. Well-known instruments of storage management (RAID and LVM) are explained as far as cutting-edge technologies for multi-level caching and data tiering. Also, kernel-mode virtual target device  for remote access to data storage from other hosts is reviewed.
\end{abstract}
Системы хранения данных (СХД) "--- это одна из основ современного мира компьютерных технологий. С возникновением облачных сред и повсеместном внедрением виртуализации возникла необходимость в сверхскоростных хранилищах большого объема и повышенной  отказоустойчивости. Кроме того, потребовались «умные» системы, которые хранят вместо несколько копий одних и тех же данных только одну, выделяют под данные именно столько реально имеющегося дискового пространства, сколько требуется, а не сколько запросит пользователь, умеют копировать массивы данных внутри устройства без отправки их на сервер и так далее.

Для организации эффективного хранения данных в Linux применяются виртуальные блочные устройства, которые представляют собой прослойку между собственно аппаратным хранилищем данных, например, жестким диском, и приложениями. Классический случай применения таких устройств программные дисковые массивы "--- RAID. За   обслуживание  RAID в Linux отвечает подсистема md (multiplie devices), которая позволяет создавать основные типы RAID "--- простое объединение дисков (JBOD), RAID уровней 0, 1 (зеркало), 4, 5, 6, а также "--- RAID 10.
.
RAID обеспечивает объединение дисков и отказоустойчивость в пределах одного сервера, однако он не решет проблему распределения пространства массива. Для управления дисковым пространством в Linux принято использовать менеджер виртуальных томов "--- LVM. Он позволяет гибко распределять место на дисках между приложениями, поддерживает создание, удаление и изменение размера тома «на лету», а также позволяет создавать собственные массивы типа JBOD,  RAID 0 и 1. Кроме того, LVM обеспечивает создание мгновенных снимков (snapshot) томов вне зависимости от того, поддерживает ли снимки файловая система тома, причем снимки доступны не только на чтение, но и на запись.  Однако платой за эту универсальность является низкое быстродействие снимков, что ограничивает область их применения.

В основе md и LVM лежит система проецирования устройств "--- device-mapper (dm). Она предоставляет единый механизм для создания на базе простых блочных устройств более сложных, наделенных дополнительными функциями. Возможности dm расширяются при помощи модулей, называемых «mapping targets». В стандартном ядре Linux, кроме md и LVM, присутствуют модули общего назначения для следующих целей:

\begin{itemize}
  \item диагностика и тестирование\begin{itemize}
  \item создание фиксированной задержки "--- dm-delay
  \item имитация сбоев "--- dm-flakey
  \item сбор статистики об обращениях к конкретным областям устройства "--- dm-statistics
  \item имитация пустого устройства "--- dm-zero
  \item проверка цифровой подписи тома "--- dm-verity
\end{itemize}


  \item построение RAID 0, 1 "--- dm-mod
  \item шифрование тома "--- dm-crypt
  \item доступ к подсистеме md для управления массивами через интерфейс device-mapper "--- dm-raid
\end{itemize}

Кроме того, в состав ядра входят модули, применяемые в высокопроизводительных СХД уровня предприятия

\begin{itemize}
  \item доступ к хранилищу через множество путей\begin{itemize}
  \item с простым резервированием "--- dm-multipath
  \item с учетом длины очереди "--- dm-queue-length
  \item с учетом времени доступа "--- dm-service-time
  \item доступ к разным регионам хранилища через разные пути "--- dm-switch
\end{itemize}


  \item кэширование данных на твердотельных накопителях\begin{itemize}
  \item dm-cache (с версии ядра 3.9,  релиз 28 апреля 2013)
  \item bcache (с версии 3.10, релиз 30 июня 2013)
  \item flashcache (разработка Facebook, не включен в ядро)
\end{itemize}


  \item не включены в ядро, но могут быть полезны:\begin{itemize}
  \item «многослойное» хранилище из разных типов дисков "--- btier
  \item RAM-диск с периодическим сохранением информации "--- eprd
Как известно, в Linux кэшируется только доступ к файловой системе, доступ же напрямую к блочным устройствам не кэшируется, поэтому одним из способов повысить быстродействие, особенно в может быть применение указанных выше модулей. В особенности это справедливо для систем виртуализации.
\end{itemize}


\end{itemize}

В системах  виртуализации принято выделять дисковое пространство не на этапе создания виртуальной машины, а по мере необходимости (thin provisioning). Device mapper имеет подобный функционал, начиная с ядра 3.2 (релиз 4 января 2012). Модуль dm-thin-pool позволяет создавать виртуальные устройства, которые не резервируют весь предназначенный им объем в момент создания, а расходуют выделенное физическое пространство по мере заполнения самого устройства данными. Также dm-thin-pool поддерживает возврат более не используемых блоков в общий пул, если вышележащая файловая система уведомит его об этом. Кроме того, модуль dm-thin-pool  позволяет создавать виртуальное устройство на базе шаблона, в роли которого выступает другое устройство, доступное только для чтения.

Несмотря на то, что device-mapper обладает широким функционалом, нельзя не заметить, что его назначение "--- создание виртуальных устройств «высокого уровня», которые служат для управления уже имеющимся дисковым пространством и должны опираться на программный или аппаратный RAID. В то же время  device-mapper не обеспечивает доступ к созданному устройству за пределами хоста, например, по протоколам iSCSI и FC. Для этих целей в ядро Linux не так давно была включена инфраструктура LIO-target.

LIO-target "--- это разработка компании Rising Tide Systems, которая была лицензирована под  GPL и включена в ядро Linux, начиная с версии 2.6.38 (релиз 15 января 2011 г.). В 2013 году разработчики LIO-target покинули  Rising Tide Systems и основали компанию Datera, целью которой является разработка программного обеспечения для СХД.

LIO-target состоит из высокоскоростного виртуального блочного устройства с поддержкой расширенного набора команд SCSI, драйверов нижнего уровня (бэкэндов), при помощи которых виртуальное устройство отображается на реальное и драйверов верхнего уровня (фронтэндов), при помощи которых можно получить доступ к устройству из-за пределов хоста.
Поддерживаются следующие бэкэнды:

\begin{itemize}
  \item FILEIO "--- обращение к нижележащему блочному устройству как к файлу через слой виртуальной ФС.
  \item BLOCKIO "--- обращение к нижележащему блочному устройству при помощи команд SCSI.
  \item PSCSI "--- пересылка команд  SCSI физическому   устройству, например, RAID-контроллеру, без обработки.
  \item Memory Copy RAMDISK "--- блочное устройство в оперативной памяти.
Фронтэнды обеспечивают подключение к LIO-target других хостов. В настоящий момент поддерживаются интерфейсы iSCSI, FCoE, Fibre Channel, InfiniBand, IBM vSCSI, FireWare и USB. Кроме того, возможна эмуляция блочного устройства на локальной машине, а также передача такого устройства внутрь виртуальной машины под управлением \linebreak KVM (vHost).
\end{itemize}

Особенностью  LIO-target является поддержка SCSI-команд аппаратного ускорения для систем хранения данных (VAAI). Эти команды используются, в первую очередь, системами виртуализации и призваны разгрузить гипервизор при таких ресурсоемких операциях, как клонирование виртуальной машины. До недавнего времени этот набор команд был реализован только в коммерческих СХД, теперь же он доступен всем пользователям ОС Linux.

Таким образом, ОС Linux предоставляет функционал, достаточный для построения на базе конкретной аппаратной платформы надежного и высокопроизводительного хранилища, которое может быть использовано как само по себе, так и в качестве узла в распределенной системе хранения данных.

\end{document}
