\documentclass[10pt, a5paper]{article}
\usepackage{pdfpages}
\usepackage{parallel}
\usepackage[T2A]{fontenc}
\usepackage{ucs}
\usepackage[utf8x]{inputenc}
\usepackage[polish,english,russian]{babel}
\usepackage{hyperref}
\usepackage{rotating}
\usepackage[inner=2cm,top=1.8cm,outer=2cm,bottom=2.3cm,nohead]{geometry}
\usepackage{listings}
\usepackage{graphicx}
\usepackage{wrapfig}
\usepackage{longtable}
\usepackage{indentfirst}
\usepackage{array}
\newcolumntype{P}[1]{>{\raggedright\arraybackslash}p{#1}}
\frenchspacing
\usepackage{fixltx2e} %text sub- and superscripts
\usepackage{icomma} % коскі ў матэматычным рэжыме
\PreloadUnicodePage{4}

\newcommand{\longpage}{\enlargethispage{\baselineskip}}
\newcommand{\shortpage}{\enlargethispage{-\baselineskip}}

\def\switchlang#1{\expandafter\csname switchlang#1\endcsname}
\def\switchlangbe{
\let\saverefname=\refname%
\def\refname{Літаратура}%
\def\figurename{Іл.}%
}
\def\switchlangen{
\let\saverefname=\refname%
\def\refname{References}%
\def\figurename{Fig.}%
}
\def\switchlangru{
\let\saverefname=\refname%
\let\savefigurename=\figurename%
\def\refname{Литература}%
\def\figurename{Рис.}%
}

\hyphenation{admi-ni-stra-tive}
\hyphenation{ex-pe-ri-ence}
\hyphenation{fle-xi-bi-li-ty}
\hyphenation{Py-thon}
\hyphenation{ma-the-ma-ti-cal}
\hyphenation{re-ported}
\hyphenation{imp-le-menta-tions}
\hyphenation{pro-vides}
\hyphenation{en-gi-neering}
\hyphenation{com-pa-ti-bi-li-ty}
\hyphenation{im-pos-sible}
\hyphenation{desk-top}
\hyphenation{elec-tro-nic}
\hyphenation{com-pa-ny}
\hyphenation{de-ve-lop-ment}
\hyphenation{de-ve-loping}
\hyphenation{de-ve-lop}
\hyphenation{da-ta-ba-se}
\hyphenation{plat-forms}
\hyphenation{or-ga-ni-za-tion}
\hyphenation{pro-gramming}
\hyphenation{in-stru-ments}
\hyphenation{Li-nux}
\hyphenation{sour-ce}
\hyphenation{en-vi-ron-ment}
\hyphenation{Te-le-pathy}
\hyphenation{Li-nux-ov-ka}
\hyphenation{Open-BSD}
\hyphenation{Free-BSD}
\hyphenation{men-ti-on-ed}
\hyphenation{app-li-ca-tion}

\def\progref!#1!{\texttt{#1}}
\renewcommand{\arraystretch}{2} %Іначай формулы ў матрыцы зліпаюцца з лініямі
\usepackage{array}

\def\interview #1 (#2), #3, #4, #5\par{

\section[#1, #3, #4]{#1 -- #3, #4}
\def\qname{LVEE}
\def\aname{#1}
\def\q ##1\par{{\noindent \bf \qname: ##1 }\par}
\def\a{{\noindent \bf \aname: } \def\qname{L}\def\aname{#2}}
}

\def\interview* #1 (#2), #3, #4, #5\par{

\section*{#1\\{\small\rm #3, #4. #5}}

\def\qname{LVEE}
\def\aname{#1}
\def\q ##1\par{{\noindent \bf \qname: ##1 }\par}
\def\a{{\noindent \bf \aname: } \def\qname{L}\def\aname{#2}}
}

\usepackage{color}
\begin{document}
\title{Интервью с участниками}
%\author{}
\date{}
\maketitle

По традиции в сборник материалов входят интервью, в которых активные участники
сообщества open source делятся своим мнением о свободном ПО, открытых
технологиях, роли и месте GNU/Linux, рассказывают, как видят проблематику
свободных проектов. В этот раз мы решили расспросить трёх участников
конференции, какое-то время назад перебравшихся из Беларуси на территорию
Европейского Союза.

\section[Александр Боковой "--- Principal software engineer, Red Hat, Эспоо,
Финляндия]{Александр Боковой "--- Principal software\linebreak engineer, Red
Hat, Эспоо, Финляндия}
%\begin{figure}[ht]
%\centering{\includegraphics[width=4cm]{49_spons_altoros.jpg}}
%\end{figure}

{\noindent \bf LVEE: Традиционно первый вопрос "--- как ты познакомился с
открытым ПО?}

{\noindent \bf Александр Боковой:} В 1995 году. Я учился на третьем курсе БГПУ
им. Максима Танка, и одна из курсовых работ была посвящена фрактальной
геометрии. Необходимо было написать приложение, которое бы отрисовывало и в
интерактивном режиме позволяло бы исследовать множества Жюлиа для
соответствующих точек из множества Мандельброта. 

{\noindent \bf L: Звучит очень наукоёмко\ldots}

{\noindent \bf А:} Программу я писал на Паскале, и в какой"=то момент стало не
хватать стандартной памяти в 16"=битном режиме.

{\noindent \bf L: Под MS DOS.} 

{\noindent \bf А:}  Да. Вариантов использования 32"=битного режима было
немного, поскольку требовалось еще и приличный интерфейс пользователя
обеспечить. Значит, нужна была не только графическая библиотека, но и виджеты,
обработка клавиатуры и так далее. И я нашел такую библиотеку "--- SWORD,
написанную французом Эриком Николя на C++ и поставлявшуюся вместе с DJGPP.

{\noindent \bf L: А DJGPP "--- это\ldots}

{\noindent \bf А:}  DJGPP "--- это первый порт программ проекта GNU на
платформу Intel x86, сделанный еще в 1989 году DJ Delorie. Ричард Столлман
выступал на встрече Northern England Unix Users Group в компании Data General,
где работал тогда DJ Delorie, и на вопрос о переносе GCC под MS DOS, ответил,
что это невозможно, поскольку gcc слишком большая программа, а MS-DOS работает
в 16"=битном режиме. DJ понял, что это вызов и принял его. Так что в 1990 мы
уже имели компиляторы GNU, Emacs, binutils и много разных библиотек, все под
MS DOS в 32"=битном режиме "--- DJ пришлось написать свой DOS Extender для
того, чтобы компилировать gcc под MS DOS.

{\noindent \bf L: Как скоро ты осознал, что это вот "--- свободное ПО, сообщество, коллективная разработка?}

{\noindent \bf A:} Практически сразу. DJGPP поставлялся со всеми исходными текстами, в документации было
написано, где можно задавать вопросы. Я подписался на рассылки и первое время просто читал "--- и переписку,
где люди отвечали не только на вопросы об использовании тех или иных компонент системы, но и обсуждали бытовые
темы. Выглядело все это очень по"=домашнему, а если кто-то предлагал патчи, то это предполагало прежде всего
устранение необходимости патчить то же самое место в следующей версии "--- rsync еще не был написан (он появился
только в 1996), а DJGPP распространялся по FTP. На наших узких линиях (64Кбит/с на весь университет) тогда
приходилось прежде всего думать, а потом делать.

{\noindent \bf L: Итак, ты использовал DJGP. А каким образом пользователь СПО стал его разработчиком?}

{\noindent \bf A:} Со SWORD и DJGPP я и начал. В 1996 вышла вторая версия DJGPP, независящая от
коммерческих компонент для своей пересборки. Главное, что случилось с DJGPP в
1994--96 годах "--- это взрывной рост популярности, привлекший огромное число
терпеливых и общительных людей в списки рассылки. Можно было задавать вопросы и
получать ответы на них, вне зависимости от того, насколько плох был твой
английский язык. В 1995 году сделали зеркало в рассылку в виде группы USENET
\url{comp.os.msdos.djgpp}, она стала доступна на локальном NNTP"=сервере университета.

{\noindent \bf L: Не помнишь, как вообще оформилась мысль: а сделаю"=ка я
публичный патч? Или это получилось как"=то незаметно: пообсуждал, поисправлял
"--- и вдруг люди уже пользуются?}

{\noindent \bf A:} Непубличные патчи поддерживать было неудобно, поскольку сам комплект
DJGPP распространялся в виде архивов. Так что старался отправлять исправления сразу.
К тому же, библиотеки были мне нужны для работы, но не являлись главным ее содержимым.
Лицензия SWORD "--- GNU General Public License, которую я уже читал и видел в применении
к остальным компонентам GNU. 

В 1998 я вместе с Эриком работал над третьей версией SWORD. Эта работа привела
к тому, что через несколько лет я ушел из аспирантуры, так и не закончив свою
работу над диссертацией, потому что вместо работы над методикой преподавания
фрактальной геометрии сосредоточился над SWORD "--- нужда в нормальной
интерфейсной библиотеке, работающей под MS DOS и GNU/Linux на тот момент не отпала,
поскольку Qt до 2000 года выходила под неудачной с точки зрения свободного ПО и написания
программ под GNU GPL лицензией и не поддерживала MS DOS.

Правда, после ухода из аспирантуры я сосредоточился на сетевых файловых системах,
а Эрик переписал SWORD с нуля с учетом прогресса в Qt и проект был перезапущен в 2005: 
\url{http://www.erik-n.net/software/sword/}. 

На GNU/Linux я перешел где"=то в 1996--1997, практически сразу, как появился
собственный компьютер.

{\noindent \bf L: На какой дистрибутив?}

{\noindent \bf A:} Начал со Slackware. А в декабре 1999 перевел на белорусский
язык программу установки Mandrake Linux. Она вошла в Mandrake Linux Russian
Edition, а потом и в основной Mandrake Linux.

Другой проект, который <<втянул>> меня в себя в приблизительно то же время, это
Midgard, система ведения веб"=сайтов. Изначально придуманная финнами Генри
Бергиусом и Юккой Зиттингом для сайта своего реконструкторского общества в
1998, система переросла викингов и стала довольно успешно использоваться как
конструктор различных сайтов, в том числе и для интранетов. Я выступал с докладом
о Midgard на первом FOSDEM в 2001 году, а в середине 2000"=х даже интегрировал
Midgard и Samba для того, чтобы обеспечить прозрачную авторизацию в
интранет"=приложениях на Midgard в среде Active Directory.

{\noindent \bf L: И тут мы наконец подобрались к твоему участию в проекте Samba :) }

{\noindent \bf A:} Получается забавная ситуация: практически все проекты, над которыми я работал и
работаю, в той или иной мере связаны между собой. В 2001--2004 годах мы с Игорем
Вергейчиком работали над системой хранения, где требовалась поддержка различных
сетевых файловых систем и я столкнулся с необходимостью внести какие"=то
изменения в Samba.  Мы написали ряд патчей, отправили их в рассылку, часть из
них приняли, часть "--- нет.  Потом Игорь доработал Samba до поддержки Unicode.
Потом я написал поддержку множественных модулей виртуальной файловой системы. И в
2003 меня пригласили в Samba Team. Принцип был простой: мой код практически
не требовал дополнительных доработок, поэтому мне дали прямой доступ к
изменению исходного текста.

Когда в декабре 2003 мы получили заказ на разработку поддержки Active Directory
в нашей системе хранения, Эндрю Триджелл, создатель Samba, просто сказал нам:
<<Зачем пытаться добавить патчи в версию 2.0, лучше помогите мне закончить 3.0,
где я уже много добился>>. То есть, взгляды апстрима и даунстрима совпали,
получилось сделать многое. Конечно, не в тот срок, который обещал Эндрю, но в
2005 у нас был вполне работающий продукт.

{\noindent \bf L: Какие различия бросаются в глаза, если сравнивать
опенсорс"=комьюнити в СНГ с англоязычным? }

{\noindent \bf А:} Если тебе нужны какие-то изменения к существующему коду, ты их пишешь,
оформляешь патчи, отправляешь в рассылку и обсуждаешь с другими разработчиками.
Патчи могут принять сразу, могут не принять совсем, но чаще всего приходится
объяснять и находить компромисс. Работа над отдельными изменениями может
затянуться на годы. В СНГ есть разработчики свободного ПО (и их много), но
очень мало сообществ разработчиков свободного ПО как таковых. Те, кто
заинтересован, участвуют в международных проектах различного масштаба. Новые
проекты с преимущественно русскоязычным общением -- редкость, они мало кому из
разработчиков нужны. Они, безусловно, нужны пользователям, но сколько времени
разработчики могут посвятить локальным пользователям?

С другой стороны, уровень знания английского языка может препятствовать
активному участию в существующих проектах, даже если кто"=то готов написать
код, часто сталкиваешься с тем, что довести работу до конца они не могут "---
нужна документация на английском, участие в дискуссиях, причем в темпе
активности конкретного проекта, а не разработчика.  В этом смысле разница в
Европе особенно бросается в глаза, здесь проблем с английским языком среди
разработчиков свободного ПО нет, даже в традиционно неанглоязычных странах.
Английский "--- lingua franca свободного ПО.

Другой аспект взаимодействия в проектах свободного ПО, это значительно меньший
накал страстей в рассылках по сравнению с тем, что я вижу в русскоязычной
среде. 

{\noindent \bf L: О да! В этом сезоне организаторская рассылка LVEE переживала 
как раз такую драму, в традициях русскоязычного сегмента :) }

{\noindent \bf А:} Наблюдается заметное ослабление эмоций при общении с пользователями при
продвижении с востока на запад в Европе "--- если, скажем, польские пользователи
еще пишут с активным выражением своей позиции в отношении разработчиков на IRC"=каналах, 
то там, где преобладают английские или американские пользователи,
атмосфера менее накалена. В программном обеспечении есть и будут ошибки, никто
не идеален, поэтому поиск источника ошибки "--- рабочая ситуация, не требующая
перехода на личности. Почему"=то русскоязычное пространство переполнено полярным
выражением собственных эмоций.

{\noindent \bf L: Кстати, возвращаясь к теме работы над продуктами. Как бы ты
охарактеризовал свой личный опыт использования СПО в корпоративном секторе?}

{\noindent \bf A:} Мне повезло, я последние лет 15 использую свободное ПО в
рабочем окружении. В последние пять"=семь лет с этим стало совсем хорошо из"=за
активного продвижения мобильных платформ и веб"=приложений, которые вынесли из
многих компаний специализированные плагины и прочие платформо"=зависимые
клиентские компоненты.

Работа над Samba и FreeIPA предполагает, что приходится иметь дело с
проприетарной инфраструктурой и клиентским ПО, но для обеспечения собственной
жизни в корпоративной среде мне они практически не нужны. Гораздо сложнее
с проприетарным ПО на серверной стороне "--- даже если интерфейс к нему позволяет
использовать свободное ПО на клиентской стороне, доступность данных в
большинстве таких систем завязана на производителя. Это данные наших компаний,
но извлечь их в структурированном виде и перенести куда"=то еще мы часто просто
не можем.

{\noindent \bf L: Последний вопрос, о Red Hat. Как это выглядит изнутри?}

{\noindent \bf А:} По"=домашнему. В прямом смысле "--- большую часть времени
я работаю из дома. У нас небольшой офис в Эспоо, рабочее место у меня есть, но
появляюсь я в офисе нечасто, поскольку моя команда разбросана по миру. Из инструментов
общения "--- электронная почта, IRC, интернет"=телефония и видео"=конференции.
Раз или два в год получается встретиться лично, это время используется для интенсивных
дискуссий, особенно в феврале, когда в Брно (Чехия) проходит традиционная конференция
\url{devconf.cz} "--- на нее съезжаются ребята из многих команд и есть шанс обсудить
предстоящие задачи на год вперед с теми, с кем не получается пересекаться <<в эфире>>
из"=за часовых поясов.

Самым удивительным для меня четыре года назад было то, как мало информации скрыто от
посторонних глаз. Если Red Hat участвует в разработке какого-то проекта, то вся информация
доступна на сайте апстрима. Внутри только детали планов интеграции конкретных апстримных
версий в продукты компании, а весь дизайн новых функций и их разработка ведутся публично.

А моя история замкнулась "--- DJ Delorie работает в Red Hat и обеспечивает нас
работающими компиляторами вот уже более шестнадцати лет.
 
\end{document}


