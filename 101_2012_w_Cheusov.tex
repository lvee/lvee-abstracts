\documentclass[10pt, a5paper]{article}
\usepackage{pdfpages}
\usepackage{parallel}
\usepackage[T2A]{fontenc}
\usepackage{ucs}
\usepackage[utf8x]{inputenc}
\usepackage[polish,english,russian]{babel}
\usepackage{hyperref}
\usepackage{rotating}
\usepackage[inner=2cm,top=1.8cm,outer=2cm,bottom=2.3cm,nohead]{geometry}
\usepackage{listings}
\usepackage{graphicx}
\usepackage{wrapfig}
\usepackage{longtable}
\usepackage{indentfirst}
\usepackage{array}
\newcolumntype{P}[1]{>{\raggedright\arraybackslash}p{#1}}
\frenchspacing
\usepackage{fixltx2e} %text sub- and superscripts
\usepackage{icomma} % коскі ў матэматычным рэжыме
\PreloadUnicodePage{4}

\newcommand{\longpage}{\enlargethispage{\baselineskip}}
\newcommand{\shortpage}{\enlargethispage{-\baselineskip}}

\def\switchlang#1{\expandafter\csname switchlang#1\endcsname}
\def\switchlangbe{
\let\saverefname=\refname%
\def\refname{Літаратура}%
\def\figurename{Іл.}%
}
\def\switchlangen{
\let\saverefname=\refname%
\def\refname{References}%
\def\figurename{Fig.}%
}
\def\switchlangru{
\let\saverefname=\refname%
\let\savefigurename=\figurename%
\def\refname{Литература}%
\def\figurename{Рис.}%
}

\hyphenation{admi-ni-stra-tive}
\hyphenation{ex-pe-ri-ence}
\hyphenation{fle-xi-bi-li-ty}
\hyphenation{Py-thon}
\hyphenation{ma-the-ma-ti-cal}
\hyphenation{re-ported}
\hyphenation{imp-le-menta-tions}
\hyphenation{pro-vides}
\hyphenation{en-gi-neering}
\hyphenation{com-pa-ti-bi-li-ty}
\hyphenation{im-pos-sible}
\hyphenation{desk-top}
\hyphenation{elec-tro-nic}
\hyphenation{com-pa-ny}
\hyphenation{de-ve-lop-ment}
\hyphenation{de-ve-loping}
\hyphenation{de-ve-lop}
\hyphenation{da-ta-ba-se}
\hyphenation{plat-forms}
\hyphenation{or-ga-ni-za-tion}
\hyphenation{pro-gramming}
\hyphenation{in-stru-ments}
\hyphenation{Li-nux}
\hyphenation{sour-ce}
\hyphenation{en-vi-ron-ment}
\hyphenation{Te-le-pathy}
\hyphenation{Li-nux-ov-ka}
\hyphenation{Open-BSD}
\hyphenation{Free-BSD}
\hyphenation{men-ti-on-ed}
\hyphenation{app-li-ca-tion}

\def\progref!#1!{\texttt{#1}}
\renewcommand{\arraystretch}{2} %Іначай формулы ў матрыцы зліпаюцца з лініямі
\usepackage{array}

\def\interview #1 (#2), #3, #4, #5\par{

\section[#1, #3, #4]{#1 -- #3, #4}
\def\qname{LVEE}
\def\aname{#1}
\def\q ##1\par{{\noindent \bf \qname: ##1 }\par}
\def\a{{\noindent \bf \aname: } \def\qname{L}\def\aname{#2}}
}

\def\interview* #1 (#2), #3, #4, #5\par{

\section*{#1\\{\small\rm #3, #4. #5}}

\def\qname{LVEE}
\def\aname{#1}
\def\q ##1\par{{\noindent \bf \qname: ##1 }\par}
\def\a{{\noindent \bf \aname: } \def\qname{L}\def\aname{#2}}
}


\begin{document}

\title{Software security}%\footnote{Текст данных и последующих тезисов, кроме специально оговоренных случаев, доступен под лицензией Creative Commons Attribution-ShareAlike 3.0}

\author{Алексей Чеусов\footnote{Минск, Беларусь}}
\maketitle

\begin{abstract}
System techniques and methods are reviewed for securing the software in UNIX World.
\end{abstract}

Язык программирования С, на десятилетия определивший успех операционных систем класса UNIX, на протяжении последних лет все чаще
становится источником проблем в области безопасности программного
обеспечения. Появившись как язык системного программирования, язык С
широко применяется также и для разработки прикладного ПО, что, ввиду принципиальной <<небезопасности>> этого языка, приводит к многочисленным проблемам, таким как получение доступа к системе злоумышленником, повышению привилегий процесса до уровня суперпользователя, rootkit-ам, нестабильности работы ОС и т.п. То же относится и к языку программирования С++.

В ближайшее время вряд ли стоит ожидать значительного падения популярности этих языков в области разработки как системного, так и прикладного ПО, поэтому
актуальной становится разработка средств и методов борьбы с
перечисленными выше проблемами менее радикальными, чем смена языка
программирования, средствами. Таким средствам и технологиям, существующим и развивающимся в различных UNIX"=подобных системах, посвящен настоящий краткий обзор.

К числу рассматриваемых технологий защиты можно среди прочих отнести следующие:

\begin{itemize}
  \item безопасные функции strl\{cat,cpy\} для работы со строками,
  \item SSP (stack smashing protection) "--- защита от переполнения стека,
  \item ASLR (address space layout randomization) "--- рандомизация базовых адресов сегментов виртуальной памяти процесса,
  \item PIE (position independent executable) "--- позиционно"=независимые исполняемые файлы,
  \item hardened chroot "--- усиленный chroot,
  \item W\^{}X "--- защита исполняемых сегментов памяти от записи и записываемых (стек и данные) от исполнения,
  \item PaX MPROTECT "--- защита mprotect(2),
  \item PaX Segvgard "--- защита от перебора адресов сегментов памяти приложения,
  \item Information filtering "--- сокрытие информации, доступной пользователям и процессам,
  \item per-user /tmp directory "--- размещение подкаталога временных файлов в домашней директории пользователя,
  \item SUID/SGIG executables "--- избавление от исполняемых файлов с установленным битом SUID,
  \item PAM tcb "--- замена PAM unix как средство избавления от бита SUID у passwd(8)
  \item capsicum "--- расширение POSIX API для обеспечения лучшей безопасности в системе UNIX.
  \item FUSE, PUFFS "--- подсистемы для реализации файловых систем в пространстве пользователя
  \item Микроядерные ОС "--- класс операционных систем, в которых основные сервисы работают на уровне пользователя, на уровне ядра же работает лишь самое необходимодое, за счет чего достигается надежность и безопасность
  \item RUMP "--- подсистема для запуска ядерного кода в пользовательском приложении
  \item SE Linux "--- подсистема контроля доступа в Linux
  \item kauth(9) "--- подсистема авторизации ядра NetBSD
  \item jail "--- система изоляции и виртуализации FreeBSD
\end{itemize}


\end{document}




