\documentclass[10pt, a5paper]{article}
\usepackage[T2A]{fontenc}
\usepackage{ucs}
\usepackage[utf8x]{inputenc}
\usepackage[polish,english,russian]{babel}
\usepackage{hyperref}
\usepackage[inner=2cm,top=1.8cm,outer=2cm,bottom=2.3cm,nohead]{geometry}
\usepackage{listings}
\usepackage{graphicx}
\usepackage{wrapfig}
\usepackage{longtable}
\usepackage{indentfirst}
\frenchspacing
\usepackage{fixltx2e} %text sub- and superscripts
\usepackage{icomma} % коскі ў матэматычным рэжыме
\PreloadUnicodePage{4}

\newcommand{\longpage}{\enlargethispage{\baselineskip}}
\newcommand{\shortpage}{\enlargethispage{-\baselineskip}}

\def\switchlang#1{\expandafter\csname switchlang#1\endcsname}
\def\switchlangbe{
\let\saverefname=\refname%
\def\refname{Літаратура}%
\def\figurename{Іл.}%
}
\def\switchlangen{
\let\saverefname=\refname%
\def\refname{References}%
\def\figurename{Fig.}%
}
\def\switchlangru{
\let\saverefname=\refname%
\let\savefigurename=\figurename%
\def\refname{Литература}%
\def\figurename{Рис.}%
}

\hyphenation{admi-ni-stra-tive}
\hyphenation{ex-pe-ri-ence}
\hyphenation{fle-xi-bi-li-ty}
\hyphenation{Py-thon}
\hyphenation{ma-the-ma-ti-cal}
\hyphenation{re-ported}
\hyphenation{imp-le-menta-tions}
\hyphenation{pro-vides}
\hyphenation{en-gi-neering}
\hyphenation{com-pa-ti-bi-li-ty}
\hyphenation{im-pos-sible}
\hyphenation{desk-top}
\hyphenation{elec-tro-nic}
\hyphenation{com-pa-ny}
\hyphenation{de-ve-lop-ment}
\hyphenation{de-ve-loping}
\hyphenation{de-ve-lop}
\hyphenation{da-ta-ba-se}
\hyphenation{plat-forms}
\hyphenation{or-ga-ni-za-tion}
\hyphenation{pro-gramming}
\hyphenation{in-stru-ments}
\hyphenation{Li-nux}
\hyphenation{en-vi-ron-ment}
\hyphenation{Te-le-pathy}
\hyphenation{Li-nux-ov-ka}

\def\progref!#1!{\texttt{#1}}
\renewcommand{\arraystretch}{2} %Іначай формулы ў матрыцы зліпаюцца з лініямі
\usepackage{array}

\def\interview #1 (#2), #3, #4, #5\par{

\section[#1, #3, #4]{#1, #5}
\def\qname{LVEE}
\def\aname{#1}
\def\q ##1\par{{\noindent \bf \qname: ##1 }\par}
\def\a{{\noindent \bf \aname: } \def\qname{L}\def\aname{#2}}
}

\begin{document}
\title{Enterprise Storage OS (ESOS)}
\author{Александр Клыга, Minsk, Belarus}
\maketitle
\begin{abstract}
The Enterprise Storage OS (ESOS) is open-source software to build the Data Storages System within the SAN model. The ESOS is a compact Linux distribution and it is distributed under a GPL license. In my presentation the main components of the ESOS are considered including a loading method and the Data Storages System architecture.
\end{abstract}
\textbf{Enterprise Storage OS (ESOS).}

Создание и обслуживание сетей хранения данных (Storage Area Network или сокращенно SAN) одна из интересных технических задач для технических специалистов обслуживающих системы хранения данных в различных компаниях. Прежде всего это обусловлено самой моделью SAN, и техническими требованиями предъявляемыми к системе хранения данных в целом, особенно с учетом широкого использования средств визуализации и необходимостью хранить большие объемы данных.

Очень часто техническим специалистам, сопровождающим крупные системы хранения данных, приходится решать трудные задачи, связанные с расширением объема доступного дискового пространства, надежностью хранения информации и интеграции с другими СХД, в частности с распределенными файловыми системами. Именно поэтому поиск оптимального решения подобных задач является критически важным, а оптимальный вариант должен обеспечивать должный уровень надежности, гибкости и возможности гибкого расширения без сложных структурных реорганизаций.

Поэтому вполне логично, что поиск решений для оптимизации и расширения существующей корпоративной архитектуры сети хранения данных начинается с решений на рынке свободного программного обеспечения (обзор некоторых из них был представлен мною в рамках доклада «Обзор решений на рынке открытого ПО для создания СХД» на зимней сессии LVEE в феврале этого года~\cite{Kliga1}). По совету коллег несколько лет назад я обратил внимание на один очень интересный проект  Enterprise Storage OS (ESOS)~\cite{Kliga2}.

В ESOS основное внимание у меня привлекло возможность создания дисковых хранилищ для систем хранения данных модели SAN и возможности интеграции существующей SAN в объектное хранилище данных Ceph. 
С физической точки зрения отдельное дисковое хранилище в SAN представляет собой законченное техническое решение состоящее отдельных дисковых полок (с установленных в них дисками)  подключенных к контроллеру управления, а для подключения к сетевой инфраструктуре SAN  на нем устанавливаются HBA (host bus adapter) адаптеры. Управление оборудованием осуществляется через консоль или по локальной сети через специальное приложение установленное на компьютере или web – приложение.

Но прежде чем приступить к описанию всех достоинств использования ESOS в SAN, необходимо кратко рассмотреть ее  физическую и программную модели.

Физическая модель SAN состоит из трех основных компонентов: серверной фермы, сетевой инфраструктуры и дискового массива. В состав серверной фермы SAN входят сервера SAN не имеющих собственной подсистемы дисковой памяти и подключаются к сетевой инфраструктуре хранилища через специальный адаптеры HBA (host bus adapter). В состав сетевой архитектуры SAN входит активное сетевое оборудование (концентраторы, коммутаторы, мосты, маршрутизаторы)   и пассивное оборудование (кабели, коннекторы). В состав дискового массива входят  специализированные дисковые стойки (устройства состоящие из десятков дисков управляемые контроллером или сервером) подключаемые к сетевой инфраструктуру через адаптеры HBA.

Программная модель взаимодействия основных компонентов \linebreak SAN между собой базируется на клиент-серверной модели семейства протоколов Small Computer System Interface (SCSI). Согласно ее клиентские приложения устанавливаются на сервера серверной фермы SAN, сервер выступает в роли инициатора (Initiator) сессии взаимодействия с целевым устройством (Target) в дисковой подсистемой хранилища через сетевую инфраструктуру SAN. Target  обрабатывает команды адресуемые ему от Initiator и формирует ответ.

При казалось бы видимой простоте программной модели следует отметить важные моменты:

\begin{itemize}
  \item взаимодействия между Initiator и Target осуществляется блоками заданной длины (т. е. все устройства представляются в виде сетевых блочных устройств);
  \item Target это логическое устройство которому присвоен уникальный номер, и с одним Target могут взаимодействовать несколько Initiator;
  \item Initiator не взаимодействует напрямую с дисками в хранилище данных, а использует одно или несколько логических устройств с уникальным номером — LUN (logical unit number);
  \item LUN абстрагирован от физической архитектуры дискового хранилища и представляет собой виртуальный диск созданный посредством специального ПО.
\end{itemize}

Таким образом, становится очевидно, первое, что сервер SAN подключенный с сети хранения данных напрямую не управляет физическими дисками подключенными к нему, все манипуляции с ними осуществляет через LUN (один или несколько) на Target. Второе Target может быть как отдельное физическое хранилище имеющий свой уникальный физический номер, так и  может выступать как программный интерфейс к распределенной файловой системе через блочный интерфейс доступа. При этом логический номер Target присваиваться в программном обеспечении  контроллера  (сервера) управляющего дисковым хранилищем (Data Storage) или сервера на котором установлено программное обеспечение интерфейса блочного доступа к распределенной файловой системе или другому хранилищу данных.

При этом важно отметить, что сервера серверной фермы SAN не имеют собственной подсистемы дисковой памяти и с точки зрения физической архитектуры, LUN выделенный на Target для сервера,  представляет собой обычный физический диск на который будет установлена операционная система для развертывания клиентского ПО. Следовательно программное обеспечение с помощью которого создается LUN должно поддерживать эмуляцию большинства типов файловых систем в том числе гипервизоров систем виртуализации, например, VMware.

К моему удивлению все это было объединено в проекте ESOS. В его основе был положен проект SCST (Generic SCSI Target \linebreak Subsystem for Linux)~\cite{Kliga3}, дополненный всеми необходимыми инструментами управления дисками (включая поддержку RAID-массивов), пакетами драйверов поддержки инфраструктуры SAN (оборудования производства QLogic, Emulex и других вендоров), мониторинга работы сервера и компонентами для поддержки пулов хранения в формате VMFS VMware ESX/ESXi, томов Windows NTFS и дисков Linux. Что дает возможность создавать хранилища данных (Data Storage) для SAN на базе серверов модели DAS~\cite{Kliga1}.
Также в состав компонентов ESOS входят модули поддержки распределенной файловой системы Ceph (в частности используется Ceph RBD в качестве оконечных устройств хранения данных) и поддержки кластеризации (Pacemaker\&Corosync и DRBD).

Исходные коды ESOS находятся на GitHub ~\cite{Kliga4} и компилируется виде Linux-дистрибутива с очень компактным кодом и распространяется под лицензией GPL. Ключевой особенностью ESOS является его полное резидентное расположении в памяти, при этом загрузка осуществляется с USB-накопителя и не требует отдельного диска для установки дистрибутива. При этом реализованы механизмы восстановления работоспособности системы в случае сбоев.

\begin{thebibliography}{99}

\bibitem{Kliga1} А. Клыга. Обзор решений на рынке открытого ПО для создания СХД // Материалы конференции LVEE Winter 2016, Минск, 12-14 ферваля 2016 г. \url{https://lvee.org/ru/abstracts/181}
\bibitem{Kliga2} ESOS: ESOS -- Enterprise Storage OS // ESOS \url{http://www.esos-project.com}
\bibitem{Kliga3} SCST GENERIC SCSI TARGET SUBSYSTEM FOR LINUX // SCST Project \url{http://scst.sourceforge.net/}
\bibitem{Kliga4} The project ESOS on GitHub // ESOS on GitHub \url{https://github.com/parodyne/esos}
\end{thebibliography}
\end{document}
