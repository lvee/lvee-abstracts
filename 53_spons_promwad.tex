\documentclass[10pt, a5paper]{article}
\usepackage[T2A]{fontenc}
\usepackage{ucs}
\usepackage[utf8x]{inputenc}
\usepackage[polish,english,russian]{babel}
\usepackage{hyperref}
\usepackage[inner=2cm,top=1.8cm,outer=2cm,bottom=2.3cm,nohead]{geometry}
\usepackage{listings}
\usepackage{graphicx}
\usepackage{wrapfig}
\usepackage{longtable}
\usepackage{indentfirst}
\frenchspacing
\usepackage{fixltx2e} %text sub- and superscripts
\usepackage{icomma} % коскі ў матэматычным рэжыме
\PreloadUnicodePage{4}

\newcommand{\longpage}{\enlargethispage{\baselineskip}}
\newcommand{\shortpage}{\enlargethispage{-\baselineskip}}

\def\switchlang#1{\expandafter\csname switchlang#1\endcsname}
\def\switchlangbe{
\let\saverefname=\refname%
\def\refname{Літаратура}%
\def\figurename{Іл.}%
}
\def\switchlangen{
\let\saverefname=\refname%
\def\refname{References}%
\def\figurename{Fig.}%
}
\def\switchlangru{
\let\saverefname=\refname%
\let\savefigurename=\figurename%
\def\refname{Литература}%
\def\figurename{Рис.}%
}

\hyphenation{admi-ni-stra-tive}
\hyphenation{ex-pe-ri-ence}
\hyphenation{fle-xi-bi-li-ty}
\hyphenation{Py-thon}
\hyphenation{ma-the-ma-ti-cal}
\hyphenation{re-ported}
\hyphenation{imp-le-menta-tions}
\hyphenation{pro-vides}
\hyphenation{en-gi-neering}
\hyphenation{com-pa-ti-bi-li-ty}
\hyphenation{im-pos-sible}
\hyphenation{desk-top}
\hyphenation{elec-tro-nic}
\hyphenation{com-pa-ny}
\hyphenation{de-ve-lop-ment}
\hyphenation{de-ve-loping}
\hyphenation{de-ve-lop}
\hyphenation{da-ta-ba-se}
\hyphenation{plat-forms}
\hyphenation{or-ga-ni-za-tion}
\hyphenation{pro-gramming}
\hyphenation{in-stru-ments}
\hyphenation{Li-nux}
\hyphenation{en-vi-ron-ment}
\hyphenation{Te-le-pathy}
\hyphenation{Li-nux-ov-ka}

\def\progref!#1!{\texttt{#1}}
\renewcommand{\arraystretch}{2} %Іначай формулы ў матрыцы зліпаюцца з лініямі
\usepackage{array}

\def\interview #1 (#2), #3, #4, #5\par{

\section[#1, #3, #4]{#1, #5}
\def\qname{LVEE}
\def\aname{#1}
\def\q ##1\par{{\noindent \bf \qname: ##1 }\par}
\def\a{{\noindent \bf \aname: } \def\qname{L}\def\aname{#2}}
}

\begin{document}
\title{Голос спонсора: инновационная компания Promwad}
%\author{}
\date{}
\maketitle
%\begin{figure}[ht]
%\includegraphics[width=3.5cm]{53_spons_promwad.pdf}
%\end{figure}
Инновационная компания Promwad разрабатывает электронику для массового производства: от аппаратного и программного решения до дизайна корпуса. Её клиенты "--- компании из России, стран Европейского союза, США и Канады. 
%Специалисты дизайн"=центра электроники Promwad создают планшетные и мобильные ПК, приставки для цифрового ТВ, встраиваемые системы, навигационные ГЛОНАСС/GPS"=устройства, электронику для датакома, автомобилей и авионики. Один из основных векторов компании "--- разработка цифровых платформ на базе систем на кристалле, адаптация дистрибутивов Embedded Linux, создание пакетов поддержки аппаратуры (BSP) и встраиваемого ПО. 
Роман Пахолков, основатель и руководитель инновационной компании Promwad, рассказывает об истории компании, полученном опыте и планах на будущее.

\subsubsection*{Promwad: история создания, численность и структура}
Компания основана в 2004 году. Но в реальности ее история началась раньше, еще в 2002 году в процессе разработки новой универсальной измерительной платформы для серии приборов, используемых для измерений волоконно-оптических линиях связи. Вокруг этого амбициозного и сложного проекта собрались практически все белорусские специалисты по системам на кристалле (СнК) и операционной системе Embedded Linux. %Прототипы устройства были реализованы еще на StrongARM, одном из первых СнК для мобильных устройств.

%По окончании работы над проектом стало понятно, что для продуктовой компании, в стенах которой создавался новый прибор, нерентабельно содержание такого количества инженеров с глубокими специализациями. Потому и было принято решение о создании отдельного независимого бизнеса для разработки сложных электронных устройств на острие технологий.

На сегодняшний день в Promwad работает высококвалифицированная команда из 60 специалистов. В сентябре прошлого года мы открыли подразделение Promwad Mobile, нацелив его  на разработку мобильных приложений для телефонов, планшетных ПК, ридеров, автонавигаторов и др. мультимедийной техники.

%За 6 лет с момента основания компании в нашей команде собралось около 60 человек, а организационная структура выросла до 8 отделов. Также на нынешнем этапе развития сформировались предпосылки для создания небольшого холдинга, уже сегодня в него входит несколько компаний и представительств в разных точках мира, а также небольшая компания, занимающаяся OEM/ODM"=производством электроники.

%В сентябре прошлого года мы открыли подразделение Promwad Mobile, нацелив его  на разработку мобильных приложений для телефонов, планшетных ПК, ридеров, автонавигаторов и др. мультимедийной техники. Приоритет нового подразделения –Android и Linux, большой опыт и экспертные знания специалистов позволяют нам решать нетривиальные задачи для любых устройств на базе этих операционных систем.

%С момента создания и до сегодняшнего дня компания работает исключительно за счет собственных инвестиций. Все эти годы мы реинвестируем большую часть прибыли в собственное развитие, а с прошлого года – в новые подразделения и компании холдинга.

\subsubsection*{О бизнес"=модели независимого дизайн"=центра электроники}
Я глубоко убежден, что такая бизнес"=модель перспективна. Компания Promwad дорожит своей репутацией контрактного разработчика, предоставляющего услуги разработки на заказ и не претендующего на интеллектуальное владение продуктом или его частью. 
Во"=первых, такой подход позволяет участвовать в проектах на острие технологий, обогащая знания наших инженеров, позволяя им впитывать мировой опыт. Во"=вторых, изначально позиционируя себя как независимую компанию, мы построили эффективные бизнес"=процессы, отобрали лучших специалистов и заставили такой бизнес быть рентабельным только за счет разработок. В"=третьих, в условиях дефицита квалифицированных инженеров наши услуги при должном уровне сервиса постоянно растут в цене.

В компании Promwad всегда существовала четкая стратегия развития, следуя которой мы смогли выжить и стать полноценным развивающимся дизайн-центром. Ключевыми моментами этой стратегии являются следующие компоненты:
\begin{enumerate}
\item Нашей технической специализацией является проектирование аппаратуры на базе современных микропроцессоров и СнК, в основном с применением Embedded Linux, и стремление заниматься средними и крупными проектами в этих сферах.
\item Ориентация компании на привлечение лучших инженерных кадров и их постоянное развитие.
\item Наша клиентоориентированность и стремление к кооперации, не смотря на регулярные неудачи в этом процессе.
\end{enumerate}

\subsubsection*{О работе с фрилансерами. Плюсы и минусы}

%Действительно, в 2004--2005 годах мы прибегали к услугам удаленных разработчиков, т.~к. некоторые задачи требовали углубленной специализации, а содержать таких сотрудников в штате мы не могли. Но вот 
Уже более 4"=х лет мы не прибегаем к услугам внештатных специалистов, и все работы, связанные с разработкой электроники, выполняются штатными сотрудниками Promwad на полной занятости. Только такой подход позволяет получать гарантированные результаты по срокам исполнения и качеству проектов. 

Своим начинающим коллегам, стремящимся к построению подобных компаний, я бы посоветовал никогда не надеяться на услуги фрилансеров в области основной деятельности, их работа не поддается измерению и не имеет гарантированного результата.

В то же время я хочу отметить, что при решении сложных инженерных, технических и алгоритмических задач мы прибегаем к использованию внешних консультантов, экспертов в узких областях. Но непосредственная реализация задач по проектированию выполняется штатными инженерами. %Благо на сегодняшний день у нас в коллективе присутствует более 10 различных инженерных специализаций.

\subsubsection*{О планах развития компании Promwad}

%Важно иметь стратегию развития компании. Сейчас мы смотрим вперед не только на 2--3 года, но также думаем о том, как может трансформироваться наша группа компаний в 5--10"=летнем цикле. В ближайшее время мы нацелены масштабировать компанию до 60--80 инженеров на полной занятости, это позволит нам стать одним из крупнейших независимых дизайн-центров в Европе. При таком росте неизбежно возникнет такой важный вопрос, как обеспечение качества проводимых работ и их сертификация по международным стандартам. Мы активно работаем над этим задачами уже сегодня. 

%Если говорить о 2--3"=летнем цикле, то мы сконцентрируемся на дальнейшей диверсификации бизнес-портфеля и будем работать над появлением в пуле наших заказов известных транснациональных корпораций. Так, за последние 2 года мы уже прошли несколько аудитов зарубежных компаний "--- известных мировых брендов. 

%Если же говорить о нашей стратегии на ближайшие 5 лет и о горизонтальном развитии компании Promwad, то мы не изменим выбранному направлению контрактного разработчика, а лишь постараемся укрепить его распределенной по миру сетью офисов. Уже сейчас мы имеем торговые представительства в Скандинавии, Канаде, Великобритании и полноценный офис в России. Но наша цель "--- это не создание агентской сети продаж, а развитие подразделений в разных странах мира, которые будут специализироваться на определенных этапах или технологиях разработки и производства электроники.

%Если говорить о росте Promwad в вертикальном направлении, тут нужно отметить развитие холдинга и наших брендов. Тут мы оценим ситуацию в рамках 10"=летнего цикла. В первую очередь должна заработать наша идея с российским OEM/ODM"=производителем, для которого мы создали отдельную компанию. Примерно через 5 лет, в течение которых она планирует выйти не только на российские, но и мировые рынки, мы надеемся выделить отдельную нишу для уникального товара (скорее всего, это будет мультимедийное решение). Именно на этом этапе можно будет подумать о создании бренда для этой группы товаров и его раскрутке на соответствующих рынках.

В заключении хотелось бы отметить, что мы разделяем разработку, производство и бренды на отдельные бизнесы. Так, компания Promwad всегда будет заниматься контрактной разработкой на острие технологий. Мы надеемся, что наши клиенты всегда будут довольны качеством наших услуг, получая свои дивиденды от продажи успешных продуктов, а мы сможем сохранить свою репутацию контрактного разработчика. Этот тот фундамент, который позволяет нам оценивать будущее электронной индустрии свежим взглядом.
%\newpage
%\begin{figure}[h!]
%\includegraphics[width=11cm]{54_spons_anakreon.pdf}
%\end{figure}
%\end{document}


