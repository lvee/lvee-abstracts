\documentclass[10pt, a5paper]{article}
\usepackage{pdfpages}
\usepackage{parallel}
\usepackage[T2A]{fontenc}
\usepackage{ucs}
\usepackage[utf8x]{inputenc}
\usepackage[polish,english,russian]{babel}
\usepackage{hyperref}
\usepackage{rotating}
\usepackage[inner=2cm,top=1.8cm,outer=2cm,bottom=2.3cm,nohead]{geometry}
\usepackage{listings}
\usepackage{graphicx}
\usepackage{wrapfig}
\usepackage{longtable}
\usepackage{indentfirst}
\usepackage{array}
\newcolumntype{P}[1]{>{\raggedright\arraybackslash}p{#1}}
\frenchspacing
\usepackage{fixltx2e} %text sub- and superscripts
\usepackage{icomma} % коскі ў матэматычным рэжыме
\PreloadUnicodePage{4}

\newcommand{\longpage}{\enlargethispage{\baselineskip}}
\newcommand{\shortpage}{\enlargethispage{-\baselineskip}}

\def\switchlang#1{\expandafter\csname switchlang#1\endcsname}
\def\switchlangbe{
\let\saverefname=\refname%
\def\refname{Літаратура}%
\def\figurename{Іл.}%
}
\def\switchlangen{
\let\saverefname=\refname%
\def\refname{References}%
\def\figurename{Fig.}%
}
\def\switchlangru{
\let\saverefname=\refname%
\let\savefigurename=\figurename%
\def\refname{Литература}%
\def\figurename{Рис.}%
}

\hyphenation{admi-ni-stra-tive}
\hyphenation{ex-pe-ri-ence}
\hyphenation{fle-xi-bi-li-ty}
\hyphenation{Py-thon}
\hyphenation{ma-the-ma-ti-cal}
\hyphenation{re-ported}
\hyphenation{imp-le-menta-tions}
\hyphenation{pro-vides}
\hyphenation{en-gi-neering}
\hyphenation{com-pa-ti-bi-li-ty}
\hyphenation{im-pos-sible}
\hyphenation{desk-top}
\hyphenation{elec-tro-nic}
\hyphenation{com-pa-ny}
\hyphenation{de-ve-lop-ment}
\hyphenation{de-ve-loping}
\hyphenation{de-ve-lop}
\hyphenation{da-ta-ba-se}
\hyphenation{plat-forms}
\hyphenation{or-ga-ni-za-tion}
\hyphenation{pro-gramming}
\hyphenation{in-stru-ments}
\hyphenation{Li-nux}
\hyphenation{sour-ce}
\hyphenation{en-vi-ron-ment}
\hyphenation{Te-le-pathy}
\hyphenation{Li-nux-ov-ka}
\hyphenation{Open-BSD}
\hyphenation{Free-BSD}
\hyphenation{men-ti-on-ed}
\hyphenation{app-li-ca-tion}

\def\progref!#1!{\texttt{#1}}
\renewcommand{\arraystretch}{2} %Іначай формулы ў матрыцы зліпаюцца з лініямі
\usepackage{array}

\def\interview #1 (#2), #3, #4, #5\par{

\section[#1, #3, #4]{#1 -- #3, #4}
\def\qname{LVEE}
\def\aname{#1}
\def\q ##1\par{{\noindent \bf \qname: ##1 }\par}
\def\a{{\noindent \bf \aname: } \def\qname{L}\def\aname{#2}}
}

\def\interview* #1 (#2), #3, #4, #5\par{

\section*{#1\\{\small\rm #3, #4. #5}}

\def\qname{LVEE}
\def\aname{#1}
\def\q ##1\par{{\noindent \bf \qname: ##1 }\par}
\def\a{{\noindent \bf \aname: } \def\qname{L}\def\aname{#2}}
}

\begin{document}
\title{Пастильда - открытый аппаратный менеджер паролей}
\author{Иван Ларионов, Санкт-Петербург, Russian Federation}
\maketitle
\begin{abstract}
Pastilda - simple and cheap opensource device that allows a hardware safekeeping and entering usernames/passwords on all types of devices
\end{abstract}
\subsection*{Введение}

Хранение паролей "--- актуальный вопрос, волнующий пользователей и руководство IT отделов организаций. Существующие программные и аппаратные средства хранения паролей имеют ряд недостатков:

\begin{itemize}
  \item закрытый код снижает доверие и вероятность оперативного устранения уязвимостей;
  \item для автозаполнения нужно ставить дополнительный софт;
  \item после ввода мастер-пароля вся база открыта и доступна, в том числе для вредоносного ПО, что особенно актуально на недоверенных устройствах;
  \item использование мобильных приложений для хранения паролей все равно подразумевает ручной ввод с клавиатуры, например когда требуется залогиниться на стационарном ПК;
  \item автозаполнение невозможно в некоторых случаях (в bios, консоли, вход в систему)
\end{itemize}

\subsection*{Пастильда}

Устройство Пастильда призвано решить задачи безопасного хранения и ввода паролей на любых устройствах без указанных недостатков. В нем реализованы две прорывные идеи:

\begin{itemize}
  \item Пастильда является умным USB-USB переходником для клавиатуры. В пассивном режиме команды с клавиатуры транслируются насквозь, в активном "--- интерпретируются устройством.
  \item Меню устройства может быть вызвано в любой строке ввода, где расположен курсор, при переходе в активный режим.
\end{itemize}

\subsubsection*{Хранение паролей}

Для хранения и управления учетными записями с паролями используется база KeePass \url{http://keepass.info/}, хранящаяся на встроенном накопителе. С базой можно работать стандартными средствами из любой ОС, обновлять через облака. 
Для доступа к базе используется мастер-пароль и PIN-код. PIN-код позволяет при переходе в активный режим не вводить мастер-пароль, что усложняет взлом базы удаленно с подглядыванием. Мастер-пароль никогда не вводится непосредственно в ПК, а ОС не получает доступа ко всей базе.

\subsubsection*{USB}

Пастильда для ОС представляет собой составное USB-устройст\-во, содержащее MSD (функция USB-флеш для непосредственной работы с файлом базы данных через файловую систему) и HID (функция клавиатуры для транслирования команд в пассивном и активном режимах). Для клавиатуры Пастильда является хостом.

\subsubsection*{Меню}

Для отображения однострочного меню Пастильда вводит и стирает символы в активном поле ввода. Поле ввода может быть в браузере, командной строке, блокноте и так далее. Меню позволяет использовать функции: добавление записи в базу паролей, использование существующих пар логин-пароль. Для быстрого выбора среди множества записей планируется реализовать подсказки при вводе.

\subsubsection*{Open source}

Исходники опубликованы на GitHub \url{https://github.com/}\linebreak\url{thirdpin/pastilda}, причем как софт, так и аппаратная часть. Проектом активно интересуются частные лица и компании. Планируется развитие софта, разработка новой аппаратной версии.

\end{document}
